\documentclass[12pt]{article}
\usepackage[utf8]{inputenc}
\usepackage[a6paper]{geometry}
\usepackage[T2A]{fontenc}
\usepackage[russian]{babel}
\usepackage{indentfirst}
\usepackage{titlesec}
\usepackage{hyperref}

\title{Лекции о сущности религии}
\author{Людвиг Андреас фон Фейербах}
\date{Брукберг, 1 января 1851 г.}

\tolerance=10000
\hbadness=10000
\vbadness=10000

\begin{document}

\maketitle

\tableofcontents

\phantomsection
\addcontentsline{toc}{section}{Предисловие}
\section*{Предисловие}

Лекции, которые я здесь отдаю в печать, были прочитаны мною с 1 декабря 1848 г. по 2 марта 1849 г. в городе --- не в университете --- Гейдельберге по предложению тамошних студентов, но перед смешанной аудиторией. 

Я их выпускаю в качестве восьмого тома моего <<Полного собрания сочинений>>  потому что закончить это издание <<Сущностью христианства>> --- было бы бессмысленно; это совершенно не соответствовало бы тому плану и той идее, которые лежат в основе моего собрания сочинений. Соответственно этому я сделал <<Сущность христианства>> своим первым, то есть самым ранним, сочинением и поэтому сознательно начал собрание своих сочинений с <<Разъяснений и дополнений к ,,Сущности христианства``>>. Но так как <<Сущность христианства>> также должна была войти в собрание моих сочинений, то она теперь в печати появляется как мое последнее сочинение, то есть как выражение моей последней воли и мысли. Эта обманчивая видимость должна быть вскрыта, христианство должно быть поставлено на то место, которое ему принадлежит в действительности. Это я делаю в этих лекциях, которые примыкают к дополнениям первого тома; эти лекции дальше излагают, развивают и объясняют те мысли, которые выражены очень кратко в <<Сущности религии>>. 

Так как я, как известно, не христианин, то есть не принадлежу к жвачным животным, ибо, как говорит Лютер, <<христианин пережевывает пищу, как это делают овцы>>  то, хотя я и отдаю эти лекции в печать в том виде, в каком они были прочитаны, однако я их снабдил новыми доказательствами, разъяснениями и примечаниями, причем по мере возможности вычеркнул все то, что казалось мне простой жвачкой, --- так, я выпустил целую лекцию, которая относилась к моим <<Основам философии>>. Однако, я оставил первые лекции, хотя они не содержат ничего такого, что не было бы напечатано в моих других сочинениях, но что было выражено в них другими словами и отрывочно; я это сделал, исходя из того предположения, что эти мои лекции могут попасть в руки к таким людям, которые не имеют остальных моих сочинений, по крайней мере философских. 

Что эти лекции появляются лишь теперь, это никого не должно удивить. Что может быть более своевременным теперь, чем напоминание о 1848 г.? При этом напоминании я, однако, должен заметить, что эти лекции были единственным проявлением моей общественной деятельности в так называемую революционную эпоху. Во всех, как политических, так и неполитических, волнениях и совещаниях этой эпохи, свидетелем которых я был, я принимал участие лишь в качестве критического зрителя или слушателя по той простой причине, что я не могу принимать никакого деятельного участия в безуспешных, а следовательно, и бессмысленных предприятиях. Но я в самом начале этих волнений и совещаний уже предвидел или предчувствовал их исход. Известный французский литератор недавно поставил мне вопрос, почему я не принимал участия в революционном движении 1848 г.? Я ответил: Господин Тайандье! Если революция вспыхнет вновь и я приму в ней деятельное участие, тогда вы можете быть, к ужасу вашей религиозной души, уверены, что эта революция победоносная, что пришел день страшного суда над монархией и иерархией. Но, к сожалению, я не доживу до этой революции. Однако я принимаю участие в великой и победоносной революции, но в той революции, истинные действия и результаты которой обнаружатся лишь в течение веков, ибо, знайте, господин Тайандье: согласно моему учению, которое не признает никаких богов, а следовательно, также и никаких чудес в области политики, согласно моему учению, которого вы совершенно не знаете и в котором вы ничего не понимаете, --- хотя вы позволяете себе судить обо мне, вместо того, чтобы изучать меня, --- согласно моему учению, пространство и время суть основные условия всякого бытия и сущности, всякого мышления и деятельности, всякого процветания и успеха. Не потому, что парламенту не хватало религиозности, как комическим образом уверяли в баварском рейхсрате, --- большинство его членов были религиозные люди, а ведь господь бог также соображается с большинством --- революция имела столь постыдный и столь безрезультатный конец, --- но потому, что у нее не было никакого чувства места и времени. 

Мартовская революция все еще была плодом, хотя и незаконным, христианской веры. Конституционалисты верили, что стоило только монарху сказать: <<Да будет свобода, да будет ,,справедливость``!>> --- и тотчас настанут справедливость и свобода. Республиканцы верили, что стоило только пожелать республику, чтобы ее тем самым вызвать к жизни; верили, таким образом, в сотворение республики из ничего. Первые переместили в область политики христианскую веру в чудодейственное слово, вторые --- христианскую веру в чудесные действия. Но знайте, господин Тайандье, обо мне хотя бы столько, что я абсолютно неверующий. После этого как же вы можете привести в связь мои дух с духом парламента, мою сущность с сущностью мартовской революции.

\phantomsection
\addcontentsline{toc}{section}{Первая лекция}
\section*{Первая лекция}

Приступая к своим <<Лекциям о сущности религии>>  я должен прежде всего признаться в том, что только призыв, определенно выраженное желание части учащейся здесь молодежи побудили меня сделать этот шаг, преодолев свое собственное упорное сопротивление. 

Мы живем в такое время, когда нет надобности, как было когда-то в Афинах, издавать закон, гласящий, что каждый в момент восстания обязан определить, на чьей он стороне; мы живем в такое время, когда каждый --- и тот, который воображает себя наиболее беспартийным, даже против собственного сознания и воли, является, хотя бы только в теории, человеком партии; мы живем в такое время, когда политический интерес поглощает собой все остальные, и политические события держат нас в непосредственном напряжении и возбуждении; в такое время, когда --- особенно на нас, неполитических немцев возлагается обязанность забыть все ради политики. Ибо как отдельный человек ничего не достигнет и не сделает, если у него нет силы в течение некоторого времени сосредоточиться над тем, в чем он собирается чего-либо достигнуть, так и человечество в известные эпохи должно позабыть ради предстоящей ему задачи все остальные, ради одной деятельности всякую другую, если оно думает осуществить что-то дельное, что-то законченное. Правда, предмет этих лекций, религия, теснейшим образом связан с политикой, но наш главнейший интерес в настоящее время --- не теоретическая, а практическая политика; мы хотим непосредственно, действуя, участвовать в политике; нам недостает спокойствия, настроения, охоты, чтобы читать и писать, чтобы поучать и учиться. Мы достаточно долгое время занимались и довольствовались тем, что говорили и писали, мы требуем, чтобы, наконец, слово стало плотью, дух материей, довольно с нас как философского, так и политического идеализма; мы хотим теперь быть политическими материалистами. 

К этим общим, коренящимся во времени, причинам моего нежелания преподавать присоединяются еще и личные соображения. Я по натуре, --- если брать ее теоретическую сторону, --- гораздо менее предназначен быть учителем, чем мыслителем, исследователем. Учитель не устает и не должен уставать тысячу раз повторять одно и то же, --- с меня же довольно сказать что-либо один раз, если только у меня имеется сознание, что я это хорошо сказал. Меня интересует и приковывает к себе предмет лишь до тех пор, пока он мне ставит еще препятствия, пока я им целиком не овладел, пока мне с ним еще как бы приходится бороться. Но как только я его одолел, то я спешу к другому, новому предмету, ибо мой умственный взор не ограничен пределами одной специальности, одного предмета: меня интересует все человеческое. Правда, я меньше всего ученый скупец или эгоист, собирающий и сберегающий только для себя; нет, то, что я делаю и мыслю для себя, я должен делать и мыслить и для других. Но я чувствую потребность поучать чему-либо других лишь до тех пор, пока, уча их, я и сам поучаюсь. С предметом же этих лекций, с религией, я давно закончил свои счеты; в своих сочинениях я исчерпал его со всех существеннейших или, по крайней мере, труднейших сторон. Затем, я по своей природе совсем не любитель ни многописания, ни многоговорения. Я, собственно говоря, могу говорить и писать только тогда, когда предмет держит меня в состоянии аффекта, когда он меня воодушевляет. Но аффект воодушевления не зависит от воли, не регулируется по часам, не является в определенные, заранее намеченные дни и часы. Я вообще могу говорить и писать только о том, о чем, мне кажется, стоит говорить и писать. Стоит же, на мой взгляд, говорить и писать лишь о том, что не разумеется само собой или что не исчерпано уже другими. Поэтому я беру, даже и в своих писаниях, из предмета всегда лишь то, о чем нет ничего в других книгах, по крайней мере ничего, меня удовлетворяющего, исчерпывающего. Остальное я оставляю в стороне. Мой ум поэтому афористичен, в чем меня упрекают мои критики, но афористичен совсем в другом смысле и по другим совсем основаниям, чем они полагают; афористичен, потому что критичен, потому что различает сущность от видимости, необходимое --- от излишнего. Наконец, я прожил долгие годы, целых двенадцать лет, в деревенском уединении, исключительно занятый наукой и писанием, и благодаря этому утратил дар речи, устного изложения или, во всяком случае, не потрудился его в себе развить, ибо я не думал, что мне придется опять, --- говорю опять, так как я в прежние годы читал лекции в одном из Баварских университетов, --- да еще в университетском городе, устное слово сделать орудием своей деятельности. 

Время, когда я навсегда сказал <<прости>> официально академической карьере и поселился в деревне, было такое страшно печальное и мрачное время, что подобная мысль никоим образом не могла возникнуть в моей голове. Это было то время, когда все общественные отношения были до такой степени отравлены и заражены, что свободу и здоровье духа можно было сохранить, лишь отказавшись от какой бы то ни было государственной службы, от всякой публичной роли, даже роли приват-доцента, когда все продвижения на государственной службе, каждое начальственное разрешение, даже разрешение читать лекции (venia docendi) достигались лишь ценой политического сервилизма и религиозного обскурантизма, когда свободно было только печатное научное слово, но и оно было свободно в чрезвычайно ограниченных пределах и свободно не из-за уважения к науке, а, скорее, из-за низкой оценки ее, --- вследствие ее действительной или мнимой невлиятельности и безразличия для общественной жизни. Что же было делать в такое время, особенно сознавая, что питаешь мысли и настроения, враждебные господствующей правительственной системе, как не замкнуться в одиночестве и не воспользоваться печатным словом как единственным средством, позволяющим не подвергаться наглости деспотической государственной власти, конечно, налагая на себя при этом самоограничение и сохраняя самообладание. 

Впрочем, отнюдь не одно отвращение к политике загоняло меня в одиночество и обрекало лишь на писание. Я жил в непрерывной внутренней оппозиции к политической правящей системе того времени, но я также находился в оппозиции и к идейным правящим системам, то есть к философским и религиозным течениям. Чтобы дать себе ясный отчет в существе и причинах этого расхождения, я нуждался в длительном, ничем не нарушаемом досуге. Но где же можно лучше найти таковой, как не в деревне, где человек, свободный от всякой сознательной и бессознательной независимости, от расчетов, тщеславия, удовольствий, интриг и сплетен городской жизни, предоставлен только самому себе. Кто верит в то, во что верят другие, кто учит и мыслит, чему учат и что мыслят другие, кто, короче говоря, живет в научном или религиозном единении с другими, тому не нужно от них телесно отделяться, у того нет потребности в уединении; но она есть у того, кто идет своей дорогой или кто порвал со всем миром, верующим в бога, и хочет этот свой разрыв оправдать и обосновать. Для этого необходимо свободное время и свободное место. Только по незнанию человеческой природы можно думать, что в каждом месте, при всякой обстановке, в каждом положении и при всякого рода отношениях человек способен свободно мыслить и исследовать, что для этого ничего другого не требуется, как только его собственная воля. Нет, для действительно свободного, беспощадного, необычного мышления, --- если только это мышление должно быть действительно плодотворным и решающим, --- требуется и необычная, свободная жизнь, не отступающая ни перед какими препятствиями. И кто духовно хочет дойти до основы всех человеческих вещей, тот и чувственно, телесно должен опереться на эту основу. Основа же эта --- природа. Только в непосредственном общении с природой выздоравливает человек и отбрасывает от себя все надуманные сверх или противоестественные представления и фантазии. 

Но как раз тот, кто годы проводит в одиночестве, --- хотя бы и не в абстрактном одиночестве христианского анахорета, или монаха, а в одиночестве гуманном, --- и только в письменной форме поддерживает связь свою с миром, тот утрачивает охоту и способность говорить, ибо существует огромная разница между устным и письменным словом. Устное имеет дело с определенной реально-присутствующей публикой, письменное же --- с неопределенной, отсутствующей, существующей только в представлении писателя. Устная речь обращена к человеку, письменная же --- к человеческому духу, ибо люди, для которых я пишу, для меня существа, живущие лишь в духе, в представлении. Писанию не хватает поэтому всех прелестей, вольностей и, так сказать, общественных добродетелей, присущих устному слову; оно приучает человека к строгому мышлению, приучает его не говорить ничего такого, что не могло бы выдержать критики; но именно благодаря этому оно делает его несловоохотливым, ригористичным, требовательным к себе, колеблющимся в выборе слов, неспособным легко выражаться. Я обращаю, господа, ваше внимание на это, --- на то, что я лучшую часть моей жизни провел не на кафедре, а в деревне, не в университетских аудиториях, а в храме природы, не в салонах и не на аудиенциях, но в уединении моего рабочего кабинета, чтобы вы не приходили на мои лекции с ожиданиями, в которых вы почувствовали бы себя обманутыми, чтобы вы не ждали от меня красноречивого, блестящего изложения. 

Так как писательская деятельность была до сих пор единственным орудием моей общественной деятельности, так как я ей посвятил прекраснейшие часы и лучшие силы моей жизни, так как я лишь в ней выявил свой дух, ей одной обязан своим именем, своей известностью, то естественно, конечно, что я свои сочинения положил в основу этих лекций и беру их своей руководящей нитью, что я придаю своим сочинениям роль текстов, а своим устам роль комментатора и что таким образом я возлагаю на свои лекции задачу изложить, пояснить, доказать высказанное мною в сочинениях. Я считаю это тем более подходящим, что я в своих сочинениях привык выражаться с величайшей сжатостью и резкостью, что я в них ограничиваюсь лишь самым существенным и необходимым, опускаю все скучные посредствующие звенья и предоставляю собственному разумению читателя делать самоочевидные пополнения и выводы, но что именно через это я рискую вызвать величайшие недоразумения, как это достаточно доказано критиками моих писаний. Но прежде чем назвать те сочинения, которые я беру текстом своих лекций, я считаю целесообразным дать коротенький обзор всех моих литературных работ. 

Мои сочинения подразделяются на сочинения, имеющие своим предметом философию вообще, и на сочинения, которые трактуют главным образом религию или философию религии. К первым относятся: моя <<История новой философии>> от Бэкона до Спинозы; мой <<Лейбниц>>; мои <<Пьер Бэйль>>  очерк из истории философии и человечества; мои философско-критические работы и основные философские положения. К другой категории принадлежат --- мои <<Мысли о смерти и бессмертии>>; <<Сущность христианства>>; наконец, разъяснения и дополнения к <<Сущности христианства>>. Но, несмотря на это различие в моих сочинениях, все они, строго говоря, имеют одну цель, одну волю и мысль, одну тему. Эта тема есть именно религия и теология и все, что с ними связано. Я принадлежу к людям, без сравнения более предпочитающим плодотворную односторонность бесплодной, ни к чему не нужной разносторонности и многописанию; я принадлежу к людям, которые всю свою жизнь имеют в виду одну цель и на ней все сосредоточивают, которые, правда, очень много и очень многое изучают и постоянно учатся, но одному лишь учат, об одном лишь пишут, убежденные, что только это единство является необходимым условием для того, чтобы что-нибудь исчерпать до конца и провести в жизнь. Вот почему соответственно этому я во всех моих сочинениях никогда не упускаю из виду проблем религии и теологии; они были всегда главным предметом моего мышления и моей жизни, хотя, разумеется, трактовал я различно, в разное время, соответственно менявшейся у меня точке зрения. Я должен, однако, отметить, что в первом издании моей <<Истории философии>> отнюдь не из-за соображений политических, а из-за юношеских капризов и антипатий --- я выпустил в печати все, непосредственно относящееся к теологии, во втором же издании, вошедшем в мое собрание сочинений, я восполнил эти пробелы, но уже не с моей прежней, а с нынешней точки зрения. 

Первое имя, которое упоминается здесь в связи с религией и теологией, есть имя Бэкона Веруламского, которого многие не без основания называют отцом современной философии и современного естествознания. Он многим служит образцом благочестивого, христианского естествоиспытателя, ибо он торжественно признал, что не хочет применять своей критики профана (что было им сделано в области естествознания), к вопросам религии и теологии, он, дескать, только применительно к человеческим вещам неверующий, в вещах же божественных он самый верноподданнически верующий. Ему принадлежит знаменитое изречение: <<поверхностная философия уводит от бога, более глубокая приводит к нему обратно>>  --- изречение, которое, как и многие другие положения прежних мыслителей, было когда-то и в самом деле истиной, но перестало теперь уже быть таковой, хотя нашими историками, не делающими различия между прошедшим и настоящим, оно и до сих пор за таковую признается. Я показал, однако, в моем изложении, что те принципы, которые Бэкон признает в теологии, он отрицает в физике, что старый, телеологический взгляд на природу --- учение о намерениях и целях в природе --- есть необходимое последствие христианского идеализма, производящего природу от существа, действующего намеренно и сознательно, что Бэкон христианскую религию вытеснил из той ее старой позиции, охватывающей весь мир, которую она занимала у истинно верующих в средние века, что он лишь как частный человек осуществлял свой религиозный принцип, а отнюдь не как физик, не как философ, не как исторически действующая личность, и что поэтому неправильно превращать Бэкона в христианско-религиозного естествоиспытателя. Второй, интересный для философии религии человек --- это младший современник и друг Бэкона, Гоббс, знаменитый главным образом благодаря своим политическим взглядам. Он тот из современных философов, к кому впервые приложен был страшный эпитет: атеист. Ученые господа, впрочем, в прошлом веке долго спорили, был ли он действительно атеистом. Я же разрешил этот спор в том смысле, что признаю Гоббса одинаково и атеистом, и теистом, ибо Гоббс, как и весь современный мир, принимает, правда, бога, но этот гоббсовский бог все равно, что и не бог, ибо вся его действительность есть телесность, божественность же его, --- так как Гоббс не может указать для нее никаких телесных качеств, --- с принципиально-философской точки зрения есть только слово, а не сущность. Третьей значительной личностью, в религиозном отношении, однако, не представляющей существенных особенностей, является Декарт. Его отношение к религии и теологии я впервые охарактеризовал в <<Лейбнице>> и <<Бэйле>>  потому что Декарт лишь после появления моего первого тома был провозглашен образцом религиозного и, в частности, католически-религиозного философа. Но я и относительно него доказал, что Декарт-философ и Декарт-верующий --- два лица, целиком друг другу противоречащие. Наиболее значительные для философии религии, наиболее оригинальные фигуры, рассмотренные мной в этом томе, это --- Яков Беме и Спиноза; оба они отличаются от вышеназванных философов тем, что являют собою не только противоречие между верой и разумом, но оба устанавливают самостоятельный религиозно-философский принцип. Первый из них --- Яков Беме кумир философствующих теологов или теистов, другой --- Спиноза --- кумир теологизирующих философов или пантеистов. Первого его поклонники в самое последнее время прославили как вернейшее целительное средство против того яда моего учения, которое и составит содержание этих моих лекций. Я же недавно, в моем втором издании, сделал Якова Беме вторично объектом самого основательного изучения. Мое вторичное изучение привело меня, однако, к такому же выводу, как раньше, а именно, что тайна его теософии заключается, с одной стороны, в мистической натурфилософии, с другой --- в мистической психологии; так что у него не только нельзя найти опровержения моей точки зрения, но, наоборот, находится подтверждение ее, то есть точки зрения, согласно которой вся теология разлагается на учение о природе и учение о человеке. Заключением в названном томе является Спиноза. Он --- единственный из новых философов, положивший первые основы для критики и познания религии и теологии; он --- первый, который определенно выступил против теологии; он --- первый, который классическим образом формулировал мысль, что нельзя рассматривать мир как следствие или дело рук существа личного, действующего согласно своим намерениям и целям; он --- первый, который оценил природу в ее универсальном религиозно-философском значении. Я с радостью поэтому принес ему дань моего удивления и почитания; я его упрекал только в одном --- в том, что это существо, действующее не согласно поставленным целям, не соответственно своим сознанию и воле, он --- еще в плену у старых теологических представлений --- определил как совершеннейшее, как божественное существо и тем отрезал себе путь к развитию, что он рассматривал сознательное человеческое существо лишь как часть, лишь как <<модус>>  выражаясь языком Спинозы, вместо того, чтобы рассматривать его, как предельное завершение бессознательного существа. 

Антипод Спинозы --- Лейбниц; ему я посвятил отдельный том. Если Спинозе принадлежит честь превращения теологии в служанку философии, то первому германскому философу нового времени, а именно Лейбницу, наоборот, принадлежит честь и бесчестие за то, что философия попала опять под башмак теологии. Это Лейбниц выполнил в особенности в своем знаменитом труде <<Теодицея>>. Лейбниц, как известно, написал эту книгу из галантности по отношению к одной прусской королеве, потревоженной в ее вере сомнениями Бэйля. Но подлинная дама, для которой Лейбниц написал эту книгу и за которой он ухаживал, --- это теология. Тем не менее он не удовлетворил теологов. Лейбниц во всех случаях держался обеих партий и именно поэтому не дал удовлетворения ни одной. Он не хотел никого обидеть, никого задеть; его философия есть философия дипломатической галантности. Даже монады, то есть существа, из которых состоят, согласно Лейбницу, все предметы, воспринимаемые нашими чувствами, --- даже монады, не оказывают друг на друга никакого физического воздействия, только бы не повредить которой-либо из них. Но кто не хочет, хотя бы и не намеренно, обижать и задевать, тот лишен всякой энергии и всякой действенной силы, ибо нельзя шагу ступить, не раздавливая существ, нельзя выпить ни капли воды, не проглотив инфузорий. Лейбниц --- человек промежуточный между средневековьем и новым временем, он как я его назвал --- философский Тихо-де-Браге, но именно благодаря этой своей нерешительности он и до сего дня --- кумир всех нерешительных, лишенных энергии голов. Поэтому уже в своем первом издании, вышедшем в 1837 году, я сделал теологическую точку зрения Лейбница объектом критики, а вместе с ней и всю теологию вообще. Точка зрения, с которой я производил эту критику, была собственно говоря, спинозистская или абстрактно-философская и заключалась она в том, что я строго различал между теоретической и практической точками зрения человека, приурочивая первую к философии, а вторую --- к теологии и религии. Стоя на практической точке зрения, человек говорю я --- все вещи относит только к себе, к своей пользе и выгоде; стоя же на теоретической точке зрения он эти вещи относит к ним самим. Необходимо поэтому, говорю я там, проводить существенное различение между теологией и философией; кто смешивает их, смешивает по существу различные точки зрения и создает в силу этого нечто уродливое. Рецензенты этой моей работы долго останавливались на этом различении; но они упустили из виду, что уже Спиноза в своем <<Теолого-политическом трактате>> рассматривает и критикует теологию и религию с этой же точки зрения и что даже Аристотель, если бы он только сделал теологию предметом своей критики, не мог бы иначе ее критиковать. Впрочем, эта точка зрения, с которой я тогда критиковал теологию, отнюдь не есть точка зрения моих позднейших сочинений, отнюдь не есть моя последняя и абсолютная точка зрения, а лишь относительная, исторически-обусловленная. Поэтому-то я и подверг в новом издании моего сочинения <<Изложение и критика философии Лейбница>> теодецею и теологию Лейбница, равно как и его связанную с ними пневматологию или учение о духе, новому критическому рассмотрению.

\phantomsection
\addcontentsline{toc}{section}{Вторая лекция}
\section*{Вторая лекция}

Как Лейбниц --- антипод Спинозы, так антиподом Лейбница --- в теологическом отношении --- является французский ученый и скептик Пьер Бэйль. Audiatur et altera pars (следует выслушать и другую сторону) применимо не только в юриспруденции, но и в науке вообще. Соответственно этому изречению я дал место в ряду своих работ вслед за верующим или, по крайней мере, верующим в мысль немецким философом, неверующему или, во всяком случае, сомневающемуся французскому философу. Впрочем, работа эта была вызвана совсем не одним только научным, но также и практическим интересом. Как и вообще мои работы, так и мои <<Бэйль>> обязаны своим происхождением противоречию с тем временем, когда пытались насильственно отбросить запуганное человечество во тьму прошедших столетий. Моя книга о Бэйле появилась в то время, когда в Баварии и в Рейнских провинциях Пруссии вспыхнула в самых резких и безобразных формах старая борьба между католицизмом и протестантизмом. Бэйль был одним из первых превосходнейших борцов за просвещение, гуманность и терпимость, свободным от пут как католической, так и протестантской веры. Целью моей книги о Бэйле было поучить и пристыдить этим голосом прошлого введенную в заблуждение и озлобленную современность. 

Первая глава трактует о католицизме. Сущность католицизма с его монастырями, святыми, безбрачием духовенства и так далее я определил в отличив от протестантизма как противоречие между плотью и духом. Вторая глава трактует о протестантизме, --- его сущность я определил в отличив от католицизма как противоречие между верой и разумом. Третья говорит о противоречии между теологией и философией --- наукой вообще. Ибо, говорю я, для теологии истинно только то, что для нее священно, для философии священно только то, что истинно. 

Теология ведь опирается на особый принцип, на особую книгу, в которой она полагает заключенными все, по крайней мере, необходимые и полезные человеку истины. Она поэтому по необходимости узка, исключительна, нетерпима, ограничена. Философия же, наука, опирается не на определенную книгу, но черпает истину во всем целом природы и истории, она опирается на разум, который по существу универсален, но не на веру, которая по существу партикуляристична. Четвертая глава трактует о противоположности или противоречии между религией и моралью или о мыслях Бэйля об атеизме. Дело в том, что Бэйль утверждает, что человек может быть морален и без религии, ибо большинство людей неморально и живет соответственным образом, имея религию и ей вопреки, что атеизм совсем не необходимо связан с имморализмом, что поэтому государство может вполне хорошо состоять из атеистов. Это высказал Бэйль еще в 1680 г., между тем как еще год тому назад один дворянин, депутат, на объединенном прусском ландтаге не постыдился заявить, что он готов предоставить всем религиозным исповеданиям признание со стороны государства и полномочие на осуществление политических прав, всем --- только не атеистам. Пятая глава говорит специально о самостоятельности морали, ее независимости от религиозных догматов и мнений; то, что в четвертой главе доказывалось на примере истории и обыденной жизни, выводится здесь из существа самого предмета. Шестая глава трактует о противоречии между христианскими догматами и разумом, седьмая --- о значении противоречия между верой и разумом у Бэйля. Бэйль жил ведь в то время, когда вера была еще до такой степени авторитетом, что человек представлял себе возможным верить или заставлял себя верить даже в то, что он, согласно своему разуму, признавал ложным и нелепым. Восьмая глава говорит о значении и заслугах Бэйля как полемиста против религиозных предрассудков его времени; девятая, наконец, о характере Бэйля и значении его для истории философии. 

Книгой о Бэйле завершаются мои исторические работы. Позднейших, новейших философов я разбирал только как критик, но не как историк. Когда мы приступаем к новейшей философии, то мы тотчас же наталкиваемся на крупнейшее отличие новейших философов от прежних. В то время как прежние философы резко разграничивали философию и религию, даже прямо противопоставляли одну другой, полагая, что религия покоится на божественных мудрости и авторитете, философия же --- на человеческих, или, как выражался Спиноза, религия ставит своей целью пользу, благосостояние людей, философия же --- истину; в противоположность этому новейшие философы выступают с утверждением тождества философии и религии, по крайней мере что касается их содержания, их существа. Вот против этого-то тождества я и выступил. Уже в 1830 г., когда появились мои <<Мысли о смерти и бессмертии>>  я, обращаясь к одному догматику из гегелевской школы, утверждавшему, что между религией и философией существует лишь формальное различие, что философия возводит лишь в понятие то, что у религии имеется в виде представления, --- привел ему стих:

\begin{quote}

<<Сущность и форма --- одно>>; сокрушит содержание веры, 

Кто представленье ее --- форму ее --- сокрушит. 

\end{quote}
Я бросил поэтому философии Гегеля упрек в том, что она существенное в религии делает несущественным, и обратно --- несущественное --- существенным. Существо религии есть как раз то, что превращается философией в простую форму. 

Сочинение, которое в этом отношении приходится особенно упомянуть, есть маленькая, вышедшая в 1839 г. брошюра --- <<О философии и христианстве>>. В ней я высказал, что, невзирая на все попытки посредничества, различие между религией и философией неистребимо, ибо философия есть дело мышления, разума, тогда как религия --- дело душевного настроения и фантазии. Религия, однако, содержит в себе не только продиктованные настроением фантастические образы спекулятивных мыслей, как это утверждает Гегель, она в гораздо большей степени заключает в себе начало, отличное от мышления, и это начало не есть одна лишь форма, но и само существо религии. Это начало мы можем обозначить одним словом --- <<чувственность>>  ибо ведь и душевные настроения и фантазии коренятся в чувственности. Тех, кто смущается словом <<чувственность>>  так как обычное словоупотребление понимает под ним только вожделение, я прошу иметь в виду, что не только чрево, но и голова есть существо чувственное. <<Чувственность>> у меня не что другое, как истинное, не надуманное и искусственное, а действительно существующее единство материального и духовного, оно у меня поэтому то же, что действительность. Чтобы сделать только что указанное различие между религией и философией ясным и отчетливым, я напомню здесь, к примеру, учение, которое это различие особенно хорошо выявляет. Старые философы, по крайней мере часть их, учили бессмертию, но только бессмертию мыслящей части в нас, только бессмертию духа в отличие от чувственного начала в человеке. Некоторые даже определенно учили тому, что и память, воспоминание гаснут и что после смерти остается лишь чистое мышление, --- конечно, абстракция, совсем не существующая в действительности. Это бессмертие, однако, именно абстрактное, отвлеченное и потому не религиозное. Христианство отвергло поэтому это философское бессмертие и на его место поставило продолжение всего действительного, телесного человека; ибо только оно есть такое продолжение, при котором душевное настроение и фантазия находят себе пищу, но только потому, что продолжение это чувственное. Но что, в частности, можно сказать об этом учении, то можно сказать и о религии вообще. Сам бог есть существо чувственное, предмет созерцания, видения, правда, не телесного, но духовного, то есть созерцания в фантазии. Мы можем поэтому различие между философией и религией свести, короче говоря, к тому, что религия чувственна, эстетична, тогда как философия есть нечто нечувственное, абстрактное. 

Однако, хотя я и признал в своих прежних сочинениях сущностью религии в отличие от философии чувственность, но я все же не мог признать чувственности религии. Во-первых, потому, что эта чувственность противоречит действительности, она только фантастична и только плод душевных настроений. Так, чтобы держаться приведенного примера, --- плоть, которую религия противопоставляет философскому бессмертию, есть только фантастическая плоть, плоть душевного настроения, <<духовная>> плоть, то есть плоть, которая как бы и не плоть. Религия есть поэтому признание, утверждение чувственности в противоречии с чувственностью. Но, во-вторых, я не мог признать ее и потому, что в этом отношении стоял еще на точке зрения абстрактного мыслителя, что не оценил еще всего значения чувств. Во всяком случае, я его еще не уяснил себе окончательно. Истинного, полного признания чувственности я достиг, с одной стороны, через вторичное, углубленное изучение религии, с другой через чувственное изучение природы, к чему мне дала прекрасную возможность моя деревенская жизнь. Поэтому только в моих более поздних философских и религиозно-философских сочинениях борюсь я самым решительным образом как против абстрактной нечеловечности философии, так и против фантастической, призрачной человечности религии. Только в этих работах ставлю я с полным сознанием на место отвлеченного, лишь мыслимого мирового существа, именуемого богом, действительный мир или природу, на место абстрагированного от человека, лишенного чувств разумного существа философии, --- одаренного разумом, действительного, чувственного человека. 

Среди всего большого объема моих религиозно-философских сочинений наилучший обзор моей духовной карьеры, моего развития и его результатов дают мои <<Мысли о смерти и бессмертии>>  причем эту тему я трактовал трижды: в 1830 г., когда я именно с этими мыслями впервые выступил как писатель, в 1834 г. в сочинении под заголовком <<Абеляр и Элоиза>> и в 1846 г. в <<Вопросе о бессмертии с точки зрения антропологии>>. Первые размышления об этом предмете писал я как абстрактный мыслитель, вторые --- во власти противоречия между началом мышления и началом чувственности, третьи --- стоя на точке зрения мыслителя, примиренного с чувствами; или --- первые писал я как философ, вторые --- как юморист, третьи --- как человек. Тем не менее, однако, <<Мысли о смерти и бессмертии>> 1830 г. уже содержат в себе в абстрактной форме, то есть в идее, то, что мои позднейшие писания содержат в конкретном виде, то есть подробно и развито. Подобно тому, как я в своих позднейших, своих последних сочинениях предпосылал человеку природу, так уже и в этой работе полемизирую я против лишенной природы, абсолютной и, стало быть, без конца продолжающейся личности, короче говоря, --- против фантастической личности, вышедшей в беспредельность из рамок действительности, --- личности, как она обычно воспринимается верой в бога и бессмертие. Первый отдел этой работы, находящейся в полном собрании моих сочинений, называется <<Метафизическая или спекулятивная основа смерти>>. Он трактует об отношении личности к существу или природе. Пределом личности является природа, говорится там по смыслу, если не всеми словами; каждая вещь вне меня есть знак моей конечности, доказательство, что я не абсолютное существо, что я в существовании других существ имею свой предел, что я, стало быть, не бессмертное лицо. Эта истина, вначале вообще или метафизически высказанная, дальше развивается в других отделах. Следующий отдел называется <<Физическая основа смерти>>. К существу личности человека, личности вообще, говорится здесь, существенно принадлежит пространственная или временная определенность. Да, человек есть не только существо вообще пространственное, но также и существенно земное, от земли неотделимое. Как неразумно поэтому такому существу приписывать вечное неземное существование! Я выразил эту мысль в следующих стихах: 

\begin{quote}
    
Где ты родился на свет, там некогда будешь и спать ты; 

Не суждено никому лоно покинуть земли. 

\end{quote}

Третий и последний отдел имеет заголовок: <<Духовная, или психологическая, основа смерти>>. Простая основная мысль его такова: личность есть не только телесно или чувственно, но и духовно определенная и ограниченная личность; человек имеет определенное предназначение, место, задачу в великой общине человечества, в истории; но именно с этим несовместима бесконечная продолжительность существования. Человек находит свое продолжение только в своих творениях, в своих делах, осуществленных им в пределах его сферы, его исторического задания. Только это есть нравственное, этическое бессмертие. Эта мысль в третьем, и последнем, отделе есть только более развитая основная мысль моих <<Юмористически-философских афоризмов>>. Духовное, этическое, или моральное, бессмертие является единственным имеющимся у человека в его творениях. То, что он страстно любит и чем он со страстью занимается, и есть душа человека. Душа человека так же многообразна, характеризуется такими же признаками, как и сами люди. Бессмертие в старом смысле этого слова, некое вечное, беспредельное существование, годится поэтому только для неопределенной, расплывчатой души, в действительности и совсем не существующей, являющейся лишь человеческой абстракцией, или продуктом воображения. Но я эти мысли, основные мысли той работы, доказал лишь на специальном примере писателя, бессмертный дух которого есть исключительно дух его сочинений. 

В третий, и последний, раз трактовал я бессмертие в моем очерке: <<Вопрос о бессмертии с точки зрения антропологии>>. Первый отдел говорит об общей вере в бессмертие, о вере, которая встречается у всех или большинства народов, пребывающих в состоянии детства или невежества. Здесь я показываю, что верящие в бессмертие подставляют в верования народов свои собственные представления, что народы на самом деле верят не в другую, а только в эту жизнь, что жизнь мертвых есть лишь жизнь в царстве воспоминаний и живой мертвец есть лишь образ живого, олицетворенный в мертвом; я показываю далее, что если хотят личного, или индивидуального, бессмертия, то необходимо верить в него в духе простых первобытных народов, у которых человек после смерти совершенно тот же, каким он был до смерти, имеет те же страсти, занятия и потребности, ибо от них человек неотделим. Второй отдел говорит о субъективной необходимости веры в бессмертие, то есть о внутренних, психологических основаниях, порождающих в человеке веру в его бессмертие. Заключительные слова этого отдела гласят, что бессмертие есть, собственно говоря, потребность только для людей мечтательных, бездельных, от жизни убегающих в фантазию, но отнюдь не для людей деятельных, занятых явлениями действительной жизни. Третья глава трактует о <<критической вере в бессмертие>>  то есть о точке зрения, при которой уже не верят в то, что люди после смерти продолжают жить в своей прежней телесной оболочке, но критически различают между смертным и бессмертным существом человека. Однако эта вера, говорю я, сама необходимо подпадает сомнению, критике; она противоречит непосредственному чувству единства и сознанию единства, присущим человеку и с недоверием отклоняющим такое критическое разделение и расслоение человеческого существа. Последний отдел говорит, наконец, о той вере в бессмертие, которая держится среди нас и сейчас, о <<рационалистической вере в бессмертие>>  которая, при своей половинчатости и раздвоенности между верой и неверием, хотя видимо и утверждает бессмертие, в действительности, однако, его отрицает, вместо веры подставляя неверие, вместо потустороннего мира --- здешний, вместо вечности --- время, вместо божества --- природу, вместо религиозного неба --- мирское небо астрономии. 

Я дал в вышеизложенном коротенький, поверхностный обзор содержания моих мыслей о бессмертии и смерти, и дал потому, что бессмертие обычно и с полным правом образует главную составную часть религии и философии религии, я же эту веру оставлю в стороне или, во всяком случае, буду трактовать ее лишь постольку, поскольку она связана с верой в бога или, вернее, составляет с ней нечто единое. 

\phantomsection
\addcontentsline{toc}{section}{Третья лекция}
\section*{Третья лекция}

Я перехожу теперь к тем моим сочинениям, которые заключают в себе содержание и предмет этих лекций: мое учение, религию, философию, или называйте это как угодно иначе. Это мое учение в немногих словах гласит: теология есть антропология, то есть в предмете религии, который мы по-гречески называем theos'ом, по-нашему богом, выявляется не что иное, как существо человека, или: бог человека есть не что иное, как обожествленное существо человека. Следовательно, история религии, или, что то же, история бога, --- ибо как различны религии, так различны и боги, а религии различны в той мере, в какой различны люди, --- есть не что иное, как история человека. Так, чтобы пояснить и конкретизировать это утверждение на примере,  --- который, однако, больше, чем просто пример, --- греческий, римский, вообще языческий бог, как это признают даже и наши теологи и философы, есть только предмет языческой религии, существо, имеющее свое бытие только в вере и представлении язычника, но не христианского народа и человека, следовательно, есть только выражение, отображение языческого духа или существа; так и христианский бог есть только предмет христианской религии, следовательно, есть лишь характеристическое выражение христианского человеческого духа или существа. Различие между языческим богом и богом христианским есть лишь различие между языческим и христианским человеком или народом. Язычник --- патриот, христианин --- космополит, следовательно, и бог язычника патриотичен, бог же христианина --- космополитический бог, то есть язычник имел национального, ограниченного бога, ибо язычник не выходил за пределы своей национальности и нация была для него выше, чем человек; христианин же имеет универсального, всеобщего, весь мир охватывающего бога, ибо он сам вышел за пределы национальности и не ограничивает достоинство и существо человека одной определенной нацией. 

Различие между политеизмом и монотеизмом есть лишь различие между видами и родом. Видов много, но род один, ибо он есть то, в чем сходятся различные виды. Так, есть различные человеческие виды, --- расы, племена или называйте их как угодно, --- но они все принадлежат к одному роду человеческому роду. Политеизм силен лишь там, где человек не возвысился над видовым понятием человека, где он признает лишь человека своего вида, как себе равного, как существо равноправное и равноспособное. В понятии вида заключена множественность, следовательно, много богов там, где человек делает существо вида абсолютным существом. До монотеизма же человек возвышается там, где он возвышается до понятия рода, на котором все люди сходятся, в котором исчезают их видовые, их племенные, их национальные различия. Различие между единым, или, что то же, всеобщим, богом монотеистов и многими, или, что то же, особыми национальными, богами язычников, или политеистов, есть лишь различие между многими различными людьми и человеком или родом, где все едины суть. 

Видимость, наглядность, короче говоря, воспринимаемость политеистических богов чувствами есть не что иное, как воспринимаемость чувствами видовых и национальных человеческих различий: грек, например, ведь отличается видимо, очевидно от других народов; невидимость, невоспринимаемость чувствами монотеистического бога есть не что другое, как невоспринимаемость чувствами, невидимость рода, на котором все люди сошлись, но который не существует, как таковой, чувственно, очевидно, ибо существуют ведь только виды. 

Коротко говоря, различие между политеизмом и монотеизмом сводится к различию между видом и родом. Род, конечно, отличен от вида, ибо в роде мы отбрасываем все видовые различия; но отсюда еще не следует, что род обладает самостоятельной сущностью; ибо он ведь есть лишь то общее, что имеется у видов. Так же мало, как родовое понятие камня есть, так сказать, сверхминералогическое понятие, понятие, выходящее из пределов царства камней, хотя оно и одинаково отлично от понятия булыжника, известняка, плавикового шпата и не обозначает исключительно какой-либо определенный камень именно потому, что оно все их охватывает, --- так же мало и бог вообще, единый и всеобщий бог, у которого изъяты все телесные, чувственные свойства богов, которых много, исключается из существа человеческого рода; он, наоборот, есть объективированное и олицетворенное родовое понятие человечества. Или точнее выражаясь: если политеистические боги человеческие существа, то и монотеистический бог --- человеческое существо, подобно тому как человек вообще, хотя он выходит за пределы отдельных человеческих видов, которых множество, и возвышается над евреем, над греком, над индийцем, не является поэтому сверхчеловеческим существом. Нет поэтому ничего более неразумного, как считать, что христианский бог сошел с неба на землю, то есть вывести происхождение христианской религии из откровения существа, отличного от людей. Христианский бог в той же мере вышел из недр человека, как и языческий. Он только потому иной бог, чем языческий, что и христианин --- человек другой, чем язычник. 

Этот мой взгляд, или учение, согласно которому тайна теологии заключена в антропологии и согласно которому существо религии как субъективно, так и объективно ничего другого не раскрывает и не выражает, как существо человека, я развил прежде всего в моей работе --- <<Сущность христианства>>  потом в нескольких более мелких очерках, имеющих отношение к этой книге, как, например: <<Сущность веры в смысле Лютера>>  1844 г., <<Различие между языческим и христианским обожествлением человека>>  наконец, по различным поводам я касался того же во 2-м издании моей <<Истории философии>> и в моих <<Основных положениях философии>>. 

Моя точка зрения, или учение, высказанное в <<Сущности христианства>>  или, точнее говоря, мое учение, как я высказал или мог высказать его в этом сочинении, соответственно его теме, имеет, впрочем, большой пробел и дало поэтому повод к самым нелепым недоразумениям. Так как я в <<Сущности христианства>>  сообразно моей теме, отвлекался от природы, игнорировал природу, потому что само христианство ее игнорирует, потому что христианство есть идеализм и оно возглавляется богом, лишенным природного естества, верит в бога или духа, который творит мир только силою своего мышления и воли, и вне и без мышления и воли которого мир, стало быть, и не существует, так как я таким образом в <<Сущности христианства>> трактовал лишь о существе человека, с которого непосредственно и начал свое сочинение именно потому, что христианство почитает как божественные силы и существа не солнце, луну или звезды, огонь, землю, воздух, но силы, которые, в противоположность природе, лежат в основе существа человека: волю, разум, сознание, --- то думали, что я человеческое существо произвожу из ничего, превращаю его в существо, которому ничто не предшествует, и возражали этому моему мнимому обожествлению человека, ссылаясь на непосредственное чувство зависимости, на признание естественного разума и сознания, что человек не сам себя сотворил, что он --- зависимое, созданное существо, а стало быть, имеет причину своего бытия вне себя, --- что он сам и через свою голову указует на некое другое существо. Вы целиком правы, господа, сказал я, мысленно обращаясь к своим хулителям и насмешникам. Я знаю столь же хорошо, --- быть может, лучше, чем вы, --- что человеческое существо, мыслимое как существующее само по себе, ни от чего независимое и абсолютное, есть нелепость, идеалистическая химера. Но существо, которое человек считает себе предшествующим, к которому он по необходимости имеет отношение, без которого ни его существование, ни его сущность не могут быть мыслимы, это существо, господа, нечто иное, как природа, а не ваш бог. Этот пробел, остававшийся в <<Сущности христианства>>  я восполнил впервые в 1845 году в маленькой, но содержательной работе <<Сущность религии>>  работе, которая, как уже показывает заголовок, тем отличается от <<Сущности христианства>>  что трактует не только о сущности христианской религии самой по себе, но и о сущности религии вообще, следовательно, также о дохристианских, языческих, естественных религиях. Здесь я, соответственно моей теме, имел уже гораздо больший простор, а поэтому и возможность сбросить ту видимость идеалистической односторонности, к которой я дал повод в <<Сущности христианства>> обвинять меня моим некритическим критикам; здесь я имел достаточно места, чтобы восполнить все пробелы <<Сущности христианства>>. Конечно, я и здесь не восполнил их, как это само собой разумеется, в духе теологии, теистической или теологической философии. Точнее всего задача и взаимоотношение этих двух сочинений представляется в таком виде. Теологи или вообще теисты различают между физическими и моральными свойствами бога, --- бог же, как уже сказано, есть имя, которым вообще называют предмет религии. Бог, говорит, например, Лейбниц, должен быть рассматриваем в двояком качестве: физически, как творец мира, морально, как монарх, как законодатель людей. По своим физическим свойствам, из которых главнейшее есть могущество, бог есть, таким образом, причина физических существ, природы; по своим моральным свойствам, из которых главнейшее есть доброта, бог есть причина моральных существ, людей. В <<Сущности христианства>> предметом моим был лишь бог, как моральное существо, по необходимости я не мог дать поэтому в <<Сущности христианства>> цельную картину моего воззрения и учения. Другую половину бога, там опущенную, его физические свойства, я должен был поэтому представить в другом сочинении, но мог ее представить сообразно теме, объективно, только в таком сочинении, где речь заходит и об естественной религии, то есть о такой религии, которая имеет главным своим предметом физического бога. Как я показал уже в <<Сущности христианства>>  бог, рассматриваемый в отношении своих моральных и духовных свойств, бог, стало быть, как моральное существо, есть не что иное, как обожествленное и нашедшее свое предметное выражение духовное существо человека, --- теология, следовательно, есть в действительности, в ее последнем основании и конечном выводе лишь антропология; так, в <<Сущности религии>> я показал, что физический бог или бог, рассматриваемый только как причина природы, звезд, деревьев, камней, животных, людей, --- поскольку и они суть естественные физические существа, не выражает ничего другого, как обожествленное, олицетворенное существо природы, так что тайна физико-теологии есть лишь физика или физиология, физиология в данном случае не в том более узком смысле этого слова, который она сейчас имеет, но в его старом универсальном смысле, означавшем вообще естествознание. Поэтому, если я раньше выразил свое учение в формуле: теология есть антропология, то теперь для полноты я должен прибавить: и физиология. 

Мое учение или воззрение может быть поэтому выражено в двух словах: природа и человек. С моей точки зрения, существо, предшествующее человеку, существо, являющееся причиной или основой человека, которому он обязан своим происхождением и существованием, есть и называется не бог --- мистическое, неопределенное, многозначащее слово, а природа --- слово и существо ясное, чувственное, недвусмысленное. Существо же, в котором природа делается личным, сознательным, разумным существом, есть и называется у меня человек. Бессознательное существо природы есть, с моей точки зрения, существо вечное, не имеющее происхождения, первое существо, но первое по времени, а не по рангу, физически, но не морально первое существо; сознательное, человеческое существо есть второе по времени своего возникновения, но по рангу первое существо. Это мое учение, поскольку оно имеет своим последним пунктом природу, апеллирует к ее истинности и, поскольку выдвигает ее против теологии и философии, представлено только что упомянутой работой, но в связи с положительным, историческим предметом: 

естественной религией, ибо я развиваю все свои учения и мысли не в голубом тумане абстракции, а на твердой почве исторических, действительных, от моего мышления независимых предметов и явлений, --- так, например, мой взгляд на природу или учение о природе на основе естественной религии. 

Я дал, впрочем, в этой работе не только изложение сущности естественной религии, но в то же время и краткий очерк всего хода развития религии, начиная с ее первых зачатков и вплоть до ее завершения в идеалистической религии христианства. Поэтому она заключает в себе не что другое, как сжатую духовную, или философскую, историю религии человечества. Я подчеркиваю эпитет: духовную, ибо дать настоящую, форменную историю религии, такую историю, где бы различные религии были одна за другой перечислены и пересчитаны, обыкновенно к тому же еще и иерархически распределены по очень произвольным отличительным признакам, --- повторяю, дать такое историческое описание не входило в мои цели. За исключением большого подразделения на религию естественную и духовную, или человеческую, я гораздо больше интересовался тем, что в религиях есть одинакового, тождественного, общего, чем их --- часто такими мелкими и произвольными --- отличиями. Вообще в этой работе моя задача состояла исключительно в том, чтобы уловить сущность религии, а историю ее лишь постольку, поскольку без нее нельзя понять религии. И даже сущность религии я прослеживал в этом сочинении, как и вообще в моих работах, совсем не из одних теоретических или спекулятивных соображений, но также в значительной мере из соображений практических. Как прежде так и теперь я интересуюсь религией, главным образом лишь в той мере, в какой она является --- хотя бы в воображении --- основой человеческой жизни, основой морали и политики. 

Для меня, как прежде, так и теперь, важнее всего осветить темную сущность религии факелом разума, дабы человек мог перестать, наконец, быть добычей, игрушкой всех тех человеконенавистнических сил, которые испокон века, еще и до сих пор пользуются тьмой религии для угнетения людей. Моя цель была доказать, что силы, пред которыми человек склоняется в религии и которых он боится, которым он решается даже приносить кровавые человеческие жертвы, чтобы расположить их к себе, что эти силы --- не что иное, как создание его собственного, несвободного, боязливого духа и невежественного, необразованного ума, доказать, что существо, которое человек противопоставляет себе в религии и теологии, как совершенно другое, от него отличное, есть его собственное существо, дабы человек, так как он ведь помимо своего сознания постоянно дает господствовать над собою и определять себя своему собственному существу, впредь сознательно сделал бы свое собственное, человеческое существо законом и определяющей основой, целью и масштабом своей морали и политики. И так и будет, и так и должно случиться. Если до сих пор непознанная религия, религиозная тьма была верховным принципом политики и морали, то отныне или, по крайней мере, когда-нибудь в будущем определять судьбу людей будет религия познанная, растворенная в людях. 

Именно эта цель --- познание религии для содействия человеческой свободе, самодеятельности, любви и счастью --- определила также размер моей исторической трактовки в религии. Все, что для этой цели было безразлично, я оставлял в стороне. Исторические описания различных религий и народных мифологий без познания религии можно встретить в бесчисленных книгах. Но так же, как я писал, так же я буду и читать лекции. Цель моих сочинений, как и моих лекций, это превратить людей из теологов в антропологов, из теофилов в филантропов, из кандидатов потустороннего мира --- в студентов мира здешнего, из религиозных и политических камердинеров небесной и земной монархии и аристократии --- в свободных и исполненных самосознания граждан земли. Моя цель поэтому меньше всего отрицательная, отрицающая, она положительная, да и отрицаю я только для того, чтобы затем утверждать; я отрицаю лишь фантастическое, призрачное существо теологии и религии, чтобы утвердить действительное существо человека. Ни с одним словом не творили столько злоупотреблений в новейшее время, как со словом отрицательный. Если я в сфере познания, науки что-нибудь отрицаю, то для этого я должен привести основания. Основания же учат, проливают свет, дают мне познания; каждое научное отрицание есть положительный духовный акт. Конечно, вывод из моего учения тот, что бога нет, то есть нет абстрактного, нечувственного существа, отличного от природы и людей и вершащего судьбы мира и человечества по своему собственному благоусмотрению; но это отрицание есть лишь вывод из познания существа бога, из познания, что это существо выражает не что иное, как, с одной стороны, существо природы, а с другой существо человека. Правда, это учение можно назвать атеизмом, ибо ведь все на свете, говорят, должно носить свою кличку, но не следует при этом забывать, что этим именем еще ничего не сказано, как не сказано и противоположным именем теизма. Теос, бог есть голое имя, выражающее все возможное, и содержание его бывает столь же различно, как различны времена и люди; поэтому все дело в том, что кто понимает под именем бога. Так, например, в восемнадцатом веке христианское правоверие замыкало значение этого слова в такие педантически узкие границы, что даже Платон слыл атеистом, ибо он не учил о сотворении мира из ничего и, стало быть, недостаточно отделял творца от его творений. Так и Спиноза в семнадцатом и восемнадцатом веках почти единогласно объявлен атеистом, так что, если память мне не изменяет, в одном латинском словаре восемнадцатого века атеист переводится даже словами assecia Spinozae (последователь Спинозы); однако девятнадцатый век вычеркнул Спинозу из рядов атеистов. Так меняются времена, а с ними вместе и боги людей. Как мало сказано словами: <<есть бог>>  или <<я верую в бога>>  так же мало сказано и словами: <<бога нет>>  или <<я не верую в бога>>. Все дело в том, каковы содержание, основа, дух теизма и каковы содержание, основа, дух атеизма. Я перехожу, однако, к самому предмету, то есть, к моему сочинению о <<Сущности религии>>  которое я положил в основу этих лекций.

\phantomsection
\addcontentsline{toc}{section}{Четвертая лекция}
\section*{Четвертая лекция}

Первый параграф в <<Сущности религии>> вкратце гласит: Основу религии составляет чувство зависимости человека; в первоначальном смысле природа и есть предмет этого чувства зависимости; природа есть, таким образом, первый объект религии. Содержание этого параграфа распадается на две части. Одна часть объясняет субъективное происхождение или основу религии, другая характеризует первый, или первоначальный, объект религии. Сначала поговорим о первой. Так называемые спекулятивные философы издевались над тем, что я чувство зависимости объявляю источником религии. Слова <<чувство зависимости>> находятся у них на плохом счету с тех пор, как Гегель против Шлейермахера, --- который, как известно, объявил чувство зависимости сущностью религии, --- пустил остроту, что соответственно этому и у собаки должна быть религия, ибо она чувствует себя зависимой от своего господина. Впрочем, так называемые спекулятивные философы --- это те философы, которые не свои понятия сообразуют с вещами, а, наоборот, скорее вещи --- с понятиями. И поэтому совершенно безразлично, удовлетворяет ли мое объяснение спекулятивных философов; дело идет только о том, отвечает ли оно своему предмету, своей сути. А приведенное объяснение им отвечает. 

Если мы рассмотрим религии так называемых дикарей, о которых нам сообщают путешественники, как равно и религии культурных народов, если мы заглянем в нашу собственную, непосредственно и без обмана нашему наблюдению доступную душу, то мы не найдем другого, соответствующего и широко захватывающего психологического объяснения религии, кроме чувства или сознания зависимости. Древние атеисты и даже очень многие как древние, так и новейшие теисты объявляли причиной религии страх, который, однако, ведь не что иное, как самое распространенное, бросающееся в глаза проявление чувства зависимости, Общеизвестно изречение римского поэта: Primus in orbe deos fecit timor, то есть страх первый сотворил в мире богов. У римлян даже слово: страх, inetus, имеет значение религии, и, наоборот, слово religio иногда означает страх, боязнь; потому dies religiosus, религиозный день означал у них то же, что несчастливый день, день, которого боятся. Даже наше немецкое Ehrfurcht --- выражение высочайшего, религиозного почитания составлено, как показывает само слово, из Ehre (почитание) и Furcht (боязнь). 

Объяснение религии из страха подтверждается прежде всего тем наблюдением, что почти все или во всяком случае очень многие первобытные народы делают предметом своей религии вызывающие страх и ужас явления или действия природы. Более примитивные, например народы Африки, Северной Азии и Америки <<боятся>>, --- как это приводит Мейнерс из описаний путешествий в своей <<Всеобщей критической истории религий>> --- <<рек в тех местах, где они образуют опасные водовороты или пороги. Когда они проезжают по таким местам, то просят о пощаде или прощении или ударяют себя в грудь и бросают разгневанным божествам умилостивительные жертвы. Многие негритянские царьки, избравшие море своим фетишем, до такой степени боятся его, что не осмеливаются даже на него взглянуть, не то что по нем проехать, потому что они верят, что лицезрение этого страшного божества убьет их на месте>>. Так, по словам В. Марсдена в его <<Естественном и гражданском описании острова Суматры>>; редшанги, живущие глубже в стране, жертвуют морю, когда они его в первый раз видят, пироги и сладкое печенье и просят его не причинять им вреда. Правда, готтентоты, как выражаются авторы путешествий, теистически настроенные и не могущие выйти за пределы своих религиозных представлений, верят в высшее существо, но не почитают его; они, наоборот, почитают, <<злого духа>>  который, по их мнению, является виновником всех бед, их постигающих на свете. Я должен, однако, заметить, что известия, сообщаемые авторами путешествий, по крайней мере авторами более ранними, о религиозных представлениях готтентотов, как и вообще дикарей, весьма противоречивы. Также и в Индии имеются местности, <<где большая часть обитателей не отправляет других религиозных служб, как только служб злым духам\dots Каждая из этих злых сил имеет свое особое название, и ей воздаются тем большие почести, чем она представляется страшнее и могущественнее>> (Штур, <<Религиозные системы языческих народов Востока>>). Точно так же и американские племена, даже такие, которые, по сообщениям наблюдателей-теистов, признают <<высшее существо>>  почитают только <<злых духов>>  или существа, которым они приписывают все худое и злое, все болезни и горести, которые их постигают, --- почитают, чтобы через это почитание их смягчить, а стало быть, из страха. Римляне в числе предметов своего религиозного почитания имели даже болезни и эпидемии, лихорадку, хлебную ржу, в честь которой они ежегодно справляли праздник, детоубийство под именем Орбоны, несчастье, словом, предметы, почитание которых не имело, очевидно, другого основания, кроме страха, как это уже отмечали сами древние, например, Плиний Старший, и другой цели, кроме как сделать их безвредными, что также уже было отмечено древними, например Геллием, который говорит, что одних богов почитали и чествовали, чтобы они приносили пользу, других --- примиряли с собой и смягчали, чтобы они не навредили. Даже самый страх имел в Риме свой храм, также и в Спарте, где, впрочем, по крайней мере по свидетельству Плутарха, он имел значение моральное, значение страха постыдных, дурных поступков. 

Объяснение религии из страха подтверждается, далее, тем обстоятельством, что даже у духовно выше стоящих народов высшее божество есть олицетворение явлений природы, вызывающих в людях высшую степень страха, божество грозы, молнии и грома. Есть даже народы, у которых нет для бога другого слова, как гром, у которых, стало быть, религия --- не что иное, как потрясающее впечатление, которое производит природа на человека своим громом при посредстве слуха, органа страха. Даже у гениальных греков, как известно, высший бог есть просто громовержец. Точно так же и у древних германцев, по крайней мере северогерманцев, равно как и у финнов и латышей, старейшим и первым, наиболее почитаемым богом был бог Торр (Thorr) или Донар (Donar), то есть бог грома. Если английский философ Гоббс выводит разум из ушей, потому что он отождествляет разум с слышимым словом, то можно, и с гораздо большим правом на основании приведенных фактов, согласно которым гром вбил людям веру в бога, признать барабанную перепонку в ухе местом резонанса для религиозных чувств и ухо маткой, из которой выходят боги. В самом деле, если бы у человека были только глаза и руки, вкус и обоняние, то он не имел бы религии, потому что все эти чувства суть органы критики и скепсиса. Единственное чувство, теряющееся среди лабиринта уха в царстве духов или призраков прошедшего или будущего, единственное мистическое и религиозное чувство страха, есть слух, как это уже верно отметили древние, говоря: <<свидетель, который видел, стоит более, чем тысячи свидетелей, которые слышали>>  и <<глаза надежнее, чем уши>>  или <<то, что видишь, вернее, чем то, что слышишь>>. Поэтому и последняя, наиболее духовная, религия --- христианская --- сознательно опирается только на слово, как она говорит: на божие слово, и, следовательно, на слух. <<Вера, --- говорит Лютер, --- возникает при слушании проповеди о господе>>. <<Только слух, --- говорит он в другом месте, --- требуется в церкви господа>>. Отсюда, кстати сказать, ясно, как поверхностно подходить к религии, особенно к ее первопричинам, с пустыми фразами об абсолютном, сверхчувственном и бесконечном, и делать так, как будто бы человек не обладает никакими чувствами, так что они не принимаются в расчет, когда речь идет о религии. Без чувств всегда бесчувственно-бессмысленна речь человека. Однако вернемся от этого вводного замечания к нашему изложению. 

Объяснение происхождения религии из страха подтверждается далее и тем, что даже и христиане, которые, по крайней мере теоретически, приписывают религии совершенно сверхчувственное, божественное происхождение и характер, настраиваются религиозно главным образом в тех случаях, в те моменты жизни, когда в человеке возбуждается страх. Когда, например, его величество, царствующий король Пруссии, который нынешними благочестивыми христианами зовется <<христианским королем>> по преимуществу и как таковой почитается, когда он созвал объединенный ландтаг, то распорядился, чтобы во всех церквах призывалось содействие божественного существа. Каковы, однако, были мотивы этого религиозного душевного движения и распоряжения его величества? Одна только боязнь, что злые тенденции нового времени могут пагубно повлиять на те планы и соображения, которые имелись в виду при образовании объединенного ландтага, этого мастерского произведения христианско-германского государственного искусства. Когда --- чтобы взять другой пример --- несколько лет тому назад случился неурожай, то во всех христианских церквах искренно и горячо молили господа бога, чтобы он дал свое благословение; тогда были даже установлены особые молитвенные и покаянные дни. Какова же была причина? Боязнь, голода. Именно поэтому бывает также, что христиане готовы свалить на неверующих и <<безбожников>> все напасти, и поэтому же --- впрочем, разумеется, исключительно из христианской любви и заботливости о душах --- они испытывают величайшее злорадство, когда с <<безбожниками>> случается несчастье, ибо христиане верят, что те через это обратятся к богу, станут верующими и религиозно настроенными. Вообще христианские теологи и ученые, правда, порицают, по крайней мере с кафедры и в писаниях, когда явление, подобное только что приведенным, рассматривается как характерное для религиозного убеждения; но для религии, по крайней мере религии в обычном или, вернее, в историческом смысле этого слова, господствующем в мире, характерно не то, что имеет значение в книгах, а что имеет значение в жизни. Христиане только тем отличаются от так называемых язычников или некультурных народов, что они причины тех явлений, которые вызывают их религиозный страх, возводят не к отдельным божествам, а к особым свойствам их бога. Они обращаются не к злым богам; но они обращаются к своему богу, когда он --- как они верят разгневан, или дабы он на них не разгневался и не наказывал их злом и несчастием. Таким образом, подобно тому как злые боги являются почти единственными объектами почитания у примитивных народов, подобно тому и разгневанный или злой бог есть главнейший предмет почитания христианских народов. А, стало быть, и у них главнейшая причина религии есть страх \hyperlink{1}{(1)}\hypertarget{b1}{}. В подтверждение этого объяснения я привожу, наконец, еще и то, что христиане или религиозные философы и теологи упрекали Спинозу, стоиков, вообще пантеистов, у которых бог есть не что иное, строго говоря, как только чистая сущность природы, --- что их бог не есть бог, то есть не настоящий религиозный бог, ибо он не является предметом любви и страха, а только предметом холодного, бесстрастного ума. Поэтому, если они и отвергали объяснение возникновения религии из страха, дававшееся древними атеистами, то косвенно они все же тем самым признавали, что страх есть, по крайней мере, существенная составная часть религии. 

Тем не менее страх не есть полное, достаточное основание, объясняющее религию, но не только из одних тех соображений, которые приводятся некоторыми, что страх, дескать, есть преходящий аффект; потому что ведь предмет страха по крайней мере в представлении остается; ведь специфическая черта страха есть та, что он действует и вне пределов настоящего момента, что он дрожит и перед возможным будущим злом, но потому, что вслед за страхом, когда опасность минуты прошла, наступает аффект противоположный, и это чувство, противоположное страху, имеет связь с тем же предметом, в чем можно убедиться при малейшем внимании и размышлении. Это чувство есть чувство освобождения от опасности, от страха и трепета, чувство восторга, радости, любви, благодарности. Явления природы, возбуждающие страх и ужас, относятся большей частью к наиболее благодетельным по своим последствиям. Бог, который своей молнией поражает деревья, зверей и людей, тот же бог освежает своими дождевыми потоками поля и луга. Откуда зло, оттуда приходит и добро, откуда страх, оттуда и радость. Почему бы в своем душевном настроении человеку не объединить того, что само имеет в природе одну и ту же причину? Только народы, живущие одним сегодняшним моментом, слишком слабые, тупые или легкомысленные, чтобы связывать различные впечатления, имеют поэтому к своей матери божьей один лишь страх и предметами своего религиозного почитания одних только злых, страшных богов. Иначе у народа, который из-за впечатлений от предмета, вызывающих минутный страх и ужас, не забывает его добрых благодетельных свойств. Здесь предмет страха делается также и предметом почитания, любви, благодарности. Так, у древних германцев, по крайней мере у северогерманцев, бог Торр, громовержец, <<благодетельный, добрый боец за людей>> и <<покровитель земледелия, бог мягкий, расположенный к людям>> (В. Мюллер, <<История и система древнегерманской религии>>), потому что он, бог грозы, одновременно и бог оплодотворяющего дождя и солнечного света. Было бы поэтому в высшей степени односторонне, даже несправедливо по отношению к религии, если бы я сделал страх единственной причиной, объясняющей религию. Я существенно отличаюсь от прежних атеистов, а также пантеистов, имевших в этом отношении взгляды, одинаковые с атеистами, как например, Спиноза, именно тем, что я беру для объяснения религии не только отрицательные, но и положительные мотивы, не только невежество и страх, но и чувства, противоположные страху, --- положительные чувства радости, благодарности, любви и почитания, что я утверждаю, что обожествляет как страх, так и любовь, радость, почитание. <<Ощущения нужды и опасности, которые преодолены>>, --- говорю я в моих комментариях к <<Сущности религии>>  --- <<совсем иные, чем ощущения нужды или опасности, имеющиеся в наличности или предстоящие. В одном случае я устанавливаю свое отношение к предмету, в другом я устанавливаю отношение предмета во мне; в одном --- я пою хвалебные песни, в другом --- жалобные; там я благодарю, здесь я прошу. Ощущение нужды практично, телеологично, чувство благодарности поэтично, эстетично. Ощущение нужды преходяще, чувство же благодарности длительно; оно завязывает узы любви и дружбы. Ощущение нужды --- грубо, чувство благодарности --- благородное чувство; одно почитает свой предмет лишь в несчастье, другое также и в счастье>>. Здесь мы имеем психологическое объяснение религии не только с ее дурной, но и с ее благородной стороны. Но если я не хочу и не могу назвать ни страх, ни радость или любовь единой объясняющей причиной религии, то какое другое обозначение найду я --- характерное, универсальное, охватывающее обе стороны, --- как не чувство зависимости? Страх есть чувство смерти, радость --- чувство жизни. Страх есть чувство зависимости от предмета, без которого или благодаря которому я ничто, предмета, во власти которого меня уничтожить. Радость, любовь, благодарность есть чувство зависимости от предмета, благодаря которому я что-то собой представляю, который дает мне чувство, сознание, что я благодаря ему живу, благодаря ему существую. Так как я благодаря природе или богу живу и существую, то я люблю его; так как я благодаря природе страдаю и погибаю, то я боюсь и страшусь ее. Короче говоря, кто человеку дает средства или источники жизненного счастья, того он любит, а кто у него эти средства берет или имеет силу их взять, того он боится. Но и то и другое объединяется в предмете религии, --- то, что является источником жизни, в своем отрицании, когда его у меня нет, --- есть источник смерти. <<Все исходит от бога, --- говорится у Сираха, --- счастье и несчастье, жизнь и смерть, бедность и богатство>>. <<Идолов, --- говорится в книге Баруха, --- не следует принимать за богов или их так называть, ибо они не могут ни наказывать, ни помогать\dots Они не могут царей ни проклясть, ни благословить>>. И точно так же Коран обращается в 26 суре к служителям идолов: <<Слышат ли они (идолы) вас, когда вы их призываете? Или могут ли они вам чем-либо быть полезны или в чем повредить?>>. Это значит: только то есть предмет религиозного почитания, только то есть бог, что может проклинать и благословлять, вредить или оказывать пользу, убивать и воскрешать, радовать и ужасать. 

Чувство зависимости есть поэтому единственно верное, универсальное название и понятие для обозначения и объяснения психологической и субъективной основы религии. Правда, в действительности не существует чувства зависимости как такового, а всегда только определенные, особые чувства, --- как, например (возьмем примеры из естественной религии), чувство голода, нездоровья, страха смерти, печаль при пасмурной и радость при ясной погоде, скорбь по затраченным напрасно усилиям, по надеждам, не сбывшимся в результате разрушительных явлений природы, --- в чем человек чувствует себя зависимым; но задача, коренящаяся в природе мышления и речи, в том и заключается, чтобы частные явления действительности сводить к таким общим названиям и понятиям. 

Исправив и дополнив объяснение религии из страха, я должен еще упомянуть о другом психологическом объяснении религии. Греческие философы говорили, что изумление перед закономерностью движения небесных светил породило религию, то есть почитание самих звезд или существа, управляющего их движением. Однако ясно без дальнейших замечаний, что это объяснение религии имеет отношение к небу, но не к земле, к глазу, но не к другим чувствам, только к теории, но отнюдь не к практике человека. Конечно, звезды были причиной и предметом почитания, но совсем не как объекты теоретических, астрономических наблюдений, а поскольку они рассматривались как силы, властвующие над жизнью человека, и, стало быть, поскольку они были предметами человеческих страха и надежд. Как раз на примере звезд мы отчетливо видим, что только тогда существо или вещь являются объектом религии, когда они являются предметом, причиной страха смерти или радости жизни, когда они, стало быть, являются объектом чувства зависимости. Правильно говорится поэтому в одном французском сочинении, вышедшем в 1768 г., <<De I'origine des principes relig'ieux>> (<<О происхождении религиозных принципов>>): <<Гром и непогода, бедствия войны, чума и голодовка, эпидемии и смерть в большей мере убедили человека в существовании бога, то есть более религиозно настроили, более убедили в его зависимости и конечности, чем постоянная гармония природы и все доказательства Кларка и Лейбница>>. Простой и постоянный порядок не приковывает к себе внимание человека. Только события, граничащие с чудом, могут его вновь оживить. Я никогда не слышал, чтобы народ говорил: бог наказывает пьяницу, потому что он теряет свой разум и здоровье. Но как часто я слышал, как крестьяне моей деревни утверждали: бог наказывает пьяниц, потому что один пьяный сломал себе ногу, когда собрался идти домой. 

\phantomsection
\addcontentsline{toc}{section}{Пятая лекция}
\section*{Пятая лекция}

Историческими примерами мы подкрепили сведение религии к чувству зависимости. Но это положение на взгляд здравомыслящего человека подкрепляется и само из себя; ибо само собой очевидно, что религия есть лишь признак или свойство существа, которое необходимо устанавливает отношение к другому существу, которое --- не бог, то есть не лишенное потребностей, независимое, бесконечное существо. Чувство зависимости и чувство конечности поэтому едины суть. Но самое чувствительное, самое больное чувство конечности для человека есть чувство или сознание, что он когда-нибудь и в самом деле кончится, что он умрет. Если бы человек не умирал, если бы он жил вечно, если бы, таким образом, не было смерти, то не было бы и религии. Ничего нет, --- говорит Софокл в <<Антигоне>>  --- сильнее человека; он пересекает моря, буравит землю, укрощает зверей, защищает себя от жары и дождя, от всего находит средства, только смерти не может избежать. Человек и смертный, бог и бессмертный --- у древних одно и то же. Только могила человека, --- говорю я поэтому в моих пояснениях к <<Сущности религии>>  --- есть место рождения богов. Чувственный знак или пример этой связи смерти с религией мы имеем в том, что в седой древности могилы умерших были одновременно и храмами богов, что, далее, у большинства народов служение мертвым, умершим, есть существенная часть религии, у некоторых даже единственная, вся религия в целом; но ведь мысль о моих умерших предшественниках как раз и есть то, что мне, живущему, всего более напоминает о моей будущей смерти. <<Никогда, --- говорит языческий философ Сенека в своих письмах, --- никогда душевное настроение смертного не бывает божественнее (или, говоря нашим языком, религиознее), чем когда он думает о своей смертности и знает, что человек для того и живет, чтобы когда-нибудь умереть>>. И в Ветхом Завете говорится: <<Господи, научи же меня, что меня должен когда-нибудь постигнуть конец и жизнь моя имеет свою цель, и я должен уйти!>>;  <<Научи нас подумать над тем, что мы должны умереть, дабы мы поумнели>>;  <<Подумай о нем: как он умер, так и ты должен умереть>>;  <<Сегодня царь, а завтра --- мертв>>. Религиозная же мысль --- и совершенно независимо от представления о боге --- есть мысль о смерти, ибо здесь я сознаю свою конечность. Но если ясно, что нет религии без смерти, то ясно также, что характерным выражением для основы религии является чувство зависимости \hyperlink{2}{(2)}\hypertarget{b2}{}; ибо что сильнее, резче внушает мне сознание или чувство, что я не от одного себя завишу, что я не могу так долго жить, как хочу, --- как не именно смерть? Но я должен сейчас же наперед заметить, что для меня чувство зависимости не составляет всей религии, что оно для меня лишь происхождение, лишь базис, лишь основа религии; ибо в религии человек ищет одновременно и средства против того, от чего он чувствует себя зависимым. Так, средством против смерти является вера в бессмертие. И единственное религиозное желание, единственная молитва, которую грубый, первобытный человек обращает к своему божеству, есть молитва качинских татар, обращенная к солнцу: <<Не убивай меня>>.

Я перехожу теперь ко второй части параграфа, к первому объекту религии. Мне не нужно много тратить по этому поводу слов, так как теперь почти общепризнано, что старейшая или первая религия людей есть естественная религия, что даже позднейшие духовные и политические боги народов, боги греков и римлян, были сначала только существами природы. Так один, хотя он впоследствии преимущественно политическое существо, а именно --- бог войны, первоначально не что иное, подобно Зевсу греков, Юпитеру римлян, как небо, поэтому солнце называется его глазом. Природа поэтому была и служит до сих пор у первобытных народов предметом религиозного почитания, совсем не как символ или орудие существа или бога, спрятавшегося за спиною природы, а как таковая, как природа. 

Содержание второго параграфа, коротко говоря, есть то, что религия, хотя и присуща существу человека или враждебна ему, но не религия в смысле теологии или теизма, подлинной веры в бога, а только религия, поскольку она не выражает ничего другого, как чувство конечности или зависимости человека от природы. 

Я должен к этому параграфу прежде всего заметить, что я здесь различаю между религией и теизмом, верой в существо, отличное от природы и от человека, хотя в своей предыдущей лекции я сказал, что предмет религии вообще называется богом. И в самом деле, теизм, теология, вера в бога до такой степени отождествили себя с религией, что не иметь бога, не иметь теологического существа, не иметь религии --- у нас равнозначно. Но здесь речь идет о первоначальных элементах религии. Именно теизм, теология вырвали человека из связи с миром, изолировали и сделали его высокомерным существом, <<Я>>  возвышающимся над природой. И только уже стоя на этой точке зрения, религия отождествляет себя с теологией, с верой в неестественное и сверхъестественное существо, как в истинное и божеское. Первоначально, однако, религия ничего другого не выражает, как ощущение человеком его связи, его единства с природой и миром. 

В моей <<Сущности христианства>> я высказал, что тайны религии могут найти свое разрешение и свое разъяснение не только в антропологии, но также и в патологии. По этому поводу чуждые природе теологи и философы пришли в ужас. Но что представляет собой естественная религия в ее празднествах и обычаях, имеющих отношение к важнейшим явлениям природы и их выражающих, как не эстетическую патологию? Часто также и очень неэстетическую. 

Что другое представляют собой эти весенние, летние, осенние и зимние празднества, встречаемые нами в древних религиях, как не воспроизведение различных впечатлений, которые оказывают на человека различные явления и действия природы? Горе и печаль по поводу смерти человека или по поводу убывания света и тепла, радость по поводу рождения человека, по поводу возвращения света и тепла после холодных дней зимы или по поводу урожая, страх и ужас перед явлениями природы и в самом деле страшными или, по крайней мере, кажущимися страшными человеку, как, например, при солнечных и лунных затмениях, --- все эти простые, естественные ощущения и аффекты являются субъективным содержанием естественной религии. Религия первоначально не представляет ничего отдельного, различающегося от человеческого существа. Лишь с течением времени, лишь в позднейшем своем развитии являет она собой что-то отдельное, выступает с особыми претензиями. И только против этой вызывающей, высокомерной духовной религии, которая именно поэтому имеет своим представителем особое официальное сословие, иду я войной. Я сам --- хотя и атеист --- признаю себя открыто сторонником религии в указанном смысле, естественной религии. Я ненавижу тот идеализм, который вырывает человека из природы; я не стыжусь моей зависимости от природы, я открыто признаю, что действия природы не только влияют на мою поверхность, на мою кожу, на мое тело, но и на мою сердцевину, мою душу, что воздух, который я вдыхаю при ясной погоде, действует благотворно не только на мои легкие, но и на мою голову, что свет солнца просветляет не только мои глаза, но и мой дух и мое сердце. И я не нахожу, чтобы эта зависимость оказывалась в каком-либо противоречии, как это полагают христиане, с моим существом, и не надеюсь потому ни на какое избавление от этого противоречия. Я знаю также, что я конечное, смертное существо, что я когда-нибудь не буду существовать. Но я полагаю это совершенно естественным, и именно поэтому я вполне примиряюсь с этой мыслью. 

Я утверждаю, далее, в своих сочинениях и докажу это в этих лекциях, что в религии человек опредмечивает свое собственное существо. Это положение подтверждают уже вами факты естественной религии. Ибо что другое запечатлели мы в празднествах естественной религии, --- а в ее празднествах именно и дает себя знать самым непререкаемым образом у древних, чувственных, простых народов сущность их религии, --- что другое запечатлели, как не ощущения и впечатления, которые оказывает на человека природа в ее важнейших проявлениях и в важнейшие периоды времени? Французские философы ничего другого не видели в религиях древности, как физику и астрономию. Это утверждение верно, если понимать под ним --- в противоположность философам не научную физику или астрономию, а только эстетическую физику и астрономию; в первоначальных элементах древних религий мы лишь опредмечивали ощущения, впечатления, производимые на человека предметами физики и астрономии, до тех пор, пока эти предметы не сделались для него объектами науки. Правда, к религиозному воззрению на природу еще у древних народов, а именно у касты жрецов, которой ведь одной у древних народов были доступны наука и ученость, присоединялись еще и наблюдения, следовательно, элементы науки; но их нельзя сделать первичным текстом естественной религии. Если я, впрочем, мое воззрение отождествляю с естественной религией, то я прошу не забывать, что и естественно-природной религии уже присущ элемент, которого я не признаю; ибо хотя предметом естественно-природной религии является лишь природа, как уже показывает само название, но все же человеку, стоящему на своей первоначальной точке зрения, точке зрения естественной религии, природа является не предметом, не такою, какова она есть в действительности, а лишь какою она представляется некультурному и неопытному уму, фантазии, духу, так что поэтому уже и здесь человек имеет сверхъестественные желания, а следовательно, и ставит природе сверхъ- или, что то же, неестественные требования. Или иными, более отчетливыми словами: уже и естественная религия не свободна от предрассудков, ибо от природы, то есть без образования, все люди, как Берно говорит Спиноза, подвержены предрассудкам. И я не хочу поэтому взвалить на себя подозрение, будто если я говорю в защиту естественной религии, то я поэтому хочу также говорить и в защиту религиозного предрассудка. Я не признаю естественной религии как-либо иначе, в ином каком-либо объеме, в ином каком-либо смысле, чем в том, в котором я вообще признаю религию, также и христианскую; я признаю лишь ее простую основную истину. Но эта истина только та, что человек зависим от природы, что он должен с природой жить в согласии, что он, даже исходя из своей высшей, духовной точки зрения, не должен забывать, что он дитя и член природы, что он должен природу, --- и как основу и источник своего существования, и как основу и источник своего духовного и. телесного здоровья, --- всегда почитать, считать священной, ибо только через ее посредство человек освобождается от болезненных, взвинченных требований и желаний, как, например, от сверхъестественного желания бессмертия. <<Станьте близки к природе, признайте ее матерью;
тогда в землю спокойно опуститесь вы в некий день>>. Как я в <<Сущности христианства>>  определяя человека целью для человека, ни в малой мере не хочу обожествлять его, как это мне глупым образом приписывали, обожествлять, то есть делать богом в смысле теологически-религиозной веры, которую я ведь разлагаю на ее человеческие антитеологические элементы, так же мало хочу я обожествлять природу, в смысле теологии или пантеизма, когда я полагаю ее основой человеческого существования, существом, от которого человек должен себя сознавать зависимым, неотделимым. Как я человеческую личность могу почитать и любить, не обожествляя ее, не игнорируя даже ее ошибок и недостатков, так же точно могу я признавать природу существом, без которого я ничто, и при этом не забывать, что у нее недостает сердца, разума и сознания, которые она обретает только в человеке, и не впадать, стало быть, при этом в ошибку естественной религии и философского пантеизма, делавших природу богом. Истинная образованность и истинная задача человека заключаются в том, чтобы брать вещи и трактовать их так, как они есть, и делать из них не больше, но и не меньше того, что они есть. Естественная же религия, пантеизм, делает из природы слишком много, как, наоборот, идеализм, теизм, христианство делают из нее слишком мало, сводя на нет. Наша задача состоит в том, чтобы избежать крайностей, превосходных степеней или преувеличений религиозного чувства и рассматривать природу, обращаться с ней и почитать ее такою, какова она есть, --- как нашу мать. Как нашей родной матери оказываем мы должное ей уважение, и как нам не нужно, чтобы ее почитать, забывать о границах ее индивидуальности, ее женского существа вообще, как мы в отношении к нашей родной матери не остаемся просто на точке зрения ребенка, а относимся к ней с свободным взрослым сознанием, так же точно должны мы смотреть и на природу не глазами религиозных детей, а глазами взрослого человека, исполненного самосознания. Древние народы, которые от избытка религиозного аффекта и смиренного чувства почитали все возможное богом, которые почти на все смотрели религиозными глазами, называли и родителей богами, как это, например, значится в одной гноме Менандра. Но как для нас родители не являются ничем, потому что они перестали быть для нас богами, потому что мы не наделяем их, как древние римляне и персы, правом власти над жизнью и смертью ребенка, следовательно привилегией божества, так же точно и природа, вообще всякий предмет не превращается в ничто, в предмет ничтожный только потому, что мы лишили его божественного ореола. Наоборот, предмет лишь тогда обретает свое настоящее, ему присущее достоинство, когда у него отнимают этот священный ореол; потому что до тех пор, пока какая-либо вещь или существо является предметом религиозного почитания, до тех пор оно рядится в чужие перья, а именно в павлиньи перья человеческой фантазии. 

Содержание третьего параграфа заключается в том, что бытие и существо человека, поскольку он определенный человек, находится также в зависимости только от определенной природы, от природы его страны, и поэтому он по необходимости и с полным правом делает природу своего отечества предметом своей религии. 

К этому параграфу я не имею ничего другого добавить, кроме того, что если неудивительно, что люди почитают природу вообще, то чему же удивляться, зачем сожалеть или смеяться над тем, что они религиозно почитают в особенности ту природу, в которой они живут и действуют, которой одной только они обязаны своим своеобразным, индивидуальным существом, следовательно --- природу своего отечества. Если по этому поводу их порицать или высмеивать, то надо вообще высмеять и отвергнуть религию; ибо если чувство зависимости есть основа религии, предметом же чувства зависимости является природа как существо, от которого зависит жизнь, существование человека, то совершенно естественно также, что не природа вообще или как таковая, а природа данной страны составляет предмет религиозного почитания, ибо только данной стране обязан я своей жизнью, своим существом. Я ведь сам не человек вообще, а данный, определенный, особенный человек. Так, я человек, говорящий и думающий по-немецки, --- ведь в действительности не существует языка вообще, а только тот или иной язык. И эта определенность характера моего существа, моей жизни неотделимо зависит от данной почвы, данного климата; особенно же это относится к древним народам, так что нет ничего смешного в том, что они религиозно почитали свои горы, свои, реки, своих животных. Это тем менее удивительно, что древним, примитивным народам по недостатку опыта и образования их страна представлялась всей землей или по меньшей мере центром земли. Наконец, это тем менее удивительно у древних народов, живших замкнуто, когда даже у народов современных, цивилизованных, живущих среди грандиозного мирового оборота, патриотизм все еще играет религиозную роль. Ведь даже французы имеют поговорку: <<Господь бог --- добрый француз>>  и даже в наши дни не стыдятся немцы, которые поистине не имеют основания, по крайней мере в политическом отношении, быть гордыми своим отечеством, --- говорить о немецком боге. Не без основания говорю я поэтому в одном примечании к <<Сущности христианства>>  что до тех пор, пока есть много народов, до тех пор будет и много богов; ибо бог какого-либо народа, по крайней мере его подлинный бог, которого, конечно, следует отличать от бога его догматиков и философов религии, есть не что иное, как его национальное чувство, национальный point d'honneur сознание чести. Этим сознанием чести для древних, примитивных народов была их страна. Древние персы, например, как сообщает Геродот, расценивали даже другие народы исключительно по степени отдаленности их страны от Персии: чем ближе, тем выше, чем отдаленнее, тем ниже. А египтяне, по свидетельству Диодора, видели в тине Нила первичную и основную материю животной и даже человеческой жизни. 

\phantomsection
\addcontentsline{toc}{section}{Шестая лекция}
\section*{Шестая лекция}

Конец последней лекции был, в противоположность христианскому супранатурализму, оправданием и обоснованием точки зрения естественной религии, а именно той точки зрения, что определенный и ограниченный человек почитает только определенную и ограниченную природу --- горы, реки, деревья, животных и растения своей страны. Как самую парадоксальную часть этого культа я сделал культ животных предметом следующего параграфа и оправдал его тем, что животные --- существа, человеку необходимые, без которых он не может обойтись, что от них зависит его человеческое существование, что только при их помощи он поднялся на высоту культуры, что человек, однако, почитает богом то, от чего зависит его существование, что поэтому в предмете своего почитания, а стало быть, и в животных он выявляет лишь ту ценность, которую он придает себе и своей жизни. 

Много спорили о том, были ли и в каком смысле и на каком основании животные предметом религиозного почитания. Что касается первого вопроса, самого факта почитания животных, то речь о нем заходила главным образом при рассмотрении религии древних египтян, и на этот вопрос отвечали как <<да>>  так и <<нет>>. Но если мы прочтем, что нам новейшие путешественники рассказывают как очевидцы, то нам не покажется невероятным, что древние египтяне, если против этого нет каких-либо особых противопоказаний, так же точно почитали или по крайней мере могли почитать животных, как почитали их еще недавно или почитают даже сейчас народы в Азии, Африке, Америке. Так, например, по словам Марциуса в его <<Правовом состоянии первоначальных обитателей Бразилии>>  ламы почитаются священными многими перуанцами, другие же молятся маисовому растению. Так, бык есть предмет поклонения у индусов. <<Ему ежегодно оказывают божеские почести, его украшают лентами и цветами, падают перед ним ниц. У них много деревень, где быка содержат, как живого идола, и если он умирает, то хоронят его с большими почестями>>. Точно так же <<все змеи священны для индуса. Есть служители идолов, которые являются до такой степени слепыми рабами своих предрассудков, что они считают за счастье быть укушенными змеей. Они считают тогда это своим предназначением и думают затем только о том, чтобы как можно радостнее закончить свою жизнь, ибо они верят, что на том свете займут какой-либо очень важный пост при дворе змеиного бога>> (Энциклопедия Эрша и Грубера, статья <<Индостан>>). Благочестивые буддисты и еще более яйны или джайны --- индийская секта, родственная буддистам, --- считают каждое убиение малейшего насекомого смертным грехом, равнозначащим убийству человека (Болен, <<Древняя Индия>>  т. 1). Джайны устраивают <<форменные лазареты для животных, даже для низших и наиболее презираемых пород, и оплачивают деньгами бедных людей для того, чтобы они устраивали ночевки в таких местах, предназначенных для насекомых, и давали им кусать себя. Многие носят постоянно кусок полотна, прикрывающий рот, чтобы не проглотить летающей букашки и не отнять у нее таким образом жизни. Некоторые проводят мягкой губкой по тому месту, на которое они хотят стать, дабы не раздавить самомалейшего животного. Или они носят с собой мешочки с мукой или сахаром или сосуд с медом чтобы поделиться ими с муравьями или другими животными>> (Энциклопедия Эрша и Грубера, ст. <<Джайны>>). <<Жители Тибета также щадят клопов, вшей и блох не менее, чем ручных и полезных животных. В Аве с домашними животными обращаются, как с собственными детьми. Женщина, у которой умер попугай, кричала, плача: <<мой сын умер, мой сын умер!>> И она велела похоронить его торжественно, будто своего сына>> (Мейнерс, <<Всеобщая критическая история всех религий>>). Удивительно, что, как замечает этот же ученый, большинство пород животных, которых в древнем Египте и на Востоке вообще почитали как богов, до сих пор признают христианские и магометанские жители этих стран неприкосновенными. Христианские копты, например, устраивают госпитали для кошек и делают завещательные распоряжения, чтобы коршуны и другие птицы получали в определенные сроки корм. Жители Суматры, по словам В. Марсдена в его <<Описании острова Суматры>>  питают такое религиозное почтение к аллигаторам и тиграм, что вместо того, чтобы уничтожать их, дают им уничтожать себя. Тигров они не решаются даже называть их обычным именем, но называют их своими предками или стариками, <<либо потому, что они сами их за таковых считают, либо чтобы им таким образом польстить. Когда европеец велит менее суеверным лицам поставить западни, то они приходят ночью на места и проделывают некоторые церемонии, чтобы убедить животное, если оно поймано или чует приманку, что западня поставлена не ими и не с их согласия>>. 

После того, как я некоторыми примерами подтверждаю факт обожествления и почитания животных, я перехожу к причине и смыслу этих явлений. Я свел причину их также к чувству зависимости. Животные были для человека необходимыми существами; без них он не мог существовать, не говоря уже о том, чтобы существовать как человек. Необходимость же есть то, от чего я завишу; поскольку поэтому природа вообще, как основной принцип человеческого существования, сделалась предметом религии, постольку могла и должна была сделаться предметом религиозного почитания и природа животного царства. Я рассматривал поэтому культ животных главным образом лишь в связи с тем временем, когда он имел свое историческое оправдание, в связи со временем начинающейся культуры, когда животные имели наибольшее значение для человека. Но разве малое значение имеет животное даже еще и для нас, смеющихся над культом животных? К чему способен охотник без охотничьей собаки, пастух --- без овчарки, крестьянин --- без быка? Не есть ли навоз душа хозяйства, а стало быть, не является ли бык и у нас еще, как это было у древних народов, высшим принципом, богом агрикультуры? Зачем же нам смеяться над древними народами, если они религиозно почитали то, что для нас, людей рационалистических, еще имеет величайшую цену? Не ставим ли и мы еще во многих случаях животное выше человека? Не имеет ли еще в христианско-германских государствах конь для армии большую ценность, чем всадник, для крестьянина бык --- большую ценность, чем батрак? И в качестве исторического примера я привел в настоящем параграфе одно место из <<Зенд-Авесты>>. <<Зенд-Авеста>> в ее настоящем виде есть, разумеется, лишь позже составленная и искаженная религиозная книга древних персов. Так вот там значится, --- правда, в старом, ненадежном переводе Клейкера, в части, называющейся <<Вендидад>>: --- <<Мир существует благодаря уму собаки\dots Если бы собака не охраняла улиц, то разбойники и волки расхитили бы все имущество>>. Именно по причине этой своей важности, но, разумеется, также и благодаря религиозным предрассудкам, собака в законах именно этой самой <<Зенд-Авесты>>  в качестве стража-охранителя от хищных зверей, <<не только приравнивается к человеку, но ей отдается даже предпочтение при удовлетворении ее потребностей>>. Так, например, говорится: <<Кто увидит какую-либо голодную собаку, обязан ее накормить лучшими кушаньями>>. <<Если сука со щенятами заблудится, то глава селения, где она нашлась, обязан взять ее и накормить; если он этого не сделает, то наказуется изувечением тела>>. Человек имеет поэтому меньше ценности, чем собака; впрочем еще худшие постановления, ставящие человека ниже животного, находим мы в религии египтян. <<Кто, значится у Диодора, --- убьет одно из этих (а именно священных) животных, подлежит смерти. Если это была кошка или ибис, то он должен во всяком случае умереть, все равно, убил ли он животное преднамеренно или случайно; сбегается толпа и расправляется с виновным самым жестоким образом>>. 

Однако против этого объяснения почитания животных их незаменимостью и необходимостью говорят как будто даже и приведенные мною примеры. Тигры, змеи, вши, блохи --- какие же это необходимейшие для человека животные? Ведь необходимые животные только те, что полезны. <<Если в общем, --- замечает Мейнерс в своем указанном сочинении, --- полезным животным больше поклонялись, чем вредным, то отсюда нельзя заключить, что полезность животных была причиною их божеского почитания. Полезные животные чтутся не в соответствии с их полезностью и вредные --- не в соответствии с их вредностью. Как неизвестны и не поддаются исследованию те обстоятельства, которые были благоприятны одному животному здесь, другому --- там, так необъяснимы и противоречивы и многие явления в культе животных. Так, например, негры в Сенегале и Гамбии чтут и щадят тигров, тогда как в царстве Анте и других соседних царствах вознаграждают тех, кто убьет тигра>>. В самом деле, в области религии мы оказываемся прежде всего в хаосе величайших и запутаннейших противоречий. Тем не менее, несмотря на это, при более глубоко идущем наблюдении их можно свести к мотивам страха и любви, которые, однако, сообразно различию людей, направляются на самые различные предметы и сводятся к чувству зависимости. Если даже какое-либо животное не приносит действительных естественно-исторически доказуемых пользы или вреда, то человек все же в своем религиозном воображении связывает с ним часто по случайному, нам неизвестному поводу суеверные действия \hyperlink{3}{(3)}\hypertarget{b3}{}. Каких только чудодейственных, лечебных сил не приписывали драгоценным камням! На каком основании? Из суеверия. Таким образом, внутренние мотивы почитания одинаковы, их проявления различаются только тем, что почитание одних предметов основывается на воображаемых пользе или вреде, существующих лишь в области веры или суеверия, тогда как почитание других основывается на действительной их благотворности или полезности, пагубности или вредоносности. Короче говоря, счастье или несчастье, благо или горе, болезнь или здоровье, жизнь или смерть от одних предметов религиозного почитания зависят и на самом деле, по-настоящему, от других же --- лишь в воображении, в вере, в представлении. 

Сверх того, я хочу заметить по настоящему поводу, когда различные свойства и многообразие религиозных предметов кажутся противоречащими приводимому мною объяснению возникновения религии, что я бесконечно далек от того, чтобы сводить религию, как и вообще какой-либо предмет, к чему-либо одностороннему, абстрактному. Когда я думаю о каком-либо предмете, то я всегда имею его перед своими глазами в его целокупности. Мое чувство зависимости не есть теологическое, шлейермахеровское, туманное, неопределенное, отвлеченное чувство. Мое чувство зависимости имеет глаза и уши, руки и ноги, мое чувство зависимости есть лишь человек, чувствующий себя зависимым, видящий себя зависимым, короче говоря --- сознающий себя всесторонне и во всех смыслах зависимым 

То же, от чего человек зависит, от чего он чувствует себя зависимым, от чего он знает свою зависимость, есть природа, предмет чувств. Совершенно поэтому в порядке вещей, что все впечатления, производимые природой на человека при посредстве чувств, хотя бы то были только впечатления идиосинкразии, могут сделаться мотивами религиозного почитания, и на самом деле таковыми делаются, что предметами религии делаются и те предметы, которые затрагивают лишь теоретические чувства и не имеют к человеку того непосредственного практического отношения, которое и заключает в себе истинные мотивы страха и любви. Даже в том случае, если какое-либо существо природы делается предметом религиозного почитания, для того ли, чтобы быть обезвреженным, если оно страшно или вредно, или чтобы получить благодарность за свою доброту, если оно благодетельно и полезно, даже и в этом случае оно имеет ведь еще и другие стороны, которые равным образом попадают в поле зрения и в сознание человека и становятся поэтому моментами религии. Если перс почитает собаку за ее бдительность и верность, за это ее, так сказать, политическое и моральное значение и необходимость для человека, то ведь собака является предметом оценки и предметом созерцания не in abstracto только, как страж, но и со всеми своими другими сторонами, природными качествами, в своем целом, в своей совокупности, и естественно поэтому, что и эти качества являются силами, принимающими участие в создании религиозного предмета. Так в <<Зенд-Авесте>> определенно приводятся еще и другие качества собаки, а не только ее полезность и бдительность. <<У нее, --- говорится, например, там, --- восемь удивительных качеств: она подобна аторну (жрецу), воину, земледельцу --- источнику богатств, птице, разбойнику, зверю, злой женщине, юноше. Как жрец, она ест то, что найдет\dots как жрец, она идет ко всем, кто ее ищет\dots собака много спит, как юноша, и, как юноша, пылка в действии>> и так далее. Так цветок лотоса (Nymphaea Lotus), который был главным предметом почитания у древних египтян и индийцев и до сих пор почитается почти на всем Востоке, есть не только полезное растение, --- ибо его корни съедобны и в особенности в прежние времена были главною пищей египтян, --- но также и один из прекраснейших водяных цветков. И если у разумного и практического, способного к культуре народа основой религиозного почитания являются только рациональные свойства предмета, имеющие значение для человеческого существования и образованности, то у народа с противоположным характером мотивами религиозного почитания могут сделаться свойства предмета только иррациональные, для человеческого существования и культуры безразличные, даже курьезные. Могут даже почитаться вещи и существа, для почитания которых нельзя привести другого основания, кроме особой симпатии или идиосинкразии. Если религия есть не что иное, как психология и антропология, то само собой разумеется, что идиосинкразия и симпатия играют в ней также роль. Все странные и бросающиеся в глаза явления в существе природы, все, что приковывает и поражает глаз человека, что изумляет и пленяет его слух, что воспламеняет его фантазию, возбуждает его удивление, что действует на его душевное состояние особым, необычным, необъяснимым образом, все это играет определенную роль при возникновении религии и может дать основу и самый предмет для религиозного почитания. <<Мы с почтением взираем, --- говорит Сенека в своих письмах, --- на верховья (то есть истоки) более значительных рек. Мы воздвигаем алтари ручью, внезапно с силою выбивающемуся из прикрытия. Мы почитаем источники теплых вод, и некоторые озера для нас священны, потому что они темные и неизмеримо глубокие>>. <<Реки почитаются,- говорит Максим Тирский в своей восьмой диссертации, --- либо за их полезность, как Нил у египтян, либо за их красоту, как Пеней у фессалийцев, либо за их величину, как Истр у скифов>>  либо по каким-нибудь другим побуждениям. <<Ребенок, --- говорит Клауберг, немецкий, хотя и по латыни писавший философ семнадцатого века, даровитый ученик Декарта, --- всего более привлекается и захватывается светлыми и блестящими предметами. Вот причина, почему варварские народы дали себя увлечь культом солнца и небесных тел и другим подобным же кумиропочитанием>>. 

Но хотя все эти впечатления, аффекты и настроения, то есть такие элементы религии, как отблеск света на камнях, --- ведь и камни почитаются, жуткость ночи, темнота и тишина леса, глубина и неизмеримость моря, бросающиеся в глаза своеобразие и причудливость, миловидность и устрашающий облик животных, --- хотя все они являются при объяснении и понимании религии величинами, принимаемыми в расчет и соображение, все же человек еще не находится здесь на почве истории, он пребывает в состоянии детства, как и отдельный человек не является еще историческим лицом, хотя он и делается таковым впоследствии, когда он без разбора, без критики дает господствовать над собой впечатлениям и аффектам, от которых только и заимствует своих богов. Такие боги --- только падающие звезды, метеоры религии. Лишь тогда, когда человек начинает обращаться к таким свойствам предметов, которые постоянно, длительно напоминают человеку об его зависимости от природы, которые непрестанно чувствительным образом дают ему ощутить, что он без природы ничего не может, что он ничто, когда он эти свойства делает предметом своего почитания, лишь тогда возвышается он до настоящей, постоянной, исторической религии со сформированным культом. Так, например, солнце является лишь там предметом настоящего культа, где оно почитается не ради своего блеска, своего сияния, одного только своего поражающего глаз существа, но где оно почитается, как высший принцип земледелия, как мера времени, как причина естественного и гражданского порядка, как очевидная, бесспорная основа человеческой жизни, короче говоря, где оно почитается ради его необходимости, его благотворности \hyperlink{4}{(4)}\hypertarget{b4}{}. Лишь там, где культурно-исторический элемент в предмете вступает в поле зрения человека, лишь там и религия, или одна из ее ветвей, составляет характерный исторический момент, объект, интересующий исследователя истории и религии. Это относится и к культу животных. Хотя в религии почитание охватывает также и животных, безразличных с точки зрения культурно-исторической, тем не менее почитание культурно-исторических животных есть все же та характерная, та разумная сторона в ней, которую надлежит отметить; причина, почему почитаются другие животные, почему вообще почитаются предметы и свойства, не обусловливающие и не обосновывающие существования человека и его человеческих черт, находится, как было уже отмечено, также не вне пределов культа предметов, достойных почитания по соображениям гуманности. Предметы природы, которые наиболее необходимы, наиболее важны, наиболее оказывают влияние, наиболее вызывают в человеке чувство зависимости от них, имеют также все свойства, которые действуют сильнейшим образом на зрение и на душевное состояние и вызывают изумление, преклонение и все другие подобного же рода аффекты и настроения. <<Приветствуем тебя, --- говорится поэтому в ,,Феноменах`` Арата в обращении к Зевсу, к богу, к причине небесных явлений, --- приветствуем тебя, отец, ты --- великое чудо (то есть великое существо, вызывающее изумление и поклонение), ты --- великое утешение людей>>. Мы имеем таким образом в одном и том же предмете соединенными обе только что указанные черты. Но предметом религии, предметом культа является не thauma, чудо, a oneiar, утешение, защита, то есть не существо, поскольку оно есть предмет удивления, а существо, составляющее предмет страха и надежды; оно является предметом культа не за свои качества, вызывающие удивление и изумление, а за свои качества, обосновывающие и сохраняющие человеческое существование и возбуждающие чувство зависимости. 

То же самое применимо и к культу различных животных, сколько бы богов-животных ни было обязано своим существованием thauma, лишенному критики глазению, бессмысленному удивлению, не знающему границ произволу религиозных предрассудков. Мы не должны поэтому удивляться и стыдиться, что человек почитал животных, ибо человек любил и почитал в них только себя самого; он по крайней мере там, где культ животных представляет собой культурно-историческое явление, --- почитал животных только за их услуги перед человечеством, а стало быть, перед ним самим, не из животных, а из гуманных побуждений. 

Что человек, почитая животных, почитает себя, тому пример мы имеем в том, как человек в настоящее время еще расценивает животных. Охотник ценит только животных, пригодных для охоты, крестьянин --- нужных для земледелия, то есть охотник ценит в животном процесс охоты, составляющий его собственное существо, крестьянин --- только хозяйство, являющееся его собственной душой и практическим божеством. Поэтому и в культе животных мы имеем доказательство и пример того утверждения, что в религии человек опредмечивает лишь собственное существо. Сколь различны люди, сколь различны их потребности, сколь различна их существенная, характеризующая их точка зрения, столь же различны --- по крайней мере у народов, принадлежащих к истории культуры животные, которых они главным образом почитают, так что по качеству животных, которые служили предметом почитания, можно судить и о самом качестве людей, их почитавших. Так, собака, замечает Роде в своем сочинении <<Священная сага и вся религиозная система древних бактрийцев, мидян и парсов или зендского народа>>  <<была для парсов, живших сначала одним только скотоводством, важнейшей опорой в борьбе против звериного ариманского мира, то есть против волков и других хищных животных, поэтому наказывался смертью тот, кто убивал пригодную собаку или беременную суку. Египтянину не нужно было бояться при обработке своей земли ни волков, ни других хищных зверей. Крысы и мыши, вредившие ему, были орудиями Тифона, поэтому кошка исполняла у него ту роль, которая возложена была на собаку у зендского народа>>. Но не только культурную практику народа, а и его теоретическое существо, его духовную точку зрения вообще объективирует нам культ животных --- естественный культ вообще; ибо там, где человек почитает животных и растения, там он еще не человек, подобный нам, там он отождествляет себя с животными и растениями, там они для него частью человеческие, частью сверхчеловеческие существа. Так, например, в <<Зенд-Авесте>> собака подчинена законам наравне с человеком. <<Если собака укусит домашнее животное или человека, то для первого раза ей отрезают правое ухо, для второго --- левое, для третьего правую ногу, для четвертого --- левую, для пятого --- хвост>>. Так, троглодиты, согласно Диодору, называли быка и корову, барана и овцу отцом и матерью, потому что они постоянно получали свою ежедневную пищу от них, а не от своих кровных родителей. Так, индейцы в Гватемале, по сообщению Мейнерса, верят, подобно африканским неграм, что жизнь каждого человека неразрывно связана с жизнью определенного животного и что если братское животное будет убито, то и человек должен умереть. Так, и Сакунтала говорит цветам: <<Я чувствую к этим растениям любовь сестры>>. Прекрасный пример отличия человеческого существа, стоящего на точке зрения восточного почитания природы, от человеческого существа, имеющего нашу точку зрения, дает нам анекдот, рассказанный В. Джонсом, что, когда однажды у него лежал на столе цветок лотоса для исследования, пришел к нему чужеземец из Непала, и как только он увидал цветок, он благоговейно пал ниц. Какая разница между человеком, набожно падающим ниц перед цветком, и человеком, смотрящим на цветок только с точки зрения ботаники! 

\phantomsection
\addcontentsline{toc}{section}{Седьмая лекция}
\section*{Седьмая лекция}

Утверждение, что человек в животных почитает самого себя, утверждение, не опрокидываемое даже тем культом животных, который не имеет под собой культурно-исторических, рациональных оснований, который своим существованием обязан лишь страху или даже особым случайностям или идиосинкразии, ибо где человек почитает какое-либо существо без основания, там он опредмечивает в нем лишь собственное неразумие и безумие, --- с этим утверждением, говорю я, мы пришли к самому важному положению параграфа, к положению, что человек почитает богом то, от чего он считает зависимой свою жизнь или в зависимость от чего он верит; что именно поэтому в предмете почитания сказывается, обнаруживается лишь та цена, которую он придает своей жизни, себе вообще, что, стало быть, почитание бога зависит от почитания человека, Это положение есть, правда, лишь предвосхищение, антиципация результата и дальнейшего содержания этих лекций; но так как оно встречается уже в этом параграфе, так как оно чрезвычайно важно для всего развития моего взгляда на религию, то пусть оно станет предметом рассмотрения уже по данному поводу, когда речь идет о культе животных, который, поскольку в основе его лежит разумный смысл, именно и подтверждает и делает очевидной истину этого положения. 

Подводя итог изложенному, скажем: там, где культ животных возвышается до значения культурного момента, явления из истории религий, достойного упоминания, --- там он имеет человеческую, эгоистическую, основу. Я употребляю к ужасу лицемерных теологов и фантастов-философов слово <<эгоизм>> для обозначения основы и сущности религии. Некритические критики, цепляющиеся за слова, высокомудро высосали поэтому из моей философии, что ее результатом является эгоизм, и что именно поэтому я и не проник в сущность религии. Но если я слово <<эгоизм>>  --- заметьте, --- употребляю в значении философского или универсального принципа, то понимаю я под ним не эгоизм в обыкновенном смысле этого слова, как это может усмотреть всякий, хоть немного способный к критике, из тех сочетаний, из той связи, из того противоположения, в которых я употребляю слово <<эгоизм>>; употребляю же я его в противоположение к теологии или вере в бога, в понимании которой, если эта вера строга и последовательна, каждая любовь, раз она не имеет своею целью и предметом бога, даже и любовь к другим людям, есть эгоизм; я понимаю поэтому под этим словом не эгоизм человека по отношению к человеку, нравственный эгоизм, не тот эгоизм, который во всем, что он делает, даже как будто для других, соблюдает лишь свою выгоду, не тот эгоизм, который является характерной чертой филистера и буржуа и составляет прямую противоположность всякому дерзанию в мышлении и действии, всякому воодушевлению, всякой гениальности и любви. Я понимаю под эгоизмом человека соответствующее его природе, а стало быть, и разуму, --- ибо разум человека ведь не что иное, как сознательная природа его, --- его самопризнание, самоутверждение по отношению ко всем неестественным и бесчеловечным требованиям, которые предъявляют к нему теологическое лицемерие, религиозная и спекулятивная фантастика, политическая грубость и деспотизм. Я понимаю под эгоизмом эгоизм необходимый, неизбежный, не моральный, как я уже сказал, а метафизический, то есть эгоизм, основывающийся на существе человека без его ведома и воли, тот эгоизм, без которого человек не может жить: ибо для того, чтобы жить, я должен постоянно присваивать себе то, что мне полезно, и отстранять то, что мне враждебно и вредно, тот эгоизм, стало быть, который коренится в самом организме, в усвоении усвояемой материи и в выбрасывании неусвояемой. Я понимаю под эгоизмом любовь человека к самому себе, то есть любовь к человеческому существу, ту любовь, которая есть импульс к удовлетворению и развитию всех тех влечений и наклонностей, без удовлетворения и развития которых человек не есть настоящий, совершенный человек и не может им быть; я понимаю под эгоизмом любовь индивидуума к себе подобным индивидуумам, --- ибо что я без них, что я без любви к существам, мне подобным? --- любовь индивидуума к самому себе лишь постольку, поскольку всякая любовь к предмету, к существу есть косвенно любовь к самому себе, потому что я ведь могу любить лишь то, что отвечает моему идеалу, моему чувству, моему существу. Короче говоря, я понимаю под эгоизмом тот инстинкт самосохранения, в силу которого человек не приносит в жертву себя, своего разума, своего чувства, своего тела духовным --- если взять примеры из ближе всего нам знакомого культа животных --- духовным ослам и баранам, политическим волкам и тиграм, философским сверчкам и совам, тот инстинкт разума, который говорит человеку, что глупо, бессмысленно из религиозного самоотрицания давать вшам, блохам и клопам высасывать кровь из тела и разум из головы, давать отравлять себя гадюкам и змеям, поедать себя --- тиграм и волкам; тот инстинкт разума, который, когда даже человек заблудится или опустится до почитания животных, кричит ему: почитай только тех животных, в которых ты почитаешь самого себя, животных, которые тебе полезны, тебе нужны; ибо даже животных, которых ты почитаешь, не имея для их почитания разумного основания, ты почитаешь ведь только потому, что ты, по крайней мере, веришь, ты воображаешь, что почитание их не без пользы для тебя. Впрочем, выражение <<польза>> как я уже объяснял в моих комментариях и дополнениях к <<Сущности религии>>  вульгарно, неподходяще, противоречит религиозному смыслу; ибо полезна и вещь; но то, что есть бог, предмет религиозного почитания, не есть вещь, а есть существо; полезный же есть выражение простой потребляемости, способности быть использованным, выражение пассивности; но деятельность, жизнь есть существенное свойство богов, как это уже верно отметил Плутарх. Религиозное понятие полезности есть выражение благодетельности; ибо только благодетельность, а не полезность внушает мне чувства благодарности, почитания, любви, и только эти чувства по своей природе, по своему действию религиозны. Природа вообще, растения и животные в особенности почитаются за их благодетельность --- если выражаться языком религиозным или поэтическим, за их полезность --- если говорить на языке не религиозном, а обыденном или прозаическом, за их необходимость, невозможность без них существовать выражаясь философски. 

Культ животных поэтому, --- по крайней мере там, где он имеет разумный религиозный смысл, --- имеет принцип, общий со всяким культом; или: то, что в глазах людей, сколько-нибудь соображающих, возвышает животных до предмета религиозного почитания, то, что является основой их почитания, то же самое есть основа почитания и каждого другого предмета; основа же эта есть полезность или благодетельность. Боги людей различаются только сообразно различию благодеяний, оказываемых богами людям, различаются только сообразно различию влечений и потребностей человека, удовлетворяемых ими; предметы религии различаются только сообразно различию способностей или сил человеческого существа, к которым они имеют касательство. Так, например, Аполлон есть врач психических, моральных болезней, Асклепий-врач болезней физических, телесных. Но основа почитания, принцип их божественности, то, что их делает богами, это --- их отношение к человеку, их полезность, их благодетельность, это --- человеческий эгоизм; ибо если я не люблю себя прежде всего, не почитаю себя, то как могу я любить и почитать то, что мне полезно и благодетельно? Как я могу любить врача, если я не люблю здоровья? учителя, если я не хочу удовлетворить мою жажду знания? Как я могу почитать свет, если у меня нет глаз, которые ищут света, нуждаются в свете? Как я могу превозносить и восхвалять творца или источник моей жизни, если я сам себя презираю? Как поклоняться и признавать объективно высшее существо, если я не имею в себе субъективно высшего существа? Как признавать бога вне себя, если я сам себе, правда, на иной манер, --- не являюсь богом? Как верить во внешнего бога, не предполагая существования внутреннего, субъективного? 

Но что такое это высшее существо в человеке, от которого зависят все другие высшие существа, все боги вне его? Это --- понятие, включающее в себя все его человеческие влечения, потребности, наклонности, это вообще существование, жизнь человека, ибо она ведь охватывает собой все. Только потому делает человек богом или божественным существом то, от чего зависит его жизнь, что для него его жизнь есть божественное существо, божественное благо или предмет. Когда человек говорит: <<жизнь не есть высшее из благ>>  то он берет жизнь в узком, подчиненном смысле, тогда человек стоит на точке зрения несчастия, разрыва, а отнюдь не на точке зрения нормальной жизни; он, правда, отвергает тогда, презирает жизнь, но презирает только потому, что его жизни не хватает свойств или благ, существенно принадлежащих нормальной жизни, потому, что это уже не жизнь. Так, например, когда человек лишен свободы, когда он раб чужого произвола, то он может, он должен эту жизнь презирать, но только потому, что она недостаточная, ничтожная жизнь, жизнь, которой не достает существенного условия и свойства человеческой жизни, самопроизвольного движения и самоопределения. На этом основывается и самоубийство. Самоубийца посягает не на свою жизнь; она у него уже отнята. Поэтому он убивает себя; он разрушает лишь видимость; он отбрасывает лишь скорлупу, из которой уже давно, по его ли вине или без вины с его стороны, выедено нутро. Но в здоровом, закономерном состоянии и если разуметь под жизнью понятие, заключающее в себе все блага, имеющие существенное отношение к человеку, жизнь поистине есть высшее благо, высшая сущность человека. 

Подобно тому, как я для всех моих главных и основных положений привожу эмпирические, исторические примеры и подтверждения, ибо я хочу лишь осознать и высказать то, что думают и говорят другие, люди вообще, так и для этого утверждения я привожу в моих дополнениях и комментариях к <<Сущности религии>> некоторые места из Аристотеля, Плутарха, Гомера и Лютера. Несколькими цитатами я не хочу, разумеется, доказать истинности того или иного изречения, в чем меня упрекали смешные некритические критики. Я люблю краткость, в немногих словах я высказываю то, что другие высказывают в целых фолиантах. Но, разумеется, большинство ученых и философов отличается тем, что они только тогда убеждаются в вескости довода, когда он преподносится им в виде фолианта или, по крайней мере, основательно толстой книги. Эти несколько цитат приводятся вместо множества, они имеют всеобщее значение, они могут быть подтверждены тысячью и еще тысячью ученых цитат; но все эти тысячи не скажут больше, по крайней мере качественно, чем эти несколько мест. Что, однако, бесконечно больше ученой цитаты, что имеет бесконечно большее значение, так это практика, жизнь. И жизнь подтверждает нам на каждом шагу, который мы делаем, при каждом взгляде, который мы на нее бросаем, истину того положения, что жизнь для людей есть высшее благо. И точно так же она прежде всего подтверждает религию и ее историю; ибо подобно тому, как философия в конечном счете есть не что иное, как искусство мышления, так и религия есть в конечном счете не что иное, как искусство жизни, которое поэтому приносит в наше поле зрения и в наше сознание не что иное, как силы и влечения, непосредственно движущие жизнь человека. Эта истина и есть общий, всеохватывающий принцип всех религий. Только потому, что жизнь бессознательно и непроизвольно в силу необходимости является человеку божественным благом и существом, человек сознательно делает в религии богом то, от чего в действительности или только в воображении зависит происхождение и сохранение этого божественного блага. Всякое удовлетворение влечения, будь это влечение низшего или высшего порядка, физическое или духовное, практическое или теоретическое, представляет для человека божественное наслаждение, и только поэтому чтит он предметы или существа, от которых это удовлетворение зависит, чтит как чудесные, достойные поклонения, божественные существа. \emph{Народ, не имеющий духовных влечений, не имеет и духовных богов.} Народ, для которого разум, как субъект, то есть как человеческая сила и деятельность, не есть нечто божественное, никогда и ни в каком случае не сделает предметом своего почитания, своим богом порождение разума. Как могу я мудрость сделать богиней в виде Минервы, если для меня мудрость уже сама по себе ни есть божественное существо? Как могу я вообще обожествлять существо, от которого зависит моя жизнь, если для меня жизнь не представляет ничего божественного? Только различие в человеческих влечениях, потребностях, способностях, только это различие в них и их размещение по рангам определяет поэтому различие и размещение по рангам и среди богов и религий. Масштаб, критерий божества, и именно поэтому источник богов, человек имеет у себя и в себе. Что этому критерию соответствует, есть бог, что ему противоречит, бога не представляет. Но этот критерий есть эгоизм в распространительном смысле этого слова. 

Отношение предмета к человеку, удовлетворение потребности, необходимость, благотворность --- вот причины, почему человек делает какой-либо предмет своим богом. Абсолютное существо есть для человека, помимо его собственного сознания, сам человек, так называемые абсолютные существа, боги --- относительные, от человека зависимые существа, они для него лишь постольку боги, поскольку они служат его существу, поскольку они ему полезны, помогают, соответствуют, словом --- благотворны. Почему высмеивали греки богов египтян: крокодилов, кошек, ибисов? Потому, что боги египтян не соответствовали существу, потребностям греков. В чем же заключалась причина, что для них только греческие боги имели значение богов? В богах ли самих по себе? Нет! Она была в греках; в богах --- лишь косвенно, лишь постольку, поскольку они были существами, соответствующими существу греков \hyperlink{5}{(5)}\hypertarget{b5}{}. Но почему христиане отвергали языческих греческих и римских богов? Потому, что их религиозный вкус изменился, потому, что языческие боги не давали им того, чего они хотели. Почему, таким образом, для них только их бог есть бог? Потому, что он есть сущность их существа, им подобен, потому, что он соответствует их потребностям, их желаниям, их представлениям. 

Мы отправлялись сначала от самых общих и обычных явлений религии и отсюда уже перешли к чувству зависимости; но сейчас мы вернулись через и за пределы чувства зависимости назад и открыли в качестве последнего субъективного основания религии человеческий эгоизм в указанном смысле, хотя и эгоизм в самом вульгарном и обычном смысле этого слова играет не подчиненную роль в религия. Но я его оставляю в стороне. Спрашивается только, отвечает ли истине это объяснение основы и сущности религии и ее предметов, богов, объяснение, абсолютно противоречащее обычным сверхчувственным и сверхчеловеческим, то есть фантастическим объяснениям религиозных основ? Попал ли я в точку этим словом, правильно ли высказал то, что человечество имеет в мыслях, когда почитает богов? Я, правда, привел уже достаточно доводов и примеров, но так как предмет слишком важен и так как ученых можно побить только их собственным оружием, то есть цитатами, то я приведу еще несколько. <<Растение, дерево, --- плодами которых питались, говорит Роде в своем, уже упомянутом, сочинении, касаясь религии древних индийцев и персов, --- почитались, и к ним обращались с просьбой приносить впредь еще больше плодов. Почитались животные, чьим молоком и мясом питались; вода, делавшая землю плодородной; огонь, так как он согревал и светил, и солнце со всеми прочими звездами, потому что их благотворное влияние на всю жизнь не могло ускользнуть от внимания и самого тупого человека>>. Автор также уже упоминавшейся работы <<De I'origine des principes religieux>> (<<О происхождении религиозных принципов>>) приводит из Histoire des Incas de Perou, par Garcillaso de la Vega (<<История Инков Перу>>  сочиненная Гарсиласо де ла Вега), --- сочинение, которое я, к сожалению, не мог раздобыть, --- следующее место: <<Жители Чинчи говорили Инке, что они не хотят признавать ни Инку своим королем, ни солнце своим богом, что у них уже есть бог, которому они поклоняются, что их общий бог  --- море, которое совсем не то, что солнце, оно дает им множество рыбы для пропитания, тогда как солнце не делает им никакого добра, его же исключительный жар им только тягостен, так что им нечем от него попользоваться>>. Таким образом, они, по их собственному признанию, почитали море за бога, потому что оно было источником их пропитания; они думали, как тот греческий комик, который говорит: <<то, что меня питает, то я и считаю моим богом>>. Известная поговорка <<чей хлеб я ем, того и песню пою>>  имеет поэтому значение и в религии. Уже самый язык дает нам подтверждение этого. Almus, например, означает питающий, поэтому он является главным образом эпитетом Цереры, которая именно поэтому и любима, ценима, прекрасна, священна. <<Изо всех божеств, о которых повествует мифология, --- говорит Диодор, --- ни одно не почитается людьми так высоко, как те два, которые своими благодетельнейшими изобретениями так исключительно выслужились перед человечеством: Дионисий --- введением в употребление прекраснейшего напитка и Деметра доставлением превосходнейшей твердой пищи>>. Эразм в своих <<Пословицах>> делает к пословице древних: <<человек для людей есть бог>>  замечание: <<Древность верила, что быть богом означает оказывать пользу смертным>>. Подобное же замечание делает филолог Иог. Мейен в одном примечании к <<Энеиде>> Вергилия. Древние, говорит он, оказывали тем, кто сделал благодетельные изобретения, божеские почести после их смерти, ибо они были убеждены, что бог --- не что иное, как то, что приносит пользу людям. <<На каком основании, --- замечает Овидий в своих ,,Посланиях с Понта``  --- станем мы почитать богов, если отнимем у них волю к тому, чтобы оказывать пользу или помогать? Если Юпитер остался глухим к моим мольбам, зачем мне закалывать жертвенное животное перед его храмом? Если море не дает мне покоя во время моего морского путешествия, то зачем мне ни за что, ни про что воскурять фимиам Нептуну? Если Церера не исполняет желания трудолюбивого селянина, то зачем же она будет получать внутренности поросой свиньи?>> Стало быть, только польза или благодеяние возвеличивают людей и богов! <<Бог для смертного тот, --- говорит Плиний Старший, --- кто помогает смертному>>. По Геллию, даже сам Юпитер имеет свое имя lovena от iuvando, то есть оказывать помощь или пользу, в противоположность nocere --- вредить. В цицероновском сочинении <<Об обязанностях>> говорится: <<Непосредственно после богов самые полезные человеку существа --- это люди>>; стало быть, боги --- существа, человеку наиболее полезные. И точно так же говорит Эразм в своих <<Пословицах>>: <<Пословица ,,един бог, но много друзей`` напоминает нам о том, что нам нужно приобретать как можно более друзей, ибо после богов более всего они могут оказать нам помощь>>. В своем сочинении <<О природе богов>> Цицерон (или, вернее, эпикуреец Веллей, но в данном случае это безразлично) признает абсурдным утверждение Персея, что полезные и целебные вещи считаются богами, и бросает Продику за подобное утверждение упрек в том, что тот упраздняет религию; но одновременно же он упрекает и Эпикура, что тот, отказывая божеству в самом божественном, в самом чудесном: в доброте, в благодетельности, тем самым в корне подрывает религию, ибо как можно, говорит он, почитать богов, если от них не получаешь ничего доброго и ничего доброго нельзя ожидать? Религиозность, pietas, это --- справедливость по отношению к богам, но как можно чувствовать себя обязанным по отношению к тем, от кого ничего не получаешь? В богах почитаем мы, --- говорит Квинтилиан в своих <<Ораторских наставлениях>>  --- прежде всего величественность их природы, затем могущество, присущее каждому из них, и те изобретения, которые оказали пользу человеку. Квинтилиан различает здесь между могуществом и величием богов и их благодеяниями, но это различение исчезает, если взглянуть поглубже, ибо, чем величественнее и могущественнее какое-либо существо, тем более способно оно также оказать другим пользу, и наоборот. Высшее могущество совпадает с высшей благодетельностью. Поэтому почти у всех народов бог небесных сил есть наивысший, самый возвышенный, самый величественный бог над всеми богами, ибо действия и благодеяния неба превосходят все остальные действия и благодеяния, потому что --- они самые всеобщие, всеохватывающие, самые грандиозные, самые необходимые. Так, у римлян Юпитер зовется: optimus, maxi-mus, то есть <<за свои благодеяния>>  как замечает сам Цицерон, зовется <<лучшим или самым благим>>  <<за свою же силу>> или <<мощь>> --- <<величайшим>> или <<наивысшим богом>>. Подобное же разделение, как и у Квинтилиана, мы находим у Плутарха в его сочинении <<Влюбленный>>. <<Хвала богам основывается главным образом на их dynamis, то есть могуществе, и opheleia, то есть полезности или благодетельности>>; но, как уже сказано, оба понятия сливаются в одно, ибо, чем больше какое-либо существо является самим по себе и для себя, тем больше им оно может быть и для других. Чем кто больше, тем он больше может быть и другим полезным, правда, однако, быть также и вредным. Поэтому Плутарх говорит в своих <<Застольных речах>>: <<Люди более всего обожествляют предметы самой всеобщей, на все распространяющейся полезности, как то: вода, свет, времена года>>. 

\phantomsection
\addcontentsline{toc}{section}{Восьмая лекция}
\section*{Восьмая лекция}

Когда последние остатки языческой религии были разрушены или, по крайней мере, лишены своего политического значения и почета, когда, между прочим, должна была быть удалена с того места, где она до тех пор стояла, статуя богини победы, Симмах написал сочинение в защиту старой, исторической религии, а также в защиту культа Виктории. Среди своих защитительных доводов он сослался также и на utilitas, пользу, как на вернейший отличительный признак божества. Никто не станет отрицать, говорит он далее, что почитать следует то божество, которое признается желательным. Это значит: только то есть предмет религии, почитания, что является предметом человеческих желаний, но только доброе, полезное, благодетельное есть то, чего желают. Поэтому образованные люди среди классических язычников, а именно --- греков, определяли в качестве существенного свойства и условия божества --- доброту, благодетельность, филантропию. <<Ни один бог, --- говорит Сократ в платоновском ,,Теэтете''  --- не настроен враждебно к людям>>. <<Какова у богов, --- говорит Сенека в своих письмах, --- основа их благодетельности? Их природа. Вера в то, что они хотят вредить --- заблуждение; они этого даже не могут>>. <<Бог, говорит он там же, --- не ищет себе слуг; он сам служит человеческому роду>>. <<Так же нелепо, --- говорит Плутарх в своем сочинении о противоречиях стоиков, --- отрицать за богами их бессмертие, как и попечение и любовь к людям или благодетельность>>. <<Под богом, --- говорит в том же сочинении у Плутарха Антипатр из Тарсиса, --- понимаем мы блаженное, бессмертное и к людям благодетельное существо>>. Поэтому боги у греков, по крайней мере лучшие, зовутся <<подателями благ>>  далее --- soteres, то есть, спасителями несущими счастье, избавителями. Греческая религия не имеет даже специального, самостоятельного злого бога, как, например, египтяне, имевшие своего Тифона, и персы --- своего Аримана. 

Отцы церкви высмеивали язычников за то, что те делали благодетельные или полезные вещи и существа предметом своего почитания или религии. Легкомысленные греки, говорит, например, Юлий Фирмик, считают богами все существа, оказывающие или оказавшие им какое-либо благодеяние. Он упрекает их, между прочим, в том, что пенаты происходят от слова: penus, что означает не что иное, как пищу. Язычники, говорит он, как люди, под жизнью ничего другого не понимавшие, как свободу поесть и попить, сделали богами свои средства пропитания. Он упрекает их и за культ Весты, ибо она не что иное, как домашний огонь, который на очаге служит для ежедневного употребления, и поэтому должна была бы иметь своими жрецами поваров вместо дев. Но сколько ни порицали и ни высмеивали отцы церкви и вообще христиане язычников за то, что те оказывали божеские почести полезным вещам --- огню, воде, солнцу, луне, --- именно из-за их столь благодетельного действия на человека, они все же не порицали их за принцип или основу этого почитания, а только за предмет их почитания, не за то, стало быть, что они сделали благодетельность или полезность основой почитания, религии, а за то, что они сделали предметом почитания не надлежащее существо, что они не почитали то существо, от которого исходят все благодетельные, человеку полезные свойства и действия природы, то существо, которое одно может помочь человеку, может сделать его счастливым, которое есть бог, существо от природы отличное, духовное, невидимое, всемогущее, являющееся творцом или создателем природы. Силу вредить или оказывать пользу, приносить счастье и горе, здоровье и болезнь, жизнь и смерть, ту силу, которую язычники, сообразно своему чувственному взгляду на вещи, разделили между многими различными вещами, христиане, в силу своего нечувственного, абстрактного мышления, объединили в одном существе, так что это единое, что христиане называют богом, есть единственно страшное и могучее, единственно любящее и благодетельное существо. Или: если мы благодетельность сделаем единственным существенным свойством божества, то мы скажем: блага, которые язычники распределяли между различными предметами действительности, христиане собрали в одном предмете, поэтому христиане определяют бога как понятие, включающее в себя всех богов. Но в определении, то есть в понимании самого божества, в принципе, то есть в сущности или основе, они не отличаются от язычников. В Библии, точно так же в <<Ветхом>>  как и <<Новом Завете>>  слово <<бог>> во всех тех местах, где говорится: <<я буду вашим богом или вам богом>> или где оно имеет вслед за собой родительный падеж, как, например, <<я бог Авраама>>  это слово значит --- благодетель. Августин в четвертой книге своего <<Града божия>> говорит: <<Если счастье есть божество, за которое его считают римляне, то почему не почитают они его одно, почему не стекаются они в один только его храм? Ибо кто не хочет быть счастливым? Кто желает чего-либо, как не для того, чтобы сделаться через это счастливым? Чего, как не счастья, добиваются от бога? Однако счастье не есть бог, а дар бога. Ищут, стало быть, того бога, который может его дать>>. В том же сочинении Августин говорит по поводу платоновских демонов: <<Какие бы бессмертные и блаженные ни жили в небесных жилищах, но если они нас не любят и не хотят, чтобы мы были блаженны, то нам их нечего почитать>>. Поэтому только то, что любит человека и желает его блаженства, есть предмет почитания для человека, предмет религии. Лютер в своем толковании некоторых глав пятой книги Моисея говорит: <<Так разум описывает бога, что он есть то, что человеку оказывает помощь, приносит пользу, идет ему на благо. Таким образом поступали язычники. Римляне наделали много богов различных просьб и помощи ради\dots Сколько было нужды, добра и угодий на земле, столько же создавалось богов, вплоть до того, что даже растения и чеснок сделались богами\dots Мы при папстве тоже понаделали богов, всякая болезнь или нужда имела собственного спасителя или бога. Беременные женщины, когда они были на сносях, призывали святую Маргариту, она была их богиней\dots Святой Христофор оказывал, говорят, помощь тем, кто был при последнем издыхании. Таким образом, каждый приписывает имя бога тому, из чего он более всего извлекает себе добра\dots Поэтому я еще раз говорю: разум знает до некоторой степени, что бог может помочь и помогает, но настоящего бога он не может найти\dots Истинный бог в писании зовется спаситель в нужде и податель всего доброго>>. В другом месте говорит он об язычниках: <<Хотя они и ошибаются в личности бога, благодаря идолопоклонству (то есть вместо того, чтобы обращаться к истинному богу, они обращаются к ложным), но все же налицо то служение, которое подобает истинному богу, то есть призывание его, и то, что они ожидают от него всего доброго и помощи>>. Таким образом субъективный принцип язычников --- истинный, или субъективно они правы, поскольку они под богом мыслят себе нечто, что только благостно, благодетельно, но объективно, то есть в предмете, они ошибаются. 

Христиане восставали поэтому особенно против богов языческой философии, а именно против бога стоиков, эпикурейцев, аристотеликов, потому что они упраздняли провидение, словами ли или на деле, потому что они устраняли те свойства, которые одни, как я уже ранее отмечал, составляют основу религии, те свойства, которые имеют отношение к счастью и к горю человека. Так, например, Мосгейм, ученый теолог восемнадцатого века, в своих примечаниях к <<Интеллектуальной системе>> Кедворта --- философско-теологической работе, направленной против атеизма, --- говорит о боге Аристотеля, <<что он человеческому роду нисколько не полезен и не вреден а поэтому не заслуживает, собственно говоря, особого культа. Аристотель верил, что мир так же необходимо и вечно существует, как и бог. Поэтому он и небо считает таким же неизменным, как и бога. Отсюда следует, что бог не свободен, стало быть, бесполезно его призывать; ибо, если мир движется согласно вечным законам и ни в каком случае не может быть изменен, то я не вижу, какую помощь мы можем ждать от бога. (Мы видим на этом примере, замечу мимоходом, что вера в бога и вера в чудо, разъединенные современным рационализмом, тождественны, как это мы увидим позже.) Аристотель лишь на словах оставляет бога, на деле же он его упраздняет. Аристотелевский бог --- бездеятелен, как и эпикурейский, его энергия, то есть его деятельность, заключается лишь в бессмертной жизни и созерцании или спекуляции. Долой бога, который живет лишь для себя и чья сущность состоит в одном только мышлении! Ибо, как может человек от такого бога ожидать себе утешения и защиты?>> Приведенные до сих пор изречения выражают, впрочем, не религиозное или теологическое умонастроение этих одиночек, они выражают умонастроение теологов и христиан, умонастроение самой христианской религии и теологии; к ним можно было бы присоединить бесчисленное множество подобных же изречений. Но зачем это ненужное и скучное множество? Я замечу еще только, что благочестивые, верующие язычники, даже из среды философов, также уже полемизировали против бесполезных, философских богов; так, например, платоники против стоического бога, стоики, которые по сравнению с эпикурейцами были верующими язычниками, --- против бога эпикурейцев. Так, например, стоик в сочинении Цицерона <<О природе богов>> говорит: <<Даже варвары и даже столь осмеянные египтяне не почитают животное, если оно не оказывает благодеяния; у вашего же бездельного бога нельзя указать ни благодеяний, ни даже поступков, действия. Но именно поэтому он бог только по имени>>. 

Но если приведенные до сих пор в виде документов цитаты имеют универсальное значение, если высказанные в них взгляды проходят через все религии и теологии, то кто сможет отрицать, что человеческий эгоизм есть основной принцип религии и теологии. Ибо, если быть достойным поклонения и почитания, а следовательно, если божеское достоинство существа зависит исключительно от отношения существа к благосостоянию человека, если только существо, благодетельное для человека, полезное ему, божественно, то основание божественности существа покоится ведь исключительно в эгоизме человека, который все относит лишь к себе и сообразно этому отношению ценит. Если я, впрочем, делаю эгоизм основой и сущностью религии, то я ей этим не делаю упрека, по крайней мере не делаю его в принципе, безусловно. Я делаю ей упрек лишь тогда, когда этот эгоизм совсем вульгарен, как, например, в телеологии, где религия делает своей сущностью отношение предмета, а именно природы, к человеку, и именно поэтому принимает по отношению к природе неограниченно эгоистический, презирающий природу характер, или где, выходя за пределы необходимого, естественно обосновываемого эгоизма, эгоизм является неестественным и сверхъестественным, фантастическим, как в христианской вере в чудо и бессмертие. 

Против этого моего понимания и объяснения религии теологические лицемеры и спекулятивные фантасты, которые вещи и людей рассматривают только с точки зрения своих самодельных понятий и воображаемых представлений, которые никогда не спускаются ни со своего амвона, ни со своей кафедры, этих искусственных высот их духовного и спекулятивного самомнения, чтобы встать на один уровень с рассматриваемыми ими вещами; эти лицемеры и фантасты возражают, что я, --- который в противоположность этим духовным и спекулятивным господам привык, прежде чем судить о вещах, сначала себя с ними отождествить, почувствовать свою общность с ними и близость, --- что я делаю существом религии частные, то есть второстепенные, случайные ее явления. Сущность религии, возражают мне эти лицемеры, фантасты и спекулянты, никогда не бросавшие своего взора на действительную жизнь человека, есть скорее нечто противоположное тому, что я делаю сущностью религии, не самоутверждение, не эгоизм, но растворение в абсолютном, бесконечном, божественном или --- как еще гласят пустые фразы --- самоотрицание, самоотвержение, самопожертвование человека. Правда, есть достаточно явлений религии, которые, по крайней мере по видимости опровергают мой взгляд на религию и оправдывают противоположный. Это --- случаи отказа в удовлетворении самых естественных и самых могучих влечений, умерщвление плоти и ее, --- как выражаются христиане, --- злых страстей, духовная и телесная кастрация, самоистязания и самоумерщвления, покаяние и самобичевание, играющие роль почти во всех религиях. Так, мы уже видели, что фанатические почитатели змей в Индии дают себя кусать змеям; фанатические и энтузиастические индийские и тибетские почитатели животных дают из религиозного самоотрицания клопам, вшам и блохам высасывать свою или других людей кровь из тела и разум из головы. 

Я с удовольствием присоединю к этим примерам еще и другие, чтобы дать моим противникам против меня оружие в руки. Египтяне жертвовали ко благу своих священных животных благом людей. Так, при пожарах в Египте заботились гораздо более о спасении кошек, чем о том, чтобы потушить огонь. Эта заботливость мне невольно напоминает того королевско-прусского комиссара полиции, который несколько лет тому назад, в воскресенье, во время богослужения, из-за истинно прусского христианского отрицания людей запретил тушение пожара. А вот Диодор сообщает: <<Когда однажды египтян постиг голод, то, говорят, многие увидели себя вынужденными поедать друг друга, но никто не обвинялся в том, что он съел священное животное>>. Как благочестиво, как божественно! Из любви к освященному религией миру животных люди поедают друг друга! Максим Тирский в своей восьмой диссертации рассказывает, что одна египтянка, взрастившая молодого крокодила вместе со своим юным сыном, не оплакивала последнего, когда крокодил, подросши, съел его, а, наоборот, счастливая, радовалась, что он сделался жертвою домашнего бога; а Геродот повествует, что одна египтянка даже совокуплялась с козлом \hyperlink{6}{(6)}\hypertarget{b6}{}. Можно ли, спрошу я философов и теологов, которые, разумеется, не на практике, а только в теории отвергают человеческую любовь к себе, как принцип религии, морали и философии, --- можно ли дальше идти в самопрезрении и в самоотвержении, чем эти египтянки? Один англичанин проезжал однажды в Индии, как рассказывается в примечаниях к книге <<Индусские законы или законодательство Ману>> Гюттнера мимо лесной чащи. Вдруг выскочил тигр и схватил маленького громко вскрикнувшего мальчика. Англичанин был вне себя от ужаса и страха, индус же спокоен. <<Как, --- сказал англичанин, --- можете вы оставаться столь спокойным?>>. Индус отвечал: <<Великий бог хотел этого>>. Можно ли представить себе большее самоотречение, как то, чтобы дать тигру загрызть мальчика, дать, не выражая никакого чувства и не производя никакого действия, благочестиво доверяя и веря, что все, что случается, исходит от бога, а что исходит от бога, то делается ко благу? Карфагеняне, как известно, жертвовали во время нужды и опасности своему богу, Молоху, то, что есть наиболее любимого у человека, --- своих детей. Против значения этого и других приведенных примеров нельзя выставлять тот довод, что человек в религиозном самоотрицании должен отрицать не других, а себя самого; ибо не подлежит сомнению, что очень многим матерям и отцам легче принести в жертву себя, чем своих детей. Что карфагеняне не были лишены чувства любви к своим детям следует из того, что, как рассказывает Диодор, они пытались одно время вместо своих детей приносить в жертву чужих. Но жрецы Молоха также плохо встретили эту, хотя и в высшей степени ограниченную и призрачную, попытку гуманизировать культ Молоха, как еще и поныне плохо встречают спекулятивные и религиозные сторонники божественной бесчеловечности желание гуманизировать религию. <<Есть почитатели божества, --- говорят индийцы, как это значится в предписаниях Ману, --- чтущие его жертвоприношениями, самоистязанием, ревностным благочестием, исследованием писания, подавлением страстей и строгим образом жизни. Некоторые жертвуют своим дыханием и насильственно отгоняют его прочь от его естественного пути, другие, наоборот, своим дыханием выжимают кверху газы, скопившиеся внизу, а некоторые, высоко ценящие обе эти силы, запирают оба отверстая для выхода>>. Какое преодоление себя, повернуть нижнюю часть человеческого тела кверху и подавить естественное, но, разумеется, эгоистическое влечение человека к выходу и к свободе от всякого давления! Ни один народ так не отличался в самоистязаниях и самобичеваниях, ни один не проделал таких чудес религиозной гимнастики, как индусы. <<Некоторые истязуют себя, --- рассказывает Зоннерат в своем <<Путешествии в Ост-Индию и Китай>> об индийских самоистязателях, непрестанными ударами розог или приказывают приковать себя цепью к стволу какого-либо дерева и остаются до самой своей смерти к нему прикованными. Другие полагают нужным оставаться всю жизнь в трудном положении, например держать кулаки постоянно сжатыми, так что ногти их, которые они никогда не обрезают, с течением времени прорастают их руки. Еще другие держат постоянно руки крест-накрест на груди или вытянутыми над головой, так что, в конце концов, они ими больше не могут пользоваться. Многие заживо закапывают себя в землю и вдыхают свежий воздух только через маленькое отверстие>>. Индийцы, достигшие высшей ступени религиозного совершенства, <<ложатся даже в колею, чтобы быть раздавленным той колесницей, на которой везут по праздникам колоссальное изображение разрушительного божества (Шивы)>>. Можно ли чего большего требовать? И все же мы, эгоистические европейцы, скорее бы согласились на эти пытки, чем на то религиозное самоотрицание, с которым индиец пьет коровью мочу для очищения от своих грехов и считает почетным самоубийство, при котором он покрывается коровьим навозом и затем сжигается. 

Но что нас больше всего, как христиан, интересует, так это те самоистязания, то самоотрицание, которые возлагали на себя древнейшие христиане. Так, например, Симеон Столпник провел не менее тридцати лет на столбе, а св. Антоний одно время пролежал даже в гробу и довел религиозное подавление человеческих желаний и всякого самочинного проявления плоти до того, что не сбрасывал у себя с тела неприятных насекомых, никогда не мылся и не чистился. И о благочестивой Сильвании, интересным знакомством с которой я обязан, впрочем, только <<Истории культуры>> Кольба, рассказывают, что эта <<чистая душа в возрасте 60 лет не мыла ни рук, ни лица, ни какой-либо другой части своего тела, за исключением кончиков пальцев, когда она принимала святое причастие>>. Какой нужен для этого героический сверхнатурализм и сверхгуманизм, чтобы преодолеть естественное влечение к чистоте, чтобы отказаться от благодетельного, разумеется, эгоистического чувства, связанного с освобождением тела от всякой нечисти. Я выставляю эти примеры против религиозных абсолютистов; они не могут отвести их как заблуждения и глупости. 

Правда, приведенные примеры, это --- порождения религиозного бессмыслия и религиозного безумия. Но безумие, глупость, сумасшествие являются также принадлежностью психологии или антропологии, как и философии и истории религий, ибо в религии не действуют и не выявляются какие-либо другие силы, причины, основания, чем в антропологии вообще. Считает же религиозный человек именно болезни, как телесные, так и душевные, чудесными, божественными явлениями. Так еще и до сих пор, как замечает Лихтенштедт в своих <<Причинах большой смертности годовалых детей>> в России <<суеверие рассматривает многие болезни детей, особенно, когда они выражаются в судорогах, как что-то священное и неприкосновенное>>. Всякого рода сумасшедшие и юродивые и поныне считаются у многих народов за боговдохновенных людей, за святых. К тому же, как ни бессмысленны отмеченные нами виды человеческого самоотрицания, они --- необходимое следствие того принципа, который и сейчас еще держится в головах наших теологов, философов и вообще верующих. Раз я выставляю своим принципом самоотрицание или растворение в фантастическом существе религии и теологии, то я не вижу, почему бы мне не отрицать, как всякое другое влечение, и естественное стремление к передвижению, желание смыть грязь со своего тела, желание ходить выпрямившись, а не ползать на четвереньках, как это делали многие святые. Все эти влечения с точки зрения теологии --- эгоистические по своей натуре; ибо их удовлетворение сопряжено с удовольствием, с хорошим самочувствием. Стремление стоять прямо имеет своим источником человеческую гордость и высокомерие и находится поэтому в прямом противоречии с тем верноподданничеством, которое нам предписывает теология. 

Все те, кто изгоняет из религии принцип эгоизма, --- в широком, как я должен постоянно повторять, --- смысле этого слова, являются в основе своего существа, как бы они ни затушевали это философскими фразами, религиозными фанатиками, они еще сегодня, если не телесно, то духовно, стоят на точке зрения христианских святых-столпников, они еще и сегодня, но теоретически, а не чувственно, как делали древние и еще в настоящее время делают чувственные первобытные народы, приносят своему богу в жертву человека; еще и сегодня не смывают они из религиозного предубеждения и предрассудка грязь со своих глаз и своей головы, хотя в противоречие со святой Сильванией --- их идеалом, из непоследовательности и грубого эгоизма (ибо грязь в глазу, по крайней мере духовном, не так тягостна, потому что не так очевидна, как на остальном теле), --- они и удаляют эту грязь со своего тела. Если бы они вымыли свои глаза в холодной воде природы и действительности, то они бы поняли, что самоотрицание, при всем его религиозном значении, не есть сущность религии, но что только они сами глядят на человека, а потому и на религию, ослепленными глазами. С высоты своей кафедры и своего амвона они не замечают той эгоистической цели, которая лежит в основе этого самоотрицания, а именно не видят, что люди в практической жизни вообще умнее, чем теологи на амвоне и профессора на кафедре, а потому и в религии следуют не за философствованием о религии, а за своим разумным инстинктом, который предохраняет их от бессмыслицы религиозного самоотрицания, и даже тогда, когда они в эту бессмыслицу впадают, подсовывают ей все-таки еще человеческий смысл и цель. 

Почему же человек отрицает себя в религии? Чтобы приобрести благоволение к себе своих богов, которые все ему предоставляют, что он только пожелает. Строгостью своего подвижничества <<можно богов заставить удовлетворить всякую просьбу и даже исполнить тотчас же то, что есть в мыслях>> (Болен, <<Древняя Индия>> т. 1). Человек отрицает себя, стало быть, не для того, чтобы себя отрицать --- такое отрицание, где оно встречается, есть чистейшее религиозное безумие и бессмыслица, --- он отрицает себя, по крайней мере, там, где человек владеет своими человеческими чувствами, чтобы через это отрицание себя утвердить. Отрицание есть лишь форма, средство самоутверждения любви к себе. Всего отчетливее в религии это выявляется в жертве. 

\phantomsection
\addcontentsline{toc}{section}{Девятая лекция}
\section*{Девятая лекция}

Жертва есть предмет, на примере которого делается очевидным, что самоотрицание в религии есть лишь средство, лишь не прямая форма и способ самоутверждения. Жертва есть отчуждение блага, дорогого для человека. Но так как высшее и самое дорогое благо в глазах человека есть жизнь, так как высшему можно жертвовать только высшим, только этим его почтить, то жертва там, где лежащее в основании ее понятие реализуется полностью, есть отрицание, уничтожение живого существа и, так как высшее живое существо есть человек, отрицание человека. Мы в данном случае имеем опять, независимо от цели человеческого жертвоприношения, о которой нам надлежит говорить, доказательство того, что для человека нет ничего выше жизни, что жизнь по своему рангу равняется с богами; ибо в основании жертвы лежит --- по крайней мере вообще говоря --- сопоставление равного с равным; богам приносится лишь то, что одинаково с ними по смыслу, что им подобно; человек жертвует поэтому своей жизнью только для богов, ибо в глазах богов, как и людей, жизнь есть высшее, самое чудесное, самое божественное благо --- стало быть, благо, которому боги не могут противостоять, которое подчиняет волю богов человеку. 

Отрицание, или уничтожение, заключающееся в жертве, не есть, однако, отрицание бессмысленное, оно имеет, наоборот, весьма определенную, эгоистическую цель и основание. Человек жертвует только человеком --- высшим существом, чтобы поблагодарить за высшее, в его понимании счастье или отвратить высшее несчастье --- действительное или предполагаемое, --- ибо примирительная жертва не имеет самостоятельной цели и смысла; ведь примиряются с богами только потому, что они именно --- те существа, от которых зависит счастье и несчастье, так что отвратить гнев богов означает не что другое, как отвратить от себя несчастие, приобрести благоволение или милость богов, означает не что другое, как приобрести все хорошее и желательное. Вот несколько примеров, чтобы подтвердить как самый факт, так и указанный смысл человеческого жертвоприношения. Я начинаю с немцев и нам всего более родственных племен, хотя как раз германцы принадлежат к числу тех, которым немецкие ученые приписывают самую мягкую форму человеческих жертвоприношений. А именно они говорят, что человеческие жертвоприношения у них были лишь казнями преступников, а следовательно, карательными и в то же время примирительными жертвами богам, оскорбленным преступлениями. Прочие человеческие жертвоприношения происходили лишь по ошибке и благодаря вырождению. Но если и допустить --- доказательств чему, однако, не имеется, что первоначально приносились в жертву только преступники, то от такого жестокого бога, от бога, который наслаждается мучениями преступника, от <<князя виселицы>>  как называется Один, можно ожидать особых жестокостей и человеческих жертв. Только в патриотическом эгоизме немецких ученых заключается причина, почему немцы, которые, собственно говоря, до сего дня сохраняют в себе изрядную дозу варварской грубости под святым покровом христианской веры, должны были составить исключение из числа прочих народов. Однако, к делу. Согласно одной норвежской саге в Швеции при короле Домальде были <<неурожай и голод. Поселяне принесли в жертву много быков, но это не помогло. Шведы решили тогда принести Одину в жертву короля, дабы вернулись плодородие и хорошие бремена. Они зарезали и принесли его в жертву и обмазали его кровью все стены и сиденья в доме бога, и тогда наступили в стране лучшие времена>>. <<Большинство людей испробовало жертвоприношение как исполнение обета, данного при начале воины, --- на случай победы. Для готов и скандинавов вообще лучшей жертвой считался человек, который на войне первым попался в плен. Саксы, франки, герулы также верили, что человеческие жертвы умилостивляют их богов. Саксы приносили богам свои воинские жертвы, предавая жертвуемых мучительно тягостным наказаниям, и точно так же тулиты (скандинавы) жертвовали богу войны своих первых военнопленных, предавая их изысканной казни>> (в энциклопедии Эрша и Грубера, статья Ф. Вахтера <<Жертвоприношение>>). Галлы, как рассказывает Цезарь, когда они страдали от тяжелых болезней или подвергались опасностям войны, приносили в жертву людей, веря, что боги только в том случае могут с ними примириться, если за жизнь одного человека приносится в жертву жизнь другого. Так же и наши восточные соседи, например <<эсты, приносили страшным богам человеческие жертвы. Человеческие жертвы служили предметом торга для купцов, причем тщательно исследовалось, не имеют ли приносимые в жертву телесных недостатков, что делало бы их негодными для жертвоприношения>> (К. Эккерманн <<Учебник истории религии>>  т. 4: <<Религия чудского племени>>). И славяне также --- по крайней мере те, которые жили около Балтийского моря, --- приносили в жертву своему главному божеству Свантовиту <<ежегодно, а также при чрезвычайных обстоятельствах христианина, потому что жрец, совершавший жертвоприношение, говорил, что Сванговит и другие славянские, боги будут в высшей степени порадованы христианской кровью>> (Вахтер, указанное место). Даже римляне и греки запачкали себя кровью религиозных человеческих жертвоприношений. Так, например, как рассказывает Плутарх, фемистокл перед Саламинской битвой принес в жертву Вакху сыроядцу трех знатных персидских юношей, правда, лишь скрепя сердце, лишь вынужденный к тому прорицателем Евфрандитом, который сулил победу и счастье грекам лишь ценой этой жертвы. А в Риме еще во времена Плиния Старшего заживо погребались многие пленные на скотном рынке. Восточные народы жертвовали богам даже собственных дочерей и сыновей, то есть существа, за жизнь которых, как замечает Юстин по поводу карфагенских человеческих жертвоприношении, при Других обстоятельствах больше всего молят богов. Даже израильтяне <<проливали невинную кровь>>  как значится в Библии, <<кровь своих сыновей и дочерей, которых они приносили в жертву богам Ханаана>>. Но не только кумирам-богам, но и самому господу богу принес в жертву свою дочь Иеффай, правда, в результате необдуманного рокового обета, гласившего, что он, если одержит победу, готов принести в жертву того, кто раньше всего ему выйдет навстречу из дверей его дома; и, по несчастью, это было его собственное дитя, его дочь, которая ему прежде всего встретилась; но он не мог бы, как уже отмечали многие ученые, прийти к мысли принести в жертву свою дочь, если бы человеческие жертвоприношения не признавались? Однако из всех религиозных палачей и мучителей людей особенно выделялись древние мексиканцы жестокостью и бесчисленным количеством человеческих жертв, которые часто исчислялись в пять, даже в двадцать тысяч человек в один день. 

Как до новейшего времени сохранились почти вся религиозная бессмыслица и все ужасы древности, так сохранились и кровавые человеческие жертвоприношения. В 1791 г., например, как рассказывается в примечаниях к <<Индусской книге законов>>  --- был найден однажды утром в одном храме Дия или Шивы обезглавленный харри, то есть человек, принадлежавший к низшей касте: он был убит для предотвращения большого несчастья. А некоторые дикие племена мараттов кормят и откармливают даже прекраснейших мальчиков и девочек, как убойный скот, для принесения их в жертву в особые праздники. Даже и столь сантиментальные, так нежно заботящиеся о жизни насекомых жители Индии сбрасывают во времена больших несчастий, подобных войне или голоду, знатнейших браминов вниз с пагод, чтобы таким образом утихомирить гнев богов. <<В Тонкине (дальняя Индия) ежегодно, ---  как рассказывает Мейнерс в своей ,,Всеобщей истории религий`` со слов путешественников, --- отравляют детей ядом, дабы боги благословили поля и даровали богатый урожай, или рассекают одного из детей пополам, чтобы смягчить богов или побудить их не вредить остальным. В Лаосе не выстраивают даже богам храма, пока не закопают в фундамент тех, кто раньше всего прошел мимо, и тем как бы не освятят почвы>>. <<У некоторых негритянских народов до сих пор еще приносятся в жертву многие сотни и тысячи пленных в призрачной надежде, что такими жертвами можно вернее всего снискать милость богов и через них --- победу над врагами. В других местностях Африки закалывают то детей, то взрослых мужчин, чтобы добиться таким образом исцеления больных царьков или продления их жизни>> (Мейнерс). Кханды в Гондване, вновь открытое племя первобытных обитателей Индии, приносят в жертву, как это сообщается в журнале <<Ausland>> (<<Чужие края>>) за 1849 г., своему верховному богу --- богу земли, Бера Пенну, от которого, как они верят, зависит процветание людей, животных и полей, регулярно каждый год людей, а кроме того, также и в случаях несчастий, если, например, тигр загрыз какого-либо ребенка, дабы примирить с собой разгневанных богов. Жители южных островов Океании также вплоть до новейшего времени приносили человеческие жертвы, приносят их отчасти и сейчас. 

Христианскую религию обычно восхваляют за то, что она уничтожила человеческие жертвоприношения. Но она только заменила кровавые человеческие жертвоприношения другого рода жертвами, --- подставив на место телесных человеческих жертвоприношений психическое, духовное, то жертвоприношение, которое, хотя и не в чувственном смысле, но фактически на самом деле составляет еще человеческое жертвоприношение \hyperlink{7}{(7)}\hypertarget{b7}{}. Поэтому люди, которые держатся за видимость, верят, что христианская религия принесла в мир что-то существенно иное, чем религия языческая, но это только одна видимость. Вот пример: христианская церковь отвергла самооскопление, хотя в Библии, действительно или только по видимости, --- говорится даже в его защиту; по крайней мере, великий отец церкви Ориген, который, конечно, был столь же учен, как и нынешние господа теологи, понял ее так, что счел обязанным самого себя оскопить; христианская церковь и религия, говорю я, строжайше запретили телесное самооскопление, предписываемое языческой религией, но запретили ли они самооскопление духовное. Отнюдь нет. Они во все времена говорили в защиту морального, духовного, психического самооскопления. Даже Лютер ставит безбрачное состояние выше брачного. Но в чем различие между телесным и духовным уничтожением органа? Различия нет; в одном случае я отнимаю у органа его телесное, анатомическое, в другом --- его душевное существование и значение. Имею ли я орган, или не употребляю его для предназначенной природою цели, убиваю ли я его телесно или духовно, --- это одно и то же. Но это различие между языческим и христианским самооскоплением есть различие между языческой и христианской человеческой жертвой вообще. Христианская религия не имеет, правда, на своей совести, телесных, анатомических жертв, но имеет достаточно психических. Там, где человек представляет себе как идеал существо абстрактное, от действительного существа отличное, почему бы ему там не удалить от себя все, не пытаться сбросить с себя все, что этой его цели, его идеалу противоречит. Богу, который не есть чувственное существо, человек по необходимости жертвует именно своей чувственностью; ибо бог --- эту мысль мы дальше специально разовьем --- есть не что иное, как цель, как идеал человека. Бог, который не является ни моральным, ни практическим образцом для человека, который не есть то, чем сам человек должен и хочет быть, есть бог лишь по имени. Короче говоря, христианская религия, --- то есть как религия, покоящаяся на теологической вере, --- не отличаясь вообще по своему принципу от других религий, не отличается от них и в этом пункте. Как христианство на место видимого, чувственного, телесного бога подставило невидимого, точно так же на место видимой, очевидной человеческой жертвы подставила она человеческую жертву невидимую, нечувственную, но от того не менее действительную. 

Из приведенных примеров мы видим, что даже самое бессмысленное и страшное отрицание человека, религиозное убийство, имеет человеческую, или эгоистическую, цель. Даже тогда, когда человек совершает религиозное человекоубийство не над другими, а над самим собой, когда он расстается со всеми человеческими благами, отвергает все чувственные и человеческие радости, --- это отвержение есть лишь средство приобрести и вкусить небесного или божественного блаженства. Так обстоит дело у христиан. Христианин жертвует собой, отрицает себя только ради того, чтобы обрести блаженство. Он приносит себя в жертву богу, это значит: он приносит в жертву все земные, преходящие радости, потому что они не удовлетворяют его сверхъестественному чувству христианина, он приносит их в жертву небесному царству радости. И так же у обитателей Индии. Так, например, в книге законов Ману говорится: <<Если брамин начнет избегать всех чувственных удовольствий, он достигнет блаженства на этом свете, блаженства, которое продолжится и после смерти>>. <<Если брамин свое телесное здоровье\dots незаметным образом расшатал и сделался равнодушным к горю и страху, то он чрезвычайно возвысится в своем божественном существе>>. Быть единым с богом, сделаться самому богом, --- вот к чему стремится брамин, подвергая себя лишениям и самоотрицанию; но это фантастическое самоотчуждение связано в то же время в высочайшей степени с чувством своего <<Я>>  самоудовлетворением. Брамины --- высокомернейшие люди на свете, они в своих собственных глазах --- земные боги, перед которыми все остальные люди --- ничто. Религиозное смирение, смирение перед богом, возмещается, вообще говоря, всегда духовным высокомерием по отношению к людям. Уже самое отчуждение от чувств, отказ от того, чтобы что-либо видеть, осязать, обонять, к чему стремится индиец, --- связано с фантастическим наслаждением. В мемуарах Бернье мы читаем о браминах: <<Они так глубоко погружаются в состояние экстаза, что в течение многих часов остаются бесчувственными; в это время, как они утверждают, они видят самого бога во образе блестящего, неописуемого света и ощущают неизъяснимое блаженство и совершенное презрение к миру и отчуждение от него. Я это слышал от одного из браминов, утверждавшего, что он в это состояние экстаза может погрузиться всякий раз, когда пожелает>>. Вообще известно, как близко родственны друг другу жестокость и сладострастие. Но если уже из-за высших форм жертвоприношения вырисовывается человеческий эгоизм, как его цель, то он еще больше бросается в глаза при низших формах жертвоприношения. <<Народы Америки, Сибири и Африки, занимающиеся рыбной ловлей и охотой, жертвуют часть из полученной добычи богам или духам убитых животных, но обыкновенно они жертвуют только тогда, когда случается нужда; так, они жертвуют целых животных, когда проезжают по опасным дорогам и рекам. Камчадалы обычно приносят богам от своего рыбного улова только головы и хвосты, которых они сами не едят. По свидетельству Степана Крашенинникова в его описании Камчатки лучшей жертвой у камчадалов считаются тряпки, надетые на шест. Древние славяне бросали в огонь лишь худшие части жертвенных животных. Лучшие они съедали либо сами, либо отдавали жрецам. Все татарские и монгольские племена в Сибири, в губерниях Оренбургской, Казанской и Астраханской отдают богам от жертвенных животных, будь то лошади и коровы или овцы и северные олени, либо только одни кости и рога, либо, самое большее, вместе с костями и рогами еще и головы или нос и уши, ноги и кишки. Негры в Африке также не предоставляют богам ничего, кроме шкур и рогов>> (Мейнерс в указанном месте). У классических народов, римлян и греков, существовали, правда, holokausta, то есть жертвоприношения, при которых все жертвенное животное после того, как с него сдиралась шкура, сожигалось в честь богов; но обычно богам давали лишь часть, лучшие же куски съедались. Известно, --- правда, различно толкуемое, место из Гезиода, в котором говорится, что хитрый Прометей научил людей оставлять мясо жертвенных животных себе, богам же жертвовать лишь кости. В противоречии с этой скудостью жертвы находятся, видимо, те расточительные жертвы, которые греки и римляне иногда приносили богам. Так, Александр пожертвовал после победы над лакедемонянами гекатомбу, а его мать Олимпия обычно жертвовала 1000 быков. Точно так же и римляне, чтобы победить, или после одержанной победы, жертвовали сотни быков или весь весенний приплод телят и ягнят, коз и свинец. После смерти Тиберия римляне были до такой степени рады новому властителю, что они, как рассказывает Светоний, пожертвовали богам в первые три месяца правления Калигулы свыше 160000 штук скота. Мейнерс в своем указанном сочинении по поводу этих блестящих жертв замечает: <<Не делает чести грекам и римлянам, что они превзошли все остальные известные народы многочисленностью жертв, и еще меньше, что величайшая расточительность по части жертвоприношений как раз пришлась на то самое время, когда у них всего больше процветали науки и искусства>>. В высшей степени характерно для направления философии новейшего времени замечание одного философа из гегелевской школы к этому суждению Мейнерса, --- замечание, сделанное им в его <<Естественной религии>>. Вот оно. <<Но и Мейнерсу делает мало чести непонимание того, что гекатомба, то есть такое отчуждение собственного имущества, такое равнодушие к пользе есть празднество, в высшей степени достойное как божества, так и человека>>. Да. Празднество в высшей степени достойное, если придерживаться современного спиритуалистического взгляда на религию, который смысл религии видит лишь в ее бессмыслице и поэтому признает более достойным человека пожертвовать сотню и тысячу быков богам, ни в чем не нуждающимся, чем употребить их ко благу нуждающегося человека. 

Но даже эти жертвоприношения, на которые ссылаются религиозный аристократизм и сибаритизм для своей защиты, подтверждают точку зрения, мною развитую. То, что я говорил об ощущении нужды и о чувстве радости по поводу избавления от нужды, целиком объясняет и различные явления, присущие жертвоприношениям. Великий страх, великая радость несут за собой и великие жертвы; оба аффекта безмерны, трансцендентны, чрезвычайны; оба аффекта поэтому являются и психологическими причинами, вызывающими представления о чрезвычайных существах, о богах. Безмерные жертвы приносятся только в состоянии безмерных страха и радости. Не богам на Олимпе, не вне- и сверхчеловеческим существам; нет! только аффектам страха и радости приносили греки и римляне в жертву гекатомбы. При обычном ходе вещей, когда человек не возвышается над обычным, заурядным эгоизмом, он и жертвы приносит эгоистические, в духе обыкновеннейшего эгоизма; но в чрезвычайные моменты и именно поэтому в состоянии чрезвычайного, а не повседневного аффекта он и жертвы приносит чрезвычайные \hyperlink{8}{(8)}\hypertarget{b8}{}.

Под влиянием страха человек обещает все, что у него есть: в опьянении радости, по крайней мере в состоянии первоначального опьянения, пока он еще не вошел в обычную колею повседневного эгоизма, он это обещание выполняет. Страх и радость --- коммунистические аффекты, но они коммунисты из эгоизма. Скупые и дрянные жертвоприношения не отличаются поэтому принципиально от жертвоприношений щедрых и блестящих. Впрочем, этим отнюдь не исчерпывается различие между гекатомбами греков и хвостами рыб, рогами, когтями и костями, приносимыми богам некультурными народами. Как различны люди, так же различны и их религии, и как различны их религии, так же различны и их жертвы. Человек в религии дает удовлетворение не другим каким-либо существам; он дает в ней удовлетворение своему собственному существу. Необразованный человек не имеет других потребностей и интересов, кроме желудочных; его истинный бог поэтому --- его желудок. Для ложных, мнимых богов, для богов, существующих лишь в его воображении, у него поэтому нет ничего, кроме того, что оставил ему его желудок, --- хвосты и головы рыб, рога, шкуры и кости. Наоборот, образованный человек имеет эстетические желания и потребности; он не хочет есть без разбора, что только заполняет его желудок и утоляет его голод; он хочет есть изысканную пищу; он хочет приятное обонять, видеть, слышать; короче говоря, у него есть эстетическое чувство. Поэтому народ, имеющий своими богами эстетические чувства, приносит естественно и эстетические жертвы, жертвы, которые ласкают зрение и слух. Точно так же народ, склонный к роскоши, приносит и роскошные жертвы. Как далеко заходят чувства народа, столь же далеко идут его боги. Где чувство, взор человека не поднимается до звезд, там у человека нет и небесных тел в качестве богов его, а где человек, подобно остякам и самоедам, ест без отвращения даже падаль, с аппетитом вкушает мертвых китов, там и боги его безвкусные, не эстетические, противные идолы. Поэтому если рассматривать гекатомбы греков и римлян в этом смысле, растворяющем религию в человеке, если рассматривать их как жертвы, которые они приносили своим собственным чувствам, то можно и в самом деле признать, что им делает честь, что они воздавали должное не одному только низменному своекорыстию и чувству утилитарности. 

Мы рассматривали до сих пор только религиозные жертвы в собственном смысле этого слова; но история религии знакомит нас и с другого рода жертвами, которые мы в отличие от жертв религиозных в собственном смысле этого слова можем назвать моральными. Это --- случаи добровольного самопожертвования ко благу других людей, ко благу государства, отечества. Человек, правда, и здесь приносит себя в жертву богам, чтобы утишить их гнев, но характерным для этого рода жертвы является все же моральный или патриотический героизм. Так, например, у римлян принесли себя в жертву за отечество оба Деция, у карфагенян --- оба Филена, велевшие себя --- так по крайней мере рассказывают --- заживо похоронить во время одного спора о границах между Карфагеном и Киреной и тем давшие возможность получить большую прирезку к карфагенским владениям, и точно так же поступил суффет Гамилькар, бросившийся в огонь для умилостивления богов, за что, так же как и оба Филена, был обоготворен карфагенянами, у греков --- Спертий, Кодр, легендарный Менэкей. Но эти жертвы меньше всего оправдывают точку зрения того сверхъестественного, фантастического отрицания человека, которое религиозные и спекулятивные абсолютисты делают сущностью религии; ибо как раз все эти случаи самоотрицания имеют своими очевидными содержанием и целью утверждение человеческих целей и желаний, только при этом отрицание и утверждение, жертва и эгоизм выпадают на различных лиц. Но лица, для которых я приношу себя в жертву, ведь это --- мои сограждане, мои земляки. У меня тот же интерес, что и у них; это --- мое собственное желание, чтобы отечество мое было спасено. Я жертвую поэтому своей жизнью не чужому, отличному от меня, теологическому существу, я жертвую своему собственному существу, своим собственным желаниям, своей собственной воле, --- только бы знать, что мое отечество спасено. Точно так же, как истинные боги, которым греки и римляне приносили свои великолепные жертвы, не были богами вне человека, а только их изощренными в искусстве чувствами, их эстетическим вкусом, их роскошью, их любовью к зрелищам, так и истинное божество, которому Копр, Деций, Гамилькар, Филены принесли себя в жертву, было единственно любовью к отечеству; но любовь к отечеству не исключает любви к себе; мои собственные радость и горе теснейшим образом связаны с его радостью и горем. Поэтому, как рассказывает Геродот, приносимый в жертву у персов мог просить не только добра для себя, но и <<для всех персов, ибо ведь и он сам значится среди всех персов>>. Таким образом, если я и прошу только для своего отечества, то я все же одновременно прошу и для себя; ибо ведь в нормальные времена мое и других благо теснейшим образом друг с другом связаны. 

Только в исключительных несчастных случаях должен отдельный одиночный человек приносить себя в жертву всеобщему, то есть большинству. Но нелепо исключительный ненормальный случай брать за норму, самоотрицание делать безусловным, универсальным принципом и законом, как будто бы всеобщее и одиночное является чем-то существенно различающимся между собой, как будто бы всеобщее не состоит само как раз из одиночек, как будто поэтому государство, объединение людей, не погибло бы, если бы каждый человек выполнил на самом себе требование спекулятивных, религиозных и политических абсолютистов, требование самоотрицания, самоумерщвления. Только эгоизм сплачивает государства; \emph{только там распадаются государства, где эгоизм одного сословия, одного класса или отдельных людей не признает равноправным эгоизма других людей, других сословий}. И даже там, где я свою любовь вывожу за пределы моего отечества и распространяю на всех людей вообще, даже из всечеловеческой любви не исключена любовь к самому себе; ибо я ведь люблю в людях мое существо, мой род; они ведь --- плоть от моей плоти и кровь от моей крови. Но если любовь к себе неотделима от всякой любви, если она вообще составляет необходимый, неотменимый универсальный закон и принцип, то это должна подтвердить и религия. И она в самом деле подтверждает это на каждой странице своей истории. Всюду, где человек борется с эгоизмом в широком смысле этого слова --- в религии ли, в философии ли, или в политике, --- он впадает в чистейшую нелепость и безумие; ибо смысл, лежащий в основе всех человеческих влечений, стремлений, действий, есть удовлетворение человеческого существа, удовлетворение человеческого эгоизма. 

\phantomsection
\addcontentsline{toc}{section}{Десятая лекция}
\section*{Десятая лекция}

До сих нор предмет моих лекций и положенных в основу их параграфов заключался в том, что чувство зависимости есть основа и источник религии, заложенные в человеке, предметом же этого чувства зависимости до тех пор, пока оно не искажено сверхфизической спекуляцией и рассуждением, является природа; потому что мы живем, действуем и существуем в природе; она есть то, что охватывает человека; она есть то, через отнятие чего у человека отнимается и его бытие; она есть то, благодаря чему он существует, от чего он зависит во всех своих действиях, в каждом своем шаге. Оторвать человека от природы равносильно тому, как если бы захотеть отделить глаз от света, легкое от воздуха, желудок от средств питания и сделать их себе довлеющими существами. Но то, от чего человек зависит, что  является властью над смертью и жизнью, источником страха и радости, --- есть бог человека и называется богом. Чувство же зависимости, вследствие того факта, что человек почитает природу, вообще бога только за его благодетельность, --- а если и за его вредоносность и распространяемый им ужас, то лишь для того, чтобы отвратить от себя эту его вредоносность, --- привело нас к эгоизму, как к последней скрытой основе религии. Для устранения недоразумений и для более глубокого обоснования этого предмета скажу еще вот что. Чувство зависимости кажется противоречащим эгоизму; ибо в эгоизме я подчиняю предмет себе, в чувстве же зависимости --- себя предмету; в эгоизме я чувствую себя чем-то важным, значительным, в чувстве же зависимости я ощущаю свое ничтожество перед более могущественным. Исследуем еще только страх, это самое яркое выражение чувства зависимости. Почему раб боится своего господина, первобытный человек --- бога грома и молнии? Потому, что господин держит в своих руках жизнь раба, а бог грома --- жизнь человека вообще. Чего же он, стало быть, боится? Потери своей жизни. Он боится лишь из эгоизма, из любви к самому себе, к своей жизни. Где нет эгоизма, нет и чувства зависимости. Кому безразлична жизнь, для кого она ничто, для того ничто и то, от чего она зависит; он его не боится и ничего от него не ждет, и поэтому в его равнодушии нет точки опоры, на которой могло бы утвердиться чувство зависимости. Если я, например, люблю свободное движение, то я чувствую себя зависимым от того, кто может его меня лишить или предоставить его мне, кто меня может запереть или отпустить на волю гулять, ибо я часто хочу гулять, но этого не могу сделать, так как более могущественное существо мне в этом препятствует; если же я равнодушен к тому, заперт я или свободен, нахожусь ли в своей комнате или на воле, то я не чувствую себя зависимым от того, кто меня запирает, ибо он ни своим разрешением, ни своим запретом свободного движения не проявляет по отношению ко мне власти, вызывающей во мне радость или устрашающей меня и, стало быть, порождающей во мне чувство зависимости, ибо желание гулять не имеет надо мною власти. Внешняя власть предполагает, стало быть, внутреннюю, психическую власть, эгоистический мотив и интерес, без которого она для меня ничто, не проявляет надо мною власти, не внушает мне чувства зависимости. Зависимость от другого существа есть в действительности только зависимость от моего собственного существа, от моих собственных влечений, желаний и интересов. Чувство зависимости, поэтому, есть не что иное, как посредственное, извращенное или отрицательное чувство моего <<Я>>  не непосредственное, однако, чувство моего <<Я>>, а переданное мне через посредство того предмета, от которого я чувствую себя зависимым. 

Зависим ведь я только от существ, в которых я нуждаюсь для поддержания своего существования, без которых я не могу того, что я хочу мочь, которые имеют власть дать мне то, чего я желаю, в чем я имею потребность, а я сам при этом не имею власти дать себе это. Где нет потребности, нет и чувства зависимости; если бы человек для своего существования не нуждался в природе, то он не чувствовал бы себя от нее зависимым и, следовательно, не делал бы ее предметом религиозного почитания. И чем больше я нуждаюсь в предмете, тем больше я чувствую себя от него зависимым, тем больше власти имеет он надо мной; но эта власть предмета --- нечто производное, она есть следствие власти моей потребности. Потребность есть столь же слуга, сколь и госпожа своего предмета, столь же смиренна, как и высокомерна или заносчива, она нуждается в предмете, без него она несчастна, в этом заключается ее верноподданство, ее самопожертвование, отсутствие эгоизма. Но она нуждается в нем, чтобы получить в нем удовлетворение, чтобы его использовать, чтобы его наилучшим образом употребить; в этом заключается ее властолюбие или ее эгоизм. Эти противоречивые или противоположные свойства имеет в себе и чувство зависимости, ибо оно не что иное, как потребность в предмете, дошедшая до сознания, или превратившаяся в чувство. Так, голод есть не что иное, как доведенная до моего ощущения и поэтому до сознания потребность моего желудка в питании; не что иное, стало быть, как чувство моей зависимости от средств пропитания. Этой амфиболической, то есть двусмысленной и в самом деле двусторонней природой чувства зависимости объясняется и то явление, по поводу которого так часто удивлялись, потому что не могли подыскать ему разумного объяснения, а именно, что люди могли религиозно почитать животных и растения, которых они ведь уничтожали, поедали. А ведь христиане едят даже своего бога. Потребность, вынуждающая меня съедать какой-либо предмет, имеет в себе две стороны: она подчиняет столько же меня предмету, как и предмет мне, и таким образом она столько же религиозна, как и нерелигиозна. Или если мы расчленим потребность на ее составные части, на ее --- как выражаются современные философы --- моменты, то мы различим в ней недостаток и потребление предмета, ибо ведь в состав потребности в каком-либо предмете входит потребление его, --- потребность есть не что иное, как потребность в потреблении. Потребление предмета, конечно, акт легкомысленный, --- или во всяком случае я могу его так рассматривать, поедая предмет, --- но потребность, то есть чувство недостатка, страстная жажда обладания, чувство зависимости от предмета --- религиозна, смиренна, фантастична, склонна к обожествлению. Пока что-нибудь является для меня только предметом самого желания, оно для меня высшее, фантазия расцвечивает его самыми блестящими красками, моя потребность возвышает его до седьмого неба; но как только я его имею, потребляю, оно как имеющееся налицо теряет все свои религиозные прелести и иллюзии, становится чем-то обыденным; отсюда обычное наблюдение, что все грубо чувственные люди, то есть живущие минутными чувствами и впечатлениями, в нужде, в несчастии, то есть в моменты, когда они в чем-нибудь нуждаются, склонны все отдавать, способны к самопожертвованию, все обещают, но как только они недостающее или желанное получают, делаются неблагодарными, своекорыстными, все забывают; отсюда поговорка: <<нужда учит молиться>>; отсюда происходит то шокирующее благочестивых людей явление, что люди, вообще говоря, религиозны лишь в нужде, недостатке, несчастии. 

Поэтому то обстоятельство, что люди почитают, как религиозные предметы, вещи или существа, которых они поедают, не только не редко и не удивительно, но оно способно наглядно показать нам природу религиозного чувства зависимости с его обеих взаимно противоположных сторон. Различие между христианским и языческим чувством зависимости есть лишь различие в его предметах, различие, заключающееся в том, что предмет языческого чувства зависимости --- определенный, действительный, чувственный; предмет христианского, если отвлечься от воплощенного, съедобного бога, неограниченный, всеобщий, только мыслимый или представляемый, поэтому телесно не потребляемый или не годный к использованию; тем не менее, однако, он --- предмет потребления именно потому, что для христиан предмет потребности, предмет чувства зависимости есть предмет только другого рода потребления, ибо он также и предмет другого рода потребности; ибо христианин желает получить от своего бога не так называемую временную, но вечную жизнь, и удовлетворяет ею не непосредственно чувственную или телесную потребность, но потребность духовную, душевную. <<Мы пользуемся вещами --- говорит отец церкви Августин в своем сочинении о ,,Граде божием``  --- или употребляем в дело вещи, которых мы требуем или ищем не ради них самих, а ради чего-то другого, наслаждаемся мы тем, что ни к чему другому не относится, что радует нас само по себе. Поэтому земное есть предмет использования, вечное же, бог, есть предмет наслаждения>>. Но если мы и оставим в силе это различие и сделаем его признаком, отличающим язычество от христианства, причем в одном случае предметы религии --- боги --- являются предметами использования, в другом же --- предмет есть только предмет наслаждения, то все же мы встретим в христианстве те же явления, те же противоположности, которые мы признали находящимися в природе потребности, чувства зависимости, но которые христианам бросаются в глаза только в религии язычников, а отнюдь не в их собственной, ибо христианский бог, как предмет наслаждения в смысле августиновского различения между наслаждением и использованием, есть в такой же мере предмет эгоизма, как и предмет телесного потребления у язычников, который тем не менее является предметом религии. Противоречие, заключающееся в том, что человек почитает за бога то, что он съедает, противоречие, которое, однако, как только что показано, так же свойственно христианскому чувству зависимости и только благодаря природе своего предмета не так бросается в глаза, --- это противоречие многие народы выражают действительно крайне наивным, даже трогательным образом. При богослужении, при вкушении причастия оно также и здесь сильнее бросается в глаза. <<Не взыщи с нас, --- говорят американские индейцы медведю, которого они уложили, --- за то, что мы тебя убили. Ты --- разумный и понимаешь, что наши дети голодают. Они любят тебя и хотят тебя съесть. Разве не честь для тебя быть съеденным детьми великого капитана?>>. <<Шарлевуа рассказывает про других, что у них тот, кто уложил медведя, всовывает мертвому животному в пасть закуренную трубку, дует в нее с широкого конца, наполняет глотку медведя дымом и просит затем, чтобы медведь не мстил за происшедшее. Во время трапезы, за которой съедают медведя, его голову, раскрашенную всевозможными красками, ставят на возвышенном месте, где голова и принимает от всех гостей моления и хвалебные песни>> (Мейнерс, в указанном месте). Древние финны при разрезывании медведя на куски пели следующую песню: <<Ты --- дорогое, побежденное, тяжело раненное лесное животное, принеси нашим хижинам здоровье и добычу, сторицей, как ты это любишь делать, и позаботься, приходя к нам, о наших нуждах\dots Я буду тебя постоянно чтить и ждать от тебя добычи, дабы я мог не забывать моей хорошей песни в честь медведя>> (Пенпант, <<Арктическая зоология>>). Мы видим, таким образом, что животное, которое убито и съедено, может тем не менее быть одновременно и почитаемо, и что, наоборот, предмет почитания является в то же время и предметом, который едят, что, следовательно, религиозное чувство зависимости заключает в себе и выражает собой как эгоистическое возвышение человека над предметом, поскольку он является предметом потребления, так и смиренное подчинение предмету, поскольку он является предметом потребности. 

А теперь от этого длинного, отнюдь не случайного, но необходимого, самим предметом оправданного изложения развития чувства зависимости и эгоизма я возвращаюсь к природе, к первому предмету этого чувства зависимости. Я уже заметил, что целью моего трактата о <<Сущности религии>>  а следовательно, и этих лекций, является не что другое, как доказательство того, что естественный бог, или бог, которого человек отличает от своего собственного существа, которого он считает основой или причиной этого существа, есть не что иное, как сама природа, человеческий же бог, или духовный бог, или тот бог, которому человек приписывает человеческие свойства, сознание и волю, которого он мыслит себе, как существо себе подобное, которого он отличает от природы, как существа, лишенного воли и сознания, есть не что другое, как сам человек. Но я также уже отметил, что я заимствую свои мысли не из тумана беспочвенных спекуляций, а произвожу их всегда от исторических, эмпирических явлений, что я кроме того, именно поэтому не представляю своих мыслей --- по крайней мере с самого начала и непосредственно --- в общей форме, а делаю их наглядными всегда на действительных случаях, примерах, представляю и развиваю их в конкретной форме. Задачей <<Сущности христианства>>  а именно ее первой части, было показать, что природа есть первоначальное, первое и последнее существо, за пределы которого мы не можем выйти, не теряясь в области фантазии и беспредметной спекуляции, что мы должны на ней остановиться, что мы не можем между ней и собой установить посредничество отличного от нее существа, духа, существа мыслящего, и ее из него выводить, что поэтому, если мы природу производим от духа, то это наше создание имеет значение лишь субъективной, формальной, научной дедукции, но отнюдь не значение действительного, предметного созидания и происхождения. Но эту задачу, эту мысль я связал с действительным явлением, которое уже выразило эту мысль, или, по крайней мере, в основе которого эта мысль лежит, с естественной религией, с немудрствующим, простым, непосредственным человеческим смыслом, производящим природу не от духовного, неестественного и сверхъестественного существа, а рассматривающим ее, как первое, как само божественное существо. 

Человек, исповедующий естественную религию, почитает ведь природу не только, как существо, благодаря которому он существует и без которого он не может жить, не может что-либо делать, он почитает и рассматривает природу также и как существо, благодаря которому он первоначально возник, и именно поэтому --- как альфу и омегу человека. Но если природа почитается и рассматривается как существо, производящее человека, то при этом сама природа рассматривается, как не произведенная, не созданная; ибо человек как мы после еще ближе увидим, выходит за пределы природы, выводит ее из другого существа только тогда, когда он свое собственное существо не в состоянии объяснять из природы. Если мы поэтому первоначально, исходя из практической точки зрения, --- так как человек не может без природы жить и существовать, так как он обязан ей благами своего наличного существования --- если повторяем --- мы первоначально видели, как природа сделалась предметом религии, то теперь она выступает на наших глазах как предмет религии и с теоретической точки зрения. Для человека, стоящего на точке зрения естественной религии, природа не только практически первое, но и теоретически первое существо, то есть то существо, из которого он выводит свое происхождение. Так, например, индейцы считают и до сих пор землю своей общей матерью. Они верят, что они сотворены в ее недрах. Они называют себя поэтому metoktheniake, то есть землерожденные (Хеккевельдер, <<Индейские племена>>). Среди древних индейцев некоторые считали море своим главным божеством и называли его Mamacacha, то есть своей матерью, другие, как, например, колласы, верили в то, <<что предки их рода произошли из великого болота на острове Титикака. Другие свое происхождение приурочивали к большому колодцу, из которого дескать вышел их предок. Иные опять заверяли, что их предки рождены в известных оврагах и ущельях скал; поэтому они все эти места считали священными и приносили им жертвы. Одно племя приписывало причину своего существования реке, поэтому никто не смел удить в ней рыбу, ибо они считали этих рыб своими братьями>> (Баумгартен во <<Всеобщей истории народов и стран Америки>> делает по этому поводу следующее справедливое замечание: <<Так как они различные вещи делали причиной своего происхождения, то они поэтому имели и различные божества, которым поклонялись>>). Гренландцы верят, что <<первый гренландец вырос из земли и сделался, после того как получил себе женщину, родоначальником всех остальных гренландцев>> (Бастгольм, <<О человеке в его диком и первобытном состоянии>>). Точно так же смотрели на землю и почитали ее, как мать людей, греки и римляне. Исследователи языка производят самое слово Erde (земля) от Ord, которое на англосаксонском языке означает, примерно, принцип или начало, и слово Teutsch (немецкий) от Tud, Tit, Teut, Thiud, Theotisc, что значит земной или землерожденный. До какой же степени мы, немцы, благодаря христианству, которое указывает нам на небо, как на наше отечество, оказались неверны нашему происхождению, нашей матери и непохожи на нее! Я должен еще заметить, что среди греков многие, а именно более ранние философы, считали людей и животных происшедшими под влиянием солнечной теплоты либо из земли, либо из воды, либо из обеих вместе, тогда как другие рассматривали их как ниоткуда непроисшедших, как равно вечных с природой или миром. Замечательно также, что религия, или, вернее, мифология греков, а равно и германцев, по крайней мере северогерманцев, --- обе эти религии, в частности последняя, были первоначально естественными религиями, --- замечательно, что они не только людей, но и богов производили из природы очевидное доказательство того, что боги и люди едины суть, что боги вместе с людьми живут и умирают. Так у Гомера Okeanos --- море --- есть производитель, отец богов и людей. У Гезиода, напротив того, земля --- мать Урана, неба и в сочетании с ним --- мать богов. У Софокла поэтому земля называется высшим или величайшим божеством. У северогерманцев великан Имир, <<очевидно, еще не раскрывшаяся совокупность элементов и сил природы>> (Мюллер в указанном месте), предшествует происхождению богов. У римлян, как и у греков, земля называется матерью богов. Августин в своем <<Граде божием>> смеется над тем, что боги --- землерожденные, и заключает отсюда, что правы те, которые считали богов прежними людьми. Конечно, боги, включая и богов Августина, вышли лишь из земли, и если они и не были людьми в смысле Эвгемера, то они все же не существовали раньше, чем люди. С тем же правом, с каким земля называется матерью богов, у Гомера сон называется укротителем людей и богов, ибо боги --- существа, имеющие бытие лишь для и чрез людей; они поэтому не бдят над человеком, когда он спит, и если люди спят, то спят также и боги, то есть с сознанием людей угасает и существование богов. 

Моя задача в <<Сущности религии>> заключалась не в чем другом, как в том, чтобы защитить, оправдать, обосновать естественную религию, по крайней мере лежащую в ее основе истину против теистических объяснений природы и выводов из них. Я выполнил эту задачу по всем направлениям не меньше чем в двадцати параграфах, с 6 по 26. Прежде, однако, чем приступить к их содержанию, я должен предварительно заметить, что --- как это, впрочем, само собой разумеется --- ход изложения в истории религии соответствует моему ходу изложения в психологии, в философии, в развитии человечества вообще. Как природа для меня есть первый предмет в религии, точно так же и в психологии, в философии вообще чувственное есть первое; но первое не только в смысле спекулятивной философии, в которой первое означает то, за пределы чего надо выйти, но и первое в смысле невыводимого из другого, через себя самого существующего и истинного. Точно так же, как я не могу вывести чувственное из духовного, я не могу вывести и природу из бога; ибо духовное --- ничто вне и без чувственного, дух есть лишь сущность, смысл, дух наших чувств; бог же не что иное, как дух, мыслимый в общем, дух, взятый независимо от различия между моим и твоим. Поэтому точно так же, как я не могу вывести тело из моего духа, --- ибо, чтобы пояснить сейчас же примером, я должен сначала есть или иметь возможность есть, прежде чем думать, но не сначала думать, чтобы потом есть: я могу есть, не думая, как этому служат доказательством животные, но не могу думать, не вкушая пищи, --- точно так же как я не могу вывести чувство из моей способности к мышлению, из разума, ибо разум предполагает существование чувства, а не чувство --- существование разума, отказываем же мы животным в разуме, но не в чувствах, --- точно так же и еще того меньше могу я вывести природу из бога. Истинности и существенности или божественности природы, от которой отправляются философия религии и история религии, отвечает поэтому истинность и существенность чувств, от которой отправляются психология, антропология, философия вообще. И точно так же, как природа не есть преходящая истина в истории религии, точно так же не преходяща истинность чувств в философии. Напротив того, чувства --- остающаяся основа даже в тех случаях, когда они исчезают в абстракциях разума, по крайней мере в глазах тех, которые, как только начинают мыслить, перестают думать о чувствах, забывают, что человек мыслит лишь при посредстве своей чувственно существующей головы, что разум имеет прочную чувственную почву под собой в голове, в мозгу, в том месте, которое является средоточием чувств. 

Естественная религия демонстрирует перед нами истинность чувств, а та философия, которая понимает себя как антропологию, демонстрирует перед нами истинность естественной религии. Первая вера человека есть вера в истинность чувств, а не вера, противоречащая чувствам, как теистическая и христианская вера. Вера в бога, то есть в нечувственное существо, --- мало того, в существо, исключающее, отрицающее все чувственное, как недостойное, --- есть нечто меньше всего непосредственно достоверное, в противоположность тому, что утверждал теизм. Первые непосредственно достоверные существа и именно поэтому и первые боги людей --- это чувственные предметы. Цезарь о религии германцев говорит так: они почитают только те существа, которых они видят и от которых они получают очевидные благодеяния. Это столь оспариваемое место у Цезаря применимо ко всем естественным религиям. Человек первоначально верит только в существование того, что обнаруживает свое бытие чувственными, ощутимыми действиями и признаками. Первыми евангелиями, первыми и неподдельными, неискаженными через обман жрецов источниками религии человека являются его чувства. Или, правильнее, эти его чувства --- сами его первые боги; ибо вера во внешних, чувственных богов зависит ведь исключительно от веры в истинность и божественность чувств; в тех богах, которые являются чувственными существами, человек обожествляет только собственные чувства. Тем, что я почитаю свет божественным существом, этим и в этом я выражаю лишь --- разумеется, косвенно и бессознательно --- признание божественности глаза. Свет или солнце, или луна есть бог, предмет для глаза, а не носа; культ носа --- небесные благоухания. Глаз делает богов существами света, блеска, видимости, то есть он обожествляет лишь предметы, видимые глазом: созвездия, солнце, луна не имеют ведь для человека другого бытия, как через глаза; они не даны другим чувствам, то есть глаз обожествляет лишь свое собственное существо; боги других чувств для него идолы или, вернее, для него совсем не существуют. Орган же обоняния человека обожествляет благоухания. Уже Скалигер говорит в своих <<Опытах>> против Кардана: <<запах есть нечто божественное --- odor divina res est --- и, что он таков, показали древние своими религиозными церемониями, веря, что курениями можно подготовить воздух и помещение к принятию божества>>. Язычники верили, верят отчасти и до сих пор, что боги живут и питаются теми благоуханиями, которые поднимаются от приносимых жертв, что, стало быть, благоухания являются составными частями богов, а следовательно, боги --- существа, состоящие из одного лишь благоухания и пара. По крайней мере, если бы человек обладал одним только органом обоняния и никаким другим, он составил бы божественное существо из одного благоухания, оставив в стороне все прочие свойства, доставляемые другими чувствами. Так, каждое чувство обожествляет только само себя. Короче говоря, истина естественной религии опирается исключительно на истину чувств. Так с <<Сущностью религии>> сплетаются <<Основные положения философии>>. Если я, впрочем, говорю в защиту естественной религии, --- так как и поскольку она опирается на истину чувств, --- то я этим отнюдь не говорю в защиту того, как она этими чувствами пользуется, как она смотрит на природу и ее почитает. Естественная религия опирается лишь на видимость, создаваемую чувствами, или, вернее, на то впечатление, которое эта чувственная видимость производит на душевное настроение и на фантазию людей. Отсюда проистекает вера древних народов в то, что их страна есть мир или во всяком случае центр мира, что солнце движется, а земля находится в покое, что земля --- плоская, как тарелка, и омывается океаном. 

\phantomsection
\addcontentsline{toc}{section}{Одиннадцатая лекция}
\section*{Одиннадцатая лекция}

Я уже объяснял, что значение параграфов, служащих текстом для этих моих лекций, заключается исключительно в том, чтобы научно оправдать, обосновать то, что здравый смысл древних и еще нынешних первобытных народов выразил, фактически, если и несознательно, почитая природу, как божественное существо, а именно, что она существо первое, первичное, непроизводное. Но прежде всего я должен посчитаться с двумя возражениями. 

Мне могут, во-первых, возразить: как ты, неверующий, хочешь оправдать естественную религию? Не становишься ли ты тем самым на раскритикованную тобой так резко точку зрения тех философов, которые оправдывают символ веры христианства, с тою лишь разницей, что ты хочешь оправдать догмат естественной религии, веру в природу? Я на это отвечаю: природа для меня совсем не потому первична, что естественная религия на нее так смотрит и ее так почитает, а скорее, наоборот, из того, что она есть первичное, непосредственное, я заключаю, что она и должна была представиться такою первичному, непосредственному, а следовательно, и родственному природе сознанию народов. Или иначе говоря: тот факт, что люди почитают природу как бога, отнюдь не служит мне в то же время и доказательством истинности лежащего в основании этого факта сознания; но я нахожу в этом факте подтверждение впечатления, производимого природой на меня, как на чувственное существо; я нахожу в нем подтверждение тех оснований, которые побуждают меня как существо интеллектуальное, обладающее философской культурой, придавать природе если не то же значение, которое ей приписывает естественная религия, --- ибо я ничего не обожествляю, а следовательно, не обожествляю и природы, --- то во всяком случае аналогичное, подобное, видоизмененное лишь при посредстве естественных наук и философии. Конечно, я симпатизирую религиозным почитателям природы; я --- ее страстный поклонник и почитатель; я понимаю не из книг, не на основании ученых доказательств, а из моих непосредственных представлений и впечатлений от природы, что древние народы и даже еще и современные народы могут почитать ее как бога. Я еще и сейчас нахожу в своем чувстве, ощущаю в своем сердце, как оно охватывается природой, нахожу и сейчас в своем уме аргументы в пользу ее божественности или ее обожествления. Я отсюда заключаю, ибо ведь и солнце, огне и звездопоклонники --- такие же люди, как и я, что и их побуждают к обожествлению природы сходные с моими мотивы (хотя и видоизмененные соответственно их точке зрения). Я не делаю заключений, подобно историкам, от прошлого к настоящему, а заключаю от настоящего к прошлому. Я считаю настоящее ключом к прошлому, а не наоборот, на том простом основании, что ведь я хотя и бессознательно и непроизвольно, но постоянно измеряю, оцениваю, познаю прошлое исключительно со своей нынешней точки зрения, поэтому каждое время имеет другую историю прошлого, хотя это прошлое само по себе мертво и неизменно. Я признаю поэтому естественную религию не в силу того, что она для меня внешний авторитет, а исключительно потому, что я и посейчас нахожу в самом себе ее мотивы, те основания, которые и в настоящее время сделали бы меня человеком, обожествляющим природу, если бы власть естественной религии не спасовала перед властью культуры, естественных наук, философии. Это кажется дерзким; но то, чего человек не познает из самого себя, того он совсем не познает. Кто изнутри себя и на себе не чувствует, почему люди могли обожествлять солнце, луну, растения и животных, тот не поймет и исторического факта обожествления природы, хотя бы он и прочел и написал груду книг об естественной религии. 

Второе возражение таково: ты говоришь о природе, не давая нам определения того, что такое природа, не говоря нам, что ты под природой понимаешь. Спиноза придает однозначный смысл словам <<природа>>  или <<бог>>. Употребляешь ли ты и это слово в том неопределенном смысле, в котором ты нам можешь легко доказать, что природа есть первичное существо, причем ты под природой ничего другого не понимаешь, как бога? Я на это отвечаю немногими словами: я понимаю под природой совокупность всех чувственных сил, вещей и существ, которые человек отличает от себя как нечеловеческое; я разумею вообще под природой, как я это уже высказал на одной из своих первых лекций, подобно Спинозе, не существо, живущее и действующее, будучи наделенным волей и разумом, как сверхъестественный бог, но только существо, действующее сообразно необходимости своей природы; однако оно для меня не бог, как для Спинозы, то есть не одновременно сверхъестественное, сверхчувственное, отвлеченное, тайное и простое существо, но существо многообразное, человеку доступное, действительное, всеми чувствами воспринимаемое. Или, беря слово практически, природа есть все то, что для человека --- независимо от сверхъестественных внушений теистической веры --- представляется непосредственно, чувственно, как основа и предмет его жизни. Природа есть свет, электричество, магнетизм, воздух, вода, огонь, земля, животное, растение, человек, поскольку он является существом, непроизвольно и бессознательно действующим. Под словом <<природа>> я не разумею ничего более, ничего мистического, ничего туманного, ничего теологического. Я апеллирую этим словом к чувствам. Юпитер есть все, что ты видишь, говорил один древний; природа, говорю я, есть все, что ты видишь и что не является делом человеческих рук и мыслей. Или, если вникнуть в анатомию природы, природа есть существо или совокупность существ и вещей, чьи проявления, обнаружения или действия, в которых проявляется и существует их бытие, имеют свое основание не в мыслях, или намерениях и решениях воли, но в астрономических или космических, механических, химических, физических, физиологических или органических силах или причинах. 

Содержанием параграфов 6 и 7, взятых мною в качестве темы этой лекции, является защита и оправдание язычников, защита против упреков, делаемых им христианами, и имеет оно отношение к ранее высказанному утверждению, что христианская религия отличается от языческой не своим принципом, не отличительным признаком божества, а только тем, что она имеет своим богом не определенный предмет природы, даже не природу вообще, а существо, от природы отличное. Христиане, по крайней мере разумные из их числа, порицали язычников не за то, что они восторгались красотой и прелестью природы, а за то, что они причину их приписывали самой природе, что они воздавали хвалу за их благодетельные свойства земле, воде, огню, солнцу, луне, тогда как ведь они получили эти свойства лишь от творца природы, и его, стало быть, одного надлежит почитать, бояться, восхвалять. Солнце, земля, вода, правда, являются причиной того, что существуют растения и животные, которыми живут в свою очередь люди, но это причины --- подчиненные, причины, которые сами произведены, истинная же причина, это --- первая причина. Против этих обвинений я защищаю язычников, причем вначале оставляю открытым вопрос, существует ли первая причина, как христиане ее себе представляют; я защищаю при помощи примера или, вернее, сравнения, взятого из круга христианских представлений. Адам --- первый человек; он в ряду людей --- то же, что первая причина в ряду естественных причин или предметов; мои родители, деды и так далее в такой же мере дети Адама, как причины в природе суть действия первой причины; только Адам не имеет отца, как и первая причина не имеет причины. Но тем не менее я почитаю и люблю как моего отца не Адама; Адам охватывает всех людей, из него вытравлена всякая индивидуальность; Адам в такой же степени отец негра, как и белого, славянина, как и германца, француза, как и немца; я же не человек вообще; мое бытие, мое существо индивидуально, я принадлежу к кавказской расе и в ней --- опять-таки к определенной ветви, к немецкой. Причина моего существа, по необходимости, и сама индивидуальна, определенна, но это именно мои родители, дед и бабушка, короче --- ближайшие ко мне поколения или люди. Если я дальше обращаюсь назад, то я теряю из глаз все следы моего существования; я не нахожу тех свойств, из которых я мог бы вывести свои собственные свойства. Человек семнадцатого века не мог бы ни в каком случае, если бы его и не отделяло время, быть отцом человека девятнадцатого века, ибо слишком велико качественное расстояние, различие в нравах, привычках, представлениях, направлении ума, которое отражается даже и физически. Поэтому, как человек со своим почитанием останавливается на своих ближайших предках, как на причинах своего существования, а не восходит до первого предка, потому что он не находит в этом предке содержащейся и представленной своей от него неотделимой индивидуальности, так же точно останавливается он и на чувственных естественных существах, как на причинах своего существования. Я есмь то, что я есмь только в данной природе, в природе, какова она сейчас, какова она на человеческой памяти. Только существам, которых я вижу, чувствую, --- или если я их сам не вижу и не чувствую, то все же они сами по себе существа видимые, чувствуемые или как-нибудь иначе ощущаемые, --- только им обязан я своим существованием, я, который представляю собой чувственное существо, который без чувств погружаюсь в небытие. Если эта природа и создана, если ей и предшествовала природа другого рода или характера, то обязан я все же своим существованием только природе данного рода и характера, в которой я живу, с характером которой уживается и характер моего существования. Если и допустить, что существует первая причина в теологическом смысле, то все же должны были сначала быть солнце, земля, вода, короче говоря, природа, и природа данного рода, прежде чем я явился на свет; ибо без солнца, без земли я сам ничто; мое существование предполагает существование природы. 

Зачем мне выходить за пределы природы? Я бы имел на то право, если бы сам был существом, стоящим над природой. Но я до такой степени мало являюсь сверхъестественным существом, что даже не являюсь существом сверх или надземным; ибо земля есть абсолютное мерило моего существа; я не только стою своими ногами на земле, я думаю и чувствую только с точки зрения земли, только соответственно тому положению, которое Земля занимает во вселенной; я, конечно, поднимаю свои взоры до самого отдаленного неба; но я вижу все вещи в свете и в масштабе земли. Короче говоря, то, что я земное существо, что я не житель Венеры, Меркурия, Урана, составляет, как говорят философы, мою субстанцию, мою основную сущность. Если, стало быть, Земля и возникла, то все же я обязан только ей, только ее происхождению моим происхождением; потому что только существование земли есть основа человеческого существования, только ее существо есть основа человеческого существования. Земля есть планета, человек --- планетное существо, существо, жизненный путь которого возможен и действителен только в пределах жизненного пути планеты. Но земля отличается от других планет. Это ее отличие обосновывает ее своеобразное, самостоятельное существо, ее индивидуальность, и эта ее индивидуальность есть соль земли. Если мы и примем, и с полным правом примем, что одна и та же причина, сила или субстанция создала планеты, то все же эта сила, от действия которой образовалась Земля, была другой, чем та, от действия которой образовались Меркурий или Уран, то есть столь своеобразно определенная, что в результате ее действия получилась именно эта, а не какая-либо другая планета. Этой индивидуальной причине, которую нельзя отличить от существа Земли, обязан человек своим существованием. Революционный толчок, который вырвал Землю из ее мистического растворения в общей основной материи Солнца, планет и комет, --- революция, которая, как выражается Кант в своей превосходной теории неба, имела свое основание в <<различии родов элементов>>  этот отрыв или толчок и был тем, от чего и до сих пор ведут свое происхождение движение нашей крови и колебания наших нервов. Первая причина есть всеобщая причина, причина всех вещей без различия; но причина, которая производит без различия все, на самом деле не производит ничего, она только понятие, мысленное существо, имеющее только логическое и метафизическое, но отнюдь не физическое значение, существо, из которого я, данное индивидуальное существо, ни в каком случае не могу себя вывести. Первой причиной, --- первой, я всегда прибавляю, <<в смысле теологов>> --- хотят поставить предел движению причин до бесконечности. Это движение причин до бесконечности можно лучше всего пояснить на уже приведенном примере происхождения человека. Я имею причиной моего существования моего отца, мой отец --- своего отца и так далее Но могу ли я идти до бесконечности? Всегда ли человека производил на свет человек? Решаю ли я этим вопрос о происхождении человека, не отодвигаю ли его только тем, что подвигаюсь от отца к отцу? Не должен ли я прийти к первому человеку или к первой человеческой паре? Но откуда явилась эта пара? 

Но так же точно обстоит дело и со всеми другими вещами и существами, составляющими данный чувственный мир. Одно предполагает другое; одно зависит от другого; все вещи и существа конечны, все произошли одно из другого. Но откуда, спрашивает теист, первое в этой цепи, в этом ряду? Мы должны поэтому сделать скачок вон из этого ряда к тому первому, которое, будучи само без начала, является началом всех возникших, будучи само без конца, или бесконечным, является основой всех конечных существ. Это --- одно из обыкновеннейших доказательств существования бога, одно из доказательств, которое называют космологическим и по-разному выражают, например так: все, что есть, или мир, --- переменчиво, временно, имеет свое происхождение, случайно; но случайное предполагает наличность необходимого, конечное бесконечного, временное --- вечного; это бесконечное, это вечное есть бог. Или можно еще выразить так: все, что есть, все-чувственное, действительное, есть причина определенных действий, но причина, которая сама явилась результатом действий, которая сама, в свою очередь, имеет причину, и так далее; поэтому необходимой потребностью нашего разума является остановка, наконец, на причине, которая не имеет за собой больше причины, которая не вызвана ничьим действием, которая, как выражаются некоторые философы, есть причина самой себя или явилась из самой себя. Древние философы и теологи определяли поэтому конечное, не божественное, как то, что происходит из другого, а бесконечное, бога --- как то, что происходит от или из самого себя. 

Но против этого вывода можно заметить следующее. Если продвижение причин до бесконечности в вопросе о происхождении людей, даже Земли, и противоречит разуму и мы не можем любое состояние Земли вывести из предыдущего ее состояния, но должны подойти, наконец, к той точке, когда человек вышел из природы, а Земля --- из планетной массы или из той основной массы, которую можно как угодно называть, --- то, во всяком случае, это продвижение нисколько не противоречит разуму, сформировавшемуся на представлении о мире, не противоречит тогда, когда это продвижение относится или применяется к природе, или миру вообще. Только ограниченность человека и его склонность к упрощению ради удобства подставляют ему вместо времени вечность, вместо непрекращающегося никогда движения от причины к причине безначальность, вместо не знающей устали природы --- неподвижное божество, вместо вечного движения --- вечный покой. Правда, мне, имеющему дело с настоящим, неразумно, бесцельно, скучно и даже невозможно мыслить или только представлять себе безначальность и бесконечность мира; но эта необходимость, существующая для меня, --- оборвать этот бесконечный пробег не есть еще доказательство того, что этот пробег действительно обрывается, что существуют действительное начало и действительный конец. Даже в сфере отражающихся на человеческом сознании исторических явлений, даже явлений, вызванных самим человеком, можем мы наблюдать, как человек --- частью, правда, по незнанию, но частью из простого стремления к сокращению и упрощению ради удобства --- обрывает исторические исследования: вместо многих имен, многих причин, которые слишком долго и слишком обременительно было бы прослеживать и которые часто ускользают от взора людей, --- ставит одну причину, одно имя. Как человек во главе какого-либо изобретения, основания государства, постройки города, возникновения народа ставит имя одной какой-либо личности, хотя масса неизвестных имен и личностей принимала в нем участие, точно так же ставит он и во главе мира имя бога, равно как по той же причине и все изобретатели, основатели городов и государств считались определенно за богов. Поэтому большинство древних имен исторических и мифических людей, героев и богов являются именами коллективными, ставшими, однако, именами собственными. Даже самое слово <<бог>> первоначально, как, разумеется, и все имена, не есть имя собственное, но всеобщее или родовое имя  \hyperlink{9}{(9)}\hypertarget{b9}{}. Даже в Библии греческое слово theos и еврейское слово elohim употребляются для обозначения других предметов, кроме бога. Так, князья и начальствующие лица называются богами, дьявол --- богом этого мира, даже живот --- богом людей или, по крайней мере, некоторых людей; от этого места в Библии даже Лютер приходит в ужас. <<Кто когда-либо, --- говорит он, --- слыхал такую речь, что живот есть бог? Если бы раньше не говорил так Павел, я бы не смел так говорить, ибо я не знаю более постыдных слов. Не горестно ли, что постыдный, вонючий, грязный живот может называться богом?>> 

И даже при философском определении, гласящем, что бог есть наиреальнейшее, то есть наисовершеннейшее существо, совокупность всех совершенств, бог есть, собственно говоря, коллективное имя; ибо достаточно мне из различных свойств, сконцентрированных в боге, выделить лишь их различия, чтобы они произвели на меня впечатление разных вещей или существ, и чтобы я нашел, что слово бог такое же неопределенное, коллективное или собирательное слово, как, например, слово овощи, зерно, народ. 

Ведь каждое свойство бога есть само бог, как утверждает теология или теологическая философия, каждое свойство бога может быть поэтому поставлено вместо бога. Даже в повседневной жизни вместо бога говорят божественное провидение, божественная мудрость, божественное всемогущество. Но свойства бога имеют весьма различную и даже противоречивую природу. Ограничимся рассмотрением лишь самых популярных свойств. Как различны могущество, мудрость, доброта, справедливость! Можно быть могущественным без мудрости и мудрым без могущества, добрым без справедливости и справедливым без доброты. Fiat justitia, pereat mundus (правосудие должно совершиться, хотя бы погиб мир); мир может погибнуть, только бы jus, только бы право сохраняло свое значение, --- изречение юриспруденции, справедливости; но в этом выражении, характерном для правосудия, нет, без всякого сомнения, ни искры доброты и даже мудрости; ибо не человек существует для справедливости и для правосудия, но правосудие для человека. Поэтому, если я представляю себе могущество бога, могущество, которое, если захочет, может меня уничтожить, или представляю себе справедливость бога в смысле только что приведенного изречения, то я представляю себе в виде бога совсем другое существо, я в самом деле имею совсем другого бога, чем если бы я представлял себе только его доброту. Поэтому совсем не так велико различие между политеизмом и монотеизмом, как это кажется. И в едином боге ввиду множества и разнообразия его свойств имеется много богов. Различие не больше того, которое существует между словом собирательным и коллективным. Или, вернее, оно таково: при политеизме бог --- открыто, очевидно --- есть только собирательное слово; при монотеизме же отпадают чувственные признаки, исчезает видимость политеизма; но существо, сама вещь остается. Поэтому различные свойства единого бога вели при христианах столько же не только догматических, но и кровавых боев друг с другом, как и многочисленные боги на Олимпе Гомера. 

Древние теологи, мистики и философы говорили, что бог объемлет в себе все, что есть в мире, но то, что в мире многообразно, рассеяно, разрознено, чувственно поделено между различными существами, то в боге имеется в простом, нечувственном, едином виде. Здесь мы отчетливо высказали, что человек в боге объединяет существенные свойства многих различных вещей и существ в одно существо, в одно имя, что человек представляет себе в боге первоначально себя или, поистине, --- неотличное от мира существо, но в этом существе тот же мир, только на иной лад, отличный от чувственного воззрения; то, что он представляет себе в мире или в своем чувственном воззрении протяженным в пространстве и во времени, телесным, то он мыслит в боге не протяженным в пространстве и во времени, бестелесным. В вечности он берет только в одно короткое родовое имя или понятие бесконечный ряд времен, в его полной протяженности непостижимый, в вездесущности --- только бесконечность пространства; из субъективных вполне обоснованных соображений, он, принимая вечность, обрывает бесконечно скучный для него счет с рядами чисел, множащихся до бесконечности. Но из этого обрывания, из этой скуки тянущегося до бесконечности ряда времен и пространств, из самих противоречий, связанных в нашем представлении или в абстракции с понятием вечного времени, бесконечного пространства, отнюдь не следует необходимость действительного начала или конца мира, пространства, времени; это --- в природе нашего мышления, языка, к этому приводит нас необходимость жизни, что мы всюду пользуемся знаками сокращения, что мы всюду ставим на место воззрения понятие, на место предмета --- значок, слово, на место конкретного абстрактное, на место множества --- единое, стало быть, на место многих различных причин --- одну причину, на место многих различных личностей --- одну личность как представителя, заместителя прочих. И правы поэтому те, кто утверждают, что разум, --- по крайней мере до тех пор, пока он некритически, не различая, принимает свое существо за существо мира, за объективное, абсолютное существо, пока он не облагорожен мировоззрением, --- необходимо приводит к идее божества. Не нужно только эту необходимость, эту идею отдельно выдвигать, изолировать ее, обособлять от других явлений, идей и представлений, которые основываются на той же необходимости, которые мы тем не менее познаем как субъективные, то есть основывающиеся только на своеобразной природе представления, мышления, речи, и которым не приписывается никакого объективного значения и существования, никакого существования вне нас. 

Та же необходимость, которая побудила человека поставить имя одной личности на место ряда личностей и даже поколений и родов, которая побудила его вместо созерцаемой величины поставить число, на место чисел --- буквы, которая побудила его вместо груши, яблока, вишни говорить --- плоды, вместо геллера, пфеннига, крейцера, гроша, гульдена, талера --- просто деньги, вместо <<дай мне этот нож, эту книгу>> говорить <<дай мне эту вещь>>  --- эта же необходимость побудила его также на место многих причин, совместно действовавших при возникновении мира, если мы мыслим себе его возникшим, и при его сохранении поставить одну причину, одно существо, одно имя. Но именно поэтому это единое существо столь же субъективно, то есть существует только в человеке, только в природе его представления, мышления, речи и на них основывается, как и вещь, деньги, плоды. Что идея или родовое понятие божества в его метафизическом значении покоится на той же необходимости, на тех же соображениях, что и идея или понятие вещи, плодов, доказывается уже тем, что у политеистов боги являются не чем другим, как представленными в виде существ собирательными или родовыми именами и понятиями. Так, римляне чтобы остаться при старых примерах --- имели богиню денег --- Pecunia, и даже различные главные сорта или роды денег, медные и серебряные деньги, они делали богами. Они имели бога Aesculanus или Aerinus, то есть бога бронзовых или медных денег, бога Argentinus, то есть серебряного бога. Имели они также и богиню плодов --- Рошопа. Если у римлян и греков не встретишь всех родовых имен и понятий в качестве богов, то это происходит только от того, что именно римляне --- эти эгоистические святоши --- обожествляли лишь то, что одновременно обозначает какое-либо отношение к человеческому эгоизму; поэтому римлянами почитался даже бог навоза, deus Stercutius, дабы удобрение принесло пользу полям. Но навоз --- родовое понятие; ведь есть различные виды навоза: голубиный навоз, лошадиный, коровий и так далее.

Обратимся теперь к другому пункту, который нам нужно развить против обычного вывода относительно первой, уже не имеющей причин, причины. Все, что есть, зависимо, или, как другие это формулируют, имеет основание своего существования вне себя, существует не от себя и не через себя само, предполагает поэтому наличность существа, независимого от других, имеющего основание своего существования в самом себе, безусловно необходимого, существа, которое есть, потому что оно есть. Против этого довода я привожу опять пример человека, ибо ведь в конечном счете только человек есть то, от чего исходит человек и чью зависимость и чье происхождение он берет за образец зависимости и происхождения всех чувственных предметов. Конечно, я завишу от моих родителей, родителей моих родителей и так далее; конечно, я брошен в мир не самим собой; меня бы не было, если бы другие не существовали раньше меня; тем не менее, однако, я --- отличное и независимое от моих родителей существо; то, что я собой представляю, я представляю не только благодаря другим, но и благодаря себе самому; я, конечно, стою на плечах моих предков, но и на плечах их я стою все же еще и на своих собственных ногах; я, конечно, без моего ведома и желания зачат и рожден; но появился я на свет не без влечения к самостоятельности, к свободе, к эмансипации от моей зависимости от материнского тела (влечения, которое, разумеется, сейчас мной не сознается); короче говоря, я произведен на свет, я зависим или был зависим от моих родителей; но я сам также отец, сам также муж, и то, что я произошел, что я был когда-то ребенком и зависел телесно и духовно от своих родителей, находится бесконечно далеко позади моего настоящего самосознания. Одно несомненно: сколько бы мои родители не имели сознательно или бессознательно влияния на меня, --- какое мне дело до прошлого? Сейчас я своего отца и свою мать чувствую лишь в себе самом, сейчас мне не поможет никакое другое существо, ни даже сам бог, если я себе сам не помогу; я действую по собственной инициативе, как хочу. Пеленки, которыми заботливость моих родителей обвязала мое тело, давно сгнили; зачем же мне и мой дух оставлять в тех путах, которые давно сбросили мои ноги? 

\phantomsection
\addcontentsline{toc}{section}{Двенадцатая лекция}
\section*{Двенадцатая лекция}

В моей последней лекции я показал на примере человека одно из первых и обыкновеннейших доказательств, так называемое космологическое доказательство, бытия божьего, основывающееся на том, что все в мире конечно и зависимо, а потому предполагает существование вне себя чего-то бесконечного и независимого. Вывод был тот, что человек хотя первоначально и сын, но в то же время и отец, хотя и следствие, но в то же время и причина, хотя зависим, но в то же время и самостоятелен. Но то, что относится к человеку, относится, разумеется, --- с тем само собой понятным различием, которое вообще имеется налицо между человеком и другими существами, --- и к этим существам. Каждое существо, несмотря на свою зависимость от других, принадлежит себе, самостоятельно; каждое существо имеет основу своего существования в самом себе; --- ибо для чего бы иначе оно существовало? каждое существо произошло при условиях и из причин --- каковы бы они ни были, которые не могли дать другого существа, как именно это; каждое существо произошло из сочетания причин, которого бы, то есть сочетания, не было, если бы не было этого существа. Каждое существо столько же следствие, как и причина. Не было бы рыбы, если бы не было воды, но и воды бы не было, если бы не было рыб, или, по крайней мере, животных, которые могли бы в ней жить, как рыбы. Рыбы --- существа, зависимые от воды; они не могут существовать без воды; они предполагают ее существование; но причина их зависимости находится в них самих, в их индивидуальной природе, делающей именно воду их потребностью, их стихией. 

У природы нет ни начала ни конца. Все в ней находится во взаимодействии, все относительно, все одновременно является действием и причиной, все в ней всесторонне и взаимно; она не упирается в монархическую верхушку; она --- республика. Кто привык к монархическому управлению, разумеется, не может помыслить себе государство, общественную совместную жизнь людей без монарха; и так же не мыслит себе природы без бога тот, кто с детских лет привык к этому представлению. Но природа не менее мыслима без бога, без вне и сверхъестественного существа, чем государство или народ без монарха-идола, стоящего вне народа и над ним. \emph{И как республика составляет историческую задачу, практическую цель человечества, точно так же и теоретическую цель человека составляет признание строя природы республиканским, не перенесение управления природой во вне ее, но обоснование его ее собственной сущностью}. Нет ничего более бессмысленного, как превращать природу в одностороннее следствие и этому следствию противопоставлять одностороннюю причину в виде внеестественного существа, не являющегося в свою очередь следствием другого существа. Ведь раз я не могу удержаться и все дальше и дальше размышляю и фантазирую, не останавливаясь на одной лишь природе и не находя удовлетворения ненасытному стремлению моего ума искать причины во всестороннем и взаимном действии природы, --- то что удержит меня пойти также и дальше бога? Зачем мне здесь останавливаться? Отчего не поставить вопрос об основе бога или его причине? И нет ли, когда речь идет о боге, того же соотношения вещей, которое мы видим в сцеплении естественных причин и следствий и которое мы только что захотели устранить допущением бога? Разве бог, когда я его мыслю как причину мира, не зависит от мира? Разве есть причина бездействия? Что останется от бога вообще, если я оставлю мир в стороне или отброшу его мыслью? Где его --- бога --- могущество, если он ничего не творит, его мудрость, если нет мира, в управлении которым и состоит его мудрость? Где его благость, если нет ничего, по отношению к чему он был бы благ? Где его сознание, если нет предмета, по отношению к которому он бы себя сознавал? Где его бесконечность, если нет ничего конечного, ибо ведь он бесконечен, лишь будучи противоположен этому конечному? Поэтому, если я выкидываю мир, то мне ничего не остается от бога. Почему же не хотим мы остановиться на мире, перескочить через который или выйти из пределов которого мы не в состоянии, ибо даже представление о боге и принятие его бытия отбрасывает пас назад к миру, так как с устранением природы, мира мы уничтожаем всю действительность, а стало быть и действительность бога, поскольку он мыслится, как причина мира? 

Поэтому трудности, которые встают перед нашим духовным взором по вопросу о начале мира, не разрешаются нами тем, что мы допускаем существование бога, существа, стоящего вне мира, а только отодвигаются или отбрасываются в сторону, или затушевываются. Всего разумнее поэтому принять, что мир был и будет вечно, что он, стало быть, имеет в себе самом основу своего существования. <<Нельзя, --- говорит Кант в своих лекциях по философии религии, --- отделаться от мысли, но в то же время нельзя и допустить, что существо, которое мы представляем себе как высшее среди всех возможных существ, как бы говорит самому себе: я существую от вечности и в вечность; кроме меня нет ничего, за исключением того, что по моей воле является чем-то; но откуда же я сам?>> Это значит, другими словами: откуда же бог, что заставляет меня на нем остановиться? Ничто; наоборот, я должен спросить себя о его происхождении. И это происхождение не тайна; причина первой и всеобщей причины вещей в смысле теистов, теологов, так называемых спекулятивных философов, есть человеческий рассудок. Разум восходит от единичного и особенного ко всеобщему, от конкретного к абстрактному, от определенного к неопределенному. Так же точно разум восходит от действительных, определенных, особых причин до тех пор и так далеко, пока не дойдет до понятия причины как таковой --- той причины, которая не порождает определенных, особых следствий. Бог не есть, по крайней мере непосредственно, как утверждают теисты, причина молнии и грома, лета и зимы, дождя и солнечного света, огня и воды, солнца и луны; все эти вещи и явления имеют лишь определенные, особые, чувственные причины; бог же --- только всеобщая причина, причина причин; он есть причина, которая не является определенной, чувственной, действительной причиной, причина, абстрагированная от всякого чувственного вещества и материала, от всяких специальных предназначений, то есть он есть причина вообще, понятие причины как олицетворенной сущности, которая стала самостоятельной. Подобно тому, как разум олицетворяет понятие существа, у которого отняты все определенные свойства действительных существ, в виде единой сущности, --- точно так же олицетворяет он понятие причины, лишенное всех действительных, определенных, причинных качеств, в виде первой причины. Как вообще с точки зрения разума, отвлекающегося от чувств, субъективно и логически совершенно правильно человек предполагает род раньше, чем индивидуумов, цвет как таковой раньше, --- чем цвет определенный, человечество --- раньше, чем человека, --- точно так же предполагает он и раньше причин причину как таковую. \emph{Бог есть основа мира, это значит: причина как таковая есть основа причин; если нет причины, то нет и причин; первое в логике, в распорядке разума есть причина, второе --- подчиненное причине, или виды причины; короче говоря, первая причина редуцируется, сводится к понятию причины и понятие причины сводится к разуму, который всеобщее выводит из особых конкретных вещей и затем сообразно своей природе это выведенное из них всеобщее предпосылает им, как первое}. Но именно поэтому, так как первая причина есть простое понятие, принадлежащее разуму, или его сущность, не имеющая предметного существования, она и не является причиной моей жизни и бытия; причина вообще мне не помощь; причина моей жизни есть понятие, включающее в себя многие, различные, определенные причины; причина, например, того, что я дышу, субъективно легкое, объективно --- воздух; причина того, что я вижу, объективно --- свет, субъективно --- глаз. Я обращаюсь поэтому опять от безотрадной, абстрактной темы первой, ничего не производящей причины к природе, к совокупности действительных причин, чтобы еще раз более отрадным способом доказать, что мы должны остановиться на природе, как на последнем основании нашего существования, что все выходящие за пределы природы выведения этого существования от неестественного существа --- только фантазия или самообман. Эти доказательства частью прямые, частью косвенные; одни взяты из природы и имеют непосредственное отношение к ее существу; другие показывают те противоречия, которые заключаются в допущении обратного, те нелепые выводы, которые из этого допущения следуют. 

\emph{Наш мир, и отнюдь не только мир политический и социальный, но и мир, живущий духовными интересами, мир ученый, есть мир наизнанку. Торжество нашего образования, нашей культуры состояло большей частью в возможно большем отдалении и уклонении от природы; торжество нашей науки, нашей учености --- в возможно большем отдалении и уклонении от простой и очевидной истины.} Так, всеобщее основное положение нашего мира, вывороченного наизнанку, гласит, что бог обнаруживает себя в природе, между тем как должно бы гласить обратно, что природа, по крайней мере первоначально, предстоит человеку, как божество, что природа производит на человека впечатление, именуемое им богом, впечатление, которое в его сознании отлагается под именем бога, которое он опредмечивает. Так, всеобщее учение нашего вывороченного наизнанку мира гласит, что природа произошла от бога, между тем как должно бы гласить, напротив, что бог произошел от природы, что бог из природы выведен, что он составляет от нее абстрагированное, произведенное понятие; ибо все предикаты, то есть все свойства или определения, все реальности, как говорят философы, то есть все существенные свойства или совершенства, которые собраны в боге, или совокупностью которых он является, все эти божественные предикаты, стало быть, которые --- поскольку они не заимствованы у человека --- почерпнуты из источника природы, опредмечивают и представляют собой, наглядно показывают нам не что иное, как существо природы, или --- короче говоря --- природу. Разница лишь та, что бог есть абстрактное, то есть созданное мыслью, природа же --- конкретное, то есть действительное существо, но их сущность и содержание одно и то же; \emph{бог есть абстрактная природа, то есть природа, отвлеченная от чувственного созерцания, мыслимая, превращенная в объект, в существо рассудка; 
природа в собственном смысле есть чувственная, действительная природа, как ее нам непосредственно обнаруживают и представляют чувства}. Обращаясь же к определению свойств божественного существа, мы найдем, что все они коренятся только в природе, что они имеют смысл и разумное основание только в том случае, если они сводятся к природе. Одна существенная черта бога заключается в том, что он --- существо могущественное и даже самое могущественное, в позднейших представлениях --- всемогущее. Могущество есть даже первое определяющее свойство божества или, вернее, первое божество. Но что такое это могущество, что оно выражает? Не что иное, как мощь явлений природы; потому-то, как на это уже указывалось в первых лекциях, молния и гром в качестве тех явлений, которые производят на человека самое мощное, самое страшное впечатление, представляют собой действие наивысшего, самого могущественного бога или даже тождественны с ним. Даже в <<Ветхом Завете>> гром есть голос бога, а молния во многих местах называется <<лицом божиим>>. Но что такое бог, голос которого есть гром и лицо которого --- молния, как не существо природы, или молнии и грома? Даже у христианских теистов могущество, несмотря на духовность их бога, означает не что другое, как могущество чувственное, могущество природы. Так, например, христианский поэт Триллер в своих <<Поэтических размышлениях>> говорит: 

\begin{quote}
    
Ведь у тебя --- сознайся в том! 

От страха сердце замирает, 

Когда гремит могучий гром 

И в небе молния сверкает. 

Откуда в сердце этот страх? 

Кто поселил его? --- Сознанье, 

Что бог тебя, свое созданье, 

Грозой повергнуть может в прах. 

И нет поэтому сомненья,

Что бог глаголет к нам из туч,

Что гром и молния --- знаменья 

Того, что наш господь могуч. 

\end{quote}


Но и там, где могущество природы не представляется столь явственным чувству христиан, как в данном случае молния и гром представлялись принадлежавшему к духовенству Триллеру, оно, это могущество, все же является основой. Так, христианские теисты, существо которых есть абстракция и именно поэтому отдаление от правды природы, производили причину движения в природе от могущества или всемогущества бога, --- так как они превращали эту природу в мертвую, инертную массу или материю. Бог, --- говорили они, --- насадил, внедрил, сообщил движение материи, которая сама по себе неподвижна, и именно поэтому дивились они чудовищному могуществу бога, --- могуществу, силой которого он приводит в движение эту чудовищную массу или машину. Но не абстрагировано ли это могущество, силою которого бог приводит в движение тела или материю, не выведено ли оно от той силы или мощи, с которой одно тело сообщает движение другому, находящемуся в покое. Дипломатичные теисты отрицали, правда, чтобы бог привел в движение материю толчком, непосредственным прикосновением, он, дескать --- дух, он все это осуществил одной своей волей. Но как бог представляется не простым духом, но в то же время и существом и к тому же существом материальным, чувственным, хотя и скрыто материальным, скрыто чувственным, точно так же он и не создал движение одной своей волей. Воля ничто без могущества, без положительной, материальной возможности. Ведь сами теисты определенно отличают в боге могущество от воли и разума. Но что же такое это отличное от воли и разума могущество, как не могущество природы? 

Представление о могуществе, как о божественном основном свойстве или божестве, получается или развивается в человеке, особенно при сравнении действий природы с действиями человека. Человек не может создавать травы и деревья, производить бурю, делать погоду, не может сверкать молнией и греметь, подобно грому. <<Неподражаемой>> называет поэтому Виргилий стрелу юпитеровой молнии, а Салмонея в греческой мифологии потому и поражает молния Юпитера, что он дерзнул пожелать сверкать и греметь, как Юпитер. Эти действия природы превосходят силы человека, они не в его власти. 

Именно поэтому существо, производящее эти действия и явления, есть для него существо сверхчеловеческое и, как сверхчеловеческое существо, божественное. Но все эти действия и явления выражают не что иное, как могущество природы. Правда, христиане, теисты, приписывают эти действия богу посредственно, то есть сводя их происхождение к богу, как существу, отличному от природы, действующему при посредстве воли, разума, сознания; но это лишь объяснение, а здесь речь идет не о том, является ли дух причиною этих явлений или нет, может ли он или не может ею быть, а только о том, что те явления и действия природы, которые даже христианин, по крайней мере рационалистический, просвещенный христианин, не считает непосредственными действиями бога, а действиями бога только постольку, поскольку дело идет об их первоначальном происхождении, по их же действительному существу и свойствам полагает их действиями природы, --- что эти явления и действия природы являются оригиналом, с которого человек первоначально берет свое определение и понятие сверхчеловеческой божественной мощи и силы. Вот пример. Когда молния убивает человека, то христианин говорит или думает, что это произошло не от случая или не как следствие простого устройства природы, а в результате божественного решения; ибо <<ни один воробей не упадет с крыши без воли божией>>. Бог хотел, чтобы он умер и именно таким образом. Божья воля есть конечная или первая причина смерти, ближайшая же есть молния; молния, согласно древней вере, есть средство, при помощи которого сам бог убил человека, но, согласно же современной вере, она --- посредствующая причина, произведшая смерть по воле бога или, по крайней мере, с его разрешения (соизволения). Но сокрушающая, убивающая, испепеляющая сила есть собственная сила молнии, подобно тому, как сила или действие мышьяка, которым я убиваю человека, не есть следствие моей воли, моей силы, а есть сила или действие, присущие мышьяку. Мы отличаем таким образом с теистической, или христианской, точки зрения силу вещей от силы или, вернее, воли бога; мы не считаем действия и, следовательно, свойства --- ибо мы, ведь, познаем свойства вещей только из их действий --- электричества, магнетизма, воздуха, воды, огня свойствами и действиями бога; мы не говорим: бог горит и греет, но говорим: огонь горит и греет, мы не говорим и не думаем: бог мочит, но мочит вода, не бог гремит и сверкает, но гром гремит и сверкает молния и так далее. Но как раз именно эти отличные от бога как духовного существа, как его мыслит себе христианин, явления, свойства и действия природы и являются теми, от которых человек берет свое представление о божественной, сверхчеловеческой мощи и ради которых он почитает природу как бога до тех пор, пока остается верен своему первоначальному простому пониманию, не раскалывающему природу на бога и мир. 

Употребляя выражение <<сверхчеловеческий>>  я не могу удержаться от того, чтобы не ввернуть одного замечания. Одна из обычнейших ламентаций религиозных и ученых плакальщиков по поводу атеизма состоит в том, что атеизм разрушает или игнорирует существенную потребность человека, а именно потребность его признавать и почитать что-нибудь, стоящее над ним, что именно поэтому он делает человека существом эгоистическим и высокомерным. \emph{Однако атеизм, уничтожая теологическое нечто, стоящее над человеком, не уничтожает тем самым моральной инстанции, над ним стоящей}. Моральное высшее, стоящее над ним, есть идеал, который каждый человек себе должен ставить, чтобы стать чем-то дельным; но этот идеал есть --- и должен быть --- человеческим идеалом и целью. Естественное высшее, стоящее над человеком, есть сама природа, в особенности небесные силы, от которых зависит наше существование, наша Земля; ведь сама Земля есть составная часть их, и то, чем она является, она является только сообразно тому положению, которое она занимает в солнечной системе. Даже религиозное сверхземное и сверхчеловеческое существо обязано своим происхождением всего только чувственному, оптическому бытию над нами неба и небесных тел. Юлиан у Кирилла доказывает божественность светил небесных тем, что каждый воздымает руки к небу, когда молится или клянется, или как-нибудь вообще призывает имя божества. Ведь даже христиане помещают своего <<духовного, вездесущего>> бога на небо; и они помещают его на небо по тем же основаниям, по каким первоначально небо само слыло за бога. Аристон из Хиоса, ученик Зенона, основателя стоицизма, говорил: <<Физическое (природа) над нами и проходит мимо нас, ибо невозможно и бесполезно познать его>>. Но это физическое есть главным образом небесное. Предметы астрономии и метеорологии были теми, которые прежде всего возбудили интерес естествоиспытателей и натурфилософов. Так, Сократ отвергал физику как нечто превосходящее человеческие силы, и вел людей от физики к этике; но под этой физикой он понимал главным образом астрономию и метеорологию; отсюда известное изречение, что он <<философию свел с неба на землю>>  отсюда и то, что он всякое философствование, превосходящее силы и предназначение человека, называл meteorologein (то есть занятием небесными, сверхземными вещами). 

Но как могущество, сверхчеловечность, высшее или верховное, над нами находящееся существо, --- у римлян боги называются superi, --- так и другие предикаты божества, как вечность, бесконечность, --- первоначально определения природы. Так, например, у Гомера бесконечность есть предикат моря и Земли, у философа Анаксимена --- предикат воздуха, в <<Зенд-Авесте>> вечность и бессмертие --- Солнца и звезд. Даже величайший философ древности Аристотель в противоположность бренности и изменчивости земного приписывает неизменность и вечность небу и небесным телам. И даже христианин умозаключает (то есть выводит) из величия и бесконечности мира или природы о величии и бесконечности бога, хотя тотчас же вслед за тем --- из вполне понятного, но не подлежащего здесь нашему рассмотрению основания --- заставляет эти свойства мира исчезнуть перед свойствами бога. Так, например, Шейхцер в своем <<Естествознании Иова>> говорит в согласии с бесчисленными другими христианами: <<Его (бога) бесконечное величие указует не только непостижимая величина мира и мировых тел, но и самая малая пылинка>>. А в своей <<Физике или естествознании>> тот же ученый и благочестивый естествоиспытатель говорит: <<Бесконечная мудрость и мощь творца явствует не только из infinite magnis (бесконечно больших величин), из массы всего мира и тех больших тел, которые обращаются в свободном небе\dots но также и из infinite parvis (бесконечно малых величин), из пылинок и мельчайших животных\dots Каждая пылинка объемлет собой бесконечное число мельчайших миров>>. Понятие бесконечности совпадает с понятием всеохватывающей всеобщности или универсальности. Бог не есть особое и потому конечное, ограниченное той или другой нацией, тем или другим местом существо, но он также и не природа. Солнце, Луна, небо, Земля и море общи всем, говорит один греческий философ, а один римский поэт (Овидий) говорит: природа никому не присвоила ни солнце, ни воздух, ни воду. <<У бога нет лицеприятия>>  но нет ее и у природы. Земля производит свои плоды не только для той или другой избранной личности или нации; Солнце светит над головами не одних христиан, евреев, оно освещает всех людей без различия. Именно благодаря этой бесконечности и всеобщности природы и не могли древние евреи, считавшие себя за избранный богом, то есть за единственно правомочный, народ, верившие, что мир создан только ради них, евреев, не могли понять, почему блага жизни предоставлены не им лишь одним, но также и идолопоклонникам. На вопрос, почему бог не уничтожает служения идолам, еврейские ученые отвечали, что он бы уничтожил идолопоклонников, если бы они не почитали вещи, необходимые миру; но так как они почитают солнце, луну, звезды, воду, огонь, то зачем богу уничтожать мир из-за нескольких глупцов? то есть на самом деле: бог должен допустить существование причин и предметов идолопоклонства, потому что без них не могли бы существовать евреи \hyperlink{10}{(10)}\hypertarget{b10}{}. 

Мы имеем в данном случае интересный пример некоторых существенных черт, характерных для религии. Прежде всего пример противоречия между теорией и практикой, верой и жизнью, противоречия, которое встречается в каждой религии. С их теорией, с их верой в прямом противоречии находилась та естественная общность земли, света, воздуха, которая имелась у евреев с идолопоклонниками; так как они с язычниками не хотели иметь ничего общего и согласно их религии не должны были ничего общего иметь, то они и блага жизни не должны бы были иметь с ними общие. Если бы они были последовательны, то они должны бы были либо язычников, либо себя лишить пользования этими благами, дабы не иметь ничего общего с нечестивыми язычниками. Во-вторых, мы в данном случае имеем пример того, что природа куда либеральнее бога религий, что соответствующая природе точка зрения человека или естественное воззрение гораздо универсальнее, чем точка зрения религиозная, которая отделяет человека от человека, христианина от иудея, иудея от язычника, что, следовательно, единство человеческого рода, любовь, простирающаяся на всех людей, опирается отнюдь не на понятие небесного отца или, как современные философы переводят это выражение, не на понятие духа, но столько же или еще лучше опирается на природу и первоначально только на нее и опиралась. Поэтому всеобщая любовь к человечеству ведет свое происхождение совсем не со времен лишь христианства. Уже языческие философы учили этой любви; но бог языческих философов был не что иное, как мир или природа. 

Христиане, наоборот, имели ту же веру, как и евреи; они так же верили и говорили, что мир создан ради них, христиан, и ради них сохраняется; поэтому они так же мало могли последовательно объяснить существование неверующих и вообще язычников, как и евреи; ибо если мир существует только ради христиан, то зачем и почему существуют другие люди --- не христиане, не верящие в христианского бога? Христианским богом можно объяснить существование только христиан, но отнюдь не язычников и не неверующих людей. Бог, который дает восходить солнцу над праведными и неправедными, над верующими и неверующими, над христианами и язычниками, есть бог, равнодушный к этим религиозным различиям, не желающий о них ничего знать, он в действительности не что иное, как природа \hyperlink{11}{(11)}\hypertarget{b11}{}. Если поэтому в Библии значится: бог дает своему солнцу восходить над добрыми и злыми, то мы в этих словах имеем следы или доказательства религиозного естественного воззрения, или под добрыми и злыми понимаются люди, различающиеся между собой лишь морально, но отнюдь не догматически, ибо догматический библейский бог строго различает между козлищами и овцами, между христианами, с одной стороны, и евреями и язычниками --- с другой, между верующими и неверующими: одним он сулит ад, другим --- небо, одним он обещает вечную жизнь и счастье, другим --- вечное бедствие и смерть. Но именно поэтому нельзя бытие этих обреченных на ничто людей выводить от бога; мы можем это себе объяснить, мы можем вообще уйти от тысячи противоречий, затруднений, осложнений и непоследовательностей, в которые нас запутала религиозная вера, только в том случае, если мы признаем, что первоначальный бог был лишь существом, производным от природы, и если мы поэтому сознательно на место мистического, подлежащего различным толкованиям имени и существа бога поставим имя и существо природы.

\phantomsection
\addcontentsline{toc}{section}{Тринадцатая лекция}
\section*{Тринадцатая лекция}

То, что во вчерашней лекции я говорил о могуществе, о вечности, о сверхчеловечности, о бесконечности и универсальности бога, что они взяты из природы и первоначально выражали лишь ее свойства, то же применимо и к свойствам моральным. Благость бога заимствована от существа и явлений природы, полезных, благих, благодетельных для человека, внушающих ему чувство и сознание того, что жизнь, существование есть благо, счастье. Благость бога есть лишь облагороженная фантазией, поэзией чувства, олицетворенная, обретенная самостоятельность, как особое свойство или сущность, в активной форме выраженная и понятая полезность природы и способность ее быть использованной. Но так как природа есть в то же время и причина влияний, человеку враждебных, для него вредных, то он эту причину превращает в самостоятельное существо и обожествляет в виде злого бога. Эта противоположность встречается почти во всех религиях; но в этом отношении самая знаменитая религия --- персидская, которая ставит во главу своей веры двух враждебных друг другу богов: Ормузда, бога, или причину, всех существ, благодетельных для человека --- полезных животных, отрадных явлений, подобных свету, дню, теплоте, --- и Аримана, бога, или причину, тьмы, пагубной жары, вредных животных. 

Христианская религия, представления которой в области веры почти целиком заимствованы из персидского, вообще восточного миросозерцания, также имеет, собственно говоря, двух богов, из коих, однако, только один преимущественно или исключительно бог, тогда как другой называется сатаной или дьяволом. И даже в тех случаях, когда злые, вредные действия природы не производятся от самостоятельной, личной причины, дьявола, они производятся во всяком случае от божьего гнева. Но бог в гневе, или гневный бог, --- не что иное, как злой бог. Мы здесь имеем опять пример того, что между политеизмом и монотеизмом нет существенной разницы. Политеист верит в добрых и злых богов, монотеист же воплощает злых богов в гневе бога, а добрых --- в благости бога и верит в одного бога, но этот один, одновременно и добрый и злой, гневный бог --- бог противоположных свойств. Но гнев бога есть не что иное, как правосудие бога, представленное, олицетворенное в виде чувства, страсти. Ведь гнев и в человеке первоначально и сам по себе есть не что иное, как страстное чувство справедливости или мщения. Человек приходит в гнев тогда, когда ему --- в действительности или только в его представлении --- причинена обида, несправедливость. Гнев есть возмущение человека против деспотических посягательств, которые другое существо позволяет себе по отношению к нему. Но как благость бога заимствована лишь от благих действий природы, так и справедливость заимствована первоначально от злых, вредных, пагубных действий природы. Представление о правосудии создается таким образом посредством рефлексии. Человек --- эгоист; он по отношению к себе бесконечно добр и верит тому, что все должно ему служить только на пользу, что не должно и не может быть зла; но он встречает противоречия этому своему эгоизму и вере; он верит поэтому, что зло постигнет его только в том случае, если он погрешил против существа или существ, от которых он производит все доброе и благодетельное, и если он тем самым привел их в гневное состояние. Он объясняет себе поэтому зло природы как наказание, которое бог возложил на человека ввиду содеянного человеком по отношению к ному проступка или несправедливости, Отсюда и вера христиан в то, что природа была некогда раем, где не существовало ничего враждебного и вредного человеку, но что этот рай погиб в результате грехопадения и вызванного им гнева божьего. Но это объяснение теологически извращенное. Первоначально гнев или суд бога в отличие от его благости заимствован и выведен из пагубных и вредных явлений природы. Не потому этот человек убит молнией, что бог наказывает, что он справедлив, гневен, рассержен, а, наоборот, из того, что он убит молнией мы заключаем, что причиной этого смертного случая является разгневанное, наказующее, злое существо. Таков первоначальный ход человеческой мысли \hyperlink{12}{(12)}\hypertarget{b12}{}. Но как благость и справедливость бога заимствованы и выведены из благотворных и вредных явлений природы, точно так же заимствована и выведена и мудрость только из природы и, в частности, из того порядка, в котором явления природы следуют одно за другим, из связи естественных причин и следствий. 

Но как и указанные уже физические или метафизические и моральные свойства бога, так и прочие, более неопределенные или отрицательные заимствованы из природы. Бог невидим; но и воздух невидим. Именно поэтому почти у всех сколько-нибудь умственно развитых народов воздух, дыхание, дуновение тождественны с духом. И сам бог в свою очередь не отличается от духа, то есть от воздуха, как того существа, которое в примитивном чувственном представлении одно обусловливает собой жизнь людей или, вернее, поддерживает ее и является ее причиной. Если поэтому значится: ты не должен делать себе изображения бога, то отсюда еще не следует, что под богом понимается дух в нашем смысле, то есть мыслящее, имеющее желания, познающее существо. Кто может себе сделать изображение из воздуха? Не удивляйся, возражает Минупий Феликс на упрек, делаемый язычниками в том, что бог христиан не может быть ни показан, ни видим, --- не удивляйся, если ты бога не видишь, ветер и воздух также невидимы, хотя они все и толкают в разные стороны, двигают, потрясают. Бог неуловим, неосязаем. Но разве уловим, осязаем воздух, хотя он может быть взвешен физиками? Разве уловим, осязаем свет? Поддается ли свет, воздух пластическому изображению, то есть в виде индивидуальной, телесной фигуры? Как ошибочно поэтому делать из того, что народы не имеют изображений, статуй, а следовательно, и храмов своего бога, умозаключение, что они почитают существо духовное в нашем смысле этого слова? Они почитают природу в целом или частично, еще не очеловечив ее, еще не заключив ее по крайней мере в определенную человеческую фигуру и форму; вот причина, почему они не имеют человеческих изображений и статуй для предметов своего религиозного почитания. 

Я не могу себе представить бога в ограниченных формах, изображениях, понятиях; но могу ли я мир, вселенную представить себе в них? Кто может создать изображение природы, по крайней мере отвечающее ее существу? Каждое изображение ведь взято лишь с части мира, как же я могу представить соответствующим образом целое в части? Бог не есть существо, ограниченное во времени и в пространстве; но ограничен ли мир? Приурочен ли мир к данному месту, к данному времени, не находится ли он повсеместно и во все времена? Мир ли находится во времени или, не вернее ли, время в мире? Не есть ли время только форма мира, способ, которым следуют друг за другом отдельные существа и явления мира? Как, следовательно, могу я приписывать миру начало во времени? Время ли является предпосылкой для мира или, не вернее ли, мир для времени? Мир есть вода, время есть движение воды; но разве вода не предшествует своему движению, согласно природе вещей? Разве движение воды не предполагает уже существование воды? Разве движение воды не есть следствие ее своеобразной природы и свойств? Не так же ли глупо, стало быть, мыслить себе мир происшедшим во времени, как если бы я мыслил себе существо какой-либо вещи возникшим как одно из следствий этого существа? Не так же ли бессмысленно мыслить себе какую-либо точку во времени как начало мира, как представлять себе падение воды как ее происхождение? Не видим ли мы, однако, из вышесказанного, что существо и свойства мира и существо и свойства бога одни и те же, что бог не отличается от мира, что бог есть лишь понятие, абстрагированное от мира, что бог есть лишь мир в мыслях, мир же --- лишь бог в действительности или действительный бог, что бесконечность бога взята лишь из бесконечности мира, вечность бога --- лишь из вечности мира, могущество и великолепие бога --- лишь из могущества и великолепия природы, что они отсюда произошли, что они отсюда выведены? 

Различие между богом и миром есть лишь различие между духом и чувством, мышлением и представлением; мир как предмет чувств, а именно телесных чувств, подобных грубому чувству осязания, есть так называемый мир, тогда как мир как предмет мысли, мышления, выводящего общее из чувств, есть бог. Но подобно тому, как то всеобщее, которое разум выводит из чувственных вещей, есть, хотя и не непосредственно, все же посредственно чувственное, чувственное по существу, соответственно предмету, хотя и не соответственно форме (ибо понятие о человеке есть нечто чувственное через посредство человека, понятие о дереве --- через посредство тех деревьев, которые мне указываются чувствами), так и существо бога, хотя оно и есть только мыслимое, выведенное существо мира, есть посредственно все-таки чувственное существо. Бог, правда, не есть чувственное существо, как какое-либо видимо или ощутимо ограниченное тело, как камень, растение, животное, но если бы только из-за этого явилось желание отказать существу бога в чувственности, то в таком случае следовало бы отказать в ней и воздуху, и свету. Даже тогда, когда человек думает, что он в своем представлении о боге возвышается над природой, когда он, по крайней мере в своем воображении, мыслит себе, подобно христианам и в особенности так называемым рационалистическим христианам, бога как существо, лишенное всех чувственных свойств, нечувственное, бестелесное; даже и в этом случае, по крайней мере, основой духовного бога является представление чувственного существа. Кто вообще может мыслить себе нечто в виде существа, не мысля себе его в то же время в виде существа чувственного, хотя бы он устранил от него все ограничения и свойства осязаемо чувственного существа? Различие между существом бога и существом чувственных предметов есть лишь различие между родом и видами или индивидуумами. 

Бог есть так же мало то или другое существо, как цвет вообще есть тот или иной цвет, как человек вообще есть тот или другой человек: ибо в родовом понятии человека я отвлекаюсь от различий человеческих видов и отдельных людей, в родовом понятии цвета отвлекаюсь от отдельных, различных цветов. Так и в существе бога я отвлекаюсь от различий и свойств многих различных чувственных существ, я мыслю его себе только, как существо вообще; но именно потому, что понятие божественного существа заимствовано лишь от чувственных существ, которые имеются в мире, что оно есть лишь родовое понятие, мы постоянно подсовываем этому общему понятию образы чувственных существ, мы представляем себе существо бога то как существо природы в целом, то как существо света, или огня, или человека, в особенности старого почтенного человека, подобно тому, как перед нами при каждом родовом понятии витает образ тех индивидуумов, от которых мы его абстрагировали. И так же, как с существом бога, обстоит дело и с его существованием, как это само собой разумеется, ибо существование нельзя ведь отделить от существа. Даже тогда, когда бог представляется в виде существа, которое, так как оно само есть дух, то и существует только для человеческого духа, и делается для человека предметом только в том случае, если он возвышается над чувствами и отвлекает от чувственных существ свой дух, даже и в этом случае в основе существования бога лежит истина чувственного существования, истина природы. Бог должен существовать не только в мышлении, в духе, но также и вне духа, независимо от нашего мышления, он должен быть существом, отличным от нашего духа, от наших мыслей и представлений. То, что он существо от нас независимое, вне нас существующее, объективное, подчеркивается со всей силой. Но не признается ли тем самым даже в боге, где якобы должно отвлечься от всего чувственного, истина чувственного бытия, не делается ли признания, что нет никакого бытия вне чувственного бытия. Разве мы имеем другой признак, другой критерий существования вне нас, существования, независимого от мышления, кроме чувственности? Не есть ли существование без чувственности голая мысль, призрак существования? Существование бога или существование в том виде, как оно приписывается богу, отличается от существования чувственных существ вне нас лишь в той мере, в какой существо бога отличается от чувственных существ согласно только что данному объяснению. Существование, которое признается за богом, есть существование отвлеченное, родовое понятие существования, из которого удалены все особые и индивидуальные свойства или признаки. Это существование, конечно, духовно, абстрактно, как всякое общее понятие, являющееся чем-то абстрактным, чем-то духовным; тем не менее, оно не что иное, как чувственное существование, мыслимое только как таковое. 

В этом мы имеем разрешение тех затруднений, которые понятие существования представляло для философов и теологов, как это показывают так называемые <<доказательства>> бытия божьего, разрешение тех противоречий, которые встречаются в объяснениях существования бога и в представлениях о нем; теперь мы понимаем, почему богу приписывают духовное существование, но при этом представляют себе в то же время это духовное существование, как чувственное, даже местное, как существование на небе; короче говоря, противоречие, спор между духом и чувственностью в представлении о существовании бога, двусмысленность, мистическая его неопределенность объясняются просто тем, что они абстрагированы, взяты от чувственного существования действительных предметов и существ, но что именно поэтому в это абстрактное существование по необходимости подставляется образ существования чувственного, подобно тому, как постоянно образ чувственного существа подставляется в существо бога. Но если, как мы до сих пор видели, все свойства, существенные или действительные качества, которые вместе составляют существо бога, заимствованы у природы, если существо, существование, свойства природы являются оригиналом, соответственно которому человек составил себе образ бога, или, беря глубже, если бог и мир, или природа, отличаются друг от друга лишь так, как родовое понятие отличается от индивидуумов, так что природа, как предмет чувственного представления, есть природа в собственном смысле слова, а богом является природа, которая в отличие от чувственности и отвлеченная от своей материальности и телесности составляет предмет духа, мышления, --- если все это так, то ясно само собой, и тем самым уже также доказано, что природа произошла не от бога, что действительное существо произошло не от абстрактного, что телесное, материальное существо --- не от духовного. Выводить природу из бога, --- все равно что желать вывести оригинал из изображения, из копии, вещь из мысли об этой вещи. 

Как это ни ошибочно, но на этой ошибке покоится тайна теологии. Предметы в теологии мыслятся и являются желательными, не потому что она есть, а наоборот они есть, потому что они мыслятся и являются желательными. Мир существует, потому что бог его мыслил и желал, потому что до сих пор еще бог его мыслит и желает. Идея, мысль абстрагирована не от предмета ее, а наоборот --- мысль есть творящее, есть причина того предмета, о котором она мыслит. Но именно это учение --- суть христианской теологии и философии --- есть ложь, в которой строй природы выворочен наизнанку. Но как приходит человек к этому ошибочному взгляду? Я уже говорил, касаясь первой причины, что человек, и субъективно с полным правом, --- по крайней мере до тех пор с полным правом, пока он не разобрался в своем собственном существе, предпосылает род, то есть в данном случае родовое понятие, видам и индивидуумам, выражаясь философским языком: абстрактное --- конкретному. Этим объясняются и разрешаются все затруднения и противоречия, которые имеют место при сотворении, при объяснении мира богом. 

Человек при помощи своей способности к абстракции извлекает из природы, из действительности то, что подобно, равно в предметах, обще им, отделяет это от предметов, друг другу подобных или имеющих одинаковую сущность, и превращает, в отличие от них, в качестве самостоятельного существа в их сущность. Так, например, человек выводит из чувственных предметов пространство и время, как общие понятия или формы, в которых все эти предметы друг с другом сходятся, ибо все они протяженны и изменчивы, все существуют один вне другого и один после другого. Так, каждая точка земли находится вне другой точки и каждая точка в движении земли чередуется с другой; там, где сейчас находится данная точка, там в следующий момент окажется другая. Но хотя человек абстрагировал пространство и время от пространственных и временных вещей, однако их же он предпосылает этим последним как первые причины и условия их существования. Он мыслит себе поэтому мир, то есть совокупность всех действительных вещей, вещество, содержание мира, возникшим в пространстве и во времени. Даже у Гегеля материя возникает не только в пространстве и времени, но и из пространства и времени. Именно потому, что человек предпосылает время и пространство действительным вещам, приписывает самостоятельное существование общим понятиям, произведенным от отдельных предметов, в философии --- в виде общих сущностей, в религии политеистической --- в виде богов, в монотеистической --- в виде отдельных свойств бога, именно поэтому он сделал также и пространство и время богом или отождествил их с богом. Даже знаменитый христианский математик и астроном Ньютон называет еще пространство неизмеримостью бога, его чувствилищем, то есть органом, при помощи которого бог является присущим всем вещам, при помощи которого он воспринимает все вещи. Ньютон рассматривает также пространство и время, <<как следствия бытия божьего, ибо бесконечное существо находится повсюду, а стало быть, существует это неизмеримое пространство; вечное существо существует от вечности, а стало быть, и в самом деле существует вечная длительность>>. Непонятно также, почему бы время, отделенное от временных вещей, не могло быть отождествлено с богом; ибо абстрактное время, в котором нет различия между теперь и тогда (так как ведь отсутствует различающее содержание), не отличимо от мертвой, неподвижной вечности. И сама вечность не что иное, как родовое понятие времени, абстрактное время, время, взятое вне зависимости от временных различий. Неудивительно поэтому, что религия сделала время одним из свойств бога или самостоятельным богом. Так, индийский бог Кришна в Бхагаватгите сделал время, разумеется среди бесчисленных прочих вещей, своим предикатом, почетным титулом, говоря: я есмь время, которое все сохраняет и все разрушает \hyperlink{13}{(13)}\hypertarget{b13}{}. Так и у греков, и у римлян время обожествлено под именем Хроноса и Сатурна. В персидской же религии Заруаноакарана, то есть несозданное время, стоит во главе, как первое, высшее существо. Точно так же и у вавилонян, и у финикийцев бог времени или, как его также называли, владыка времени, царь вечности, был высшим богом. 

Мы видим на этом примере, как человек соразмерно или в согласии с природой своей абстрагирующей деятельности создает общие понятия, но в противоречии с природой действительных вещей предпосылает чувственным вещам общие понятия, представления, или созерцания пространства и времени, как их называет Кант, как условия или, вернее, первопричины и элементы их существования, --- того не соображая, что в действительности происходит как раз обратное, что не вещи предполагают существование пространства и времени, а, наоборот, пространство и время предполагают наличность вещей, ибо пространство, или протяженность, предполагает наличность чего-то, что протяженно, и время --- движение: ведь время --- лишь понятие, производное от движения, --- предполагает наличность чего-то, что движется. Все пространственно и временно; все протяженно и движется; пусть так; но протяженность и движение различаются в той мере, в какой различаются протяженные и движущиеся вещи. Все планеты обращаются вокруг Солнца; но каждая имеет свое собственное движение, одна обращается в более короткое время, другая --- в более долгое: чем ближе к Солнцу, тем быстрее, чем дальше от него, тем медленнее. Все животные движутся, хотя не все передвигаются с места на место; но как бесконечно разнообразно это движение! И каждый вид движения соответствует строению, образу жизни, короче говоря, индивидуальной сущности движения. Как же я стану объяснять и выводить это многообразие из времени и пространства, из одной лишь протяженности и движения? Протяженность и движение ведь зависят от чего-то, от тела, от существа, которое протяженно и движется. Поэтому то, что для человека или, по крайней мере, для его абстрагирующей деятельности является первым, для природы или в природе есть последнее; но так как человек делает субъективное объективным, то есть то, что для него есть первое, делает первым в себе, или по природе, то он также пространство и время делает первыми основными сущностями природы, превращает общее, то есть абстрактное, в основное существо действительности; следовательно, и существо, имеющее общие понятия, мыслящее, духовное существо он превращает в первое существо, в существо, которое не только по рангу, но и по времени предшествует всем остальным существам и которое является основой и причиной всех существ. 

Вопрос о том, сотворил ли бог мир, вопрос вообще об отношении бога к миру, есть вопрос об отношении духовного к чувственному, общего или абстрактного к действительному, рода к индивидуумам; поэтому один вопрос не может быть решен без другого; ибо ведь бог --- не что иное, как совокупность всех родовых понятий. Я, правда, только что уже пояснил этот вопрос, беря понятие пространства и времени, но вопрос этот нуждается еще в дальнейшем обсуждении. Я замечу, однако, что этот вопрос принадлежит к числу важнейших и в то же время труднейших вопросов человеческого познания и философии, что, явствует уже из того, что вся история философии вращается, в сущности говоря, вокруг этого вопроса, что спор стоиков и эпикурейцев, платоников и аристотеликов, скептиков и догматиков в древней философии, номиналистов и реалистов в средние века, идеалистов и реалистов, или эмпириков, в новейшее время сводится всецело к этому вопросу. Но это один из труднейших вопросов не только потому, что философы, а именно новейшие, внесли в эту материю бесконечную путаницу самым произвольным употреблением слов, но также и потому, что природа языка, природа самого мышления, которое ведь никак неотделимо от языка, держит нас в плену и путает, ибо каждое слово выражает нечто общее, и многим поэтому представляется, что уже язык, который не дает выразить ничего единичного, служит доказательством ничтожности единичного и чувственного. Наконец, на этот вопрос и его решение существенное влияние оказало различие людей по духу, по их занятиям, их наклонностям, даже по их темпераменту. Например, люди, отдающие больше времени практике жизни, чем кабинетным занятиям, предпочитающие природу библиотекам, люди, чьи занятия и наклонности влекут их к наблюдению, к созерцанию действительных существ, будут этот вопрос всегда решать в духе номиналистов, признающих за общим лишь субъективное существование, существование в языке, в представлениях человека; наоборот, люди противоположных занятий и качеств будут решать его в противоположном смысле, в духе реалистов, признающих за общим самодовлеющее существование, существование, независимое от мышления и языка человека. 

\phantomsection
\addcontentsline{toc}{section}{Четырнадцатая лекция}
\section*{Четырнадцатая лекция}

В конце вчерашней лекции я говорил о том, что отношение бога к миру сводится к отношению родового понятия к индивидууму, что вопрос, есть ли бог, является лишь вопросом о том, имеет ли общее понятие самостоятельное существование. Но это не только один из самых трудных вопросов, но также один из важнейших, ибо лишь от него зависит бытие или небытие бога. У многих их вера в бога зависит только от этого вопроса; существование их бога опирается только на существование родовых или общих понятий. Если нет бога, говорят они, то никакое общее понятие не представляет истины, то нет мудрости, нет добродетели, нет справедливости, нет закона, нет общественности; тогда все становится чистым произволом, все возвращается в хаос, даже в ничто. На это следует тотчас же заметить, что если и нет мудрости, справедливости, добродетели в теологическом смысле, то отсюда еще отнюдь не следует, что таковых не имеется в человеческом и разумном смысле. Чтобы признать значение за общими понятиями, для этого нет необходимости их обожествлять, превращать в существа, отличные от индивидуумов или особей. Как мне не нужно для того, чтобы гнушаться порока, превращать его в самостоятельное существо в виде дьявола, по примеру древних христианских теологов, имевших для каждого порока своего особого дьявола (так, например, для пьянства --- дьявола пьянства, для обжорства-обжорного дьявола, для зависти-дьявола зависти, для скупости --- дьявола скупости, для страсти к игре --- игорного дьявола, одно время даже для новомодного покроя брюк --- особого брючного дьявола), --- так же точно не нужно мне для того, чтобы любить их, представлять себе добродетель, мудрость, справедливость в виде богов или что то же --- свойств бога. 

Если я ставлю себе какую-либо цель, если, например, я возлагаю на себя осуществление добродетели постоянства или выдержки, то нужно ли мне для этого, чтобы не выпускать ее из глаз, сооружать ей алтари и храмы, подобно тому как это делали римляне, превращавшие добродетель в богиню и даже делавшие богинями отдельные добродетели? Должна ли она --- добродетель --- быть вообще самостоятельным существом, чтобы проявлять надо мной свою силу, чтобы иметь власть надо мной? Не имеет ли она ценность и в том случае, когда она является свойством человека? Ведь я сам хочу быть стойким; я больше не хочу подчиняться смене тех впечатлений, которым меня подвергает моя мягкость и чувственность, я сам себе противен в качестве мягкого, чувствительного, переменчивого, капризного человека; стойкий человек и является для меня поэтому целью. Пока я еще не стоек, я различаю, разумеется, стойкость от самого себя, я ставлю ее над собой как идеал, олицетворяю ее; быть может, даже обращаюсь к ней в одиноком разговоре с собой, как будто бы она была существом самим по себе, следовательно, отношусь к ней так же, как христианин к своему богу, как римлянин к своей богине добродетели; но я знаю, что я ее олицетворяю, и, тем не менее, она не теряет для меня ценности, ибо я ведь лично в ней заинтересован, я имею в самом себе, в своем эгоизме, в своем стремлении к счастью, в своем чувстве чести, которому противоречит мягкость, открытая всем впечатлениям и переменам, достаточно оснований, чтобы быть стойким. И то же самое относится и ко всем другим добродетелям или силам человека, как то --- разум, воля, мудрость, ценность которых и реальное значение поэтому для меня не теряется, вообще не уничтожается оттого, что я рассматриваю их лишь как свойства человека и знаю их в качестве таковых, не обожествляю, не превращаю в самостоятельные существа. То же, что относится к человеческим добродетелям и силам, относится и ко всем общим или родовым понятиям; они существуют не вне вещей или существ, не отдельно, не независимо от индивидуумов, от которых мы их произвели. Субъектом, то есть существом, имеющим бытие, является всегда лишь индивидуум, род же --- лишь предикат, лишь свойство. Но именно предикат, свойство индивидуума, отделяет нечувственное мышление, абстракцию от индивидуума, делает эту абстракцию предметом самим по себе, рассматривает ее в этом отвлечении как сущность индивидуумов, определяет различия индивидуумов между собой лишь как индивидуальные, то есть, в данном случае, случайные, безразличные, несущественные различия, так что для мышления, для духа все индивидуумы сливаются, собственно говоря, в один индивидуум или в одно понятие, и мышление приписывает себе всю сущность, на долю же чувственного воззрения, которое раскрывает нам индивидуумов как индивидуумов, то есть в их множественности, разнообразии, индивидуальности, оставляет лишь оболочку, так что то, что в действительности является субъектом, существом, мышление делает предикатом, свойством, простою модой или манерой родового понятия, и, наоборот, то, что в действительности является лишь свойством, предикатом, оно делает существом. 

Кроме приведенных примеров возьмем для уяснения предмета еще один, и притом чувственный, пример. Каждый человек имеет голову, разумеется, человеческую голову, то есть голову с человеческими свойствами; потому что ведь и животные имеют головы, хотя голова не входит в состав понятия, характеризующего животных вообще, ибо есть животные, еще не имеющие развитой головы, головы в собственном смысле слова, и даже у высших животных голова служит лишь низшим потребностям, не имеет самостоятельного значения и достоинства; поэтому, собственно, голова отступает на задний план перед пастью. Голова есть, следовательно, отличительный признак для всех людей, общая, существенная черта или предикат человека; существо, которое выходит из чрева матери без ног и без рук, есть все же человек, но существо без головы не человек. Но следует ли из этого, что у всех людей лишь одна голова? А ведь единство головы есть необходимое следствие единства рода, который человек превращает в самостоятельное существо в своем абстрактном, то есть нечувственном, мышлении. Но разве чувство мне не говорит, что каждый человек имеет свою голову, что есть столько же голов, сколько и людей, а стало быть, не существует генеральной или общей головы, а есть только индивидуальные головы? --- что голова, голова вообще как родовое понятие, голова, из которой я устранил все индивидуальные особенности и отличительные признаки, существует лишь в моей голове, вне же моей головы имеются только головы? Но что существенно для этой, моей головы? То ли, что она голова вообще или что она есть данная, определенная голова? Что она есть данная голова, потому что, кто возьмет у меня мою голову, тот вообще не оставит мне больше головы. И не голова вообще, а только действительная, индивидуальная голова действует, творит, мыслит. Слово: индивидуальный, разумеется, двусмысленно, ибо мы понимаем под ним также и безразличное, случайное, незначительное своеобразие, отличающее часто одного человека от другого. Поэтому необходимо сначала, чтобы уяснить себе значение индивидуальности, противопоставить человека или, чтобы остаться при нашем примере, голову человека голове животного, взять индивидуальность человеческой головы в отличие от животной. Но и далее, при сравнении одной человеческой головы с другой, хотя и имеются индивидуальные различия в том смысле, при котором индивидуальное обозначает лишь безразлично своеобразное, существенным является то, что каждый человек имеет свою собственную, данную, определенную, чувственную, видимую, индивидуальную голову. Голова как родовое понятие, как общий атрибут, или отличительный признак человека, не имеет, стало быть, другого значения, другого смысла, кроме того, что все люди сходятся на том, что каждый имеет голову. Если я, однако, несмотря на это общее согласие, отрицаю, чтобы люди имели одну голову, между тем как единство головы является необходимым следствием того представления, что единство рода в отличие от индивидуумов есть нечто существующее, самостоятельное, а в особенности того представления, что все люди имеют один разум; если я утверждаю, что имеется столько же голов, сколько и индивидуумов; если я голову отождествляю с индивидуумом, их друг от друга не отличаю или не отделяю, --- следует ли отсюда, что я отрицаю значение и существование головы, что я делаю человека безголовым существом? Наоборот, вместо одной головы я получаю много голов, и если четыре глаза видят больше, чем два, то и много голов сделают бесконечно больше, чем одна; поэтому вместо того, чтобы что-либо проиграть, я только выиграл. Если поэтому я уничтожаю различие между родом и индивидуумом, если я оставляю его существовать только в мышлении, в различении, в абстрагировании, то я не отрицаю в силу этого значение родового понятия; я утверждаю только, что род существует лишь как индивидуум или предикат индивидуума \hyperlink{14}{(14)}\hypertarget{b14}{}. Я не отрицаю, --- сошлюсь опять на прежние примеры, --- мудрость, добро, красоту; я отрицаю лишь, что они в качестве этих родовых понятий являются существами, в виде ли богов, или свойств бога, или в виде платоновских идей, или гегелевских самополагающихся понятий; я утверждаю только, что они существуют лишь в мудрых, добрых, прекрасных индивидуумах и, следовательно, как уже сказано, являются лишь свойствами индивидуальных существ, что они не являются сами по себе существами, а только атрибутами или определениями индивидуальности, что эти общие понятия предполагают существование индивидуальности, а не наоборот \hyperlink{15}{(15)}\hypertarget{b15}{}. 

Теизм как раз основывается на том, что он родовые понятия, по крайней мере содержание их, которое он называет богом, предпосылает в качестве источника их происхождения действительным вещам, что он не общее производит от индивидуумов, а, наоборот, индивидуумов от общего. Но общее, как таковое, родовое понятие существует в мышлении и для мышления; поэтому и получается, что человек приходит к мысли и вере в то, что мир вышел из идей, из мыслей духовного существа. Если стоять на точке зрения мышления, отвлекающегося от чувств, то представится как нельзя более естественным этот ход мысли; ибо духу, абстрагирующему от чувств, ближе абстрактное, духовное, только мыслимое, чем чувственное; оно для него и более раннее и высшее, чем чувство, поэтому для него совершенно естественно выводить чувственное из духовного, действительное из мыслимого. Мы встречаем этот ход мысли даже и у современных спекулятивных философов. Они еще и поныне творят из своей головы мир, как когда-то это делал христианский бог. 

Вера в то или иное представление о том, что мир, природа вообще создана мыслящим или духовным существом, имеет еще и другое основание, кроме только что приведенного, которое мы можем называть философским или спекулятивным в отличие от того популярного, к которому мы сейчас переходим. Вот оно. Человек создает вещи, существующие вне его, им предшествовала в человеке мысль о них, набросок, понятие, и в основе их лежит намерение, цель. Если человек строит дом, то он имеет в голове идею, образ, согласно которому он строит, который он осуществляет, превращает или переводит в камень и дерево, находящиеся вне его, и при этом он имеет также цель: он строит себе дом для жилья или беседку, или фабричное здание; короче говоря, он строит себе дом для той или другой цели. И эта цель определяет идею дома, которую я набрасываю в моей голове: ибо дом для данной цели я мыслю себе иным, чем дом для другой цели. Вообще человек есть существо, действующее согласно известным целям; он ничего не делает без цели. Но цель есть, вообще говоря, не что иное, как волевое представление --- представление, которое не должно остаться представлением или мыслью и которое я поэтому реализую, то есть осуществляю, при посредстве инструментов своего тела. Короче говоря, человек создает если не из своего духа, то, во всяком случае, при помощи своего духа, если не из своих мыслей, то, во всяком случае, при помощи своих мыслей и согласно им, вещи, которые именно поэтому даже внешним образом имеют на себе печать намеренности, планомерности и целесообразности. Но человек мыслит обо всем по себе; он переносит поэтому представление о своих собственных созданиях на создания или действия природы; он рассматривает мир как жилой дом, мастерскую, часы, короче говоря, как продукт человеческого мастерства. Так как он не различает продукты природы и продукты мастерства, в лучшем случае различая их только как разновидности, то он предполагает как причину их человеческое, ставящее себе цели, мыслящее существо. Но так как продукты и действия природы выходят в то же время далеко за пределы человеческих сил, бесконечно превышают их, то он мыслит себе человеческую по своему существу причину в то же время и как сверхчеловеческое существо, как существо, которое имеет те же свойства, что и люди: разум, волю, силу, для осуществления своих мыслей, но в бесконечно больших размерах, бесконечно превышающих масштаб человеческих сил и способностей, и называет он это существо богом. 

Доказательство бытия божьего, опирающееся на эти способы наблюдения или познания природы, называется физико-теологическим или телеологическим доказательством, то есть таким, которое почерпнуто из целесообразности природы, ибо это доказательство ссылается главным образом на так называемые цели природы. Цели предполагают наличность разума, намерения, сознания; а так как, говорится в этом доказательстве, природа, вселенная, материя слепа, действует без разума, без сознания, то она предполагает, что имеется первоначально духовное существо, ее создавшее или во всяком случае организовавшее и оформившее согласно своим целям. Это доказательство приводилось уже древними верующими философами, платониками и стоиками, в христианские же времена повторялось до того, что набило оскомину. Это самое популярное и с известной точки зрения наиболее ясное и убедительное доказательство --- доказательство простого, то есть необразованного, о природе ничего не знающего человеческого разума; оно поэтому есть единственное, по крайней мере единственное теоретическое, основание и опора теизма в народе. Мы должны, однако, против этого вывода указать прежде всего на то, что хотя представлению о целях в природе и соответствует нечто конкретное или действительное, все же выражение или понятие цели в применении к природе неподходяще. То именно, что человек называет целесообразностью природы и как таковую постигает, есть в действительности не что иное, как единство мира, гармония причин и следствий, вообще та взаимная связь, в которой все в природе существует и действует. Как слова только тогда имеют смысл и разумное содержание, когда они стоят друг с другом в необходимом соотношении, точно так же только необходимое соотношение, в котором друг к Другу стоят существа или явления природы, производит на человека впечатление разумности и намеренности. Стоики в своих доказательствах разумной причины, мира против представления, --- разумеется, неразумного, --- о том, что мир обязан своим существованием случаю, случайному скоплению атомов, то есть бесконечно малых твердых и неделимых тел, пользовались сравнением, говоря, что это было бы то же самое, как если бы из случайного сочетания букв хотели объяснить происхождение духовного труда, например, исторических книг Энния. Однако, хотя мир и не обязан своим существованием случаю, но нам по этой причине не приходится еще мыслить для него человеческого или человекоподобного автора. Чувственные вещи (не буквы или литеры, которые еще должны сначала быть набраны существующим вне их наборщиком, ибо они не находятся друг к другу ни в каких необходимых отношениях) --- вещи в природе притягиваются друг к другу, друг в друге нуждаются и друг друга желают, ибо одна не может быть без другой, следовательно, они вступают во взаимоотношения друг к другу по собственному почину, соединяются друг с другом вследствие действия своих собственных сил, как, например, кислород с водородом, образуя воду, или с азотом, образуя воздух, и тем кладут основание тому достойному удивления взаимоотношению, которое человек, еще не заглядывавший в существо природы и судящий обо всем по себе, объясняет себе как творение существа, действующего и созидающего по известным планам и сообразно известным целям. Что люди больше всего полагали возможным рассматривать как доказательство существования разумного и духовного творца мира, так это не только так называемую внутреннюю органическую целесообразность, согласно которой органы тела отвечают своим функциям или отправлениям, но также и главным образом и так называемую внешнюю целесообразность, в силу которой неорганическая природа обладает такими свойствами или, как выражаются теисты, так устроена, что животные и люди могут в ней жить и притом самым приятным, самым комфортабельным образом. 

Если бы Земля находилась ближе к Солнцу или дальше от него и температура поднялась бы до точки кипения воды или упала бы ниже точки замерзания, то все бы высохло от жары или оцепенело от холода. Как мудро поэтому господь бог рассчитал, на каком расстоянии земля должна находиться от солнца, чтобы животные и люди могли на ней жить! И как милостиво позаботился он всюду о нуждах живущего! Даже в самых печальных, неплодородных, холодных местностях есть все же еще, по крайней мере, мхи, лишаи, кустарники и известные животные, служащие пищей человеку. И как явственно, как очевидно проявляет себя благость и мудрость бога в богатстве более теплых стран! Как позаботился там господь бог об удовлетворении вкусовых ощущений человека! Какие лакомства произрастают там на кустах и деревьях! Там имеется сахарный тростник, там рис, которым, как говорят, в одном Китае питается до ста млн. людей, там инбирь, ананасы, кофейное дерево, чайный и перечный кусты, шоколадное дерево, откуда к нам приходит шоколад, мускатный орешник, гвоздичное дерево, ванильный куст, кокосовая пальма, кору которой, как говорит один современный набожный популярный ботаник, <<благое провидение снабдило всюду выступами, имеющими форму полумесяца, с помощью которых человеку облегчается возможность взлезть на высокое дерево, чтобы раздобыть драгоценные плоды и утоляющий напиток, ими доставляемый>>. Мы, однако, заметим по этому поводу следующее, и прежде всего относительно первого пункта. Органическая жизнь не случайно появилась на земле, среди неорганической природы; нет, органическая и неорганическая жизнь тесно связаны друг с другом. Что же я такое, если отправляться от органической жизни, без внешнего мира? Так же, как легкое принадлежит мне, мне принадлежит и воздух; как мне принадлежит глаз, точно так же и свет; ибо что такое легкое без воздуха, глаз без света? Свет существует не для того, чтобы глаз видел, но глаз существует потому, что есть свет; точно так же и воздух существует не для того, чтобы его вдыхать, но его вдыхают потому, что есть воздух, потому что без воздуха не могло бы быть жизни. Существует необходимое взаимоотношение между органическим и неорганическим. Мало того, это взаимоотношение само есть основа, есть сущность жизни. Поэтому у нас нет никакого основания воображать, что если бы человек имел больше чувств или органов, он познавал бы также больше свойств или вещей природы. 

Их не больше во внешнем мире, как в неорганической, так и в органической природе. У человека как раз столько чувств, сколько именно необходимо, чтобы воспринимать мир в его целостности, в его совокупности. Подобно тому, как человек, организм, произошел не из воды или земли, как верили древние, вообще не из какого-либо отдельного единичного элемента или из какого-либо рода предметов, которому соответствует то или другое чувство, а обязан своим существованием и происхождением взаимодействию всей природы в целом, так и чувства его не ограничены определенными родами или видами телесных качеств или сил, а охватывают всю природу. Природа не прячется, напротив, она навязывается человеку со всей силой, и так сказать, бесстыдством. Как воздух проникает к нам через рот и нос и все поры тела, так же точно и предметы или свойства природы, восприятие которых нашими теперешними чувствами мы не улавливаем, дали бы себя почувствовать через соответствующие чувства, если бы только такие предметы и свойства действительно имелись. Однако вернемся опять назад. 

Разумеется, жизнь погасла бы на земле, по крайней мере, эта жизнь, которая на ней сейчас есть, если бы Земля заступила место Меркурия, но ведь тогда и Земля не была бы уже больше Землей, то есть данной, индивидуальной, от других планет отличающейся планетой, какою она является в настоящее время. Земля есть то, что она есть, только на том месте, которое она занимает в солнечной системе, и она не потому помещена на это место, чтобы люди и животные могли на ней жить, а потому, что она --- соответственно своей первоначальной природе --- по необходимости занимает это место, потому что она вообще наделена такими свойствами, какими она обладает сейчас, поэтому произошли и живут на ней такие органические существа, какие мы встречаем на Земле. Мы видим и на Земле, что особые страны или пояса Земли производят и особых, только им присущих, животных и растения, например, жаркие страны производят самые горячие темпераменты, самые горячительные напитки, самые возбуждающие пряности, что, стало быть, органическая и неорганическая природа связаны одна с другой, что они неразрывны, что они в своем существе едины. Поэтому нисколько не удивительно, что мы находим на Земле условия существования и средства к жизни, подходящие для людей и животных; ибо ведь с самого начала индивидуальности Земли соответствует индивидуальность нашего существа; мы ведь не дети Сатурна или Меркурия, а земные создания, земные существа. Это ведь та же Земля, то же Солнце, тот же климат, которым, например, баобаб и обезьяна, как и негр, обязаны своим происхождением и своим существованием. Там, где имеется такая температура, при которой вода может существовать не в виде пара или льда, а в виде жидкости, где есть, следовательно, вода, которую можно пить или которая может быть впитана растениями, где есть воздух, который можно вдыхать, свет такой силы, такой меры, чтобы глаз животного или человека его мог вынести, там имеются и элементы, первоосновы и истоки животной и растительной жизни, там естественно и даже необходимо должны быть и растения, которые соответствуют животным и человеческим организмам и служат им предметом питания. Поэтому, если этому хотят удивляться, то нужно вообще удивляться существованию Земли или же ограничить свое теологическое изумление и аргументацию только первыми, так сказать, астрономическими свойствами Земли; ибо раз мы имеем эти свойства, раз мы имеем Землю как данную, индивидуальную, самостоятельную, от других мировых тел отличную планету, то в этой индивидуальности Земли нам дано и условие или, вернее, также и происхождение органических индивидуумов; ибо только индивидуальность есть принцип, основа жизни. 

В чем, однако, индивидуальность Земли имеет свою основу? В притягивании и отталкивании, которые существенным образом свойственны материи, основным элементам природы, которые человек только в своем уме от них отделяет. Материальные частицы или тела, притягиваясь друг к другу, тем самым отделяются от других, отталкиваются от них и потому образуют особое целое. Основные вещества, первичные элементы --- материю мира --- мы должны вообще себе мыслить не как нечто единообразное, не имеющее в себе различий; такая материя есть лишь человеческая абстракция, химера; существо природы, существо материи есть уже с самого начала существо, в себе дифференцированное, ибо только определенное, отличимое, индивидуальное существо есть существо действительное. Как нелеп вопрос, почему вообще что-либо существует, так же нелеп и вопрос, почему нечто является именно данным, определенным существом, почему, например, кислородный газ не имеет запаха, вкуса и более тяжел, чем атмосферный воздух, почему он при сжимании светится и под самым сильным давлением не превращается в жидкость, почему при соединении его вес выражается цифрою 8, почему при соединении с водородом он сохраняет постоянно отношение по весу, как 8 к 1, 16 к 2, 24 к 3? Эти именно свойства являются основой индивидуальности кислорода, то есть его определенности, его своеобразия, его сущности. Если я отброшу эти его свойства, отличающие его от других элементов, то я тем самым уничтожу его бытие, уничтожу его самого. Следовательно, спрашивать, почему кислород является именно этой, а не другой материей, все равно, что спрашивать, почему существует кислород. Но почему же он существует? На это я отвечу: он существует именно потому, что существует; он принадлежит к существу природы; он существует не для того, чтобы поддерживать огонь и дыхание животных, но потому, что он есть, существует огонь и жизнь. Там, где дано условие или основание для чего-нибудь, там не может не быть и следствия; где дана материя --- материал для жизни, --- там не может отсутствовать и жизнь, точно так же как, если даны кислород и горючее тело, по необходимости следует и процесс горения. 

\phantomsection
\addcontentsline{toc}{section}{Пятнадцатая лекция}
\section*{Пятнадцатая лекция}

Я ранее в своей последней лекции несколько раз уже бегло указывал на то, что явления природы, которые теист объясняет как акт сознательного существа, ставящего себе цели, могут быть объяснены физическим или естественным путем. Я, впрочем, чрезвычайно далек от того, чтобы задаваться целью объяснить происхождение и существо органической жизни при помощи этих поверхностных указаний. Мы еще стоим далеко не на почве естествознания, которое могло бы этот вопрос разрешить. Мы знаем или, по крайней море можем доподлинно знать только то, что, как мы теперь происходим и сохраняем нашу жизнь естественным путем, точно так же мы и произошли некогда естественным путем, что все теологические объяснения ничего не дают. Но и независимо от этого капитального вопроса о происхождении жизни --- есть, разумеется, много обращающих на себя внимание и удивительных явлений природы, которые именно поэтому теист подхватывает с особенною жадностью и противопоставляет натуралистам, говоря: вот вам явственное доказательство божественного провидения, ставящего себе цели. Однако с этими явлениями природы дело обстоит так же, как и с теми случаями из человеческой жизни, в которых теист усматривает очевидные доказательства существования особого, над человеком бдящего провидения и которые я разобрал уже на одном примере в моих комментариях к <<Сущности религии>>. Это всегда случаи, имеющие отношение к человеческому эгоизму, и, хотя существуют также и другие столь же удивительные явления, которым мы, однако, не колеблясь, даем естественное, причинное объяснение, мы, тем не менее, выдвигаем лишь эти явления, интересующие человеческий эгоизм, проходим мимо их сходства с теми, другими, для нас безразличными явлениями, и рассматриваем их, как доказательство особого, сознающего свои цели провидения, как, так сказать, естественные чудеса. 

<<При более низкой температуре, --- говорит Либих, --- мы выдыхаем больше углерода, чем при более высокой, и мы должны в соответственной пропорции потреблять больше или меньше углерода в пище: в Швеции больше, чем в Сицилии, в наших местностях зимой на целую восьмую больше, чем летом. Даже тогда, когда мы в холодных и теплых местностях потребляем одинаковое количество пищи по весу, бесконечная премудрость устроила так, что эта пища весьма неодинакова по количеству содержащегося в ней углерода. Плоды, потребляемые жителями южных стран, содержат в себе в свежем состоянии не более 12 процентов углерода, тогда как сало и жир потребляемые жителями полярных стран содержат от 66 до 80 процентов углерода>>. Но что это за бесконечная премудрость и сила, которая устраняет лишь последствия беды, нужды? Почему не устраняет она самой беды, не касается самой причины? Если экипаж, в котором я еду, сломается, но я при этом не сломаю себе ноги, то должен ли я причину сего приписать божественному провидению? Не могло ли бы оно прежде всего помешать поломке экипажа? Почему не предотвратят божественные мудрость и благость холод полярных стран, заставляющий даже скалы давать трещины? Разве бог не может создать рая? Что толку в божественном существе, которое помогает лишь потом, задним числом? Разве жизнь обитателей полярных стран не является, несмотря на их богатые углеродом сало и жир, в высшей степени жалкой жизнью? И как можно при подобных явлениях искать прибежища в религиозном представлении о божественной мудрости и благости, когда даже сама религия представляет себе мир, каков он есть, ввиду его противоречий божественной благости и мудрости, вышедшим из рук божиих не таким, как он есть, но принимает, что грех, дьявол исказил его, и именно поэтому рисует перспективу божественного, лучшего мира? И нельзя ли найти естественное основание для указанного явления? Почему бы нет? Бедный обитатель полярных стран, который по временам, как, например, гренландец, поддерживает свое жалкое существование даже при помощи старых мехов своей юрты и башмачных подошв, разумеется, не вкушает южных фруктов и других лакомств южных стран, но лишь по той простой причине, что они у него не произрастают; в силу горькой необходимости он вынужден довольствоваться главным образом салом и жиром тюленя и кита; однако сало и жир встречаются отнюдь не в одних только северных полярных странах. Кит только преследованиями людей отогнан на крайний север, а морской слон, за которым охотятся из-за его обильного жира, встречается также, например, и у берегов Чили. Но если бы даже и в самом деле находились особенно значительные массы углерода близ северного полюса, то и для этого явления мы могли бы указать аналогию в том опыте, который гласит, что дрова, нарубленные зимой, бывают плотнее, более тяжелы и, следовательно, более богаты горючим веществом или углеродом, чем нарубленные весной или летом, что, очевидно, происходит оттого, что в это время под влиянием света и теплоты растение не только разлагает углекислоту, то есть усваивает себе углерод и выделяет кислород, но в период наливания почек, цветения, оплодотворения полностью уничтожает углерод; поэтому в сахарном тростнике, как замечает Ж. Дюма в своем <<Опыте химической статики органических существ>>  сахар, накопившийся в стебле, оказывается совершенно исчезнувшим, когда бывает закончено цветение и оплодотворение. Тот же самый Либих, который в сале и жире бедных жителей полярных стран усматривает доказательство бесконечной божественной мудрости, объясняет, впрочем, другие столь же удивительные явления, которые равным образом могут быть объясняемы и объясняются теологически, в высшей степени простыми естественными причинами. <<Находят достойным удивления, --- говорит он, --- что виды травяных растений, семена которых служат пищей, следуют за человеком, как домашние животные. Они следуют за человеком по таким же причинам, по каким соляные растения следуют за морским берегом и солончаками, марь --- за мусорными кучами; как навозные жуки вынуждены питаться экскрементами животных, точно так же и соляные растения нуждаются в соли, мусорные растения --- в аммиаке и азотнокислых солях. Ни одно из наших хлебных растений не может дать хорошего семени, семени, дающего муку, без обильного количества фосфорнокислого горькозема и аммиака, необходимых для образования этого семени. Эти семена развиваются только в такой почве, где эти три составные части встречаются соединенными вместе, и нет почвы, ими более богатой, чем места, где животные и люди живут вместе на подобие семьи; они (хлебные растения) следуют за мочой, за экскрементами животных и людей, ибо без их составных частей они не в состоянии дать семени>>. 

Таким образом, мы имеем здесь в высшей степени удивительное и для человека важное явление, которым теист может тыкать натуралиста, как самым разительным доказательством существования особого провидения, если только он ничего не знает об естественных причинах этого явления, --- явления, сопоставленного с другими столь же удивительными явлениями, но для человека безразличными (потому что марь, встречающаяся большей частью также вблизи человеческих жилищ --- за исключением, самое большее, одного вида, листья которого употребляются для охлаждающих примочек, --- не приносит пользы ни скоту, ни людям) и объясненного из взаимоотношения между жизнью растений и животными экскрементами, из того взаимоотношения, стало быть, из которого мы вообще пытались объяснить в последней лекции явление целесообразности природы. К уже приведенному примеру я присоединяю еще другой. <<Наиболее распространены, --- говорит химик Мульдер в своей физиологической химии, --- те соли, которые\dots столь же необходимы для жизни, как и органические четыре элемента\dots Большая часть этих солей абсолютно необходима для крови и находится как в питьевой воде, так и в соках растений, служащих пищей для людей и животных; это --- факт, указывающий на тесную связь между обоими царствами природы, которые в науке слишком привыкли отделять друг от друга!>> И хотя в природе достаточно имеется явлений, физическую, естественную причину которых мы еще не открыли, нелепо, однако, раз мы какое-либо явление не в состоянии объяснить физически, естественным путем, прибегать по этому случаю к помощи теологии. То, чего мы <<еще не познали, познают наши потомки. Как неисчислимо много явлений, которые наши предки могли объяснить себе только при помощи бога и его намерений, мы объяснили теперь существом природы! Когда-то даже самое простое, естественное, необходимое объясняли лишь при посредстве телеологии и теологии. Почему люди не одинаковы, почему у них различные лица? --- спрашивает один старый теолог и отвечает на это: для того, чтобы их можно было отличить друг от друга, для того, чтобы их друг с другом не смешивать, для этого дал им бог различные лица. Мы видим в этом объяснении прекраснейший пример сущности телеологии. Человек, с одной стороны, из невежества, с другой --- из эгоистического стремления все объяснять по себе, все мыслить по своему образцу, превращает непроизвольное в произвольное, естественное --- в намеренное, необходимое --- в свободное. Что человек отличается от других людей, есть необходимое, естественное следствие его индивидуальности и его существования; ибо если бы он не был отличен, он не был бы и особым, самостоятельным, индивидуальным существом, и если бы он не был отдельным существом, индивидуумом, то он бы не существовал. Нет двух листьев на одном и том же дереве, говорит Лейбниц, которые бы вполне походили друг на друга, и он вполне прав; лишь бесконечное, необозримое разнообразие есть принцип жизни; одинаковость уничтожает необходимость существования; если меня нельзя отличить от других, то безразлично, существую я или не существую; другие заменяют меня; короче говоря я есмь потому, что я отличен; и я отличен потому, что я есмь. Уже в непроницаемости, в том, что то место, которое я занимаю, никто другой не может занимать, что я всех исключаю из своего места, в этом заключается моя самостоятельность, то что отличает меня от других. Короче говоря, у каждого человека свое собственное лицо, потому что у него своя собственная жизнь, свое существо. Но то же, что применимо к данному случаю, применимо и к бесчисленному количеству других, объясняемых человеком телеологически, с тою только разницей, что поверхностность, невежество и смехотворность телеологии в других случаях не так явственна, очевидна, как в данном примере, к которому, впрочем, можно было бы присоединить еще много других. 

Я только что сказал, что я те явления природы, которые теист объясняет телеологически, отнюдь не хочу считать объясненными вышесказанным. Я иду дальше и утверждаю, что если бы даже многие явления природы и могли бы быть объяснимы только телеологически, то отсюда бы еще далеко не следовали все выводы теологии. Я допускаю, стало быть, вместе с телеологами, что глаз может быть объяснен только при помощи существа, которое при формировании или создании глаза задавалось целью, чтобы глаз видел, что, следовательно, глаз видит не потому, что он организован таким, каков он есть, а что он организован для того, чтобы он видел. Я делаю в этом, стало быть, уступку телеологам, но я отрицаю, что отсюда вытекает наличность существа, к которому подходит имя бога, отрицаю, что мы тем самым выходим из пределов природы. Цели и средства в природе всегда лишь естественные, каким же образом могли бы они отсылать к существу сверх и внеестественному? Вы не можете объяснить себе мира без того, чтобы не принять личного духовного существа за его творца, но я вас все же прощу, будьте любезны, объясните мне, как из бога может произойти мир, каким образом дух, каким образом мысль --- действия же духа суть прежде всего только мысли --- может сотворить плоть и кровь? Я вместе с вами охотно допускаю, что цель, как цель, что цель, как вы ее себе представляете в вашей голове независимо от содержания, предмета, материи цели, указует на бога, на некий дух, но я утверждаю, что эта цель и ее автор, существо, цели ставящее и осуществляющее, так же живут только в вашей голове, как первопричина теизма есть лишь олицетворенное понятие причины, существо бога --- лишь сущность чувственных существ, освобожденная от всех особых предназначений, бытие бога --- лишь родовое понятие бытия. Ибо цели столь же различны, столь же материальны, как и орудия этих целей; как же вы можете, как же, стало быть, вы хотите отделить цели от орудий? Как отделить, например, цель глаза, зрение, от склеры, от сетчатой оболочки, от сосудистой оболочки, от водянистой влаги, от стекловидного тела и других необходимых для зрения тел? Но если вы не в состоянии отделить цель глаза от его материальных средств и органов, то как же вы хотите то существо, которое создало цель глаза, отделить и отличить от существа, создавшего эти многообразные, целям содействующие, материальные части. Может ли, однако, существо не материальное, не телесное быть причиной целей, являющихся следствием материальных, телесных средств или органов? Как можно от целей, зависящих только от материальных, телесных условий и средств, заключать к нематериальному, бестелесному существу как причине? Ведь существо, которое осуществляет свои цели только при помощи материальных средств, по необходимости само только материальное существо. Каким же образом, стало быть, являются, могут явиться творения природы доказательствами и творениями бога? Бог есть, как мы еще дальше увидим, опредмеченное существо человеческого воображения, сделавшееся самостоятельным, бог располагает всеми чудесами воображения; бог все может; он ничем не связан, как и фантазия, как и желание человека; он может из камней делать людей; он даже из ничего творит мир. 

И так как бог творит только чудеса, то и сам он в своем существе есть чудо. Бог видит без глаз, слышит без ушей, думает без головы, творит без орудий, короче говоря, он есть все и делает все, не употребляя и не имея нужных для этого делания средств и органов. Но природа слышит только при помощи ушей, видит только при помощи глаз; как же можно, стало быть, выводить природу из бога, как орган слуха из существа, которое слышит без ушей, как условия и законы природы, с которыми связаны все ее явления и действия, --- из существа, которое не связано ни с какими условиями и законами? Короче говоря, творения бога только чудеса, но не действия природы. Природа не всемогуща; она не все может; она может только то, к чему есть условия; природа, земля, например, не может произвести зимой на деревьях цветение и дать плоды; ибо не хватает нужной для этого теплоты; но богу это сделать ничего не стоит. <<Бог, --- говорит Лютер, --- может и кожу кармана превратить в золото, и из пыли сделать сплошь зерно, и воздух обратить в погреб, полный вина>>. Природа не может создать человека, если нет налицо двух различных, но равноправных организмов, мужского и женского, действующих совместно; но бог из чрева девы без содействия мужчины творит человека. <<Может ли для господа быть что-либо невозможно?>> Короче говоря, природа --- республика, результат существ или сил, взаимно друг в друге нуждающихся и производящих, совместно действующих, но равноправных. Весь животный организм --- мы этим примером характеризуем всю природу --- может быть сведен к нервам и крови. Но нерв ничто без крови, кровь ничто без нерва; в природе именно поэтому неизвестно, кто повар, кто официант, ибо все одинаково важно, одинаково существенно; здесь нет привилегий; самое простое так же важно, так же необходимо, как и высшее; пусть мои глазные нервы превосходно организованы, но если не хватает той или другой жидкости, той или другой оболочки, мой глаз все-таки не может видеть. Именно потому, что организм есть республиканское общежитие, происшедшее только из взаимодействия равноправных существ, происходит материальный вред, борьба, болезнь, смерть; но причина смерти есть и причина жизни, причина зла есть и причина блага. 

Бог, наоборот, монарх и притом абсолютный, неограниченный самодержец, он делает и может, что хочет, он <<стоит над законом>>  но свои произвольные заповеди он делает законами для своих подданных, как бы они ни противоречили их потребностям. \emph{Как в республике господствуют лишь законы, выражающие собственную волю народа, так и в природе господствуют лишь законы, соответствующие собственному существу природы}. Так, существует закон природы, имеющий силу, по крайней мере для более высоко организованных животных, что рождение детей и размножение зависит от существования и совместного действия двух в половом отношении различных индивидуумов, но это закон не деспотический; для существа высших организмов характерно то, что различие в половом отношении приводит к формированию различных самостоятельных индивидуумов, что они, следовательно, появляются на свет более трудным и опосредствованным образом, чем низшие организмы, которые размножаются, как, например, полипы, простым делением. И если для закона природы мы и не можем указать основания, то все же аналогия обязывает нас к вере или; скорее, к уверенности в том, что существует соответствующее естественное основание. Но бог предоставляет деве привилегию произвести человека без мужа, приказывает огню, чтобы он не жег, чтобы он действовал, как вода, и воде, чтобы она действовала, как огонь, чтобы она, стало быть, оказывала такое действие, которое противоречит ее природе, ее существу, как веления деспота противоречат существу его подданных. Короче говоря, бог навязывает природе свою волю, он правит абсолютно произвольно, как деспот, ожидает от людей самого неестественного. Так, например, император Фридрих Второй в своем законе об еретиках предписывал: <<Так как оскорбление величества, направленное против бога, более велико, чем направленное против людей, и так как бог взыскивает за грехи отцов с детей, то дети еретиков должны считаться неспособными занимать какие бы то ни было общественные должности и почетные места, за исключением, однако, тех из детей еретиков, которые донесли на своего отца>>. Есть ли еще более противоречащее природе человека исключение и предписание, чем это? Вильгельм Завоеватель в числе своих прочих тиранических законов издал предписание, чтобы в городах все общества расходились и огонь и свет были погашены, как только в 7 часов вечера прозвучат вечерние колокола. Может ли быть более недостойное человека, более неестественное ограничение человеческой свободы, чем это? Впрочем, подобные же предписания мы пережили еще немного лет тому назад в наших монархических государствах. Томас Пэн рассказывает, что некогда один брауншвейгский солдат, взятый в плен во время войны за независимость североамериканцев, сказал ему: <<Ах, Америка --- прекрасная, свободная страна! Она стоит того, чтобы народ за нее сражался; я понимаю это различие, ибо я знаю свою страну. Когда в моей стране государь говорит: ешьте солому, мы едим солому!>> Но может ли быть приказание, предписывающее человеку большее, более противо и сверхъестественное самоотречение, чем приказание есть солому! Не есть ли, следовательно, единоличный, монархический, по крайней мере абсолютно монархический, режим --- режим чудес как в политике, так и в природе? 

Как согласуется, однако, этот режим с существом природы? Где найдем мы в природе, где все естественно, все происходящее согласуется с существом естественных вещей, где найдем мы следы режима чудес? Желать вывести из природы бога, то есть сверхъестественное чудодейственное существо, так же неразумно, так же свидетельствует о незнакомстве не только с существом природы, но и с существом бога, как если бы я из республики или, беря уже, из главы республиканского государства, президента республики захотел создать государя, короля или императора, доказать, что и он также монарх, правитель соответственно смыслу наших государств, и что поэтому ни одно государство не может обойтись без государя. Президент вышел из крови народа; он одной сущности, одной породы с народом, он лишь олицетворенная народная воля, он не может того, чего он хочет, он выполняет лишь законы, установленные народом. Государь же есть существо, от народа особое или, вернее, отличное по своему роду, как бог отличен от мира; он --- монаршей крови; он господствует над народом не как олицетворенная воля народа, он господствует над народом как существо особое, стоящее вне народа, как бог над природой, как особое, сверхъестественное существо: именно поэтому действия обоих являются только произвольными велениями силы, чудесами, знамениями. В природе же, как уже сказано, республиканский режим. Человеческая голова есть, правда, президент моего тела, но никоим образом не абсолютный монарх или правитель божией милостью, ибо голова есть такое же существо из плоти и крови, как желудок, сердце; она сделана из той же массы, из того же органического основного вещества, из которого сделаны и прочие органы; она, правда, стоит над другими органами; она <<глава>> (caput), первое существо; но не существо иное, чем они, по роду, по происхождению; она не имеет поэтому деспотической власти; она приказывает другим органам лишь такие действия, которые отвечают их существу; именно поэтому она не безответственна, она наказуется, лишается своей власти, когда вздумает разыгрывать из себя государя и требовать от желудка, от сердца или еще от какого-либо органа того, что противоречит их природе. Короче говоря, как в республике, по крайней мере демократической, которую одну мы здесь имеем в виду, управляют лишь народные существа, а не государь, так и в природе управляют не боги, а только естественные силы, естественные законы, естественные элементы и существа. И потому, скажем мы, возвращаясь к прежнему примеру, так же нелепо выводить бога из существа, господствующего над природой, как было бы нелепо и свидетельствовало бы о недостатке разума и способности суждения из президента республики вылущивать государя или монарха. 

\phantomsection
\addcontentsline{toc}{section}{Шестнадцатая лекция}
\section*{Шестнадцатая лекция}

Вера или представление того, что бог есть творец, хранитель, правитель мира, --- представление, которое человек заимствовал из себя, из политического строя и перенес на природу, --- основывается на незнакомстве с природой; оно ведет поэтому свое происхождение от детских времен человечества, хотя бы оно и сохранилось вплоть до сегодняшнего дня, и оно лишь там уместно, лишь там оно является, по крайней мере субъективно, правдой, где человек при своем религиозном простодушии и невежестве приписывает богу все явления, все действия природы. <<Было естественно, --- говорит современный теолог-рационалист Бретшнейдер в своей работе ,,Религиозное учение о вере согласно разуму и откровению для мыслящих читателей`` --- что в древнейшие времена благочестивое: чувство рассматривало все или большую часть природных изменении как непосредственные действия богов или бога. Дело в том, что чем меньше знали природу и ее законы, тем решительнее должны были для этих изменений подыскивать сверхъестественные причины и --- следовательно --- искать волю богов. Так, у греков Юпитер был тем, кто посылал грозу, метал молнии направо и налево. Благочестивое чувство еврейского народа так же относило все или во всяком случае большую часть к богу как к непосредственной причине. Согласно Ветхому завету, Иегова дает взойти семени, оберегает жатву, доставляет хлеб, масло, вино, посылает урожайные и неурожайные годы, болезни и эпидемии; побуждает чужеземные народы к войнам, добрых награждает долгой жизнью, богатством, здоровьем и другими благами, злых наказывает болезнью, ранней смертью и так далее; выводит на небо Солнце, Луну и звезды и руководит по своему желанию всей природой, как и судьбами народов и отдельных людей>>. 

Мы должны, однако, тотчас же заметить этому рационалисту, что этот род представления основан на сущности религии, что лишь там вера в бога является еще истинной, живой, где все объясняется только теологически, а не физически. Мы находим поэтому такое представление не только у древних народов, по и у древних христиан, да и вообще у благочестивых христиан, сохранивших древние, то есть истинные, представления религии и веры в бога, у благочестивых христиан, у которых культура ума еще не победила и не разрушила религиозных представлений, и это служит явственным доказательством того, что это представление истинно религиозно. Мы встречаемся поэтому с ней и у наших реформаторов. Различие между обычным ходом вещей в природе и чудом, по их мнению, заключается лишь в том, что, когда бывает чудо, действие бога бросается в глаза, между тем как обычный ход вещей в природе имеет своей предпосылкой такое же чудесное действие бога, но только в силу своей обычности он не является таким чудом в глазах толпы. Все действия природы, по их мнению, --- действия бога; различие между чудом и действием природы, с их точки зрения, лишь то, что в одном случае бог действует в противоречии с природой, в другом же --- в согласии, по крайней мере с ее проявлениями. <<Не хлеб, --- говорит Лютер, --- а слово божие питает тело, точно так же естественно, как оно творит и сохраняет все вещи. Если есть хлеб, то бог питает хлебом и под видом хлеба так, чтобы этого не видели и полагали, что это делает хлеб. Там же, где его не имеется, там он питает без помощи хлеба одним лишь словом так же, как он это делает под видом хлеба. Вообще же все создания --- маски и переодевания бога (<<бессильные тени бога>>  --- как выражается Лютер в другом месте), которые он хочет заставить действовать с ним и помогать ему творить всяческое, что, однако, он может сделать и без их содействия, да и на самом деле делает>>. Так же говорит и Кальвин в своем <<Учреждении христианской религии>>; так, например: <<божественное провидение предстает перед нами далеко не всегда в обнаженном виде, часто оно, наоборот, облекается в форму естественных средств, оно часто помогает нам при помощи отдельного человека или неразумной твари, оно помогает нам также, не прибегая к естественному средству или даже в противоречии с природой>>  следовательно, очевидно, чудесным путем, то есть другими словами: все действия природы являются, собственно говоря, только действиями бога, все вещи --- лишь инструменты, орудия бога и притом инструменты безразличные, не инструменты, подобные инструментам природы, которая видит только при помощи глаза, но не уха, не носа, но инструменты, с которыми бог по желанию, лишь силою своей воли, сочетал те или другие действия, действия, которые он может поэтому произвести и без помощи этих инструментов. <<Бог мог бы, --- говорит Лютер в одной из проповедей, --- производить детей без помощи отца и матери\dots Но он создал людей для этого и производит на свет и вскармливает детей через посредство родителей, отца и матери. Он мог бы произвести и день без солнца, подобно тому, как первые три дня творения были день и ночь, а между тем не было тогда еще создано ни солнца, ни луны, ни звезд. Подобное бог мог бы еще сотворить\dots если бы этого захотел; но он этого не хочет делать>>. Разумеется, странное ограничение, странное <<но>>  которое значит, что он не хочет делать того, что он может делать. 

По этим изречениям древних, истинно верующих в бога, мы можем судить, как мало физика или физиология согласуются с теологией, до какой степени даже те явления, которые рационалистический теист рассматривает как цели и приводит в качестве доказательств существования бога, плохо выводятся из бога. Между таким органом, как глаз, или средством видения, и целью глаза, актом видения, существует в природе необходимая связь; в самой организации, в природе глаза заложена та особенность, что только глаз, а не какая другая часть тела может видеть; в теологии же воля бога нарушает эту необходимую связь; бог может, если захочет, сделать так, чтобы человек видел и без глаз или даже при посредстве органа, противоположного глазу, даже при посредстве органа, лишенного чувства, даже при посредстве заднего прохода. Кальвин определенно утверждает, что бог в Ветхом завете создал свет раньше солнца, дабы люди могли отсюда видеть, что благотворные явления света отнюдь не необходимо связаны с солнцем, что бог и без солнца может сделать то, что он теперь, то есть при обыкновенном, но совсем не необходимом естественном ходе вещей, делает через солнце или при посредстве солнца. 

Мы имеем в этом случае одно из наиболее убедительных доказательств того, что природа отрицает существование бога и, наоборот, существованием бога отрицается природа. Если есть бог, то зачем же существует мир, существует природа? Если есть совершенное существо, --- совершенное существо, каким его себе представляют в виде бога, --- то зачем еще несовершенное? Не отрицает ли существование совершенного существа необходимость, основу существа несовершенного? К несовершенству совершенство, правда, подходит; но как подходит к совершенству несовершенство? Смысл несовершенства заключается в совершенстве: несовершенное хочет стать совершенным, мальчик --- мужем, девочка --- женщиной, то, что находится внизу, стремится вверх, хочет подняться: но как могу я, будучи в здравом уме, из высшего существа вывести ниже его стоящее, низшее существо? Если я в уме, как могу я из разумного существа дать произойти существам, лишенным разума? Как может дух производить на свет существа, лишенные духа? Стало быть, если я себе мыслю бога и хочу правильно умозаключить и действительно заставить его мысленно что-либо создать, хотя божество есть всегда нечто непроизводительное, то спрашивается: что может бог создать кроме богов, кроме существ, которые ему равны? И если бог есть, то есть, есть существо, которое видит без помощи глаз и слышит, то есть все воспринимает, не имея ушей, то как могу я от него вывести глаза и уши? Ведь смысл, цель, существо, необходимость существования глаз и ушей есть только зрение и слух; но если уже есть существо, видящее без помощи глаз, то зачем же тогда глаза? Не исчезает ли с тем вместе основа его существования? <<Кто создал ухо, как может тот не слышать, кто создал глаз, как может он не видеть?>> Но тот, кто уже видит, зачем ему делать глаз? Глаз существует, потому что без него нет видящего существа, но он существует не потому, что есть видящее существо. Глаз является следствием страстного желания природы видеть, жажды света, потребности, необходимости глаза для жизни, по крайней мере, высшего организма. 

Часто говорилось: мир необъясним без бога; но истинно как раз обратное: если есть бог, то существование мира необъяснимо; ибо он совершенно излишен. Мир, природа лишь тогда объяснимы, мы лишь тогда находим разумное основание их существованию, если только мы таковое ищем, если мы признаем, что нет существования вне природы, нет другого, кроме телесного, естественного, чувственного существования, если мы предоставим природу самой себе, если мы, стало быть, признаем, что вопрос об основе природы совпадает с вопросом об основе существования. Но вопрос о том, почему вообще что-либо существует, есть глупый вопрос. Следовательно, мало того, что мир не имеет в боге своего основания в противоположность тому, что говорили древние теисты, но основа мира уничтожается, если бог существует. Из бога нельзя ничего вывести; все иное кроме него излишне, пусто, ничтожно; как же могу я это другое выводить из бога и желать им обосновывать? Но также верен и обратный вывод. Раз есть мир, раз этот мир есть истина и эта истина ручается за его существование, то бог есть лишь сон, лишь существо, которое человек себе вообразил, которое живет лишь в его воображении. К какому же мы присоединимся заключению? К последнему, ибо мир, природа есть нечто непосредственно, чувственно достоверное, нечто, что не подлежит сомнению. Из существования какого-либо предмета заключать к его необходимости и существенности, без сомнения, куда разумнее и вернее, чем из необходимости какого-либо существа заключать к его существованию; ибо эта необходимость, та необходимость, которая не основывается на бытии, может быть лишь субъективной, лишь воображаемой. Но вот нет же человека, нет жизни, если нет воды, нет света, нет теплоты, нет солнца, нет хлеба, если, короче говоря, нет средств существования. Мы, следовательно, имеем полное право из их бытия заключать к их необходимости, полное право заключать, что жизнь, которая без них, без неорганической природы не существует, существует только благодаря им. Мы чувствуем, мы знаем, что мы умрем от жажды, засохнем без воды; что мы умрем от голода, погибнем без еды; мы чувствуем, мы знаем, стало быть, что своеобразная сила воды и пищи, основанная на их индивидуальной природе, есть то, что оказывает на нас это благотворное действие. Зачем же мы хотим похитить у природы эту силу и приписать ее отличному от природы существу, богу? Зачем думаем мы отрицать то, о чем так отчетливо говорят нам чувства и разум, а именно, что мы только этим силам, этим существам природы обязаны нашим существованием, что нас бы не было, если бы их не было, что они --- необходимые элементы и основы нашего существования, что не бог при посредстве этих вещей, а эти вещи при посредстве их собственной силы поддерживают нас без бога; ибо зачем бы богу нуждаться в подобных небожественных, обычных средствах, как вода и хлеб, но также зачем бы и воде и хлебу иметь надобность в боге для того, чтобы проявить те действия, которые коренятся в их собственной материальной природе? 

Однако вернемся опять назад! 

В образе действий бога мы различаем три ступени или три разновидности: из них первую мы можем назвать патриархальным образом правления или действия, вторую --- деспотическим или абсолютистски-монархическим и третью можно назвать конституционно-монархическим. Первая ступень есть та, когда бог является, собственно говоря, лишь выражением аффекта, выражением удивления, поэтическим именем для каждого предмета природы, производящего на человека особенное впечатление, когда человек хотя и говорит вместо: гром гремит или дождь идет --- бог гремит, бог посылает дождь, по этот бог все же не выражает еще ничего особого, ничего от природы и ее явлений отличного, ибо в том-то и дело, что человек еще не имеет никаких познаний, никакого даже самомалейшего представления о существе и действиях природы, когда именно поэтому еще нет чудес, по крайней мере, в собственном, в нашем смысле этого слова, ибо человеку еще все представляется чудесным; чудо же выражает нечто отличное от естественного, закономерного или, по крайней мере обычного, хода вещей. Я называю это представление патриархальным, ибо оно самое древнее, самое простое, наиболее естественное для похожего на ребенка необразованного человека, ибо патриархальная форма правления есть такая, при которой правитель находится в таком же отношении к управляемым, в каком отец находится к детям, отец, который отличается не по существу от детей, а только возрастом, большей силой и большим умом, ибо и правитель природы и человечества в данном случае так же еще не отличается ничем от природы. Зевс есть бог, от которого исходят гром и молния, град и буря, ливни и метели. Он владыка, то есть очеловеченная, олицетворенная причина этих явлений; он повелевает этими действиями природы по своему желанию и произволению; постольку он, стало быть, --- однако только для нас, --- отличное от них существо, но его отличие теряется в испарениях и синеве неба; Юпитер есть так он и называется --- небо, эфир, воздух; вместо холодного воздуха, сырого воздуха поэты говорят даже --- холодный Юпитер, сырой Юпитер; его отличие истекает с каждой дождевой каплей, которая падает с неба на землю, оно улетучивается, как метеор, с каждой стрелой молния. Так, например, Плиний называет молнию то творением Юпитера, то частью Юпитера. Поэтому для римлян сама молния представляла собой нечто священное, божественное; она у них прямо называется священною молнией, священным огнем. Как мало существо этих богов, по крайней мере первоначально, отличается от существа природы, до какой степени их существо растворяется в существе природы, не имеет характера устойчивой личности, мы увидим, если ближе ознакомимся с древними религиями и заметим, как они даже те явления природы, которые в наших глазах совсем не представляются и не воспринимаются в виде личностей, существ, тем не менее почитали за богов. Так, например, персы обожествляли дни и времена дня, утро, полдень, послеполуденное время, полночь; египтяне обожествляли даже часы; греки --- kairos, благоприятный момент, движения воздуха, ветра. Также и персы, что, однако, в данном случае безразлично. Но бог, существо которого есть ветер, --- что это за улетучивающееся, преходящее существо? Или кто может бога ветра отличить от ветра? Греческие и римские исторические книги кишат чудесными историями, но эти чудеса отнюдь не имеют значения чуда в смысле монотеизма, по крайней мере развитого монотеизма, они имеют более поэтический, наивный характер, они создания скорее натуралистического, чем теологического суеверия, они совсем не такие доктринерские, преднамеренные чудеса, как чудеса монотеистические. От обмана жрецов мы, разумеется, здесь отвлекаемся. 

В монотеизме же существо бога, хотя оно первоначально есть не что иное, как отвлеченное и отделенное от чувств существо природы или мира, представляется существом, отличным от мира и его существа. Здесь поэтому идут насмарку поэтическое простодушие и патриархальный уют политеизма. Здесь начинается рефлексия, бог критически отделяется от мира; здесь во главе мира, природы, становится бог-деспот, перед волей которого сгибается и воле которого безвольно и безлично подчиняется все, деспот, который простым приказом вызвал к жизни мир. <<Как он говорит, --- значится в Библии, --- так и делается, как он приказывает, так оно и есть>>. <<Он приказывает --- и так творится>>. <<Он может творить все, что хочет>>. Стоя на этой точке зрения, хотя и не в самом начале, но при дальнейшем развитии монотеизма, человек именно потому, что он различает природу и бога, человек имеет уже представление о способах деятельности природы в отличие от способов божеских. Он верит в особые действия бога, которым он в отличие от действий природы дает имя чудес. Тем не менее, однако, и на этой точке зрения стоя, до тех пор, пока его религиозные представления еще не ограничены разумом, неверием, пока он еще живет в ненарушимой, деятельной вере, он еще представляет себе действия природы действиями бога. Остановимся по этому случаю на уже приведенном нами из Лютера примере с хлебом. Если бог сохранит человека без пищи, без хлеба, то это --- явственное чудо, ибо здесь человек, очевидно, сохраняется чудесным образом, но если бог сохраняет человека при помощи хлеба, то и здесь не в меньшей степени есть действие бога, есть чудо, ибо бог здесь действует в равной море, но только под видом хлеба; ибо не сила хлеба, а сила бога питает и сохраняет тело. Существа природы --- личины, тени, за которыми и под видом которых действует бог. Поэтому хотя различие между действиями природы и действиями бога здесь уж осознано человеком, тем не менее здесь все же еще налицо, собственно говоря, только чудеса, поступки, действия бога; ибо действия природы --- только кажущиеся действия, обычные действия и явления природы, --- только затаенные замаскированные действия бога, чудеса же в собственном смысле слова --- лишь свободные от покровов, обнаженные действия бога; в одном случае бог действует только инкогнито, в другом же --- во всем своем божественном величии. Короче говоря, как при патриархальной или политеистической точке зрения бог теряется в природе, его отличие от природы неуловимо, ничтожно, --- так здесь наоборот, при точке зрения настоящей веры в бога, теизма или монотеизма, теряется природа; ее существо стушевывается перед существом бога; за ней не признается существования собственной силы и самостоятельности. 

Здесь бог есть единственно действительное, единственно действующее и деятельное. Магометанство высказало эту мысль со всей энергией восточной фантазии и пыла. Так, например, один арабский поэт говорит: <<все, что не бог, ничто>>  а в <<Развитии понятий магометанского исповедания веры>> Эль Сенуси значится: <<невозможно, чтобы рядом с богом что-либо существовало и при том самостоятельно действовало>>. Точно так же ведется борьба против тех магометанских философов, которые утверждают, что бог не является в каждый данный момент заново действующим и творящим в мире, но что мир продолжает самодеятельно существовать благодаря той силе, которая однажды в него была вложена богом; против них ведется борьба и выдвигается утверждение: <<Ничто не имеет действенной силы кроме бога, и если взаимная связь между причиною и следствием, существование которой мы в мире признаем, заставляет нас верить в то, что это есть самодеятельность мира, то мы ошибаемся; сама эта связь есть лишь свидетельство вечно действующей силы божией>>. Но даже и философы магометанской религии высказывали это последовательное и строго религиозное отрицание самодовлеющей и самостоятельной деятельности природы. Так, арабские ортодоксальные философы и теологи, мотекаллемины, верили и учили, <<что мир творится постоянно вновь и потому является непрестанным чудом, что нет ненарушимого существа вещей, нет необходимой связи между основанием и следствием, между причиной и действием>>  --- утверждения, с необходимостью вытекающие из признания всемогущей силы воли и чудотворной деятельности божией; ибо если бог все может, то не может быть и необходимой связи между существом, или основанием, и следствием. Эти арабские ортодоксы утверждают поэтому совершенно правильно, с точки зрения теологии, что <<нет противоречия в том, если что-нибудь произойдет с вещью против ее природы, ибо то, что мы имеем обыкновение называть природою вещей, есть не что иное, как обычный ход вещей, от которого воля бога может отклониться. Не невозможно, чтобы огонь производил холод, чтобы земной круг был превращен в небесный свод, чтобы блоха могла стать такой же большой, как слон, и слон таким же малым, как блоха; каждая вещь могла бы быть иной, чем она есть>>. Эти примеры они приводят, однако, --- замечает Риттер, из сочинения которого <<О нашем знании арабской философии>> взяты эти цитаты, --- только для пояснения их тезиса, что <<если бы богу было благоугодно, то он мог бы сотворить другой мир и с тем вместе другое устройство природы>>. Или, вернее: это представление о том, что все могло бы быть иным, чем оно есть, что нет необходимой природы вещей, есть лишь следствие веры, что бог все может, что богу все возможно, что, стало быть, воле божией не противостоит естественная необходимость. И среди христиан было довольно не только теологов, но и философов, не признававших существования рядом с волею божией какой-либо естественной необходимости и отрицавших за вещами вне бога всякую причинность, всякую самодеятельность и самостоятельность. 

Но этот взгляд, хотя он и последовательно и строго религиозен, тем не менее слишком противоречит естественному человеческому разуму, слишком противоречит опыту, тому чувству, которому природа невольно дает себя знать как самодеятельная сила, чтобы человек, --- по крайней мере прислушивающийся к голосу разума и опыта, --- мог сохранить этот взгляд. Человек поэтому отказывается от этого взгляда и признает за природой самодеятельность действий; но так как для него в то же время отличное от природы существо, бог, есть существо действительное и действующее, то перед ним оказывается двоякого рода действие: действие бога и действие природы; действие природы --- как непосредственное, ближайшее; действие бога --- как посредственное, отдаленное. Бог не производит здесь непосредственных действий; он действует не без посредства подчиненных посредствующих причин, которые и являются естественными существами. Они называются подчиненными или вторыми причинами, потому что первая причина есть бог, средними или посредствующими причинами, потому что они именно те средства, при помощи и силою которых бог действует, но не средства в смысле старой веры, являющиеся лишь произвольными и безразличными инструментами в руке всемогущества, а средства в том смысле, в каком, например, глаз можно назвать средством зрения, --- средства, имеющие собственную природу и силу, необходимые средства. Но бог здесь поступает и действует не только не без участия естественных причин, он поступает только сообразно этим причинам, он поступает не как неограниченный, абсолютный монарх, который распоряжается вещами, как хочет, который делает вещь также и тем, что противно ее природе (огонь водою, пыль --- хлебным зерном, кожу --- золотом), нет, он правит здесь лишь согласно законам природы; он правит, как конституционный монарх. Король, с точки зрения конституционного, а именно английского государственного права, как это точно выражено, может управлять лишь согласно законам, и точно так же бог с точки зрения рационализма, ибо точка зрения, с которой мы сейчас имеем дело, есть не что иное, как так называемый рационализм, взятый нами здесь, однако, в самом широком смысле слова, правит только согласно законам природы. Конституционализм, как выражаются немецкие государствоведы, ставит пределы <<злоупотреблению государственной властью>>  а рационализм ставит предел злоупотреблению божественным всемогуществом и произволом, то есть чудотворчеству. Различие между конституционализмом и рационализмом в этом отношении лишь то, что рационалистический или конституционный бог может делать чудеса --- ибо рационалист не отказывает богу в возможности делать чудеса, --- но он их не делает, конституционный же монарх, или суверен, может не только злоупотребить своею властью, но и действительно ею злоупотребляет всякий раз, когда ему это захочется. Неограниченный монарх царствует и управляет или, по крайней мере, вмешивается в управление всякий раз, когда ему захочется, конституционный же монарх, напротив того, царствует, но не управляет, точно так же и конституционный или рационалистический бог, который лишь стоит во главе мира, но не вмешивается непосредственно, подобно старому абсолютному богу, в управление миром. Короче говоря, подобно тому, как конституционная монархия есть монархия, ограниченная демократией или демократическими учреждениями, так же точно рационализм есть теизм, ограниченный атеизмом или натурализмом, или космизмом, короче говоря, элементами, противоположными теизму. Или: подобно тому, как конституционная монархия есть лишь ограниченная и встретившая себе препятствия демократия, которая поэтому необходимо ведет в своем развитии к истинной и совершенной демократии, точно так же современный рационалистический теизм, или вера в бога, есть лишь ограниченный и встретивший себе препятствие, непоследовательный атеизм или натурализм. Ибо что такое бог, который поступает лишь сообразно законам природы, чьи действия суть лишь действия природные? Он лишь бог по имени, по содержанию же не отличается от природы; он бог, противоречащий понятию бога; ибо только бог, неограниченный, не связанный никакими законами, творящий чудеса, спасающий человека --- по крайней мере, соответственно его вере, его воображению --- из всех бед, есть бог. Но бог, который, например, помогает в болезнях лишь через посредство врачей и лекарств, есть бог, который не более помогает и может помочь, чем врачи и лекарства; это совершенно излишний, ненужный бог, бог, имея которого я ничего не выигрываю, по сравнению с тем, что я имел бы и без него через посредство одной лишь природы, и при потере которого я, стало быть, ничего не теряю. Одно из двух: либо абсолютная монархия, либо никакой; либо абсолютный бог, каким был бог древней веры, или полное отрицание бога. Бог, повинующийся законам природы, приспособляющийся к движению мира, каким является бог наших конституционалистов и рационалистов, --- есть нелепость \hyperlink{16}{(16)}\hypertarget{b16}{}. 

\phantomsection
\addcontentsline{toc}{section}{Семнадцатая лекция}
\section*{Семнадцатая лекция}

К содержанию последних лекций мне надо прибавить еще несколько дополнительных пояснений и замечаний. Человек отправляется от ближайшего, от настоящего, и заключает отсюда к более отдаленному; это делает атеист, это делает и теист. Различие между атеизмом, или натурализмом, вообще между учением, которое выводит природу из нее самой, то есть из естественного принципа, и теизмом, или учением, производящим природу от инородного, постороннего, от природы отличного существа, состоит лишь в том, что теист отправляется от человека и отсюда уже переходит к природе, к ней умозаключает, атеист же, или натуралист, исходит от природы и уже только от нее приходит к человеку. Атеист идет естественным путем, теист неестественным. Атеист искусству предпосылает природу, теист же природе предпосылает искусство; он выводит природу из искусства бога или, что то же, из божественного искусства; атеист предоставляет концу следовать за началом; он делает первым то, что согласно природе является более ранним, теист же делает конец началом, более позднее первым, короче говоря, он делает не естественное, бессознательное, действующее существо природы первым существом, а существо сознательное, человеческое, искусственное; он совершает поэтому ту, мною уже подвергавшуюся порицанию, ошибку, что вместо того, чтобы производить сознательное от бессознательного, он производит бессознательное от сознания. Теист, как мы уже видели при обсуждении телеологического доказательства, от того, что он смотрит на природу, на мир, как на жилище, на часы или на другое какое-либо механическое произведение искусства, умозаключает отсюда к мастеру или художнику, как их творцу. Он, стало быть, делает искусство оригиналом природы, по образцу творений человеческих мыслит он себе творения природы; отсюда-то проистекает и тот вывод, что причина, их производящая, есть существо индивидуальное, как человек есть делатель, творец. 

Это заключение, или доказательство, есть то, которое, как уже было упомянуто, наиболее убедительно действует на людей, по крайней мере, стоящих на известной ступени развития, поэтому при его посредстве миссионеры и утверждают веру в бога у некультурных народов, а христианские учителя и родители --- у детей. Но это доказательство рассматривается не только как наиболее убедительное, наиболее усвояемое, но и как наиболее безошибочное, как такое, которое без всякого сомнения является ручательством существования бога. Уже маленьким детям, говорят верующие, бог внушил этот вопрос; кто создал звезды, цветы? чтобы обратить их внимание на свое бытие. Но спрашивается, возник ли у детей этот вопрос сам собой или он был внушен им родителями? Есть во всяком случае много народов и бесчисленное множество людей, которые спрашивают не о том, откуда они произошли, а о том, откуда мы получаем пищу, чем мы живем? Так, гренландцев можно было сколько угодно спрашивать о происхождении неба и земли, --- иного ответа они не давали, как тот, что небо и земля возникли сами собой, или что они, гренландцы, об этом нисколько не думают, им бы иметь только вдоволь рыбы и тюленей. Так и жители Калифорнии не имели <<ни малейшего представления о творце природы. На вопрос, думали ли они когда-нибудь о том, кто создал солнце, луну или то, что является для них наиболее ценным, --- питахахии, они отвечали --- вара, нет>> (Циммерман: <<Записная книжка путешествий>>). Но даже если отвлечься от этого, если и в самом деле вопрос этот возник на собственной почве детского сознания, то все же он имеет безусловно наивное, детское или ребяческое значение, значение, из которого никаким образом нельзя делать христианско-теологических выводов. Ребенок спрашивает, кто сделал звезды, потому что он не знает, что такое звезды, потому что он не отличает их от свечей, которые горят в комнате его родителей и которые отлил мыловар; он спрашивает-кто сделал цветы? --- потому что он не отличает цветов от других пестрых и цветных вещей, которые он видел вокруг себя и которые сделаны человеческими руками. И если далее ответ: господь бог это сделал --- и в самом деле удовлетворяет детское сознание, то отсюда еще отнюдь не следует, что он соответствует истине: он так же мало соответствует истине, как и ответ на вопрос детей о том, кто принес рождественские подарки, что принес их младенец Христос, или ответ на вопрос детей о происхождении их сестрицы или братца, что их выудили из прекрасного и глубокого колодца, хотя ответ этот детей и удовлетворяет. 

Но как же отвечать, удовлетворяя любопытство детей? Пока дети --- еще настоящие дети, пока этот вопрос является еще детским, до тех пор необходимо давать и детский ответ; ибо ответа, соответствующего истине, они все равно не поймут; если же этого не хотят делать, то нужно детям ответить, что они это узнают лишь тогда, когда станут побольше и немного подучатся. Но когда дети становятся больше, когда они входят в разум настолько, что уже более не поверят, что дети достаются из колодца, тогда им нужно совершенно так же, как это практикуется со старыми детьми, делающими господа бога причиной всех вещей, дать понятие, представление о природе. Нужно отправляться при этом не от человека или если и от человека, то не от того, что человек создал и создание чего предполагает уже всегда наличность природы, не от человека как художника, артиста или ремесленника, а от человека как от существа природы. Нужно ребенка, как и необразованного человека, убедить прежде всего в существовании различия между искусством и жизнью; некультурные народы принимают произведения искусств за живые существа, теистически культивированные народы принимают, наоборот, живые существа за произведения искусства, мир --- за машину; им нужно на примерах показать, чем корабль отличается от рыбы, кукла --- от человека, часовой механизм от животного или живого идущего механизма. Затем надо перейти к вопросу о происхождении; посмотрите: ведь растения происходят из зародыша, животное --- из яйца, следовательно, растение из растительной, животное --- из животной материи, которая, однако, еще не есть животное. Когда мы, наконец, подвинулись настолько далеко, что показали людям наглядно происхождение, рождение на свет животных и растений, то мы можем заставить их умозаключить уже к более отдаленным вещам, уложить им в головы и сделать понятным на основании очевидных фактов, что и первые растения и животные не сделаны, не созданы, а произошли из естественных веществ и в силу естественных причин, что вообще все существа и тела мира произошли не от существа, не принадлежащего к миру или находящегося вне его, а от существа мирского, естественного. Если бы они, однако, нашли это непонятным и невероятным, то им нужно было бы возразить, что если бы человек не знал по опыту, что дети происходят естественным путем, то он считал бы это происхождение столь же невероятным и поэтому не сомневался бы в том, что господь бог делает детей, что дети непосредственно происходят от бога. 

И в самом деле, произведение на свет человека, то есть происхождение человека от человека, объявлялось бы столь же необъяснимым и непонятным, как и первое происхождение человека из природы, и поэтому как в том, так и в другом случае происходило обращение к помощи бога. Однако понятен или непонятен процесс произведения на свет человека, все равно, это --- процесс естественный, и даже он является естественным процессом не вопреки этой непонятности, а как раз благодаря ей, ибо для человека, который все меряет на свой образец, который не имеет понимания, разумения природы, самое естественное есть и самое непонятное. Ведь даже человек человеку, щедрый скупцу, нерасчетливый --- благоразумному, гениальный --- филистеру представляется чем-то непонятным; во сколько же раз непонятнее природа! Каждый понимает лишь то, что ему подобно, что ему родственно. Но как тому или другому человеку непонятный человек есть все же человек, так же точно и природа, которую мы не понимаем, потому что она противоречит тем ограниченным понятиям, которые мы о ней составили, есть, тем не менее, природа, не что-либо сверхъестественное. Сверхъестественное существует лишь в фантазии, или есть та природа, которая выходит за пределы наших ограниченных понятий, нами, о ней составленных. Как неумно поэтому делать из этих непонятных явлений в природе теологические выводы или пытаться их разрешить при помощи теологии! Физики и физиологи еще и в настоящее время не в состоянии объяснить массы явлений органической и неорганической природы. Но следует ли из этого, что эти явления не имеют своих физических и физиологических причин, как их имеют другие явления, которые мы в состоянии объяснить? Или часть природы имеет физический характер, а другая часть сверхфизический? Не есть ли, наоборот, природа единая, вся сплошь единая, всюду природа? 

А теперь обратимся ко второму замечанию. Главная причина, в силу которой человек выводит мир из бога, из духа, есть та, что он не в состоянии объяснить свои дух при помощи мира, или природы. Откуда же произошел дух? восклицают теисты, возражая атеистам: дух ведь может произойти только от духа. Эта трудность выведения духа из природы происходит, однако, только оттого, что, с одной стороны, о природе имеются слишком неуважительные представления, а с другой, о духе --- слишком высокие, преувеличенные. Если дух делают богом, то естественно, что он может быть только божественного происхождения. И уже самое утверждение, что дух не может быть выведен из природы, есть косвенное утверждение того, что дух --- не естественное, а вне и надмировое божественное существо. И на самом деле, дух, как его представляют себе теисты, необъясним из природы; ибо этот дух есть очень поздний продукт, и притом продукт человеческой фантазии и абстракции, и его так же нельзя вывести, по крайней мере непосредственно вывести из природы, как нельзя объяснить себе непосредственно из природы лейтенанта, профессора, статского советника, хотя человек из природы объясним. Но если не придавать духу большего значения, чем ему полагается, если не делать его абстрактным, отделенным от человека существом, то происхождение его из природы не представляется непонятным. Ведь дух развивается вместе с телом, с чувствами, с человеком вообще; он связан с чувствами, с головой, с телесными органами вообще; следует ли, например, телесный орган, голову, то есть череп и мозг, производить из природы, в то время как дух, находящийся в голове, то есть деятельность мозга, --- от существа совершенно другого рода, чем природа, от мысленного и фантастического существа, от бога? Что за половинчатость, что за раздвоение, что за <<шиворот-навыворот>>! Оттуда же, откуда череп, откуда мозг, оттуда же и дух, откуда орган, оттуда и его функция; ибо как можно одно отделять от другого. Следовательно, если череп, если мозг --- от природы, ее продукт, то так же точно и дух. Мы словесно различаем деятельность головы, как духовную, от других телесных отправлений, мы ограничиваем слова <<телесность>>  <<чувственность>> только особыми видами телесности и чувственности и делаем, как я уже показал в своих сочинениях деятельность, от них отличную, деятельностью абсолютно иного порядка, духовной, то есть абсолютно нечувственной и бестелесной; но ведь и духовная деятельность --- ибо что такое дух, как не духовная деятельность, получившая самостоятельное бытие благодаря человеческой фантазии и языку, как не духовная деятельность, олицетворенная в виде существа? --- и духовная деятельность есть телесная, есть головная работа; она отличается от других родов деятельности только тем, что она есть деятельность иного органа, а именно деятельность головы. Но так как мыслительная деятельность есть деятельность особого рода, которая именно потому не может быть сравнима ни с какою другою, что в этой деятельности органы, ее обусловливающие, не являются для человека непосредственным предметом его чувства и сознания, как, например, рот и желудок при еде, --- их пустоту и наполнение человек чувствует, --- глаза во время зрения, кисти рук и сами руки при ручной работе, потому что головная деятельность есть самая скрытая, удаленная, бесшумная, неуловимая, то он эту деятельность сделал абсолютно бестелесным, неорганическим, абстрактным существом, которому он дал название духа. Но так как это существо своим бытием обязано лишь незнанию людей об органических условиях деятельности мышления и фантазии, возмещающей это незнание, так как это существо есть, стало быть, лишь олицетворение человеческого незнания и фантазирования, то и на самом деле отпадают все трудности, мешающие представить себе это существо. Если дух есть деятельность человека, а не существо само по себе, если он имеет органы, не отделимые от тела, то его можно вывести лишь из существа природы, но отнюдь не из бога, ибо этот бог, или божественный дух, из которого должен выводиться дух человеческий, есть сам не что иное, как именно эта духовная деятельность, мысленно отделенная от тела и всех телесных органов, надуманная и представленная как самостоятельное существо. 

Конечно, дух есть высшее в человеке; он представляет собою элемент благородства в человеческом роде, его отличие от животного; но то, что является первым для человека, еще не есть поэтому естественно или от природы первое. Наоборот, наивысшее наисовершенное есть последнее, самое позднее. Поэтому делать дух началом, исходным моментом, есть извращение естественного порядка. Но человеческое тщеславие, самолюбие и невежество любят приписывать и первенство по времени над всеми прочими существами тому, что есть первое по качеству. Стремление человека выводить дух из бога, то есть опять-таки из духа, приписывать духу изначальное существование, существование довременное, предшествующее природе, тождественно поэтому со стремлением, которое некогда побуждало древние дворянские роды, да и вообще древние народы, мнившие себя дворянскими родами по отношению к другим народам, побуждало и побуждает многие народы, еще и до сих пор со своего бытия, со своей истории начинать бытие, историю вообще, приписывать себе непосредственно божественное происхождение. Гренландцы, когда им во что бы то ни стало хотели навязать веру и заставить признать, что должен же был кто-нибудь мир сотворить, отвечали: <<ну да, так его сотворил какой-нибудь гренландец>>. Эта мысль нам кажется по справедливости смешной. Тем не менее, однако, она основывается на том же стремлении, которое побуждает духовно одаренный, мыслящий народ, народ, преисполненный сознания своего духа, как прерогативы своего благородства, приписывать духу довременное, божественное существование, производить мир из духа. 

А теперь перейдем к третьему замечанию. Так как происхождение телесного мира из духовного существа или бога есть слишком бросающаяся в глаза невозможность, так как, сверх того, дух без тела есть очевидная абстракция человека, то некоторые верующие в бога мыслители или религиозные философы новейшего времени отказались от древнего учения о сотворении из ничего, учения, являющегося необходимым следствием представления о происхождении мира из духа, --- ибо откуда дух берет материю, телесное вещество, как не из ничего? --- отказались и сделали самого бога телесным, материальным существом именно для того, чтобы иметь возможность объяснить из него материальный мир. Короче говоря, они рассматривали божество не как просто дух, или они сделали богом не только ту часть человека, которую он называет духом, но и другую часть человека, именуемую телом, и, следовательно, они мыслили себе бога существом, состоящим из тела и духа, подобно человеку, существующему в действительности. Шеллинг и Франц Баадер признали это учение. Но родоначальниками этого учения являются некоторые старинные мистики, вроде Якова Беме, по профессии сапожника, родившегося в 1575 г. в Верхнем Лаузице, умершего в 1624 г. Этот и в самом деле примечательный человек различает в боге мрак и свет, или огонь, положительное и отрицательное, доброе и злое, мягкое и суровое, любовь и гнев, короче говоря, дух и материю, душу и тело. И ему, по видимости, по крайней мере, легко вывести мир из бога, ибо все силы, качества или явления природы, как то: холод и жар, горечь и остроту, твердость и жидкость, он вводит в понятие бога. Самое примечательное в Беме есть то, что он, так как нет света без мрака, нет духа без материи, или природы, предпосылает природу бога его духу, который только и есть бог в собственном смысле этого слова, хотя местами он и противоречит этому происхождению бога ввиду своей зависимости от христианской веры; по крайней мере, это происхождение или развитие духа из природы, или материи, он хочет рассматривать не как существующее во времени и, следовательно, не как действительное, истинное. Это учение во всяком случае разумно в том отношении --- и в этом оно сходится с атеизмом или натурализмом, --- что оно отправляется от природы и уже только оттуда переходит к человеку, причем заставляет дух развиваться из природы --- путь, совпадающий с путем природы, а следовательно, и опыта, ибо мы ведь все --- сначала материалисты, прежде чем стать идеалистами, мы все преданы телу, низшим потребностям и чувствам, прежде чем подняться до духовных потребностей и чувств; дитя сосет, спит и вперяет в мир глаза, прежде чем научиться видеть. Но это учение неразумно в том отношении, что оно этот процесс развития, этот путь природы заволакивает опять мистическим туманом теологии, что оно связывает с богом то, что противоречит понятию бога, и с природой связывает то, что противоречит природе; ибо природа телесна, материальна, чувственна, но божественная природа, в том виде, как она является составной частью бога, не должна, мол, быть таковой. Природа в боге, или божественная природа, включает в себя, правда, все то, что включает природа небожественная, то есть материальная, чувственная, но божественная природа включает это на особый манер, нечувственный, нематериальный, ибо бог есть дух или должен быть им, несмотря на свою материальность. Поэтому и здесь в конечном счете перед нами старая необъяснимость, старая трудность понимания того, каким образом из этой нематериальной, духовной, природы должна произойти действительная, телесная. 

Эта трудность устраняется лишь тогда, когда мы на место божественной природы поставим действительную природу, как она есть, когда мы телесные существа начнем производить из действительного, а не только воображаемого телесного существа. Но так же точно, как божественная природа противоречит понятию и существу природы, так же точно и бог Якова Беме противоречит понятию божества, ибо бог, который развивается и поднимается из мрака к свету, из недуховного существа до духа, не есть бог; бог есть вполне абстрактное, законченное, совершенное существо, существо, из которого выключено всякое основание, всякая необходимость развития; ибо ведь развитию подлежит лишь естественное существо. Правда, как сказано, это развитие не происходит во времени, но как можно время отделить от развития? Короче говоря, это учение мистично, это есть учение о природе, которое, однако, в то же время должно быть и учением о боге, поэтому учение, полное противоречий и смутной неясности, теистический атеизм, исполненное веры в бога отрицание бога, натуралистический сверхнатурализм или сверхнатуралистический натурализм, учение, которое именно поэтому вынуждает нас выйти из царства фантазии и мистики, в котором оно пребывает и в котором имеет свои корни, на свет действительности, и, следовательно, вместо нечувственной природы поставить чувственную, на место божественной истории историю действительную, мировую историю, вообще на место теологии антропологию. В учении Якова Беме мы опять имеем отчетливый, убедительный пример того, что бог есть лишь существо, производное от человека и природы; различие между учением Беме и обычным теистическим заключается лишь в том, что его бог есть существо, производное не только от действительных или воображаемых целей природы, то есть от тех явлений природы, которые человек объясняет себе наличностью действующего целесообразно существа-духа, но и от вещественного содержания этих целей, которые ведь все лишь материальной, телесной природы, так что Яков Беме поэтому обожествляет не только дух, но и материю. Как положение: бог есть дух --- имеет своей предпосылкой положение: дух есть бог, или божественное существо; так и положение: бог не только дух, но и телесное существо, имеет своей предпосылкой положение: материя, телесное существо, есть божественное существо, или, вернее, в этом последнем положении и заключается только истинный смысл и объяснение первого положения. Если, однако, бог, который есть дух, представляет из себя лишь олицетворенное выражение божественности духа, бог же, который есть тело, материя, равным образом является не чем иным, как олицетворенной божественностью, то есть (выражаясь философски) существенностью и истиной природы или материи, то ясно, что учение, которое демонстрирует перед нами божественность материи в боге, есть мистическое, есть ложное учение, что истинное, разумное учение, то учение, в котором это мистическое только и находит свой смысл, есть атеистическое учение, рассматривающее дух и материю самих по себе, без бога. 

Если бог есть материальное, телесное существо, как этого хотят последователи Якова Беме, то истинным доказательством этой телесности является лишь то, что бог есть также предмет наших телесных чувств. Что такое телесное существо, которое бы не было предметом тела? Ведь мы только на основании телесных впечатлений о предмете заключаем о его телесности. С этим, однако, конечно, не соглашаются материалистические теисты; они так далеко не дают спуститься своему богу в материю, чтобы его можно было также телесно уловить и увидеть; это для них слишком нечестиво, слишком небожественно. Правда, при этом своем перемещении в нечестивый материальный мир он утратил бы свое бытие, потому что там, где вступают в свои права глаза и руки, там прекращают свое существование боги. Гренландцы верят даже в то, что самого могущественного из их богов, Торнасука, мог бы убить ветер и даже простое прикосновение собаки. Но именно в силу этой боязни экспериментальной физики телесность бога Якова Беме есть лишь фантастическая, воображаемая телесность. Короче говоря, это учение, как все теологические, есть ложь, есть противоречие, Оно обожествляет природу, телесность и в то же время опять-таки упускает, отрицает то, что эту телесность только и делает истинной телесностью. Если вы хотите признать истинность телесности, то раскройте ваши чувства, признайте истинность чувств. Но вы признаете лишь истину фантазии, воображения, мистического, нечувственного мышления и представления; вы должны поэтому признать, что вы в вашем боге, несмотря на его материальность и телесность, обожествляете только вашу фантазию и силу воображения --- как орган, так и предмет этого органа. Если я отрицаю чувства, то я отрицаю и чувственное существо, то я имею всегда дело только с духовным или воображаемым существом. 

\phantomsection
\addcontentsline{toc}{section}{Восемнадцатая лекция}
\section*{Восемнадцатая лекция}

К замечаниям, сделанным в последних лекциях, я должен присовокупить еще следующее. Я сказал, что с точки зрения рационализма мы имеем бога и природу, два существа, две причины и два способа действия: один непосредственный, который приписывается действительным и естественным существам, другой --- посредственный, который приписывается богу, совершенно подобно тому, как при конституционализме господствуют или спорят о господстве две силы, народ и государь, тогда как при натурализме господствует только природа, при подлинном теизме --- только бог, так что поэтому рационализм, как и конституционализм, есть система половинчатости, противоречия, нерешительности, бесхарактерности. Я должен, однако, заметить, что уже при абсолютной вере, или том боге, который является абсолютным монархом, и даже до известной степени при политеизме --- почитайте только римских и греческих историков и поэтов, которые в высшей степени наивным образом связывают божественную деятельность с человеческой, --- выступает противоречие, заключающееся в том, что, несмотря на всепоглощающую деятельность бога, за вещами помимо бога признается самодеятельность. И встречается это противоречие там уже по той простой причине, что человек своей даже самой чрезмерной склонностью верить никогда не в состояния подавить, однако, свой естественный разум, подавить в себе <<человека>> или отказаться от них. Но этот <<человек>> приписывает внебожественным вещам или существам причинную самодеятельность. Это именно относится к западному человеку и особенно к германцу, у которого высшими понятиями являются самодеятельность, свобода и самостоятельность --- свойства, в которых он должен был бы себе отказать в том случае, если бы кроме бога не существовало ничего самодеятельного. Обитатель западных стран поэтому пресекает благодаря своей врожденной склонности к разумной самодеятельности --- выводы из своей религии, своей веры в бога, тогда как восточный человек, соответственно своей натуре, выводам из своей веры в бога не ставит никаких пределов, он лишает себя поэтому своей свободы и даже своего разума, безоговорочно подчиняется решению рока, чтобы оказать своему богу честь, что он не только первая причина, как утверждают умные, эгоистические, рационалистические люди западных стран, но и единственная причина, единственное самодеятельное и действующее существо. Я уже приводил в предпоследней лекции несколько примеров из магометанства; разумеется, есть и магометанские, вообще восточные философы и теологи, приписывающие самодеятельность и вещам, но противоположное воззрение есть господствующее или во всяком случае характеристичное. <<Бог, --- говорит, например, правоверный магометанский философ Альгазель --- цитата, которую я приобщаю к предыдущим, --- бог есть во всей природе единственная действующая причина; при ее посредстве так же возможно, чтобы огонь коснулся пакли, без того чтобы она загорелась, как и то, чтобы пакля загорелась без прикосновения огня. Нет естественного хода вещей, нет естественного закона; нет различия между чудесами и естественными явлениями>>. 

Западная теология ломает себе голову над вышеупомянутым противоречием и ломает себе голову даже в лице своих строгих и правоверных представителей. Разумеется, это противоречие лежит в существе теологии; ибо если есть бог, то не нужен мир, и наоборот. Каким же образом должны эти взаимно исключающие друг друга существа друг с другом в своей деятельности ладить и иметь возможность объединяться? Деятельность бога упраздняет деятельность мира, и, наоборот, деятельность мира упраздняет первого рода деятельность. Если я это сделал, то это сделал не бог; если это сделал бог, то я этого не сделал; одно исключает другое. Как приложимо здесь представление о средстве представление о том, что бог через мое посредство это сделал? Со средством несогласима никакая самодеятельность. Короче говоря, желание предоставить богу и миру одновременно быть и действовать приводит к самым нескладным противоречиям, к самым смешным софизмам и ухищрениям, как это достаточно доказала история теологии в учении о так называемом concursus del, участии бога в свободных поступках людей. Вот пример. <<Так как христианин, --- говорит правоверный, но именно поэтому и могущий служить примером, Кальвин в своем <<Учреждении христианской религии>>  --- самым твердым образом убежден в том, что нет ничего случайного, а все происходит согласно велению бога, то он будет всегда устремлять свои взоры к богу, как к наисовершеннейшей или первой причине вещей, подчиненным же причинам он уделит место, им соответствующее. Он не будет сомневаться, что особое провидение, простирающее свое действие на все самые мелкие явления, бдит над ним и что оно не допустит ничего, что бы не шло к его благу. Поэтому все, что удается счастливо и согласно желанию, он отнесет к одному лишь богу, он будет одного лишь бога считать причиною этого, хотя бы он испытал его благость благодаря услугам, которые оказали ему люди, или получил помощь от неодушевленных созданий. Ибо так помыслит он в сердце своем: поистине, это господь бог склонил их души сделаться орудиями его благожелательного умонастроения по отношению ко мне. Он будет поэтому, получая добро от людей, почитать и прославлять бога, как главнейшего виновника; людей же чтить и признавать как его слуг, чтить за то, что благодаря воле божией он обязан им, руками которых бог пожелал оказать ему свои благодеяния>>. 

Мы представили здесь всю нищету теологии в том виде, как она дает себя знать в этом вопросе. Если бог есть наисовершеннейшая или главная причина или, вернее, просто причина оказанного мне людьми добра --- ибо только causa praecipua есть настоящая причина, --- то как чтить мне людей, как мне чувствовать себя по отношению к ним обязанным, к ним, через посредство которых бог сказал мне добро? Ведь это не их заслуга; бог склонил их сделать мне добро, а не их собственное сердце, их собственное существо; бог мог бы мне так же точно помочь через других, даже ко мне плохо расположенных людей или через иные, нечеловеческие существа, или, наконец, мог бы мне помочь сам, без посредничества. Посредничество совершенно безразлично, совершенно несущественно; оно совершенно не способно вызвать по отношению к себе настроение благодарности, почитания, любви, так же точно, как не способен это сделать сосуд, в котором мне подают напиться, когда я изнываю от жажды. Пусть никто не найдет это сравнение неподходящим. Ведь люди, как это сказано в Библии, по отношению к богу что горшки по отношению к горшечнику. Мы видим поэтому на этом примере, как теология в противоречии с своей верой в бога как во всемогущую, все осуществляющую причину капитулирует перед естественным чувством и смыслом человека, который те существа, от которых он получает благодеяния, рассматривает и как причины этих благодеяний и поэтому чувствует себя по отношению к ним обязанным благодарностью, любовью, почтением. Мы видим, как бог и природа, любовь к богу и любовь к человеку друг другу противоречат; действие бога и действие природы и человека несовместимы друг с другом иначе, как при помощи софистики. Или бог, или природа. Третьего, среднего, их соединяющего, не существует. 

Или признайте бога и отрицайте природу, или, если вы этого не можете, если вы должны признать, по крайней мере, за ней бытие, потому что ваши чувства наперекор вашей вере навязывают вам бытие природы, то отрицайте, по крайней мере, за ней всякую причинность, всякое существо, говорите, что она простая видимость, простая маска; или признайте природу и отрицайте, что есть бог, что бог за ее спиной ведет свое существование, что бог через нее действует. И если вы признаете бога истинной причиной или причиной вообще добра --- потому что только истинная причина есть первая причина, --- то не отрицайте также и того, что причина зла, причиняемого человеку другими людьми или существами, есть бог. Но этот вывод непоследовательным образом отрицает теизм. Тот же самый Кальвин, который рассматривает людей, творящих добро, лишь как орудие бога, заявляет, что бессмысленно и безбожно делать отсюда заключение, будто если, например, убийца убьет честного человека, то он является только орудием, исполняющим решение или волю бога, будто, стало быть, все преступления происходят лишь по распоряжению бога и в силу его воли. И все же это --- необходимый вывод. Если действительные, естественные существа --- только средства, только орудия бога, то они таковы, делают ли они добро или зло. Если вы отрицаете, что человек самостоятельно, по собственному побуждению, творит добро, то отрицайте и то, что он по собственному побуждению творит дурное, зло; если вы отнимаете у человека честь благодетеля, то отнимите и позор злодея и преступника; ибо для того, чтобы делать зло, для этого нужно иметь столько же, а часто даже и более силы и могущества, чем для того, чтобы делать добро; но ведь всякая сила, всякое могущество, по вашему мнению, есть сила и могущество бога. Как смехотворно и в то же время как зло, с одной стороны, отрицать за человеком причинность, а с другой --- опять ее ему приписывать, добро дарить ему как милость, зло же вменять как преступление! 

Но в этом и состоит сущность теологии или, --- беря персонально, теолога, что он ангел но отношению к богу, но дьявол по отношению к человеку; что он доброе приписывает богу, но злое --- человеку, творению, природе. Конечно, добро, которое делает человек, приходится не только на его собственный счет, есть не только дело его собственной воли, оно также результат естественных и общественных условий, отношений, обстоятельств, при которых человек был зачат и рожден, воспитан и образован. Верить тому, что эти условия, отношения и обстоятельства и под их влиянием возникшие склонности и умонастроения имеют свою основу в намерениях и решениях бога, было бы грубейшим, глубочайшим и самым суеверным эгоизмом. Как целесообразность природы есть лишь человеческое или, вернее, теологическое выражение для бесконечной взаимной связанности, в которой в природе все друг к другу находится, так же точно воля или решение бога, в силу которого человек обладает теми или иными свойствами, влечениями, наклонностями, способностями, есть лишь антропоморфизм, простонародное человеческое выражение того взаимоотношения, при котором каждый человек сделался тем, что он есть. Это единственный разумный смысл того представления или учения, которое гласит, что человек не по собственной воле, а по воле или милости бога есть то, что он есть. Милость бога есть олицетворенный случай или олицетворенная необходимость, олицетворенное взаимоотношение, при котором люди складываются, живут и действуют. Я есмь то, что я есмь, лишь как сын девятнадцатого века, лишь как частица природы, в том виде, как она сложилась в этом веке; ибо и природа меняется, поэтому каждый век имеет свою собственную болезнь, и я не по собственной воле помещен в этот век. И, однако, как я не могу отделить мое существо от существа этого века, как я не могу мыслить себя в виде существа, вне его имеющего бытие, от него независимого, точно так же не могу я и свою волю отделить от этого существа; хочу я этого или нет, сознаю я это или не сознаю, я принимаю этот жребий или судьбу, эту необходимость быть членом этого времени; я являюсь тем, чем я являюсь от природы, помимо моей воли, в то же время и согласно моей воле; я не могу ничего другого хотеть, чем то, что я есмь, то есть, чем я являюсь в существенном или по существу. Мои безразличные случайные качества я могу себе мыслить иначе, могу желать изменить их, но не мое существо; моя воля зависит от моей природы, от моего существа, но не моя природа --- от моей воли, моя воля и без того, чтобы я это знал и хотел, направляется моим существом, но мое существо, то есть существенные свойства моей индивидуальности, не направляется моей волей, как бы я ни напрягался и ни старался себя превзойти. 

Человек может, конечно, пожалеть --- хотя его существо и не позволяет ему отделить себя от своего времени, --- что он не родился в Афинах во времена Фидия и Перикла! Но подобные желания только фантастичны; даже они определяются характером эпохи, в которую человек родился и сформировался, определяются самим существом человека, которое он даже этим фантастическим перемещением в чужие места и времена нисколько не может изменить. Ибо только в такую эпоху, которая чувствует себя близкой древней афинской жизни, и только в человеке, чье собственное существо чувствует себя влекомым к этой жизни и ее формам, может возникнуть подобное желание. И если я в самом деле в мыслях переношусь в Афины, то этим я не выхожу за пределы моего века, моего существа, что невозможно, ибо ведь я мыслю себе эти Афины только соответственно тому, что думает моя голова, только в духе моего века; это --- лишь отражение моего собственного существа, ибо каждое время думает о прошлом лишь по себе. \emph{Короче говоря, человек есть то, что он есть, что он есть по существу и чем он быть желает; он не может отщепиться от своего существа; даже его желания, выходящие за эти пределы в его фантазии, определяются этим существом, постоянно к нему возвращаются, как бы они видимо от него далеко и ни удалялись, подобно камню, который, подброшенный в высоту, возвращается на землю}. Следовательно: каков я ни есмь благодаря самодеятельности, благодаря моей работе, благодаря напряжению воли, я сделался тем, чем я есмь, лишь в соотношении с данными людьми, с данным народом, с данным местом, с данным веком, с данной природой, лишь в соотношении с данной окружающей средой, обстоятельствами, условиями, событиями, которые составляют содержание моей биографии. Это единственный разумный смысл, лежащий в основании веры, что человек обязан тем, что он есть и что он имеет, не себе, не своим заслугам, не одной своей собственной силе, а богу. Но с тем же правом, как нельзя добро относить лишь на мой счет, точно так же нельзя относить лишь на мой счет и зло; не моя вина, по крайней мере не одна моя, а вина также и обстоятельств, вина людей, с которыми я с самого начала находился в соприкосновении, вина времени, в которое я родился и получил образование, что я имею эти недостатки, эти слабости. Как каждый век имеет свои собственные болезни, точно так же имеет он и свои собственные преобладающие пороки, то есть преобладающие склонности к тому или другому, что само по себе не плохо, но делается плохим или порочным лишь благодаря своему преобладанию, благодаря тому, что оно подавляет другие  равноправные склонности или влечения. 

Этим, впрочем, отнюдь не уничтожается свобода человека, по крайней мере разумная, имеющая свое основание в природе, та свобода, которая выявляется и познается как самодеятельность, трудолюбие, упражнение, образование, самообладание, напряжение, старание; ибо век, обстоятельства и отношения, естественные условия, при которых я сложился, --- не боги, не всемогущие существа. Природа скорее, напротив, предоставляет человека самому себе; она ему не помогает, если он сам себе не помогает; она дает ему утонуть, если он не умеет плавать, бог же не дает мне погибнуть в воде, хотя бы я и не был в состоянии продержаться собственными силами и умением. Уже древние имели поговорку: <<если этого захочет бог, то ты сможешь проплыть и на сите>>. Даже животное должно само себе отыскивать средства пропитания, должно испытать всякого рода лишения, должно пустить в ход все имеющиеся в его распоряжении силы, пока оно найдет себе пищу; как должна мучиться гусеница, пока она отыщет подходящий листок, как должна страдать птица, пока она изловит насекомое или другую птицу! Но бог избавляет людей и животных от самодеятельности; ибо он о них заботится; он деятельное начало; они --- страдательное, воспринимающее. Так, по приказанию господа вороны приносили Илии <<хлеб и мясо утром и вечером>>. Но <<кто готовит пищу ворону>>? Бог, <<который дает скоту свой корм>>  как это значится в псалмах и в книге Иова, <<готовит ее молодым воронам, его призывающим>>. Поэтому разумная свобода, самостоятельность и самодеятельность людей, вообще индивидуальных существ, согласуется с природой, но не со всемогущим, всезнающим и намеренно предопределяющим богом. Все бесчисленные, отравляющие сердца и приводящие в смятение головы противоречия, затруднения и софизмы, которые создаются в теологии самодеятельностью и самочинным проявлением себя созданий, творений, несовместимых с богом, как с существом, которое действует только одно или по преимуществу одно, --- все они исчезают или по крайней мере делаются разрешимыми, если на место божества поставить природу. 

Как теисты морально плохое, злое вменяют человеку и только добро выводят из бога, точно так же вменяют они и физически плохое, зло природе, частью прямо, частью косвенно, частью открыто, частью молчаливо --- материи или неизбежной природной необходимости. Если бы не было этого плохого, то не было бы и доброго, говорят они, если бы человек не голодал, то он не имел бы никакого удовольствия от еды и никакого влечения к ней, если бы он не мог сломать себе ноги, то у него не было бы костей, он, стало быть, не мог бы ходить; если бы он при ранении не испытывал болей, то у него не было бы и побуждений предохранять себя от них; поэтому поверхностные раны гораздо более болезненны, чем глубокие. Глупо поэтому, говорят они, когда атеисты приводят зло, страдания, горести жизни как доказательства против существования благого, мудрого, всемогущего творца. И, конечно, совершенно правильно, что, если бы не было того или иного зла, не могло бы и быть того или иного добра; но эта необходимость имеет значение только для природы, а не для бога. Как бог есть существо, в котором теист мыслит себе блаженство при отсутствии неблаженства, совершенство без несовершенства, так же точно и так же необходимо связывается с богом и представление о том, что он может творить добро, не творя зла, мир без страданий и недостатков. Потому-то и верит христианин в будущий мир, в котором это действительно осуществлено, в котором действительно устранено то, что атеист приводит как опровержение божественного происхождения мира. И ведь древние христиане уже имели этот мир в раю. Если бы Адам остался в состоянии невинности, совершенства, в котором он вышел из рук бога, то его тело было бы неразрушимо и неуязвимо, природа вообще осталась бы предохраненной от всех тех зол и недостатков, которые над ней сейчас тяготеют. Все те соображения, которыми теисты оправдывают зло мира, то есть в данном случае мира естественного, не гражданского, имеют значение лишь в том случае, если принимать природу за основу существования вещей, если мыслить себе природу как первую причину, но не тогда, когда бога считают творцом мира. Поэтому и в самом деле в основе всех теодицей, всех оправданий бога находится, сознательно или бессознательно, природа как что-то самостоятельное; существом и деятельностью природы они ограничивают деятельность бога, всемогущество бога, свободу бога, --- которая могла бы ведь сотворить мир совсем иначе, чем каков он есть, --- представлением о необходимости, которая происходит ведь только от природы, только ей соответствует. Это особенно дает себя знать в господствующих представлениях о провидении. Так, например, парижский архиепископ опубликовал в 1846 г. послание, в котором он призывает верующих к молитвам, <<дабы при избрании папы никакие чужеродные влияния не оказали противодействия богомилостивым намерениям>>. Так, недавно (в январе 1849 г.) прусский король издал приказ по армии, в котором говорится: <<в истекшем году, когда Пруссия подпала власти соблазна и государственной измены, оставшись без помощи божией, моя армия сохранила свою старую славу и завоевала новую>>. Но что это за слабое существо, чьим милостивым намерениям могут противодействовать и противостоять чужеродные влияния! Что это за помощь бога, которая без помощи штыков и шрапнелей не имеет силы и успеха? Что это за всемогущество, которое для своей поддержки нуждается в военной силе? Что это за бог, который свою славу делит со славой королевско-прусской армии? Или предоставьте честь одному богу, как это делали древние теисты и христиане, верившие, что бог может побеждать без штыков и шрапнелей, что врагов можно побеждать одной лишь молитвой, что молитва, то есть сила религии, или, что то же, сила бога, всемогуща; или же отдайте честь грубости материальных сил и средств, которая вам помогла. 

На этих примерах, число которых можно было бы, впрочем, увеличить до бесконечности, ибо каждый осведомительный листок, газета дают образцы того, как безбожны даже и современные верующие в бога, как они своего бога в действительности отрицают и принижают, между тем как на словах они ему воздают хвалы --- отрицают и принижают тем, что приписывают материи, миру, человеку силу и деятельность, от бога независимую и самостоятельную, самому же богу уделяют роль праздного зрителя или инспектора, самое большее лишь помощника или спасителя в крайней нужде. Уже обычное выражение <<помощь бога, содействие бога>> характеризует этот безобразный разлад между богом и природой, ибо тот, кто мне помогает, содействует, не упраздняет моей деятельности; он только поддерживает меня, он только берет на себя часть работы, часть бремени. Но какое недостойное представление о боге: веря в бога, отказывать ему во всемогуществе, по крайней мере на деле, присоединять к нему силу природы и человека и прибегать к ее помощи! Если надо мною бдит глаз, то зачем мне самому иметь глаз, зачем остерегаться? Если бог обо мне печется, зачем мне самому о себе заботиться? Если существует благое и в то же время всемогущее существо, то что мне ограниченная мощь естественных средств и сил? Впрочем, не будем порицать людей западных стран за то, что они не доводят своей религиозной веры до ее практических выводов, что они, наоборот, самовольно устраняют эти следствия их веры, что они в действительности, на практике от своей веры отказываются; ибо только этой непоследовательности, этому практическому неверию, этому инстинктивному атеизму и эгоизму обязаны мы всем прогрессом, всеми изобретениями, которыми христиане отличаются от магометан, люди западных стран --- от восточных. \emph{Кто полагается на всемогущество бога, кто верит, что все, что происходит и существует, происходит и существует по милости божией, тот никогда не будет думать о средствах, как устранить существующее зло, --- ни как устранить естественное зло, поскольку оно не устранимо, ибо против смерти лекарств не будет найдено, ни как устранить зло общественного порядка}. <<Каждому, говорит Кальвин в уже многократно упоминавшемся мною сочинении, указывается божеством его положение и его состояние. Соломон поэтому в своем изречении: <<жребий бросается в подол, но выпадает он так, как угодно господу>> --- призывает бедных к терпению, ибо те, что недовольны своим жребием, пытаются сбросить с себя бремя, возложенное на них богом. Так и другой пророк, псалмопевец, порицает безбожников, приписывающих человеческому искусству или счастью достижение некоторыми людьми почетных мест, тогда как другие пребывают в низком состоянии>>. 

Это --- необходимое следствие веры в бога, веры в провидение, когда эта вера является не просто теоретической, бездеятельной, неверующей верой, но верой истинной, практической. Некоторые отцы церкви считали даже безбожной критикой дел божиих отрезать себе бороду. 

Совершенно верно! Борода обязана своим существованием воле и намерению бога, касающимся самых мелких подробностей; если я даю себе отрезать бороду, то я этим выражаю недовольство; я косвенно порицаю творца бороды; я восстаю против его воли; ибо бог говорит: <<да будет борода!>> тем, что он дает ей расти, а я говорю: <<борода да не будет!>> тем, что я даю ее себе отрезать. Все оставить таким, каково оно есть, --- вот необходимый вывод из веры в то, что бог правит миром, что все происходит и существует по воле божией. Каждое самовольное изменение существующего порядка вещей есть святотатственная революция. Как в абсолютистски-монархическом государстве правительство не предоставляет народу ничего и присваивает себе всю политическую деятельность, так и в религии бог не оставляет ничего на долю человека, пока бог еще абсолютное, неограниченное существо. <<Поэтому, --- говорит Лютер в своем толковании Экклезиаста, --- наилучшая и наивысшая мудрость предоставить и поручить все богу\dots дать богу действовать и управлять и все, что случается несправедливого или что приносит горе праведным людям, отдать на волю того, который, в конце концов, все точно и правильно устроит\dots Поэтому, если ты очень хочешь иметь радость, мир и хорошие дни, то подожди, чтобы их дал тебе бог>>. Но, как сказано, христиане, к своему и нашему счастью, и сообразно духу и характеру Запада, в особенности же германского племени, противопоставили выводам своих взятых с Востока религиозных учений и представлений человеческую самодеятельность, разумеется, сделав тем самым свою религию, свою теологию, которой они, однако, держатся, по крайней мере теоретически, и до настоящего дня, переплетом самых смешных противоречий, половинчатостей и софизмов, невыносимой, бесхарактерной смесью веры и неверия, теизма и атеизма. 

\phantomsection
\addcontentsline{toc}{section}{Девятнадцатая лекция}
\section*{Девятнадцатая лекция}

Камчадалы, как рассказывают и выражаются путешественники-теисты, имеют высшего бога, которого они называют Куткой и которого они считают творцом неба и земли. Им, говорят они, все сотворено, и от него все произошло. Но они считают себя гораздо умнее бога и никого не считают таким глупым, бессмысленным и бездумным, как своего Кутку. Если бы, говорят они, он был умен и рассудителен, то он создал бы мир гораздо лучше, не поставил бы так много непроходимых гор и утесов, не сотворил бы так много бешеных потоков и длительных бурь. Поэтому, когда вы въезжаете зимою на гору или с нее спускаетесь, вы не можете удержаться от того, чтобы не обругать самым ужасным образом Кутку. <<Мы справедливо приходим в ужас по поводу этих безумств>>  --- замечает по этому поводу один писатель-рационалист. Я же этому нисколько не ужасаюсь; я гораздо больше удивляюсь тому, что у христиан так мало самопознания и что они не замечают того, что они по существу нисколько не отличаются от камчадалов. Они только тем отличаются от камчадалов, что они своему гневу по поводу суровости и грубости природы дают выход не как камчадалы, в ругательствах, а в действиях. Христиане выравнивают горы или по крайней мере проводят через них проходимые, удобные дороги; они бешеным потокам противоставят плотины или отводят их; короче говоря, они, сколько только могут, меняют природу согласно своему пониманию, для своей пользы. Однако каждое такое дело есть критика природы; я не снесу горы, если я предварительно не рассержусь на то, что она существует, не предам анафеме, не прокляну ее; тем, что я ее сношу, я только превращаю это проклятие в действие. Против постоянных бурных ветров, дающих камчадалам основание проклинать их виновника, христиане, правда, не открыли прямого средства борьбы, как и вообще воздушное царство всего менее познано и побеждено; но христиане знают другие средства, которые культура дает им в руки, чтобы предохранить себя от капризов климата. В Библии, правда, говорится: <<оставайся в стране и честно прокармливай себя>>; и, однако, христиане путешествуют, разумеется, в том случае, если <<провидение>> дало им для этого средства, ездят на воды, вообще в страны, где они находят лучший, для них более подходящий климат. Но если я покидаю какое-либо место, то я его в действительности проклинаю, ругаю; я думаю или, быть может, сам говорю: здесь совсем проклятый климат; здесь я не могу дольше выдержать; здесь я погибну; итак, вон отсюда. Но когда христианин покидает свое отечество, временно или навсегда, то он практически отрицает тем самым свою веру в божественное провидение; ибо ведь это оно поместило его в этом месте, потому что оно сочло это место, несмотря на его неприятный и нездоровый климат или, вернее, благодаря ему, наиболее для него --- христианина --- подходящим и, стало быть, предопределило его жительство. 

Провидение ведь простирается на каждую особь, на каждую единицу; провидение, как его мыслят себе рационалистические теисты, провидение, простирающееся только на род, на общее, на общие естественные законы, есть провидение разве только по имени. Поэтому, если я покидаю данное место, на которое посадило меня провидение, если я сношу данную гору, которую оно, очевидно, намеренно поставило такой высоты и как раз на этом месте, если я противопоставляю плотину данному бурному потоку, который, очевидно, обязан своей силой только воле и могуществу божьему, то я отрицаю, я отвергаю своей практической деятельностью мою религиозную теорию и мою веру в то, что все, сделанное богами, хорошо, что все, созданное богами, мудро, не подлежит порицанию, не может быть превзойдено, ибо ведь бог создал все не на скорую руку, так лишь в общих чертах, а каждое в отдельности, Как могу я, стало быть, делать насильственные изменения, как могу я подчинять божественные предначертания моим человеческим целям, как могу я противопоставлять могуществу бога, выразившемуся в силе того бешеного потока, в высоте этой горы, человеческую силу? Я этого не могу, если я хочу свою веру подтвердить на деле. Когда жители Книда, рассказывает Геродот, хотели прорыть часть страны, чтобы сделать из своей страны настоящий остров, то пифия запретила им это в следующих стихах: 
\begin{quote}
    

Не смейте Истма прорывать и укреплять! 

Зевс остров, захоти он, сотворил бы сам.

\end{quote}

И когда Риму было сделано предложение отвести притоки Тибра, чтобы воспрепятствовать его разливам, то реатинцы, как рассказывает Тацит в своих <<Анналах>>  воспротивились этому, говоря, что природа --- что в данном случае, очевидно, то же, что бог, --- наилучшим образом позаботилась о человеческих интересах, дав рекам их устье, их течение, их исток, точно так же, как и их впадение. 

Поэтому все культурные средства, все изобретения, которые сделал человек, чтобы оберечь себя от насилий природы, как, например, громоотводы, последовательная религиозная вера осуждала, как посягательства против божеского управления, осуждала --- кто бы мог это подумать? --- еще даже в наше время. Когда был открыт и применен серный эфир в качестве болеутоляющего средства, то --- как мне рассказал один вполне заслуживающий доверия человек теологи одного протестантского университета, университета в Эрлангене запротестовали против этого, а именно против применения его при тяжелых родах, ибо в Библии сказано: <<в муках будешь ты рождать>>  следовательно, рождение в муках есть прямое предписание, волеизъявление бога. Таким глупым и таким одновременно дьявольским делает человека теологическая вера! Вернемся, однако, опять назад от протестантских теологов и университетов к камчадалам, у которых гораздо больше разума, ибо они совершенно правы, когда творца крутых, человеческой культуре не доступных, гор, бешеных потоков, уничтожающих посевы и нивы, постоянных бурных ветров считают за неразумное и бессмысленное существо, ибо природа слепа и лишена разума; она есть то, что она есть, и делает то, что она делает, не намеренно, не умышленно, а по необходимости, или, если мы человека, как и следует, причислим к природе ведь он тоже естественное существо, создание природы, --- то она имеет свой разум только в разуме человека. Ведь только человек своим устроительством и своими учреждениями накладывает на природу печать сознания и разума, только он, чем дальше, тем больше с течением времени преобразует Землю в разумное, человеку соответствующее жилище и когда-нибудь преобразует ее в еще более человеческое, более разумное жилище, чем то, которое существует теперь. Ведь даже климат изменяется под влиянием человеческой культуры. Что такое сейчас Германия и чем она когда-то была, даже еще во времена Цезаря! Но как согласовать такие насильственные преобразования, произведенные человеком, с верой в сверхъестественное, божественное провидение, которое все сотворило и о котором говорится: <<бог взглянул на псе, что он сотворил, и оказалось, что все было хорошо>>. 

Еще одно утверждение должен я разъяснить в немногих словах. Я сказал, что существование провидения пытались доказать главным образом такими явлениями природы, которые помогают или предохраняют против последствий существующего или естественно-необходимого зла. В особенности усмотрели поэтому доказательства особого провидения в оружии животных, которым они защищаются против своих врагов, и в защитных средствах органов человеческого и животного тела. Так, <<глаз защищен ресницами от проникновения в него мешающих ему материальных частиц, бровями --- от капающего со лба пота, глазными костями --- против поранении, а веком он может быть совсем прикрыт>>. Но почему не защищен глаз от пагубных последствий удара кулаком, брошенного камня или других влияний, разрушающих глаз или по крайней мере нарушающих силу зрения? Потому что существо, образующее глаз, не есть всемогущее и всезнающее существо, не есть бог. Если бы глаз сотворило всевидящее око и всесильная рука, то и глаз был бы защищен против всех возможных опасностей. Но существо, образовавшее глаз, не подумало при образовании его о брошенном камне, об ударе кулака и о бесчисленных других разрушительных воздействиях, потому что природа вообще не думает, а следовательно, и не знает наперед об опасностях, могущих постигнуть какой-либо орган или существо, как это знает бог. Каждое существо, каждый орган защищен лишь против определенных опасностей, против определенных воздействий, и эта защита едина с определенными свойствами этого существа, этого органа, едина с его существованием, так что без этой защиты он и существовать не мог бы. Раз что-нибудь должно существовать, то оно должно иметь и средства для существования, раз что-нибудь должно жить и хочет жить, то оно должно быть и в состоянии отстоять свою жизнь, защитить ее, стало быть, против враждебных нападений. Жизнь есть борьба, война; непосредственно вместе с жизнью дано одновременно и оружие как средство ее сохранения. 

Глупо поэтому нарочито выдвигать оружие, средства защиты и делать их доказательством существования провидения. Если жизнь необходима, то необходимо и средство сохранения жизни. Раз есть война, то есть и оружие: нет войны без оружия. Стало быть, если хотят удивляться средствам защиты органа, защиты животного, то следует удивляться существованию этого органа, этого животного. Все эти средства защиты по своей природе ограничены и составляют нечто единое со свойствами данного органа, данного существа; но именно в силу своего единства с природой существа, с природой органа, они не являются доказательствами наличия намеренно или произвольно творящего существа, и именно в силу своей ограниченности не являются доказательствами существования всемогущего и всезнающего бога, ибо бог защищает существо, защищает орган против всех, какие только возможны, опасностей. Каждое существо возникло при условиях, заключавших в себе не больше того, что было достаточно для создания данного существа, каждое существо стремится посильно отстоять себя, сохранить себя, сколько это только возможно, сколько позволяет ему его ограниченная природа; каждое существо имеет инстинкт самосохранения. Из этого инстинкта самосохранения, который, однако, един с индивидуальной природой органа или существа, но отнюдь не от всемогущего и всезнающего существа происходят оружия, средства защиты животных и органов. 

Наконец, я должен еще упомянуть об одном возражении, которое делают теисты против прежних атеистов, или натуралистов, производивших людей и животных из природы без участия бога, производивших, впрочем, способом, не являвшимся, разумеется, достаточно научным. Если природа некогда создала животных и людей, первично произведя их на свет без уже имеющихся животных и людей, то почему этого не случается больше теперь? Я отвечаю: потому что всему в природе есть свое время, потому что природа может лишь то, к чему даны необходимые условия; если поэтому теперь уже больше не происходит того, что происходило некогда, то были тогда такие условия, которые теперь отсутствуют. Но, может быть, когда-нибудь наступит время, когда природа сделает опять то же самое, когда старые породы животных и люди исчезнут и новые люди, новые породы возникнут. Вопрос, почему этого больше не происходит теперь, представляется мне таким же, как если бы спросили, почему дерево дает плоды только осенью, а цветы --- весной, разве не могло бы оно без перерыва цвести и давать плоды? Или почему данное животное лишь в данное время находится в периоде течки, разве не могло бы оно постоянно вожделеть и беременеть? Только индивидуальность, только, так сказать, неповторяемость есть соль земли, соль природы; только индивидуальность есть оплодотворяющее и творческое начало; только совершенно индивидуальные условия или отношения на земле, революции земли, которые в том виде, как они были, больше не повторились, создали органические существа, по крайней мере те, которые со времени последней геологической эпохи на земле существуют, и в том виде, как они существуют. И человек, или дух человеческий, не всегда, не во всякое время создает оригинальные вещи; нет! Всегда бывает только определенная эпоха в жизни человека, счастливейшая, наиболее благоприятная, бывают события в жизни, жизненные моменты, жизненные условия, которые позже не встретятся, не повторятся, по крайней мере в своей первичной свежести, только в эти моменты человек создает оригинальные произведения; в большинстве же других моментов он только повторяется, он только умножает свои оригинальные создания путем ординарного, обычного их воспроизведения. 

Этим замечанием я заканчиваю главу о природе. Я этим выполнил первую часть своей задачи. Эта задача заключалась в том, чтобы доказать, что человек должен вести свое происхождение не от неба, а от земли, не от бога, а от природы, что человек должен начинать свою жизнь и свое мышление вместе с природой, что природа не есть действие отличного от нее существа, но --- как говорят философы --- есть причина себя самой, что она не творение, не существо, сделанное или созданное из ничего, а существо самостоятельное, объяснимое лишь из себя и производимое лишь из себя, что происхождение органических существ, происхождение Земли, происхождение даже Солнца, если мы его мыслим себе происшедшим, всегда было естественным процессом, что мы, чтобы наглядно представить себе их происхождение и сделать его понятным, должны исходить не от человека, художника, ремесленника, мыслителя, строящего мир из своих мыслей, а от природы, как древние народы, которые, следуя своему верному природному инстинкту --- по крайней мере в своем религиозном и философском учении о происхождении мира --- сделали естественный процесс, процесс рождения, прообразом и творческим принципом мира, что как растения произошли от ростка, животное от животного, человек от человека, так и все в природе от естественного существа, ему подобного, ему родственного по своему веществу и существу, короче говоря, что природа не может быть произведена от духа, не может быть объяснена из бога, ибо все свойства бога, поскольку они не очевидно человеческие, сами взяты и произведены от природы. 

Но как ни очевидно само собой, что чувственное, телесное существо природы не может быть выведено из духовного, то есть абстрактного, существа, --- есть все же в нас нечто, что делает для нас это выведение правдоподобным и заставляет казаться естественным, даже необходимым, нечто, что противится тому, чтобы естественное, чувственное, телесное существо мыслить себе как первое, изначальное, как существо, через которое нельзя перейти; есть в нас нечто, из чего вышла вера, представление о том, что мир, природа, есть продукт духа, что она даже произошла из ничего. Но я это возражение уже устранил и разъяснил, показав, что человек из чувственного выводит общее и это общее затем предпосылает чувственному как основание. 

Поэтому способность человека к абстракции и с ней связанная сила воображения (потому что только силой своего воображения делает человек самостоятельными абстрактные, общие понятия, мыслит себе их как существа, как идеи) побуждают его выйти за пределы чувственного и производить телесный, чувственный мир от нечувственного, абстрактного существа. Но глупо эту субъективную, человеческую необходимость делать объективной потому, что человек, когда он возвысился однажды от чувственного к сверхчувственному, то есть к мыслимому, абстрактному, общему, --- затем от общего, абстрактного спускается к конкретному и это конкретное выводит из общего, глупо, не замечая этого, выводить это конкретное из общего. Что это ошибочно, явствует уже из того, что для того, чтобы телесное, материальное иметь возможность вывести из духа, приходится прибегать к пустому, фантастическому представлению о сотворении из ничего. Но если я говорю, что мир сотворен из ничего, то я этим ничего не говорю; это ничто есть пустая отговорка, при помощи которой я уклоняюсь от вопроса: откуда же взял дух недуховные, материальные, телесные вещества для мира? Это ничто, хотя оно было некогда таким же священным пунктом веры, как и существование бога, есть не что иное, как одно из бесчисленных теологических, или поповских, ухищрений и уловок, которыми в течение веков обманывали людей. Сказать, как Яков Беме и Гегель, вместо: бог создал мир из ничего, --- он создал его из себя как из духовной материи, --- значит лишь уклониться от этого <<ничего>>. Как я уже раньше показал, я таким образом не подвинусь ни на один шаг вперед, ибо как вывести из духовной материи, из бога вообще материю действительную? Сколько бы поэтому ни измышляли теологических и спекулятивных ухищрений и уловок, чтобы иметь возможность произвести мир от бога, но одно остается в силе: то, что делает мир миром, чувственное --- чувственным, материю --- материей, есть нечто, что теологически и философски не может быть откуда-нибудь выведено и объяснено, нечто непроизводное, имеющее просто свое бытие, нечто, что можно брать лишь как таковое, понимаемое лишь из себя и через посредство себя. Этим я заканчиваю первую часть своей задачи. 

Я перехожу теперь ко второй, и последней, части моей задачи, которая заключается в том, чтобы доказать, что бог отличный от природы есть не что иное, как собственное существо человека, точно так же, как в первой части мне надлежало показать, что отличный от человека бог есть не что иное, как природа или существо природы. Или: в первой части мне нужно было доказать, что существо естественной религии есть природа, что в природе и естественной религии ничто другое не раскрывается и не обнаруживается, как природа; теперь мне надлежит доказать, что в религии духа ничто другое не выражается и не раскрывается, как существо человеческого духа. Я уже в своих первых лекциях объяснял, что я в этих лекциях не касаюсь второстепенных различий в религии, что я свожу религию всего лишь к двум разновидностям или противоположностям, к естественной религии и к религии человеческой, или духовной, --- к язычеству и христианству. Я перехожу теперь поэтому от сущности естественной религии, или язычества, к сущности христианства. Но прежде, чем переходить к самому христианству, необходимо, по крайней мере вкратце, охарактеризовать переходные ступени, те побуждения, которые отвлекают человека от природы, возвращают его назад к самому себе, заставляют его искать свое спасение не вне себя, а в себе; при этом, однако, необходимо подробно коснуться и тех моментов, которые одинаково относятся и к религии духа, и к естественной религии, следовательно, к религии вообще, и которые имеют чрезвычайную важность для понимания сущности религии; эти моменты, однако, соответственно той последовательности, которой человек вынужден следовать как в речи, так и в мышлении, могут быть лишь теперь разобраны, по крайней мере, полностью. О переходе от естественной религии к теизму в собственном смысле слова, или монотеизму, говорят в <<Сущности религии>> параграфы 26-41. 

Природа есть первый предмет религии, но природа там, где она религиозно почитается, является для человека предметом не как природа, какой она является нам, а как человекоподобное или, вернее, человеческое существо. Человек, стоя на точке зрения естественной религии, молится солнцу, потому что он видит, как все от него зависит, как ни одно растение, ни одно животное, ни один человек не может без него существовать, но он все же не почитал бы его религиозно, не молился бы ему, если бы не представлял себе солнце в виде существа, которое по собственной воле, как человек, движется по небу, если бы не представлял себе действия солнца в виде добровольных даров, которые оно исключительно по доброте своей посылает земле. Если бы человек рассматривал природу такой, какова она есть, глазами, которыми мы ее рассматриваем, то отпало бы всякое побудительное основание для религиозного почитания. Ведь чувство, влекущее человека к почитанию какого-либо предмета, предполагает заранее, что предмет к этому почитанию не нечувствителен, что он, стало быть, имеет чувство, что он имеет сердце и притом человеческое сердце, чувствительное к человеческим обстоятельствам. Так, греки во время персидской войны молились ветрам, принося им жертвы, но только потому, что они смотрели на них, как на соратников, как на союзников против персов. Афиняне особенно почитали Борея, северный ветер, и просили его о содействии, но они и рассматривали его, как рассказывает Геродот, как им дружественное и даже родственное существо, ибо он имел своей женой дочь их царя Эрехтея. Но что же превращает естественный предмет в человеческое существо? Фантазия, сила воображения. Это она представляет нам существо иным, чем оно есть в действительности; это она заставляет природу являться человеку в том чарующем и заколдовывающем глаз, ослепляющем свете, для которого человеческий язык придумал выражение: божественность, божество, бог: это она, стало быть, создала богов людей. Я уже говорил, что слово <<бог>>  <<божество>> есть первоначально только общее имя, а не имя собственное, что слово <<бог>> первоначально означало не субъект, а определение, то есть не существо, а качество, подходящее или прилагаемое к каждому предмету, который представляется человеку в свете фантазии божественным существом, который производит на человека, так сказать, божественное впечатление. Всякий предмет может поэтому сделаться богом, или, что то же, предметом религиозного почитания. Я говорю, что одно и то же: бог или предмет религиозного почитания; ибо нет другого отличительного признака для божества, как религиозное почитание: бог есть то, что религиозно почитается. Но религиозно почитается предмет только в том случае, если и поскольку он является существом, предметом фантазии или силы воображения. 

\phantomsection
\addcontentsline{toc}{section}{Двадцатая лекция}
\section*{Двадцатая лекция}

Всякий предмет не только может почитаться, но и на самом деле почитается человеком за бога, или --- что то же, --- религиозно почитается. Эта точка зрения есть так называемый \underline{фетишизм}, когда человек без всякой критики и различения делает своими богами всевозможные предметы и вещи, будут ли они искусственными или естественными произведениями природы или человека. Так, например, негры в Сиерра-Леоне избирают себе своими богами рога, раковые клешни, гвозди, булыжники, раковины улиток, головы птиц, корни; носят их на шее в мешочке, убранном бусами и другими украшениями (Бастгольм, в указанном месте). <<Таитяне поклонялись флагам и вымпелам европейских кораблей, мадагаскарцы считали за богов математические инструменты, остяки оказывали религиозное почитание нюренбергским часам, имевшим фигуру медведя>> (Майнерс, в указанном месте). Но почему люди делали своими богами раковины улиток, раковые клешни, флаги и вымпела? Тому причиной фантазия, сила воображения, которая тем могущественнее, чем более велико невежество человека. Дикари не знают, что такое часы, флаг, математический инструмент; они поэтому воображают, что эти предметы представляют собой что-то иное, чем они являются в действительности; они делают из них фантастическое существо, \underline{фетиша}, бога. \emph{Теоретическая причина, или источник, религии и ее предмета, бога, есть поэтому фантазия, сила воображения}. 

Христиане обозначают теоретическую способность к религии словом вера. Религиозный и верующий для них --- одно и то же; одно и то же --- неверие и отрицание бога, или нерелигиозность. Но если мы ближе исследуем, что обозначает это слово, то окажется, что оно не что иное, как сила воображения. Вера, говорит Лютер, величайший авторитет в данной области, величайший герой веры среди немцев, как его называли, немецкий апостол Павел, <<вера>>  говорит он, например, в своем толковании первой книги Моисея, <<поистине всемогуща\dots Для верующего все возможно. Ибо вера делает так, что ничто становится чем-то, и из предметов, как бы они ни были невозможны, она делает возможное>>. Но это всемогущество веры есть лишь всемогущество фантазии, силы воображения. Символами христианской веры --- по крайней мере, согласно вере Лютера --- являются крещение и причастие. Вещество, материя крещения есть вода, материя причастия вино и хлеб, но для веры естественная вода крещения есть духовная вода, как говорит Лютер, хлеб есть тело, вино кровь господа, то есть это сила воображения превращает вино в кровь, хлеб в тело. Вера верит в чудеса, можно даже сказать, что вера и вера в чудеса едины; вера не связывает себя законами природы; вера свободна, неограничена; она полагает все возможным. <<Разве для господа что-нибудь невозможно?>> Но эта не связанная никакими законами природы сила веры, или сила бога, именно и есть сила воображения, для которой нет ничего невозможного. Вера устремляет свой взор на невидимое; <<вера относится не к тому, что видимо, говорится в Библии, --- а к тому, что невидимо>>. Но и сила воображения имеет дело не с тем, что видимо, а с тем, что невидимо. Сила воображения имеет дело только с предметами и существами, которых уже более нет, или еще нет, или которых, по крайней мере, нет налицо. <<Вера, --- говорит Лютер в указанном толковании, --- непосредственно тяготеет к тому, что еще есть сплошное ничто, и ждет, чтобы оно стало всем>>. <<Вера, --- говорит он в другом месте, уже отмеченном в моем ,,Лютере``  --- имеет дело, собственно говоря, только с будущим, а не с настоящим>>. Поэтому верующий не падает духом, если ему сейчас плохо; он надеется на лучшее будущее. Но главный предмет силы воображения есть именно будущее. Прошлое, хотя и является также предметом фантазии, не занимает нас, не интересует нас так, как будущее; ибо оно лежит позади нас; оно неизменно; оно прошло. Чего же нам о нем много заботиться? Но другое дело --- будущее, которое нам еще предстоит. И, конечно, в этом отношении Лютер совершенно прав, когда он порицает неверие в будущее, когда он порицает человека за то, что тот отчаивается, не находя выхода в настоящий момент, ибо сегодняшний день не есть день светопреставления; настоящее не есть конец истории. \emph{Все еще может сложиться по-иному, чем в настоящее время, как бы ни был печален взгляд на настоящее}. В особенности это относится к социальным и политическим делам, к делам, которые касаются человечества в его целом; ибо отдельного человека постигает, конечно, несчастье, когда надежда на улучшение или на перемену исчезает, когда <<отчаяние является обязанностью>>. 

Бог, говорят христиане, не есть предмет чувственности; его нельзя видеть, нельзя чувствовать; но он не является также, говорят, по крайней мере, правоверные христиане, и предметом разума; ибо разум опирается только на чувство; бога нельзя доказать, в него можно лишь верить, или бог не существует для чувств, для разума; он существует лишь для веры, то есть он существует лишь в воображении. Лютер говорит в своем <<Собрании церковных проповедей>>: <<Я часто говорил, что бог в отношении к людям проявляет себя таким, каким он мыслится, каким ты мыслишь бога, в какого веришь, таким ты его и имеешь. Кто его рисует себе в своем сердце милостивым или гневным, добродушным или угрюмым, тот его таким и имеет. Когда ты думаешь, что он на тебя сердится и тебя не хочет знать, то так тебе и будет. Но если ты можешь сказать: я знаю, что он хочет быть моим милостивым отцом и так далее, то ты его и имеешь таким>>. <<Каким мы его чувствуем, --- говорит он в своих ,,Проповедях о первой книге Моисея`` --- таков и он по отношению к нам. Если ты мыслишь его себе гневным и немилостивым, то он и на самом деле немилостив>>. <<Если ты его, --- говорит Лютер в своем толковании второго послания св. Петра, считаешь за бога, то он и действует для тебя, как бог>>. Это значит: бог таков, каким я его себе представляю в своей вере, каким я его воображаю; или: свойства бога зависят от свойства моей силы воображения. Но то, что относится к свойствам, относится и к бытию бога. Если я верю, что бог есть, то бог есть именно для меня; и точно так же: если я не верю, что он есть, то для меня и нет никакого бога. Короче говоря, бог есть существо, созданное воображением, существо, принадлежащее фантазии; и так как фантазия есть форма или орган поэзии, то можно также сказать: религия есть поэзия, бог есть поэтическое существо. 

Если религию представляют себе и определяют как поэзию, то напрашивается вывод, что тот, кто уничтожает религию, то есть кто разлагает ее на ее составные части, уничтожает и поэзию, и искусство вообще. И на самом деле такие выводы делались из моих разъяснений сущности религии, и по сему случаю воздымались в ужасе руки ввиду отвратительного одичания, которое вносится этим учением в человеческую жизнь, ибо оно лишает человечество поэтического воодушевления и вместе с религией разрушает поэзию. Но я был бы сумасшедшим, безумцем, если бы хотел упразднить религию в том смысле, какой мне в своих обвинениях приписывают противники. Я не упраздняю религии, не упраздняю субъективных, то есть человеческих, элементов и оснований религии, --- чувства и фантазии, стремления опредмечивать свой внутренний мир и олицетворять его, что уже заложено в природе языка и аффекта, не упраздняю потребности очеловечивать природу, делать ее предметом религиозно-философского поэтического воззрения, но делать это необходимо особым, существу природы отвечающим способом, как он известен нам благодаря естествознанию. Я упраздняю лишь предмет религии, вернее, той религии, которая существовала до сих пор; я хочу лишь, чтобы человек не привязывался больше своим сердцем к вещам, которые уже больше не соответствуют его существу и потребности и в которые он, стало быть, может верить, которые может почитать лишь в противоречии с самим собой. Есть, правда, много людей, у которых поэзия, фантазия привязаны к предметам традиционной религии и у которых, отняв эти предметы, отымаешь и всякую фантазию. Но многие --- все же не все, и что для многих необходимо, --- не есть еще необходимость сама по себе, и что сегодня необходимо, не является еще необходимым на вечные времена. Разве человеческая жизнь, история, природа не дают нам достаточно материала для поэзии? Разве живопись не имеет другого содержания, кроме того, какое она черпает в христианской религии? Я не только не упраздняю искусства, поэзии, фантазии, наоборот, я уничтожаю (aufhebe) религию лишь постольку, поскольку она является простой прозой, а не поэзией. Мы приходим, таким образом, к существенному ограничению положения: религия есть поэзия. Да, она --- поэзия, но с тем отличием от нее, от искусства вообще, что искусство не выдает свои создания за нечто другое, чем они есть на самом деле, то есть другое, чем создания искусства; религия же выдает свои вымышленные существа за существа действительные. Искусство не заставляет меня считать данный пейзаж за действительную местность, данное изображение человека --- за действительного человека, религия же хочет, чтобы я данную картину принимал за действительное существо. Простая точка зрения художника усматривает в древних статуях богов лишь произведения искусства; религиозная же точка зрения язычников в этих произведениях искусства, в этих статуях видела богов, действительные, живые существа, для которых они делали все, что постоянно делали для почитаемого и любимого действительного существа \hyperlink{17}{(17)}\hypertarget{b17}{}. Они привязывали изображения богов для того, чтобы они не убежали, они одевали и украшали их; угощали их дорогими кушаньями и напитками, укладывали их на мягких застольных диванах --- это делалось, по крайней мере, у римлян, с богами-мужчинами, богиням же не дозволялось лежать за столом, как когда-то и римлянкам, --- их купали и умащали, снабжали всеми принадлежностями человеческого туалета и тщеславия, зеркалами, полотенцами, щетками, слугами и служанками, делали им по утрам туалет, как знатным господам, услаждали их зрелищами и другими развлечениями. Августин приводит рассказ Сенеки про одного старого и дряхлого комедианта, который ежедневно давал свое представление в Капитолии, точно он мог доставлять богам удовольствие, после того как люди уже давно им были пресыщены. Именно потому, что изображения богов или статуи назывались богами и были богами, --- и скульптор или вообще тот, кто делал изображения, назывался теопойосом, то есть делателем богов; искусство скульптора называлось искусством делания богов. 

То же, что мы видим здесь у образованнейших народов древности, мы встречаем и в настоящее время у народов некультурных --- с той лишь разницей, что их боги и идолы не являют собой шедевров человеческого искусства, подобно греческим и римским. Так, например, остяки имеют своими идолами деревянные куклы с человеческими лицами. Большинство из них, однако, теперь христиане. <<И этих своих идолов они снабжают нюхательным табаком и прилагают при этом немного мочалки, убежденные, что идол, понюхавши табак, заткнет по-остяцки нос этой мочалкой. Если случится, что проезжие русские похитят этот табак ночью, когда все спят, то на утро остяки дивятся, как мог идол употребить так много табаку>> (Бастгольм, в указанном месте). Не только язычники, но и христиане были, и отчасти являются еще и до сих пор, почитателями изображении, и они считали и считают отчасти и до сих пор религиозные картины за действительные существа, даже за те самые предметы, которые на этих картинах представлены. Правда, ученые христиане отличали изображение от предмета; они говорили, что чтут только предмет через посредство изображения, а не самое изображение, чтут и молятся ему; но народ не считался с этим тонким различием. В греческой церкви, как известно, христиане в течение двух столетий даже вели друг с другом борьбу за и против почитания икон, пока, наконец, не победило почитание. Среди христиан особенно выделяются наши милые восточные соседи, русские, в качестве почитателей икон. <<Каждый русский имеет обыкновенно медное изображение св. Николая или другого какого-либо святого в своем кармане. Всюду носит он его с собой. Иногда можно видеть, как солдат или крестьянин вынимает своего медного бога из кармана, поплюет на него, вытрет его рукой, вычистит, поставит перед собой, упадет перед ним на колени, бесчисленное число раз крестясь, вздохнет и сорок раз произнесет: ,,господи помилуй``. Затем он своего бога кладет опять в карман и идет дальше>>. Далее, <<каждый русский имеет у себя дома несколько икон, перед которыми он возжигает свет. Если муж ложится спать со своей женой, то он предварительно завешивает иконы платком. Русские публичные женщины так же почтительны к своим святым. Когда они принимают гостей и хотят им отдаться, то они раньше всего закутывают свои иконы и гасят зажженные перед ними свечи>> (Штейдлин, Сборник по истории религий). Мы видим, заметим кстати, на этом примере, как легко человек обращается с моралью, исповедуя религию, с упразднением которой обычно предполагают и упразднение морали, как будто мораль не имеет своих самостоятельных основ. Ему достаточно всего только завесить изображение своего бога; но ему достаточно, если он не хочет поступать так аляповато, как поступает русская публичная женщина и русский крестьянин, ему достаточно над божественным правосудием раскинуть только плащ христианской любви, божественного милосердия, чтобы беспрепятственно делать то, что ему заблагорассудится. 

Я привел эти примеры поклонения иконам только для того, чтобы на них показать различие между искусством и религией. Обоим им обще то, что они создают изображения --- поэт создает образы из слов, живописец --- из красок, скульптор --- из дерева, камня, металла, --- но художник, если в его дело не вмешивается религия, ничего другого не требует от своих изображений, как только того, чтобы они были верны и прекрасны; он дает нам видимость действительности; но он эту видимость действительности не выдает за действительность. Религия же, напротив того, обманывает человека или, вернее, человек обманывает себя сам в религии; ибо она выдает видимость действительности за действительность; она делает из изображения живое существо, существо, которое, однако, живет лишь в воображении; в действительности же изображение есть только изображение, --- существо, которое именно поэтому есть божественное существо и божественным называется. Ибо самая сущность бога заключается в том, что он есть созданное воображением, недействительное, фантастическое существо, одновременно предполагаемое существом реальным, действительным. Религия не требует поэтому от своих изображений, подобно искусству, чтобы они были верны, соответствовали изображаемому предмету и были прекрасны; наоборот, истинно религиозные изображения --- самые некрасивые, безобразные. До тех пор, пока искусство служит религии, а не принадлежит самому себе, оно создает постоянно произведения, не могущие еще претендовать на название произведений искусства, как это доказывает история греческого и христианского искусств. Религия больше требует от своих изображений, чтобы они были полезны человеку, чтобы они помогали ему в нужде; поэтому --- так как ведь только живые существа могут помогать --- она придает своим изображениям жизнь и притом человеческую жизнь не только по видимости, по фигуре, как это делает художник, а и на деле, то есть придает им человеческое чувство, человеческие потребности и страсти, приносит сама им в дар пищу и питье. Как, впрочем, ни бессмысленно, что остяк ждет помощи от идола, который всем, что он имеет и что он собой представляет, обязан добродушию и силе воображения, ограниченности и невежеству остяка, что вообще человек ждет помощи от картин и статуй, --- в основании этой бессмыслицы лежит, однако, тот смысл, что, собственно говоря, только человек может помочь человеку, что бог, для того, чтобы помочь человеку, должен иметь человеческие чувства и, следовательно, человеческие потребности, ибо иначе он не будет иметь и сочувствия к человеческой нужде. Кто никогда не испытал, что такое голод, не поможет и в нужде голодающему. Но то, что имеет силу помочь, имеет силу и вредить. Религия, стало быть, в отличие от искусства, рассматривает изображения, создаваемые ею, как предметы чувства зависимости, как существа, которые имеют силу приносить пользу и вред, как существа, которым человек поэтому возносит хвалы и приносит жертвы, перед которыми он падает на колени, которым он молится, чтобы расположить их к себе. 

Я привел, однако, примеры поклонения иконам не для того, чтобы показать на них различие между искусством и религией только применительно к так называемым идолопоклонническим религиям; я привел их, потому что в них наглядным, доступным чувствам образом дает себя знать как сущность религии вообще, так и сущность христианской религии. Человеку приходится постоянно отправляться от чувственного, как от самого простого, бесспорного и явственного, и уже отсюда переходить к предметам сложным, абстрактным, удаленным от глаза. Различие между христианской религией и языческой заключается лишь в том, что изображения христианской религии, по крайней мере там, где она сохраняет свое отличие от язычества, где она сама не языческая и языческой не делается, являются не каменными, металлическими, деревянными или красочными изображениями, а изображениями духовными. Христианская религия опирается не на чувства, а --- как я говорил мимоходом в одной из своих первых лекций --- на слово, на слово божие, как называли Библию древние верующие христиане, которую они, как особое откровение божие, противопоставляли природе; не на силу чувственности, как язычники, которые силе чувственной любви и деторождения приписывали бытие, сотворение мира, а на силу слова; бог сказал; <<да будет свет, --- и был свет>>  да будет мир, --- и был мир. <<Слово божие, --- говорит Лютер, --- есть, следовательно, драгоценный, дорогой дар, который бог настолько высоко ценит и чтит, что он даже и небо, и землю, солнце, луну и звезды за ничто ставит по сравнению с этими словами, ибо, ведь, словом созданы все творения>>. <<Небо и земля прейдут, но мои слова не прейдут>>. Или, так как слово (субъективно для человека) имеет своим посредником слух, то можно сказать, как я уже раньше мимоходом заметил, что христианская религия опирается также на чувство, но только на слух. <<Отыми слово, --- говорит Кальвин в своем ,,Учении о христианской религии``  --- и не останется веры>>. <<Хотя человек, --- говорит он же, --- и должен серьезно обращать свой взор на созерцание божьих творений (то есть природы), но прежде всего и в особенности должен он устремить свои уши для восприятия слова, ибо образ божий, запечатленный в чудесной форме мира, недостаточно действенен>>. Именно поэтому и ратует Кальвин против всякого телесного изображения бога, ибо величие его не может быть уловлено глазом, и отвергает положение, высказанное вторым Никейским собором, гласящее, <<что бог познается не через одно лишь выслушивание слова, но и через созерцание изображений. Корнелий Агриппа фон Неттесгейм говорит в своем сочинении о недостоверности и тщете наук: <<Мы (а именно, христиане) не должны учиться из запрещенной книги изображений, но из книги божией, которая есть книга священного писания. Кто, следовательно, хочет познать бога, пусть не ищет его в изображениях живописцев и скульпторов, а изучает, как говорит Иоанн, в писании, ибо оно свидетельствует о нем. Те же, кто не могут читать, должны слушать слово писания, ибо вера их, как говорит Павел, идет от слуха. И Христос говорит у Иоанна: --- мои овцы слушают мой голос>>. <<Слово божие, --- говорит Лютер в своем толковании 18-го псалма, --- есть такое слово, что если не закрыть все другие чувства и не внимать ему одним слухом и с верой, то его нельзя восприять>>. 

Поэтому то и христианская религия устраняет чувства, кроме слуха, не делая их предметом своего почитания. Напротив того, языческий бог есть предмет и других, даже низших чувств: языческий бог, который имеет свое бытие в деревянных, каменных, красочных изображениях, который открывается и является человеку, тот может быть даже осязаем руками, но именно поэтому может быть разрушен и разбит; язычники сами разбивали своих богов и бросали их в ярости в грязь, если считали себя ими обманутыми, если не получали от них помощи. Короче говоря: языческий бог, как телесная вещь, подвержен всем возможным капризам природы и человеческого мира. Отцы церкви смеялись над язычниками за то, что те почитают за богов существа и вещи, к которым даже ласточки и другие птицы имеют так мало уважения, что их пачкают своим пометом. Христианский же бог, наоборот, не есть такое хрупкое и подверженное разрушению, такое ограниченное в своем местопребывании, запертое или могущее быть запертым в храме существо, как каменный или деревянный бог язычников; ибо он есть только словесное и мысленное существо. Слово же я не в состоянии разбить, не могу запереть его в храм, не могу его видеть глазами, осязать руками; слово есть бестелесное, духовное существо. Слово есть нечто всеобщее; слово <<дерево>> означает и обнимает собой все деревья --- грушевые, буковые, ели, дубы --- баз различия, без ограничения; но телесная, чувственная вещь, почитаемая язычником, есть нечто ограниченное и находится только в этом месте, но не в другом. Христианский бог есть поэтому существо всеобщее, вездесущее, неограниченное, бесконечное; но все эти свойства применимы и к слову. Короче говоря, сущность христианского, духовного бога, как существа, которое не может быть воспринято чувствами, которое обнаруживает свою настоящую сущность не в природе или искусстве, а в священном писании, являет нам собой не что иное как сущность слова. Или, иначе выражаясь: различия между христианским богом и языческим сводятся лишь к различию между словом и чувственными веществами, из которых состоит языческий бог. Поэтому из христианского и еврейского бога, строго говоря, не вытекает искусства, ибо всякое искусство чувственно; самое большее, что вытекает, --- это поэзия, находящая свое выражение в слове, но отнюдь не живопись и не скульптура. Наш законодатель, говорит ученый иудей Иосиф, запретил нам делать изображения, потому что он искусство делать изображения считает за нечто, не приносящее пользы ни богу, ни человеку. Но там, где бог человека не должен и не может быть представлен чувственно, в виде изображения, где чувственность исключена из всего того, что достойно почитания, из божественного, из высшего, там и искусство не в состоянии достигнуть наивысшего, там оно и вообще не может процветать, или может процветать только в противоречии с религиозным принципом. Тем не менее, однако, и христианский бог есть тоже продукт силы воображения, есть изображение, как и языческий бог, только изображение духовное, неосязаемое, изображение, каким является слово. Слово, имя есть продукт силы воображения --- разумеется, действующей разумно согласно чувственным впечатлениям, --- есть изображение предмета. В речи человек подражает природе; звук, интонация, шум, производимый предметом, есть первое, что человек подхватывает у природы, что он делает отличительным признаком, или знаком, при посредстве которого он представляет себе предмет и которым он его называет. Впрочем, это сюда не относится. В христианстве речь идет не о слове, как о выражении, изображении внешнего, а как о выражении, изображении внутреннего. 

Следовательно, так как христианский бог раскрывается и высказывается не в изображениях из камня или дерева, а также и не непосредственно в природе, а стало быть, не представляет собой ничего телесного, чувственного, а есть нечто духовное, слово же есть то же изображение, то отсюда вытекает, что и христианский, даже рационалистический, бог есть изображение силы воображения, а значит --- если поклонение изображениям есть поклонение идолам, то и духовное поклонение богу христиан есть идолопоклонство. Христианство упрекало язычество в идолопоклонстве; протестантизм упрекал в идолопоклонстве католицизм, древнее христианство, а теперь рационализм упрекает в идолопоклонстве протестантизм, по крайней мере протестантизм старый, ортодоксальный, потому что он почитал за бога человека, а стало быть, изображение бога, --- ибо человек ведь есть такое изображение --- вместо самого оригинала, вместо настоящего существа. Я же иду еще дальше и говорю: и рационализм, да и всякая религия, всякая религиозная разновидность, возглавляемая богом, то есть существом недействительным, от действительной природы, от действительного человеческого существа произведенным и от них отличным, и делающая его предметом своего поклонения, есть поклонение изображениям и, следовательно, идолопоклонство, если вообще поклонение изображениям есть, как сказано, идолопоклонство. Ибо не бог создал человека по своему образу, как значится в Библии, но человек создал бога по своему образу, как я показал это в <<Сущности христианства>>. И рационалист, исповедующий так называемую веру в мысль или разум, создает бога, которого он почитает, по своему образу; живой прообраз, оригинал рационалистического бога есть рационалистический человек. Всякий бог есть существо, созданное воображением, образ, и притом образ человека, но образ, который человек полагает вне себя и представляет себе в виде самостоятельного существа \hyperlink{18}{(18)}\hypertarget{b18}{}. Подобно тому как человек сочиняет себе богов не для того, чтобы сочинять, подобно тому как его религиозная поэзия или фантазия не является незаинтересованной, бескорыстной, так не является она и безмерной, и неограниченной, но ее закон, ее мера есть человек. Ведь сила воображения устремляется сообразно существенным свойствам человека; мрачный, боязливый, всего пугающийся человек рисует себе в своем воображении страшные существа страшных богов; жизнерадостный, веселый человек, напротив того, рисует и веселых, приветливых богов. \emph{Как различны люди, так же различны создания их воображения, их боги; правда, можно сказать и наоборот: сколь различны боги, столь же различны и люди.}

\phantomsection
\addcontentsline{toc}{section}{Двадцать первая лекция}
\section*{Двадцать первая лекция}

Прежде, чем продолжать тему вчерашней лекции, я должен предупредить одно возможное недоразумение, которого я вчера только потому не коснулся, что не хотел прерывать хода моего изложения. Я сказал, что как боги, предметы языческой веры, так и предметы веры христианской являются продуктами воображения. Отсюда можно заключить и на самом деле заключали, что библейская история как Ветхого, так и Нового завета есть чистейшая басня, чистейшее измышление. Но этот вывод ни в коем случае не оправдывает себя, ибо я утверждаю лишь, что предметы религии в том виде, в каком они являются ее предметами, суть существа, созданные воображением, а не то, что эти предметы сами по себе --- вымыслы. Так же мало, как из утверждения, что Солнце, каким его себе представляет языческая религия, а именно личным, божественным существом, а стало быть, из утверждения, что Солнце-бог есть воображаемое существо, следует, что и само Солнце есть воображаемое существо, так же мало из утверждения, что Моисей, как его представляет иудейская история религии, что Христос, как его изображает христианская религия и история религии Нового завета, являются существами, созданными воображением, --- можно заключать, что Моисей и Иисус как таковые не были историческими личностями. Ибо между личностью исторической и религиозной то же различие, как между естественным предметом как таковым и им же, каким его представляет себе религия. Религия ничего не создает из себя, иначе мы должны были бы верить в творчество из ничего, фантазия загорается лишь от естественных и исторических предметов. Как кислород не создает без горючего вещества огня, поражающего глаз  \hyperlink{19}{(19)}\hypertarget{b19}{}, так и воображение без соответствующего материала не создает своих религиозных и поэтических фигур. Но историческая личность в том виде, как она является предметом религии, --- уже более не историческая, а переделанная воображением личность. Я не отрицаю, следовательно, что был Иисус, была историческая личность, которой христианская религия обязана своим происхождением, я не отрицаю, что он пострадал за свое учение. Но я отрицаю, что этот Иисус был Христом, богом или сыном божиим, рожденным от девы, чудотворящим существом, что он исцелял больных одним своим словом, укрощал бури простым повелением, мертвых, уже близких к разложению, пробуждал и сам был пробужден от смерти, короче, я отрицаю, что он был таким, каким его представляет Библия; ибо в Библии Иисус --- не предмет простого исторического рассказа, а предмет религии, следовательно, не историческая, а религиозная личность, то есть существо, переделанное и преобразованное в существо воображения и фантазии. И неразумно или, по меньшей мере, бесплодно стремление отделить историческую истину от прибавок, искажений и преувеличений, сделанных силой воображения. У нас для этого нет исторических средств. Христос, как он изображается библейским преданием --- а мы не знаем другого, --- есть и остается созданием человеческого воображения. 

Но сила воображения, которая творит богов человека, связана первоначально с природой; явления природы, а именно те явления, от которых человек больше всего себя чувствует и сознает зависимым, и являются как раз теми, которые производят наибольшее впечатление на воображение, как я это показал на первых лекциях. Что такое жизнь без воды, огня, земли, солнца, луны? Да и какое же впечатление производят эти предметы на теоретическую способность, на фантазию! И прежде всего действует глаз, которым человек созерцает природу, а не разум, производящий опыты и наблюдения, действует исключительно сила воображения, фантазия, поэзия. Но что делает фантазия? Она творит все по образу человека; она делает природу изображением человеческого существа. <<Всюду, --- прекрасно говорит Б. Констан в своем сочинении ,,О религии``  --- где есть движение, дикарь видит и жизнь; катящийся камень кажется ему либо бегущим от него, либо его преследующим; бушующий поток бросается на него; какой-нибудь разгневанный дух живет в пенящемся водопаде; воющий ветер есть выражение страдания или угрозы; эхо от скалы --- пророчествует или дает ответ, и когда европеец показывает дикарю магнитную стрелку, то тот видит в ней существо, уведенное из своего отечества, существо, которое жадно и боязливо стремится к желанным предметам>>. Поэтому человек обожествляет лишь тем или лишь потому природу, что он ее очеловечивает, то есть он обожествляет самого себя, обожествляя природу. \emph{Природа дает лишь материал, вещество для бога; но форму, преобразующую это сырое вещество в человекоподобное и потому божественное существо, душу, --- доставляет фантазия}. Различие между язычеством и христианством, между политеизмом и монотеизмом лишь то, что политеизм делает отдельные фигуры и тела природы богами, и именно поэтому принимает чувственное, действительное, индивидуальное существо человека --- разумеется, бессознательно --- за образец и масштаб, соответственно которому фантазия очеловечивает и обожествляет предметы природы. Как человек есть телесное индивидуальное существо, так и боги политеиста являются телесными, индивидуальными существами; у него поэтому бесчисленное множество богов; у него столько же богов, сколько он замечает различных родов существ в природе. Но он идет даже дальше: он обожествляет даже отдельные разновидности. Разумеется, и это обожествление, эта религиозная схоластика связаны, главным образом, с вещами, имеющими величайшую важность для эгоизма человека; ибо именно у таких предметов человек внимательно подмечает все, своим взором следит за малейшими различиями и затем обожествляет своей фантазией. Чудесный пример этого дают нам римляне. Они имели, например, отдельных богов для каждой стадии развития, через которые с начала и до конца проходят полезнейшие для человека растения, как, например, разные виды хлеба, --- для стадии прорастания, для стадии образования колоса, для стадии наливания почки, короче говоря, для каждой бросающейся в глаза, различаемой ступени роста растения. Так, они имели и для детей множество богов: богиню Natio --- для рождения, богиню Educa --- для еды, богиню Potina --- для питья детей, бога Vagitanus --- для детей кричащих и плачущих, богиню Cunina --- для находящихся в колыбели, богиню Rumia --- для кормящихся грудью. 

Монотеист, наоборот, исходит не от действительного, чувственного человека, являющегося живым, отдельным существом, он идет изнутри наружу, он исходит от духа человека, духа, который выражается в слове, который одним лишь словом производит действие, чье простое слово способно творить. Человек, стоящий над другими, как их господин, которому они повинуются, повелевает ведь миллионами одним простым словом; ему стоит лишь приказать, чтобы его воля была исполнена другими, ему подчиненными слугами. Через посредство простого слова действующие и творящие дух и воля человека, а именно человека, деспотически, или монархически, повелевающего, есть, следовательно, то, из чего исходит монотеист, есть прообраз его фантазии, его воображения. Политеист косвенно обожествляет человеческий дух, человеческую фантазию, ибо ведь предметы природы превращаются для него в богов лишь при помощи фантазии, монотеист же обожествляет прямо, непосредственно. Монотеистический или христианский бог поэтому есть, что и надлежало доказать, в такой же мере продукт человеческой фантазии, такой же образ человеческого существа, как и политеистический, с той лишь разницей, что человеческое существо, сообразно которому христианин мыслит и творит своего бога, не есть существо осязаемое, существо уловимое, могущее быть представленным в очертаниях статуи, картины. Христианский и иудейский бог не поддается изображению; да и кто может составить себе телесный образ духа, воли, слова? Различие между монотеизмом и политеизмом заключается далее в том, что политеизм имеет своей отправной точкой и основанием чувственное воззрение, представляющее нам мир во множестве его существ, монотеизм же исходит от связи, от единства мира, от мира, каким его человек приводит к единству в своем мышлении и воображении. Есть лишь один мир и, следовательно, лишь один бог, говорит, например, Амвросий. Многие боги создания воображения, непосредственно примыкающего к чувствам; единый бог есть создание воображения, отвлеченного от чувств, связанного со способностью к абстракции. Чем больше человек находится во власти воображения, тем чувственнее его бог; так же и единый бог; чем больше человек привык к отвлеченным понятиям, тем его бог менее чувственный, более отвлеченный, хитроумный. Различие между христианским богом в том виде, в каком он является предметом для рационалиста, подвергающего свою веру размышлению, и в том виде, в каком он является предметом для правоверного христианина, заключается лишь в том, что бог рационалиста есть существо более хитроумное, отвлеченное, не чувственное, чем существо мистика или правоверного, заключается лишь в том, что рационалист силу своего воображения определяет силой абстракции, отдает первую во власть второй; старовер же дает воображению преодолеть свою силу отвлечения или способность понимания, преодолеть и властвовать над ними. Или иными словами: рационалист определяет или, лучше, ограничивает веру разумом, --- ведь --- это разум, который мы обозначаем и выражаем в обыденной речи и в обыденном мышлении как способность образовывать понятия, --- а правоверный властвует над разумом при посредстве веры. Бог староверов может все, и на самом деле делает то, что противоречит разуму; он может все, что представляет себе возможным неограниченное воображение веры, --- а для веры нет ничего невозможного, --- то есть этот бог осуществляет то, что верующий воображает; он есть лишь осуществленная, опредмеченная неограниченная сила воображения человека, цельно верующего. Рационалистический бог, наоборот, ничего не может и ничего не делает такого, что противоречило бы разуму рационалиста или, вернее, силе веры и воображения, ограниченной рационалистическим разумом. 

Тем не менее, рационализм есть также поклонение изображениям и идолопоклонство, если поклонение изображениям равносильно идолопоклонству; ибо так же, как подлинный, чувственный идолопоклонник принимает чувственное изображение за бога, за действительное существо, точно так же и рационалист считает своего бога, создание своей веры, своей силы воображения и своего разума за действительное существо, живущее вне человека. Он приходит в бешенство и впадает в фанатизм старой веры, если у него оспаривается бытие бога или --- что то же --- его бога, ибо каждый считает за бога лишь своего бога, --- если ему хотят доказать, что его бог есть лишь субъективное, то есть воображаемое, представленное, измышленное существо, что его бог есть лишь изображение его собственного, рационалистического существа, ограничивающего силу воображения силою абстракции, веру --- мышлением. Однако довольно пока что говорить о различии между рационалистами и ортодоксами, к которому мы еще позже вернемся. 

Я должен, однако, вставить еще одно замечание. Я не различал, противопоставляя друг другу язычество и христианство, веру во многих богов и веру в единого бога, я не различал между предметом языческой религии в том виде, в каком он является предметом природы, и в том, в каком он является предметом искусства; я одинаково говорил: бог язычества есть эта природа, это изображение, это дерево. Об этом скажу теперь вот что. Я говорил: сила воображения делает тела природы, солнце, луну и звезды, растения, животных, огонь, воду человеческими личными существами, но сообразно различным действиям и впечатлениям, которые производят предметы природы, она очеловечивает, олицетворяет различно. Небо, например, оплодотворяет землю дождем, освещает солнцем, оживляет теплотою. Человек представляет себе поэтому в своем воображении землю, как существо воспринимающее, женское, небо --- как существо оплодотворяющее, мужское. Религиозное искусство не имеет другой задачи, как чувственно, наглядно представить предметы природы или причины природных явлений и природных действий, какими их человек в своем религиозном воображении рисует, --- другой задачи, как осуществить создания религиозного воображения. То, во что человек верит, внутренне себе представляет, внутренне считает действительным, он хочет также и видеть вне себя, как нечто действительное. \emph{При помощи искусства --- разумеется, религиозного искусства --- хочет человек дать существование тому, что не имеет существования; религиозное искусство есть самообман, самообольщение человека; он хочет себя уверить при его помощи, что есть то, чего нет, подобно тому, как это делают верующие в бога философы, желающие нас заверить при посредстве своих искусственных доказательств бытия божия, что бог действительно есть, что действительно вне нас существует то, что есть только в нашей голове}. Что же, следовательно, есть то, чему искусство хочет дать существование? Есть ли это солнце, есть ли это земля, есть ли это небо, воздух, как причина молний и грома? Нет, эти предметы существуют, --- и что за интерес был бы для человека, а особенно религиозного человека, изображать солнце, каким оно является нашим чувствам? Нет! Религиозное искусство хочет изображать не солнце, а бога солнца, не небо, а бога небес; оно хочет изображать лишь то, что фантазия вкладывает в чувственный предмет, что, стало быть, не существует чувственно; оно хочет лишь сделать доступным нашим чувствам небо, солнце, поскольку они мыслятся, как личные существа, солнце, поскольку оно является нечувственным, а фантастическим, воображаемым существом. Главное в художественном изображении бога есть его личность, его созданное фантазией человекоподобное существо, второстепенное --- природа; естественный предмет, хотя бог и является первоначально лишь его олицетворением, служит лишь средством обозначения этого бога и придается ему как атрибут. Так, бог неба и грома, Зевс в греческой религии, хотя первоначально, как и во всех естественных религиях, он представлял одно с громом и молнией, изображается держащим в руке королевский скипетр или пылающую стрелу молнии. Первоначальное существо бога грома --- природа низведено, стало быть, до роли простого орудия лица. Тем не менее, однако, между небом как существом природы, и богом неба, представленным в произведении искусства, имеется то равенство или единство, что оба они существа чувственные, телесные; но небесный бог таков лишь в воображении, так что по сравнению с богом, по крайней мере не являющимся чувственным существом, различие между предметом искусства и природой отпадает, или по крайней мере не было необходимости это различие выдвигать. Однако вернемся опять к нашему предмету. 

Я утверждал, что сила воображения есть существенный орган религии; что бог есть воображаемое, образное существо и притом он есть образ человека; что и предметы природы, человекоподобные существа, если они рассматриваются с религиозной точки зрения, являются именно в силу этого образами человека, что и духовный бог христиан есть лишь силою воображения человека созданное, вне человека проектированное и представленное, как самостоятельное, действительное существо, как изображение человеческого существа, --- что, следовательно, предметы религии, разумеется в том виде, в каком они являются ее предметами, не существуют вне воображения. Против этого утверждения верующие, особенно теологи, ужасно протестовали и восклицали: возможно ли, что простым воображением является то, что доставляет так много утешения людям, ради чего миллионы людей даже жертвовали жизнью? Но это совсем не доказательство действительности и истинности этих предметов. Язычники также считали своих богов действительными существами, приносили им в жертву целые гекатомбы, даже жизнь свою или других людей, и все-таки христиане теперь считают, что эти боги были лишь вымышленные, воображаемые существа. Что настоящее время считает за действительность, то будущее признает фантазией, воображением. Наступит время, когда будет так же общепризнано, что предметы христианской религии были лишь фикциями, как теперь это общепризнано относительно богов язычества. Только эгоизм человека полагает своего бога за истинного, а богов других народов за воображаемые существа. Сущность силы воображения --- там где ей не выступают в противовес чувственные воззрения и разум, --- заключается именно в том, что она заставляет казаться человеку действительным то, что она ему представляет. 

Какую власть над человеком проявляет воображение, это могут наглядно показать примеры из жизни так называемых диких народов. <<Дикари в Америке и Сибири не предпринимают никакого путешествия, не совершают никакого обмена, не заключают никакого договора, если они не побуждаются к этому своими снами. Самое ценное, что у них есть, то, что они, не задумываясь, защищали бы ценой жизни, они отдают, доверившись сну. Камчадалки отдаются без сопротивления тому, кто уверит их в том, что во сне обладал ими. Один ирокезец видел сон, что ему отрезали руку, и он ее себе отрезал; другой, что он убил своего друга, и он его убил>> (Б. Констан, указанное сочинение). Может ли власть воображения быть доведена до высшей степени, чем здесь, где виденная во сне потеря руки делается основой и законом действительной потери; где виденное во сне воображаемое убийство друга делается основой и законом действительного убийства, где, стало быть, простому сну приносят в жертву свое тело, свои руки, своего друга? \hyperlink{20}{(20)}\hypertarget{b20}{} Как для дикарей в настоящее время, так и для древних народов сон имел значение божественного существа, откровения, явления бога. Даже христиане отчасти и теперь считают сны за божественные внушения. Но то, в чем бог являет свое откровение, в чем он себя проявляет, есть не что иное, как его существо. Поэтому бог, который являет свое откровение во время сна, есть не что иное, как существо сна. Но что же такое существо сна? Не ограниченная законами разума и чувственного воззрения, необузданная сила воображения или фантазия. Следует ли из того, что христиане давали себя преследовать за предметы своей веры, приносили им в жертву свое имущество и кровь, следует ли их истинность и действительность? Нисколько. Так же мало, как из того, что ирокез в угоду своему сну отсечет себе руку, следует, что он эту руку действительно потерял во сне; также мало вообще истинность снов вытекает из того, что человек, который дает над собой власть снам, приносит им в жертву истину разумного чувственного воззрения. 

Я привел сны, лишь как чувственные очевидные примеры религиозной власти воображения над человеком. Но я утверждал также, что сила воображения религии не есть свободная сила художника, но что она имеет практическую эгоистическую цель, или что сила воображения религии имеет свои корни в чувстве зависимости, что религиозная сила воображения держится главным образом за предметы, возбуждающие в человеке чувство зависимости. Чувство зависимости человека не стоит, однако, в связи с определенными только предметами. Как сердце находится постоянно в движении, непрерывно бьется, так же точно никогда в человеке, а именно в человеке, над которым властвует сила воображения, не замирает чувство зависимости, ибо при каждом шаге, который он делает, его может постигнуть беда, каждый предмет, как бы он ни был незначителен, может угрожать ему смертью. Это чувство страха, эта неизвестность, эта всегда сопровождающая человека боязнь несчастий есть корень религиозной силы воображения; и так как религиозный человек все беды, его постигающие, приписывает злым существам или духам, то страх перед привидениями и духами есть сущность религиозной силы воображения, по крайней мере у необразованных людей и народов. Чего человек боится, чего он пугается, то сейчас же фантазия превращает в злое существо, или, наоборот, что ему фантазия рисует как злое, того он боится и стремится поэтому расположить к себе при помощи религиозных средств или обезвредить. Так, например, у чиккитов в Парагвае, как это значится в <<Истории Парагвая>> Шарлевуа, <<не найдено явственного следа религии, однако они боялись демонов, которые, как они говорили, имели обыкновение являться им в ужасающих образах. Свои празднества и пирушки они начинали с того, что призывали демонов не мешать их веселью>>. Таитяне верят в то, что если кто ударится ногою о камень и ему станет больно, то это сделал один из Эатуа или сам Эатуа, то есть бог, так что про них, как это значится в третьем и последнем путешествии Кука, <<можно буквально сказать, что они при своей религиозной системе ступают всегда по заколдованной почве>>. Так и ашантии в Африке, если они ночью в темноте оступятся о камень, верят, что злой дух спрятался в камне, чтобы им причинить боль (<<Ausland>>  1849, май). Так фантазия превращает камень, о который человек по своей неосмотрительности споткнулся, в дух или в бога! Но как легко повторно спотыкается человек! На каждом шагу может приключиться с ним это несчастье. Поэтому человек, находящийся во власти своего чувства и воображения, постоянно видит себя окруженным злыми духами! Так, у североамериканских индейцев достаточно, чтобы у кого-нибудь заболели зубы или голова, чтобы это сейчас же означало, что <<духи недовольны и хотят, чтобы с ними примирились>> (Хеккевельдер, <<Известия об истории, обычаях и нравах индейских племен>>). Особенно выделяются своим страхом перед духами и привидениями народы Северной Азии, исповедующие так называемое шаманство, религию, которая состоит не в чем ином, как <<в боязни духов, изгнании духов и заклинании духов>>; они живут в непрестанной борьбе <<с враждебными духами, которые бродят по пустыне и по далеким снежным полям>> (Штур, <<Религиозная система языческих народов Востока>>). Но не одно шаманство, как это также говорит Штур, имеет свои корни в этой вере в привидения, но более или менее религии всех народов. Особенно замечательно то, что рассказывают о североамериканских дикарях. <<Как ни храбр, горд и как ни чувствует свою независимость североамериканский индеец, все же страх перед колдовством и волшебством делает его одним из самых пугливых и робких созданий>>  --- говорит Хеккевельдер. <<Невероятно, --- продолжает он далее, какое влияние оказывает на настроение вера индейцев в колдовство. Они уже не те люди в тот момент, когда их воображение охвачено мыслью, что они заколдованы. Их фантазия тогда находится постоянно в действии, рисуя самые страшные и подавляющие образы>>. Страх перед колдовством есть не что иное, как страх перед тем, что может быть причинено несчастье злым существом так называемым сверхъестественным, колдовским образом. И это суеверие, это воображение так сильно у индейцев, что они часто под влиянием <<простого воображения, что им причинен вред, что они околдованы, и на самом деле умирают>> \hyperlink{21}{(21)}\hypertarget{b21}{}. Точно так же, как и Хеккевельдер, высказывается и Вольней о североамериканских дикарях в своем описании Северной Америки: <<Страх перед злыми духами есть одно из самых распространенных и мучительных представлений; их самые бесстрашные воины в этом пункте подобны женщинам и детям; сон, какое-нибудь ночное видение в кустах, неприятный крик пугают их>>. Точно так же, как у названных народов, мы находим и у христиан самые преувеличенные представления о бедах и смертельных опасностях и описания ими опасностей, которые преследуют человека на всех путях и которые их религиозная фантазия представляет им как действия враждебного человеку злого существа или духа, дьявола, Действия, которые могут быть устранены лишь противодействием доброго, к человеку благосклонною и всемогущего бога. 

Боги, следовательно, --- создания фантазии, но создания фантазии, находящиеся в самой тесной связи с чувством зависимости, с человеческой нуждой, с человеческим эгоизмом, создания фантазии, которые в то же время являются существами, созданными чувством, существами или созданиями аффектов, в особенности страха и надежды. Человек требует от богов, как я уже это говорил при описании религиозного поклонения изображениям, чтобы они ему помогали, если он их представляет себе добрыми существами, чтобы они ему не вредили, по крайней мере не мешали его планам и радостям, если он их представляет себе злыми. Религия поэтому есть не только дело воображения, фантазии, не только дело чувства, но также и дело желания, стремления человека и его потребности устранять неприятные чувства и создавать себе приятные, получать то, чего у него нет, но что ему хотелось бы иметь, а удалять то, что он имеет, но чего иметь он не хотел бы, как, например данную беду или данный недостаток, --- короче говоря, она есть выражение стремления человека освободиться от бед, которые у него есть, или которых он опасается, и получить то добро, которое он желает, которое ему рисует его фантазия, она есть выражение так называемого стремления к счастью. 

\phantomsection
\addcontentsline{toc}{section}{Двадцать вторая лекция}
\section*{Двадцать вторая лекция}

Человек верит в богов не только потому, что у него есть фантазия и чувство, но также и потому, что у него есть стремление быть счастливым. Он верит в блаженное существо не только потому, что он имеет представление о блаженстве, но и потому, что он сам хочет быть блаженным; он верит в совершенное существо потому, что он сам хочет быть совершенным; он верит в бессмертное существо потому, что он сам не желает умереть. То, чем он сам не является, но чем он хотел бы быть, это он представляет себе имеющимся в своих богах; боги --- это желания людей, которые мыслятся как осуществленные в действительности, которые превращены в действительные существа; бог есть стремление человека к счастью, нашедшее свое удовлетворение в фантазии. Если бы человек не имел желаний, то он, несмотря на фантазию и чувство, не имел бы ни религии, ни богов. И сколь различны желания, столь же различны и боги, а желания столь же различны, сколь различны сами люди. Кто предметом своих желаний не имеет мудрости и рассудительности, кто не хочет быть мудрым и рассудительным, тот не имеет и богини мудрости предметом своей религии. Мы должны по этому случаю опять восстановить в своей памяти то, что было сказано в первых лекциях, а именно, что для того чтобы понять религию, мы должны при ее объяснении избегать всяких односторонних, ограниченных оснований или не уделять этим основаниям иного места в религии, чем то, которое они занимают в ней в действительности. Поскольку боги являются силами и притом первоначально силами природы, которые человеческое воображение переделало в человекоподобные существа, человек падает перед ними ниц; он чувствует перед ними свое ничтожество; они --- предметы его чувств ничтожества, страха, почтения, изумления, удивления, страшные или чудесные, величественные существа, производящие на человека все то впечатление, которое вообще на него производит существо или образ, наделенный волшебными силами фантазии; поскольку же они силы, исполняющие желания людей, дающие человеку то, чего он хочет и в чем он нуждается, они предметы человеческого эгоизма. Короче говоря, религия имеет по существу одну практическую цель и основание; стремление, из которого исходит религия, ее последнее основание есть стремление к счастью, и если это стремление представляет собой нечто эгоистичное, то, стало быть, --- эгоизм. Кто этого не замечает или это оспаривает, тот слеп, ибо история религии подтверждает это на каждой своей странице, она подтверждает это и на низших, и на высших стадиях религии. Пусть вспомнят при этом те свидетельства, которые в одной из прежних лекций я приводил из христианских, греческих и римских писателей. Этот пункт практически и теоретически самый важный, ибо если доказано, что бог обязан своим существованием лишь стремлению человека к счастью, но что религия удовлетворяет это стремление лишь в воображении, то необходимым следствием этого является то, что человек ищет удовлетворения этого стремления иным способом, чем религия, другими, не религиозными средствами. Вот еще несколько подтверждающих примеров. 

Тогда как прежде моя задача заключалась в том, чтобы доказать, что любовь к себе есть последнее основание религии, --- теперь моя задача более определенная: доказать, что религия имеет своей целью человеческое счастье, что человек почитает богов и молится им только для того, чтобы они исполнили его желания, чтобы он стал через них счастлив. <<Просите, --- говорится в Библии, --- и дано будет вам; ибо всякий просящий получает. Есть ли между вами такой человек, который, когда сын его попросит у него хлеба, подал бы ему камень? Итак, если вы, будучи злы, умеете даяния благие давать детям вашим, тем более отец ваш небесный дает блага просящим у него>>. <<Если бы, следовательно, кто-либо мог, --- говорит Лютер в своем собрании церковных проповедей, --- взять у самого бога мужество, дабы возыметь по отношению к нему дерзание и сказать ему от всего сердца: ты --- мой дорогой отец, --- чего бы он только не мог попросить? И в чем бы бог ему мог отказать? его собственное сердце скажет, что должно быть так, как он просит>>. Бог, таким образом, представляется в виде существа, исполняющего желания, выслушивающего просьбы. Молятся для того, чтобы получить добро, чтобы быть избавленным <<от опасностей, от нужды и от всякого рода невзгод>>. Но чем больше нужда, опасность, страх, тем сильнее дает себя знать и инстинкт самосохранения, тем живее желание быть спасенным, тем горячее молитва. Так, индейцы, как рассказывает в качестве очевидца Хеккевельдер в не раз уже упоминавшемся сочинении, обращаются при приближении бури или непогоды к Манитто воздуха (то есть к богу воздуха, к воздуху, представленному в виде личного существа), чтобы он отвратил от них все опасности; так чиппевеи на озерах Канады молятся Манитто вод, чтобы он предупредил слишком большое нарастание волн, пока они совершают переправу по воде. Так и римляне приносили жертвы бурям и морским волнам всякий раз, когда они отправлялись в море, Вулкану, богу огня, когда у них случался пожар, или чтобы пожара не было. Когда ленапы отправляются на войну, то они, по свидетельству Хеккевельдера, предварительно молятся и поют следующие строфы: <<О, я, бедный, выступающий в поход против врага! И не знаю я, вернусь ли я обратно, чтобы порадоваться объятьям моих детей и жены. О, бедное создание, чья жизнь не от него зависит, которое не имеет власти над своим телом, которое, однако, все же пытается исполнить свой долг во имя благополучия своего народа! О, ты, великий дух, там, наверху, имей сострадание к моим детям и к моей жене! Предохрани, так чтобы им не пришлось по мне горевать! Дай успеха этому моему предприятию, чтобы я, мог убить моего врага и принести знаки победы домой, моей дорогой семье и моим друзьям, так, чтобы мы друг другу порадовались. Имей сострадание ко мне и сохрани мою жизнь, и я принесу тебе жертву>>. В этой трогательной, простой молитве перед нами собраны воедино все указанные моменты религии. 

Человек не имеет власти над успехом своего предприятия. Между желанием и его осуществлением, между целью и ее исполнением лежит целая пропасть всяких трудностей и возможностей, способных не дать этой цели осуществиться. Как бы ни был превосходен мой план сражения, всякие как природные, так и человеческие происшествия --- ливень, поломка ноги, случайно запоздавшее прибытие вспомогательного корпуса и другие подобные случайности --- могут погубить мой план. Человек поэтому заполняет своей фантазией эту пропасть между целью и ее выполнением, между желанием и действительностью, заполняет при посредстве существа, от чьей воли он мыслит зависимыми все эти обстоятельства и чье благорасположение ему стоит лишь вымолить, чтобы уверенно представить себе счастливый исход предприятия, исполнение своих желаний \hyperlink{22}{(22)}\hypertarget{b22}{}. Человек не располагает своей жизнью, по крайней мере, не располагает безусловно; какая-нибудь внешняя или внутренняя причина, будь то разрыв малейшего сосуда в моей голове, может внезапно положить конец моей жизни, может, вопреки моему сознанию и воле, заставить меня расстаться с женой и детьми, с друзьями и родственниками. А между тем человек хочет жить; жизнь ведь есть совокупность всех благ! Человек превращает непроизвольно поэтому --- в силу ли своего инстинкта самосохранения, или по причине любви своей к жизни --- это желание в существо, которое может его исполнить, в существо, имеющее глаза, как и человек, чтобы видеть его слезы, и уши, как человек, чтобы слышать его жалобы; ибо природа этого желания не может выполнить; \emph{природа, какова она есть в действительности --- не личное существо, у нее нет сердца, она слепа и глуха к желаниям и жалобам людей}. 

В чем может мне помочь море, если я его представляю себе как простое скопление водяных масс, короче говоря, как то, чем оно является в действительности, когда море для нас --- просто предмет? Я могу лишь в том случае молить море, чтобы оно меня не поглотило, если я его представляю себе в виде личности, от чьей воли зависит движение моря, чью волю, чье настроение я поэтому могу склонить в свою пользу жертвами и почтительными подношениями, если, стало быть, я его представляю себе в виде бога. Поэтому отнюдь не только ограниченность человека, в силу которой он мыслит себе все на свой образец, отнюдь не его незнание только, незнакомство с тем, что такое природа, отнюдь не одна лишь его сила воображения, которая все олицетворяет; но также и душа, любовь к себе, человеческий эгоизм, или стремление к счастью, являются причиной того, что он действия и явления природы производит от желающих, духовных, индивидуальных человечески живых существ, все равно, принимает ли он существующими, когда верит во много богов, многие личные причины, или, при вере в единого бога, только одну природную причину, действующую сообразно своей воле и сознанию. \emph{Ибо только тем, что человек делает природу зависимой от бога, он делает природу зависимой от него самого, он себе природу подчиняет}. <<Может быть искуплена, --- говорится в Овидиевых <<Фастах>>  --- молния Юпитера, может быть управляем гнев жестокого>>. Если какой-нибудь предмет природы, например море, является богом, если от его воли зависят столь опасные для человека бури и волнения моря, и воля морского бога при этом определяется в пользу почитающих его людей их молитвами и жертвами, --- <<дары побеждают даже богов>>  --- то, стало быть, косвенно, то есть посредственно, движение моря зависит от человека; человек господствует над природой через бога или посредством бога. Так, некогда одна весталка, ложно обвиненная в смертоубийстве, взяла в руки решето и обратилась к Весте со словами: 

<<Веста! если я тебе постоянно служила непорочными руками, то сделай так, чтобы я этим решетом могла зачерпнуть воду из Тибра и принести к тебе в храм>>  --- и сама природа послушалась, --- как выражается Валерий Максим, смелых и необдуманных просьб жрицы, то есть вода, вопреки своей природе, не протекла сквозь решето. Так, солнце в Ветхом завете останавливается по молитве или приказу Иисуса Навина. Молитва или приказ не отличаются, впрочем, друг от друга существенно. Преодолей (или укроти, победи), говорит, например, у Виргилия божественная река Тибр Энею, гнев Юноны смиренными молитвами, преодолей, говорит ему там же Гелен, смиренными дарами могучую повелительницу. 

Молитва есть лишь смиренное приказание, но приказание в религиозной форме. Современные теологи, правда, вытравили из Библии чудо приостановки солнца и объяснили это место поэтическим оборотом речи или еще, я уже не знаю, каким образом. Но есть еще много других, столь же удивительных чудес в Библии, и поэтому решительно все равно, сохранить ли вместе с верой это чудо или, не веруя, его устранить. Так же точно по молитве Ильи падает дождь. <<Молитва праведного, --- значится в Новом завете, --- может многое. Илья молился, чтобы не было дождя, и дождя не было на земле в течение трех лет и шести лун. И он еще раз помолился, и небо послало дождь>>. И псалмопевец говорит: <<Бог исполняет волю богобоязненных>>. <<Бог, --- говорит Лютер в своем толковании второй книги Моисея, касаясь этого места Библии, --- бог делает так, как этого хочет тот, кто верует>>. И в настоящее время христиане молят во время длительной засухи о дожде, во время длительных дождей --- о солнечном свете; они верят, следовательно, хотя они это в теории и отрицают, что, воля бога, от которого, как они думают, зависит все, определяется молитвой человека, даст ли бог дождь или солнечный свет, и притом вопреки естественному ходу вещей; ибо если бы они верили, что дождь и свет солнца тогда только приостанавливаются, когда это следует согласно природе, то они бы не молились, --- молитва была бы глупостью, --- нет! они верят, что молитвой можно властвовать над природой, что природу можно с помощью горячей молитвы подчинить человеческим желаниям и потребностям. Именно поэтому человеку, по крайней мере человеку, привычному к религиозным представлениям, то учение, которое природу понимает из нее самой, которое не делает мир или природу зависимыми от воли бога, существа, к человеку благоволящего, человекоподобного, кажется безотрадным и потому ложным учением; ибо, хотя теист в теории и предполагает ложность безотрадности и поступает так, как будто он ее отвергает только благодаря доводам разума, но все же на практике, то есть на деле, в действительности ложность выводится только из безотрадности; это учение потому отвергают как ложное, что оно безотрадно, то есть неприятно, не так удобно, не так льстит человеческому эгоизму, как противоположное учение, выводящее природу из существа, определяющего ход вещей согласно молитвам и желаниям людей. <<Эпикурейцы, --- уже говорит добродушный Плутарх в своем сочинении о невозможности счастливо жить, следуя Эпикуру, --- уже тем самым наказаны, что они отрицают провидение и тем лишены отрады, которую дает вера в божественное провидение>>. <<Какое успокоение, какая отрада, --- говорит Гермоген у Плутарха в том же сочинении, заключается в представлении о том, что существа всезнающие и всемогущие так благосклонно ко мне относятся, что из-за заботы обо мне их глаз постоянно бдит надо мною как днем, так и ночью, что бы я ни делал, и что они подают мне всевозможного рода знаки, чтобы обнаружить передо мной исход каждого предприятия!>> <<Жить без бога, --- рассуждает подобным же образом английский теолог Кедворт, --- значит жить без надежды. Ибо какую надежду может возлагать человек на бесчувственную и безжизненную природу, или какое доверие к ней питать?>> И приводит при этом изречение греческого поэта Лина: <<На все можно надеяться (ни в чем не отчаиваться), ибо бог все делает с легкостью, для него ничто не служит препятствием>>. 

\emph{Вера --- представление, которого лишь потому придерживаются, если не на словах, то на деле, потому считают истинным, что оно утешительно, приятно, что оно льстит эгоизму, любви человека к себе; оно имеет также своим происхождением лишь чувство, лишь эгоизм, лишь себялюбие}. По впечатлению, которое какое-либо учение производит на человека, можно с уверенностью судить о происхождении этого учения. На что какая-либо вещь, то есть в данном случае воображаемая, представленная вещь, оказывает действие, оттуда она и происходит. Что оставляет, как говорится, холодным, равнодушным сердце, то не имеет и своего основания ни в каком сердечном или эгоистичном интересе человека. Так вот таким-то представлением, соответствующим себялюбию человека, и является представление о том, что природа действует не с неизменной необходимостью, но что выше необходимости природы стоит существо, любящее человека, человекоподобное, существо с волей и разумом, руководящее и управляющее природой так, как это полезно человеку, берущее человека под свою особую защиту, оберегающее человека от опасностей, которые ему в любой момент угрожают со стороны природы, действующей беспощадно и слепо. Я выхожу на воздух; в этот же момент падает с неба камень; в силу естественной необходимости он падает мне на голову и убивает меня, ибо я оказался как раз в направлении падения этого камня и тяжесть, в силу которой падает камень, не имеет никакого почтения ко мне, как бы я ни был знатен, как бы я ни был умен. Но бог парализует силу тяжести, уничтожает ее действие, чтобы меня спасти, потому что бог имеет более уважения к человеческой жизни, чем к законам природы, или он, по крайней мере, умеет, если не хочет сотворить чуда, так искусно и умно, так рационалистически хитро повернуть и направить обстоятельства, что камень, не нарушая законов природы, к которым рационалисты питают великое уважение, не причиняет мне вреда. Как удобно поэтому бродить под покровом небесной защиты и как тяжело и безотрадно, как это делает неверующий, непосредственно подвергаться действию наглых метеоров, града, ливней и солнечных ударов природы. Я должен, однако, сейчас же прервать ход своего изложения замечанием, что, хотя это представление о божественном провидении, равно как и другие религиозные представления, благодаря своим приятным и отрадным для сердца свойствам, отвечающим любви человека к самому себе, и вытекают из себялюбия, из сердца, --- но вытекают лишь до тех пор, пока сердце находится на службе у воображения и именно поэтому находит себе утешение в религиозных фантазиях. Ибо стоит лишь человеку раскрыть глаза, взглянуть на действительность, какова она есть, не будучи ослепленным религиозными представлениями, как сердце восстает против представления о провидении за ту пристрастность, с которой оно одного спасает, другому дает погибнуть, одних предназначает для счастья и богатства, других --- для несчастья и нищеты, за ту жестокость или, по крайней мере, за ту пассивность, с которой оно миллионы людей подвергает ужаснейшим страданиям и мукам. Кто в состоянии сочетать ужасы деспотизма, ужасы духовной иерархии, ужасы религиозной веры и суеверия, ужасы языческой и христианской уголовной юстиции, ужасы природы, подобные черной смерти, чуме, холере, с верой в божественное провидение? Верующие теологи и философы, правда, напрягли все силы своего разума, чтобы сгладить эти очевидные противоречия действительности с религиозным представлением о существовании божественного провидения; но гораздо более согласуется с сердцем, любящим правду, гораздо более даже с честью любого бога, с честью божества вообще, прямо отвергнуть его бытие, чем при помощи постыдных и смешных ухищрений, измышленных верующими теологами и философами для оправдания божественного провидения, влачить свое жалкое существование. Лучше с честью пасть, чем с бесчестьем продолжать свое бытие. Атеист дает богу с честью пасть, теист же, рационалист, напротив того, --- с бесчестием, во что бы то ни стало сохраниться! 

\phantomsection
\addcontentsline{toc}{section}{Двадцать третья лекция}
\section*{Двадцать третья лекция}

Религия имеет, таким образом, практическую цель. Она хочет тем, что она превращает действия природы в поступки, продукты в дары, будь то одного или нескольких индивидуальных человекоподобных существ, привести природу в подчинение человеку, заставить ее служить стремлению человека к счастью. Зависимость человека от природы есть поэтому, как я это доказываю в <<Сущности религии>>  основание, начало религии, но свобода от этой зависимости как в разумном, так и в неразумном смысле есть конечная цель религии. Или: божественность природы есть, правда, основание религии, но божественность человека есть конечная цель религии. \emph{Поэтому то, чего человек при развитом состоянии его разума хочет достигнуть при посредстве образования и природы, то есть существования прекрасного, счастливого, защищенного от напастей и слепых случайностей природы, того человек, находящийся в некультурном состоянии, стремится достигнуть через посредство религии}. Средством сделать природу приятной для человеческих целей и желаний в начале человеческой истории является поэтому единственно только религия. Беспомощный и не ведающий, куда ему обратиться, не располагающий никакими средствами человек не знает, как иначе себе помочь, чем молениями и связанными с ними дарами, жертвами, которыми он пытается расположить в свою пользу предмет, которого он боится, от которого чувствует себя угрожаемым и зависимым, или же волшебством, которое представляет, однако, нерелигиозную форму религии, ибо сила волшебства, то есть способность одними лишь словами, одной лишь волей господствовать над природой, сила, которую волшебник себе приписывает или которую сам проявляет, религиозный человек переносит на предмет вне себя. Впрочем, молитвы и волшебство могут быть и соединены друг с другом, причем молитвы оказываются не чем иным, как формулами заклинания и волшебства, которыми можно заставить богов даже и против их воли исполнить желания людей. Даже и у благочестивых христиан молитва не всегда имеет характер религиозного смирения, она часто выступает также и повелительно. <<Когда мы, --- говорит, например, Лютер в своем толковании первой книги Моисея, --- находимся в нужде и опасности, мы не особенно считаемся с его (бога) высоким величеством, а говорим ему прямо: помоги, господи! Так помоги же, боже! Сжалься ты, который на небесах! Тогда мы не делаем длинных предисловий>>. 

Молитва и жертва являются, стало быть, средствами, при помощи которых беспомощный и не знающий, что делать, человек стремится выпутаться из беды и подчинить себе природу. Так китайцы, рассказывает Зоннерат, во время бури на море, когда опасность требует больше всего деятельности и ловкости, молятся компасу и, молясь, гибнут вместе с ним; так тунгусы во время эпидемии набожно и с торжественными коленопреклонениями молятся болезни, чтобы она миновала их хижины; так же раньше упоминавшиеся кханды, когда вспыхивает оспа, приносят в дар божеству оспы кровь быков, овец и свиней; а жители острова Амбоина, одного из ост-индских или, точнее, Молуккских островов, <<когда начинается какая-либо жестокая эпидемия, собирают всевозможные дары и жертвы, нагружают ими корабль и пускают в море в надежде на то, что эпидемии, умилостивленные этим, последуют за принесенными им дарами и жертвами и покинут остров Амбоина>> (Мейнерс, цитируемое сочинение). Так же и так называемый идолопоклонник, вместо того, чтобы выступить против предмета, обращается даже к нему, являющемуся причиной беды, с благочестивыми молитвами, чтобы его укротить. Этого, разумеется, не делает христианин; но он в том отношении не отличается от политеиста или идолопоклонника, что, подобно им, не при помощи самодеятельности, культуры, собственного разума хочет устранить природные бедствия, сделать себе природу послушной, а при помощи молитвы, обращенной к всемогущему богу. Разумеется, мы должны здесь тотчас указать на различие между древними и современными, или между необразованными и образованными христианами, ибо первые полагались и полагаются только на всемогущество молитвы или бога; вторые же, правда, еще молятся: <<предохрани нас от бедствий, предохрани от пожара!>> на практике же не полагаются уже более на силу молитвы, а стараются оградить себя при помощи страхования своей жизни и своего имущества. Конечно, я должен сейчас же прибавить, чтобы предупредить недоразумения, что культура не всемогуща, как всемогущи религиозная вера или религиозное воображение. Как природа не в состоянии из кожи сделать золота, из пыли зерна, как это делает бог, предмет религии, так же точно не производит чудес и культура, которая овладевает природой лишь при помощи природы, то есть при помощи естественных средств. 

Бесчисленные беды, которые человек хотел устранить при помощи религиозных средств, но устранить не мог, устранило или во всяком случае смягчило образование, устранила или смягчила человеческая деятельность благодаря применению естественных средств. Религия есть поэтому выражение детства человечества. Или: в религии человек --- дитя. Дитя не может собственными силами, при помощи самодеятельности осуществить свои желания, оно обращается со своими просьбами к существам, от которых чувствует и знает свою зависимость, к родителям, чтобы при их посредстве получить то, чего оно хочет. Религия имеет свое происхождение, свое истинное место и значение лишь в период детства человечества, но период детства есть в то же время и период невежества, неопытности, необразованности или некультурности. Религии, подобные христианской, которую называют новой, возникли в более поздние времена, не были, по существу говоря, новыми религиями; они были критическими религиями; они лишь реформировали, одухотворили, приспособили к более передовым взглядам человечества религиозные представления, ведущие свое происхождение от древнейших времен. Или если даже мы будем рассматривать позднейшие религии как существенно новые, то все же период, когда возникает новая религия, есть по сравнению с более поздним временем период детства. Обратимся, например, к тому, что ближе всего к нам лежит, к тому времени, когда возник протестантизм. Какое невежество, какое суеверие, какая грубость царили тогда! Какие детские, грубые, вульгарные, суеверные представления имели тогда даже высоко просвещенные реформаторы. Но именно поэтому они ничего и не имели в своем уме, как только религиозную реформацию; все их существо, например существо Лютера, было захвачено религиозными интересами. 

Религия возникает, следовательно, лишь во тьме невежества, нужды, беспомощности, некультурности, в условиях, при которых именно поэтому сила воображения господствует над всеми другими силами, при которых человек живет с самыми взвинченными представлениями, с самыми экзальтированными душевными настроениями; но она возникает в то же время и из потребности человека в свете, в образовании или, по крайней мере, в тех целях, которые преследует образование; она сама --- не что иное, как первичная, но еще грубая, вульгарная форма образованности человеческого существа; потому-то каждая эпоха, каждая важная глава в истории культуры человечества начинается с религии. Даже и в настоящее время наши правительства, невежественные и грубые во всех более серьезных делах, прибегают, чтобы бороться с нищетою мира, к религии вместо того, чтобы прибегать к средствам помощи и образования. Поэтому все, что делается впоследствии предметом человеческой самодеятельности, образования, было первоначально предметом религии; все искусства, все науки или, вернее, первые начатки, первые элементы их, были сначала делом религии, ее представителей, жрецов, --- ибо, как только какое-либо искусство, какая-либо наука разовьются, усовершенствуются, они перестают быть религией. Так, философия, поэзия, наука о звездах, политика, правоведение, по крайней мере разрешение трудных случаев, доискивание, кто прав, кто виноват, так же, как и врачебное искусство, были некогда религиозным делом. Так, например, у древних египтян врачебное искусство имело <<религиозный астрологический характер. Как каждая часть года, так и каждая часть человеческого тела находились под влиянием особого звездного божества\dots Спор о праве, лечение не могли быть предприняты без опроса звезд>> (Е. Рет, <<Египетское и зороастровское вероучения>>). Так и в настоящее время у дикарей волшебники или колдуны, находящиеся в сношениях с духами или богами, являющиеся, стало быть, духовными лицами, жрецами дикарей, одновременно и врачеватели. И у христиан врачебное искусство или, по крайней мере сила исцеления были делом религии, веры. В Библии целительные силы сохраняются даже в частях одежды святых, героев веры, божьих людей. Я напомню здесь только об одежде Христа, к краю которой было достаточно прикоснуться, чтобы исцелиться, о платке, пропитанном потом, и о нагруднике апостола Павла, которых достаточно было, как это значится в истории апостолов, подержать над больными, чтобы заразные болезни перед ними сдавались и злые духи выходили. Религиозная медицина никоим образом не ограничивается, однако, только так называемыми сверхъестественными средствами, вроде заклинания, волшебства, молитвы, силы веры и силы бога; она применяет и естественные лечебные средства. Но в начале человеческой образованности именно эти естественные лечебные средства имели религиозное значение. Так, египтяне, у которых, как мы только что видели, медицина была частью религии, имели также и естественные лечебные средства; и как мог, в самом деле, человек довольствоваться лишь религиозными средствами, молитвой и колдовскими формулами! Его разум, как ни мало он развит, или как ни подавлен он верой, говорит ему, что нужно постоянно думать о средствах и притом о средствах, отвечающих предмету, цели; но <<книги, в которых обозначены были лечебные средства и способы лечения, у египтян причислялись к числу священных книг, поэтому все новшества строжайше воспрещались; врач, который применил новые средства и которому не посчастливилось спасти своего пациента, наказывался смертью>>. 

В этом признании египтянами обычных целебных средств священными мы имеем явственный пример того, что первые средства образования или культуры являются святынями. У христиан вода, вино и хлеб --- лишь средства для таинств; но первоначально вода как таковая считалась святыней, то есть чем-то священным, даже божественным, за свои благодетельные проявления и свойства, которые в ней находили, и которые содействовали образованию человека и его благополучию. Умывание и купание у древних народов было религиозной обязанностью и религиозным делом \hyperlink{23}{(23)}\hypertarget{b23}{}. Совестились загрязнять воды. Древние персы никогда не спускали своих вод в реку, не плевали никогда в нее. И у греков запрещалось переходить через реку с неумытыми руками, спускать свою воду в устье реки или в какой-нибудь источник. Так же или еще более священны, чем вода, были хлеб и вино, ибо для открытия их требовалось больше знаний, чем для открытия благодетельных свойств воды, уже известных животным. <<Священный хлеб>> входил в состав мистерий греческой религии. <<Даже и у нас, --- правильно замечает Гюльман в своей книге ,,Теогония, исследование о происхождении религии древности`` (Берлин 1804), --- имеется к хлебу известное религиозное чувство, благодаря которому ростовщичество хлебом объявляется, между прочим, самым ненавистным из всех видов ростовщичества и благодаря которому простой человек, когда он видит, как хлеб пропадает даром, невольно восклицает: о милый, любимый наш хлеб!>> Изобретение хлеба, как и вина, приписывалось одному из богов, потому что хлеб и вино сами считались за нечто божественное и священное. Ведь значится же даже в Библии: <<вино радует сердце человека>>. Все же благодетельное, все полезное, все отрадное, все украшающее и облагораживающее человеческую жизнь было для древних, как мы только что видели это на примере хлеба и вина, божественно, священно, религиозно. Чем невежественнее были люди, чем более лишены средств доставлять себе наслаждения, устроить себе достойное человека существование, оградить себя от невзгод природы, тем больше уважения должны были они питать к изобретателям таких средств, тем более священным считали они само средство. Поэтому для чувственных греков все, что делает человека человеком, было богом; так, например, домашний огонь объявляется благодетельным для человека существом, потому что он собирает людей вокруг очага, сближает человека с человеком. Но именно потому, что человек делал первые целебные средства, первые элементы человеческого образования и счастья святынями, именно поэтому религия была всегда в процессе развития человечества противоположностью настоящего образования, тормозом развития; потому что каждому новшеству, каждому изменению в старом, общепринятом обычае, каждому прогрессу религия оказывалась враждебной. 

Христианство явилось в мир в такое время, когда вино и хлеб и другие средства культуры были давно уже изобретены, когда уже было слишком поздно обожествлять их изобретателей, когда эти изобретения уже потеряли свое религиозное значение; христианство принесло в мир другое средство культуры: мораль, учение о нравственности; христианство хотело дать целительное средство против моральных, а не физических и политических зол, против греха. Остановимся на примере с вином, чтобы на нем пояснить отличие христианства от язычества, то есть язычества вульгарного, простонародного. Как можете вы, говорили христиане язычникам, обожествлять вино? Что это за благодеяние? Неумеренно потребленное, оно приносит смерть и гибель. Оно только в том случае благодетельно, если его потребляют умеренно, с рассудком, если его пьют без разгула; следовательно, полезность или вредоносность какой-либо вещи зависит не от нее самой, а от ее морального употребления. Христианство было в этом право. Но христианство сделало мораль религией, то есть нравственный закон --- заповедью божьей, дело человеческой самодеятельности --- делом веры. Ведь вера в христианстве есть принцип, основа учения о нравственности; <<вера есть источник добрых дел>> --- гласит известное изречение. У христианства нет бога вина, нет богини хлеба или хлебного зерна, нет Цереры, нет Посейдона, или бога моря и мореплавания; оно не знает никакого бога кузнечного дела или пиротехники, подобного Вулкану; но оно все же имеет бога вообще или, вернее, морального бога, бога искусства сделаться моральным и праведным. И с этим богом христиане и до сих пор выступают против всякого радикального, всякого основательного образования, ибо христианин не в состоянии мыслить себе никакой морали, никакой нравственной или человеческой жизни без бога; он производит поэтому мораль от бога, как языческий поэт законы и виды поэтического искусства производил от богов и богинь поэзии, как языческий кузнец и пиротехник производил технические приемы своего ремесла от Вулкана. Но как в настоящее время кузнецы и пиротехники вообще разумеют свое дело, не имея особого бога своим патроном-покровителем, так когда-нибудь и люди уразумеют искусство делаться без бога моральными и праведными. И только тогда они станут воистину моральными и праведными, когда у них не будет более бога, когда они перестанут нуждаться в религии; ведь только до тех пор, пока какое-либо искусство не совершенно, еще находится в пеленках, оно нуждается в религиозной охране. Ибо именно при посредстве религии заполняет человек пробелы своего образования, только из-за недостаточности общего образования и общей точки зрения делает он, как египетский жрец, свои ограниченные врачебные средства, свои моральные целебные средства святынями, свои ограниченные представления --- священными догматами, внушения своего собственного духа и настроения --- велениями и откровениями бога. 

Короче говоря: религия и образование противоречат друг другу, хотя образование, --- поскольку религия есть первая, старейшая форма культуры, --- и может быть названо совершенной религией, так что только истинно образованный человек есть и истинно религиозный. Однако это все же --- злоупотребление словами, ибо со словом религия постоянно связываются суеверные и негуманные представления; религия заключает в себе элементы, существенно противодействующие образованию, желая представления, обычаи, изобретения, сделанные человеком во время его детства, превратить в законы для человека, который уже вырос. Когда бог должен говорить человеку, чтобы он что-нибудь делал, подобно тому, как он приказывал израильтянам, чтобы они отправляли свою естественную нужду в особом отхожем месте, тогда человек еще стоит на точке зрения религии, но в то же время и глубочайшей некультурности; когда же человек делает что-нибудь по собственному почину, потому что ему это велит его собственная природа, его собственные разум и склонность, тогда отпадает необходимость религии, тогда ее место заступает образование. И как в настоящее время нам представляется смешным и непонятным, что требование естественного приличия было когда-то религиозным, так когда-нибудь, когда люди выйдут из состояния нашей мнимой культуры, из эпохи религиозного варварства, им будет казаться непонятным, что они веления морали и любви человеческой должны были, для того чтобы их осуществить, мыслить себе как веления бога, который за соблюдение этих велений их награждает, за несоблюдение --- наказывает. Так, Лютер говорит: 

\begin{quote}

Кто Эпикуру брат родной 

И хочет жить свинья-свинь\-ей, 

Тот пусть забудет поскорей 

Про божий суд и суд людей; 

Пусть --- как душа ни вопиет 

На жизнь загробную плюет; 

Пусть, не заботясь ни о ком, 

Все норовит тащить в свой дом; 

Пусть вечно пьет, рыгает, жрет 

И испражняется, как скот! 

\end{quote}

Мы имеем здесь разительный пример того, что культура некультурного человека есть религия и что эта культура, религия, сама есть некультурность, варварство. Религиозный человек удерживается от обжорства и пьянства не потому, что он питает к этому отвращение, не потому, что он находит в этом что-либо противоречащее человеческому существу, безобразное, животное, а из боязни наказаний, которые небесный судья за это положил в той ли, в этой ли жизни, или из любви к своему господу, --- короче говоря, из религиозных мотивов. Религия --- это то, что делает его не животным, граница между человеческим и животным состоянием; то есть в самом себе он имеет животность, вне себя и над собой --- человечность. Основой его человечности, его воздержания от пьянства, от обжорства является лишь бог, существо вне его, которое он, по крайней мере, представляет себе как от него отличное, вне его существующее; если нет бога, --- таков смысл приведенных слов Лютера, --- то я зверь, то есть именно основа и существо моей гуманности находятся вне меня. Но там, где человек имеет основу своей гуманности вне себя, в нечеловеческом, по крайней мере согласно его представлению, существе, где он, стало быть, человечен в силу нечеловеческих, религиозных оснований, там он не является еще истинно человеческим, гуманным существом. Я только тогда человек, если я действую по-человечески по собственному побуждению, когда я признаю и проявляю гуманность, как необходимое предназначение моей природы, как необходимое следствие моего субъективного существа. \emph{Религия устраняет лишь проявления зла, но не причины его; она препятствует лишь вспышкам грубости и зверства, но она не устраняет их основ, она лечит не радикально. Только там, где поступки человечности выводятся из основ, заложенных в природе человека, существует гармония между принципом и выводом, причиной и следствием, существует совершенство}. Но это делает или это ставит себе целью образование. Религия должна заменять образование, но она его не заменяет; образование же и в самом деле заменяет религию, делает ее излишней. У кого есть наука, говорил уже Гете, тот не нуждается в религии. Я ставлю вместо слова наука --- образование, ибо образование охватывает всего человека, хотя и это слово может быть оспорено, если, по крайней мере, иметь в виду то, что в настоящее время обычно понимается под образованием. Но какое слово безупречно? Не религиозными делать людей, а образовывать их, распространять образование по всем классам и сословиям --- вот, следовательно, задача времени. С религией уживаются, как это доказывает история вплоть до наших дней, величайшие жестокости, но не с образованием. Со всякой религией, покоящейся на теологической основе, --- а мы всегда имеем дело только с религией в этом смысле, --- связано суеверие; суеверие же способно ко всякой жестокости и бесчеловечности. Здесь нельзя себе помочь различием между ложной и истинной религией или религиозностью. Истинная религия, из которой удалено все дурное и жестокое, есть не что иное, как религия, ограниченная и просветленная образованием, разумом. И поэтому, когда люди, причисляющие себя к этой религии, теоретически и практически, словом и делом отвергают человеческие жертвоприношения, преследования еретиков, сожигания колдуний, смертные казни <<бедных грешников>> и тому подобные жестокости, то это приходится ставить не на счет религии, а на счет их образования, их разума, их доброты и человечности, которые они, конечно, привносят с собой и в религию. 

Против того, что я до сих пор развивал, а именно, что религия возникает лишь в древнейшие времена человечества, вообще лишь во времена грубости и некультурности и что поэтому только в такие времена она бывает в полной свежести и жизненной силе, можно привести возражение, что часто как раз самые образованные, самые ученые люди были в высшей степени религиозны. Однако это явление объясняется --- независимо от других причин, которые я приводил до сих пор, ибо здесь речь идет лишь о противоположности между религией и образованием, о противоположности, которую никто не может и не станет отрицать, ибо можно иметь религию без образования и образование без религии, --- объясняется, говорю я, тем, что вообще в человеке встречаются величайшие и друг с другом непримиримые противоречия. История человечества, в особенности же история религий, дает нам в этом отношении как раз удивительнейшие примеры, и примеры не только отдельных личностей, но и целых наций. Образованнейшие народы древности, чья творения составляют до сих пор основу научного образования, --- понимавшие искусство остроумные греки и практические, энергичные римляне, --- какого только смешного, бессмысленного религиозного суеверия ни придерживались они даже в свои лучшие времена! Римское государство само, собственно говоря, базировалось на предсказаниях, основывавшихся на качествах жертвенных животных, на молнии и других обыкновенных и необыкновенных явлениях природы, на пении, полете птиц и клевании ими пищи: римляне не предпринимали ничего важного, например, войны, если их священные курочки не имели аппетита. То есть на религиозных лжи и обмане, на которых, впрочем, и посейчас покоятся христианский трон и алтарь. Разве не заведомый обман, например, если еще и в настоящее время после результатов библейской критики, предпринятой даже теологами, Библия выдается народу за слово божие? Многими из своих религиозных обычаев и представлений греки и римляне нисколько не отличаются от грубых, некультурных народов. Можно поэтому быть в известной области образованным и умным человеком и все же в сфере религии следовать самому глупому суеверию. 

Мы находим это противоречие особенно часто в начале новейшего времени. Реформаторы философии и наук вообще были свободомыслящими и суеверными в одно и то же время; они жили среди злосчастного разлада между государством и церковью, светским и духовным, человеческим и божественным. Так называемое светское они подвергали своей критике; в церковных же и религиозных делах они были такими же верующими, как дети и женщины, смиренно подчиняли свой разум самым бессмысленным, фантастическим представлениям и догматам веры. Причину этого отвратительного явления легко понять. Религия освящает свои представления и обычаи, ставит от них в зависимость спасение людей, навязывает их человеку, как дело совести. Так они наследуются из поколения в поколение неизменными и неприкосновенными. Так, в религиозном Египте, замечает Платон в своих <<Законах>>  произведения искусства его времени и изготовленные за несколько тысячелетий до того были совершенно одинаковы, ибо каждое новшество преследовалось, а в Ост-Индии еще и в настоящее время преследуется, по свидетельству Паулинуса Сан-Бартоломео (<<Система браминов>>  1791), ни один художник и скульптор не смеет изготовить религиозное изображение, если оно не совпадает со старинными изображениями в храмах. Поэтому, в то время, как во всех других областях человек подвинулся вперед, в религии он, безнадежно слепой и безнадежно глупый, остается стоять на старом месте. Религиозные учреждения, обычаи и догматы веры продолжают еще быть священными, хотя они находятся в кричащем противоречии с прогрессировавшим разумом и облагороженным чувством человека, хотя уже давно перестали быть известными основа и смысл этих учреждений и представлений. \emph{Мы также живем еще среди этого отвратительного противоречия между религией и образованием; наши религиозные учения и обычаи находятся в величайшем антагонизме с нашей современной духовной и материальной точкой зрения. Устранить это безобразие и чрезвычайно пагубное противоречие --- вот в чем заключается в настоящее время наша задача}. Устранение этого противоречия есть необходимое условие возрождения человечества, единственное условие, так сказать, нового человечества и нового времени. Без него все политические и социальные реформы тщетны и ничтожны. Новое время нуждается в новом воззрении, в новых взглядах на первые элементы и основы человеческого существования, нуждается, --- если мы хотим сохранить слово религия, --- в новой религии! 

\phantomsection
\addcontentsline{toc}{section}{Двадцать четвертая лекция}
\section*{Двадцать четвертая лекция}

То обычное явление, что разум, по крайней мере в известных сферах жизни, уживается с суеверием, политическая свобода --- с религиозным рабством, естественнонаучное и промышленное движение вперед --- с религиозным застоем, даже ханжеством, многих привело к поверхностному взгляду и утверждению, что религия совершенно безразлична для жизни, в том числе для жизни общественной и политической. Единственно, к чему в этом отношении нужно стремиться, --- это к безусловной свободе верить, как кто хочет. Я, однако, возражаю на это, что такое положение вещей, при котором политическая свобода соединяется с религиозными предрассудками и ограниченностью, не есть надлежащее положение. \emph{Я гроша не дам за такую политическую свободу, которая оставляет человека рабом религии. Истинная свобода --- лишь там, где человек свободен также и от религиозных предрассудков; истинное образование --- лишь там, где человек возвысился над своими религиозными предрассудками и воображением. Но цель государства не может быть иной, как формировать настоящих, совершенных людей, --- совершенных, разумеется, не в смысле фантастики; поэтому государство, граждане которого, при наличности свободных политических учреждений, религиозно не свободны, не может быть истинно человеческим и свободным государством}. Не государство делает людей, а люди делают государство. Каковы люди, таково и государство. Там, где имеется налицо государство, там, разумеется, отдельные лица, сделавшиеся его членами в силу рождения или переселения, носят на себе печать этого государства; но что другое представляет собой государство по отношению к отдельным лицам, в него входящим, как не сумму и соединение уже существующих и образующих это государство людей, которые при помощи имеющихся в их распоряжении средств, при помощи созданных ими учреждений формируют вновь и позже приходящих сообразно их духу и воле. Поэтому там, где люди свободны политически и не свободны религиозно, там и государство не достигло совершенства и законченности. Но что касается второго пункта, свободы веры и совести, то, конечно, первым условием свободного государства является положение, что каждый <<может делаться праведным на свой лад>>  каждый может верить по-своему. Но это свобода весьма незавидная и бессодержательная; ибо она не что иное, как свобода или право каждому быть дураком на свой манер. Государство, как мы его понимали, до сих пор, не может, правда, ничего больше делать, как воздерживаться от всякого вмешательства в область веры, предоставить в этом отношении неограниченную свободу. Но задача человека в государстве заключается не только в том, чтобы верить, во что он хочет, но и верить в то, что разумно; вообще но только в том, чтобы верить, но чтобы и знать то, что он может и должен знать, если он хочет быть свободным и образованным человеком. Здесь нельзя удовольствоваться пределами человеческого знания. В области природы есть, правда, еще много непонятного; но тайны религии, которые скрыты от человека, человек может понять до самого основания. Именно потому, что он это может, он это должен. Наконец, совершенно поверхностна, опровергнута историей и самой обыденной, повседневной жизнью та точка зрения, что религия не имеет влияния на общественную жизнь. Эта точка зрения ведет свое происхождение лишь от нашего времени, когда религиозная вера стала химерой. Разумеется, где вера не является больше для человека истиной, она не имеет и практических последствий, она и не творит всемирно-исторических дел. Но там, где это имеет место, где вера уже только ложь, там человек находится в самом отвратительном противоречии с самим собой, там вера имеет, по крайней мере морально, пагубное действие. Но такой ложью является современная вера в бога. Устранение этой лжи есть условие нового, дееспособного человечества. 

Только что указанное явление, --- а именно, что религиозность в обычном смысле слова часто связана с противоположными свойствами, --- привело многих к гипотезе, или предположению, что существует особый орган религии или совершенно специфическое, особое религиозное чувство. Однако с большим правом можно было бы принять существование особого органа суеверия. В действительности положение, что религия, то есть вера в богов, в духов, в так называемые высшие невидимые существа, господствующие над людьми, врождена человеку, как и всякое другое чувство, --- это положение, переведенное на простой язык, означает: суеверие врождено человеку, как это уже утверждал Спиноза. \emph{Источник же и сила суеверия есть власть невежества и глупости, --- величайшая власть на земле: власть страха или чувства зависимости и, наконец, власть воображения; они из всякого зла, причину которого человек не знает, из всякого явления, --- хотя бы это было мимолетное атмосферное явление, какой-либо газ, пугающий человека потому, что он не знает, что это такое, --- делают злое существо, духа или бога; из каждого же счастливого случая, из каждой находки, из всякого добра, доставшегося ему нежданно, он делает доброе существо, доброго духа или доброго бога}. Так, например, караибы верят, что есть злой дух, действующий через посредство огнестрельного оружия, что злой дух поглощает луну при лунном затмении, что злой дух налицо даже там, где они замечают дурной запах. Но, обратно, у Гомера, когда кому-нибудь выпадает, благодаря случаю, неожиданно удача, то говорится, что бог дал ее. Вера в дьявола так же прирождена человеку или естественна, как и вера в бога, так что если допускать существование особого чувства бога или органа бога, то придется допустить существование в человеке и особого чувства дьявола или органа дьявола. 

До самого последнего времени и в самом деле вера в обоих была неразрывна; еще в восемнадцатом веке тот, кто отрицал существование дьявола, был таким же безбожником, как и тот, кто отрицал существование доброго или в собственном смысле так называемого бога. Тогдашние ученые защищали веру в дьявола с таким же остроумием, с каким современные ученые защищают веру в бога. В восемнадцатом веке ученые, даже протестантские, так же высокомерно обзывали глупостью отрицание дьявола, как в настоящее время они обзывают глупостью атеизм. Я отсылаю по этому поводу к подтверждающим цитатам из философского лексикона Вальха, имеющимся в примечаниях к моему <<Пьер Бейлю>>. Только половинчатость и бесхарактерность современного рационализма сохранила одну часть религиозной веры, а другую ликвидировала, разорвала связь между верой в добрых и злых духов или богов. Если поэтому берут под защиту религию, то есть веру в бога, на том основании, что она человечна, что почти все люди верят в бога, что человеку необходимо мыслить себе <<свободную>>  то есть человеческую, причину природы, то нужно быть настолько последовательным, настолько честным, чтобы на том же основании брать под защиту и веру в дьявола и ведьм, короче говоря, суеверия, невежество и глупость человека, \emph{ибо нет ничего более присущего человеку, нет ничего более распространенного, чем глупость, нет ничего более естественного, нет ничего более врожденного человеку, чем невежество, незнание}. 

Отрицательная теоретическая причина или, по крайней мере предпосылка, всех богов есть невежество человека, его неспособность вдуматься в природу; и чем невежественнее, чем ограниченнее, чем некультурнее человек, тем больше он сливается с природой, тем меньше он может отвлечься от себя. Так, перуанцы, видя солнечное затмение, верили, что солнце за какой-нибудь совершенный ими проступок сердится на них. Они, стало быть, верили, что солнечное затмение есть лишь следствие свободной причины, то есть недовольства, дурного расположения духа. Если же происходило лунное затмение, то они считали луну больной и были в тревоге, что она умрет, упадет затем с неба, всех их убьет и вызовет конец мира. Когда же луна опять начинала светить полным светом, то они радовались этому, как признаку ее выздоровления. Так, человек, стоя на точке зрения религии, на точке зрения, в которой имеются все корни веры в бога, приписывает даже свои болезни небесным телам. Так, индейцы на Ориноко считают даже солнце, луну и звезды живыми существами. Один из них сказал однажды Сальватору Гилии: <<Эти там наверху --- люди, как мы>>. Патагонцы верят, что звезды --- это бывшие когда-то индейцы, и что млечный путь --- это поло, на котором они охотятся за страусами; гренландцы подобным же образом верят, что солнце, луна и звезды были их предками, которые по особому случаю были перенесены на небо; точно также другие народы верили, что звезды --- это жилища или даже души великих мертвецов, которые перенесены на небо за их заслуги и там вечно сияют и блестят. Так, Светоний рассказывает, что когда после смерти Цезаря на небе появилась комета, римляне думали, что эта звезда есть душа Цезаря. Может ли человек идти дальше в претенциозности своего невежества, в очеловечении природы, в подчинении ее свободной причине, то есть человеческой силе воображения и произволу, чем если он видит в звездах своих коллег или предков или звездные ордена, которыми люди украшаются после смерти за их заслуги? В наше время верующие в бога смеются над подобными представлениями, но они не сознают того, что их вера в бога покоится на той же точке зрения, на том же основании. Разница лишь та, что они делают основой природы, или, вернее, заставляют из-за спины ее действовать не человека во плоти, не индивидуума, как я раньше указывал, индивидуума, как телесно отдельное существо, но абстрактно представленное существо человека. Сравните с тем, что в 11 и 21 лекциях я говорил о политеизме и монотеизме. Но, в сущности, все равно, вывожу ли я, как патагонцы, явления неба из настроений духа и решений воли, солнечный свет --- из добродушия и любезности солнца, его затмение --- из его злобности, его немилости к человеку, или, как христианин, верующий в бога, вывожу природу вообще из свободной причины или воли личного существа, ибо только личное существо имеет волю. 

Там, где вера в бога еще подлинная, там, где она поэтому является последовательной, строго согласованной верой, а не беспорядочной, разорванной, как современная вера в бога, там все произвольно, там нет физических законов, нет власти природы, там нагоняющие на человека страх ужасные, причиняющие несчастья явления природы выводятся из гнева бога или, что то же, дьявола, противоположные же явления природы --- из божьей благости. Но это выведение естественно необходимого из свободной причины имеет свое теоретическое основание лишь в невежестве и в силе воображения человека. Поэтому люди, приобретя некоторые познания в обыкновенных явлениях природы, усматривали главным образом в необыкновенных и неизвестных им явлениях природы, то есть в явлениях человеческого невежества, следы и доказательства произвольной или свободной причины. Так обстояло, например, дело с кометами. Так как кометы редко появлялись, так как не знали, к чему их отнести, то вплоть до начала восемнадцатого века люди и даже ученая чернь видели в них произвольные знамения, произвольные явления, которые бог производит на небе для исправления и наказания людей. Рационализм же, напротив того, отодвинул произвольную свободную причину к началу мира; кроме того, он объясняет все естественным путем, все без бога, то есть он слишком ленив, слишком беспорядочен, слишком поверхностен, чтобы восходить к началу, к принципам своих естественных, методов объяснения, к принципам своих взглядов; он слишком ленив, чтобы подумать над тем, является ли вопрос о начале мира разумным или детским вопросом, основывающимся лишь на невежестве и ограниченности человека; он не знает, как разумно ответить на этот вопрос; он наполняет, поэтому, пустоту в своей голове измышленным понятием <<свободной причины>>. Но он настолько непоследователен, что тотчас же вслед за тем отказывается от свободной причины и свободу заменяет природной необходимостью, вместо того, чтобы подобно старой вере продолжить эту первичную свободу в непрерывную цепь свобод, то есть произвольных действий, или чудес. 

Лишь бессознательность, бесхарактерность, половинчатость могут сочетать веру в бога с природой и естествознанием. Если я верю в бога, в <<свободную причину>>  то я должен также верить в то, что воля бога одна представляет собой природную необходимость, --- что вода не в силу своей природы, а по воле божьей, увлажняет, что, поэтому, она в любой момент, если бог захочет, может начать сжигать, приняв природу огня. Я верю в бога --- означает: я верю, что нет природы, нет необходимости. Или надо отказаться от веры в бога, или же от физики, астрономии, физиологии! Никто не может служить двум господам. И если берут под защиту веру в бога, то пусть берут под защиту, как я уже сказал, и веру в дьявола, духов и ведьм. Эта вера неотделима от веры в бога не только в силу своей одинаковой всеобщности, но и в силу своих одинаковых свойств и своего происхождения. Бог есть дух природы, то есть олицетворение того, как природа отражается на духе человека, или духовное изображение природы человеком, которое, однако, он мыслит себе отдельным от природы и самостоятельным существом. Так же точно дух человека, который бродит после смерти, то есть привидение, не что иное, как образ умершего человека, образ, который остается еще и после смерти, но который человек олицетворяет, представляя его себе существом, отличным от действительного, живого, телесного существа человека. Кто поэтому признает единый дух или единое привидение, великое привидение природы, тот пусть признает и другой дух или привидение человека. 

Однако я отклонился от моей настоящей задачи. Я хотел лишь сказать, что если желают принимать существование у человека особого органа для религии, то нужно прежде всего принять существование особого органа для суеверия, для невежества и лености мысли в человеке. Но есть, однако, люди, которые в одном отношении рационалистичны, являются неверующими, в другом же суеверны, именно потому, что они не выяснили себе известных вещей, потому что некоторые, часто совсем им неизвестные, влияния и причины не дают им выйти за пределы того или иного пункта. Если пожелать объяснить эти противоречия органически, то пришлось бы поэтому принять два совершенно противоположных органа или чувства в одном и том же человеке. Есть ведь люди, которые на деле отрицают, не признают, высмеивают то, что головой своей они признают, и, наоборот, отрицают головой то, что признают сердцем, которые боятся, например, привидений, отрицая в то же время существование привидений, даже сердясь на самих себя и стыдясь того, что они ночью рубашку принимают за духа, за привидение. Если для объяснения особого явления пожелать тотчас же искать прибежища в особом чувстве или органе, то пришлось бы у этих людей принять существование особого органа для священного мира духов и для страха привидений, и другой орган для нечестивого отрицания мира привидений. Нет, правда, ничего более удобного, чем выдумывать совсем особую причину для какого-нибудь необычного явления; но именно это удобство и должно возбудить в нас подозрительность относительно подобных способов объяснения. Что же касается в особенности религии, то ничто не дает нам права выводить ее из особого чувства, особой способности или органа, как показало все наше предыдущее объяснение. 

Для явственного доказательства я апеллирую к органам чувств. Ведь только на основании проявлений этого чувства, его действий, его обнаружений при посредстве органов чувств можем мы что-нибудь познавать, также в том числе и религию. Самый важный, обнаруживающий больше всего ее существо и в то же время самый бросающийся в глаза акт религии есть молитва или поклонение; ибо, ведь, поклонение есть молитва, производящая впечатление на чувства, выражающаяся в чувственных телодвижениях и знаках. Если мы взглянем, поэтому, на различные способы поклонения народов, то найдем, что религиозные чувства не отличаются от чувств, которые человек имеет и вне и без религии в настоящем смысле этого слова. У Мейнерса, который по этому поводу, как и по другим вопросам, собрал самые важные данные в своей работе, мы читаем: <<самым общим естественным выражением смирения перед высшими существами, как перед неограниченными повелителями, было падение перед ними ниц на землю. Падение на колени, хотя и не столь обычно, как падение ниц, все же весьма часто встречается у самых различных народов. Египтяне коленопреклонением почитали своих богов, как и своих царей и их доверенных лиц. Евреи не смели так же, как и нынешние магометане, садиться во время молитвы, потому что благопристойность испокон века запрещала на Востоке, чтобы подданные садились в присутствии повелителей, клиенты --- в присутствии патронов, женщины, дети и рабы --- в присутствии мужей, отцов и господ. На древнем Востоке, как и в древней Греции и Италии, с незапамятных времен подданные выражали свое почтение и преданность повелителям, рабы --- своим господам, женщины и дети --- своим мужьям и отцам тем, что они целовали им руки или колени и края одежд, или, наконец, ноги; что подчиненные делали по отношению к начальству, то делали люди по отношению к богам. Они целовали, стало быть, или руки, или колени, или ноги у изображений богов. Свободные греки и римляне позволяли себе целовать подбородок, даже уста божественных статуй>>. 

Мы видим на этом примере, что люди не имеют других выражений почитания, других способов выражения своих чувств и настроений по отношению к божественным существам, чем по отношению к существам человеческим, что поэтому настроения и чувства, которые человек питает по отношению к религиозному предмету, к богу, он питает и по отношению к нерелигиозным предметам, что, следовательно, чувства религии не представляют собой чего-то особенного, или что нет специфического религиозного чувства. Человек падает на колени перед богами; но то же самое он делает и перед своими властителями, перед теми вообще, в чьих руках его жизнь; он самым смиренным образом молит их о сострадании; короче говоря, он проявляет к людям то же почитание, что и по отношению к богам. Так, римляне с тем же благоговением или набожностью, с какою они почитали богов и отечество, почитали своих кровных родных, родителей. Набожность, благоговение есть, как говорит Цицерон, справедливость по отношению к богам, но она есть также, как он же говорит в другом месте, и справедливость по отношению к родителям \hyperlink{24}{(24)}\hypertarget{b24}{}. У римлян поэтому полагалось, как замечает Валерий Максим, то же наказание за оскорбление богов, как и родителей. Но еще выше, чем достоинство родителей, которые почитались наравне с богами, стояло, как он же замечает, величие отечества, то есть высшим богом римлян был Рим. У индусов в числе пяти великих религиозных церемоний или таинств, которые согласно законодательству Ману должен отправлять ежедневно глава семьи, имеется таинство людей, таинство гостеприимства, почитание и уважение гостей. Но в особенности Восток довел почитание государей до высшей степени религиозного сервилизма. Так, например, в Китае все подданные, даже платящие дань главы государств, обязаны трижды склонить колени перед императором и девять раз коснуться головой земли, а в известные дни месяца самые знатные мандарины появляются перед императором, и хотя бы он даже при этом сам не присутствовал, выражают такое же почтение пустому трону. И даже перед императорскими грамотами и указами полагается склоняться на колени и девять раз касаться земли головой. Японцы считают своего императора стоящим настолько высоко, что <<даже особы первого класса могут наслаждаться лишь счастьем лицезреть ноги императора, не смея в то же время бросать взоры выше>>. Именно потому, что человек, и особенно восточный человек, испытывает по отношению к своему правителю величайшее почтение, на какое лишь способен человек, именно поэтому его фантазия и сделала правителей богами, разукрасив их всеми свойствами и титулами божества. Общеизвестны гиперболические титулы султанов и китайских императоров. Но даже маленькие ост-индские князьки называются <<царями царей, братьями солнца, луны и звезд, господами прилива и отлива океана>>. У египтян также звание царей отождествлялось с божеством, и даже в такой степени, что царь Рамзес изображался поклоняющимся самому себе, как богу. 

Христиане унаследовали от восточных народов как религию, так и это обожествление государей. Свойства или титулы божества, которые имели языческие императоры, были и титулами первых христианских императоров. Еще и посейчас христиане так смиренно высказываются о своих правителях, так раболепно, как можно высказываться только о боге. Еще и сейчас титулы их государей такие же фантастические гиперболы и преувеличения, как те титулы, которыми сыздавна религиозная лесть пыталась прославить богов. Еще и сейчас различия, существующие между божественной и государевой властью --- лишь дипломатические различия ранга, но не различия по существу. В том-то и дело, что нет особого религиозного чувства, особой религиозной способности и, следовательно, особенного религиозного предмета --- предмета, который исключительно и один претендовал бы на религиозное почитание, а потому поклонение богам и идолам, религия и суеверие происходят в конечном счете из общего корня в существе человека. 

\phantomsection
\addcontentsline{toc}{section}{Двадцать пятая лекция}
\section*{Двадцать пятая лекция}

Поклонение идолами поклонение богу, сказал я в конце предыдущей лекции, имеют общие корни в естественном существе человека. Нет никакого особого органа ни для того, ни для другого. И если бы пожелали утверждать существование такового для религии, то пришлось бы с тем же правом отстаивать подобный орган и для суеверия вообще. Но откуда же происходит различие между идолопоклонством и поклонением богу? Это различие основывается исключительно на том, что чувства и выражения почитания, которые должны быть достоянием исключительно одного предмета, почитающегося <<священным>>  могут быть также обращены и к другому предмету, будь то предмет естественный, чувственный или духовный. Религия, которая высказывается в публичных исповеданиях, в определенных богослужебных формах, говорю я в <<Сущности христианства>>  есть публичное исповедание любви. Любимую женщину, проявляющую величайшую власть над мужчиной, женщину, которая в его глазах есть прекраснейшее и совершеннейшее существо, которая именно поэтому вызывает в нем чувство зависимости, чувство, что он без нее не может жить или, во всяком случае, не может быть счастливым, эту женщину избирает он предметом своей любви и, по крайней мере пока он ею не обладает, пока она остается лишь предметом его желаний и воображения, делает предметом также и величайшего почитания, --- предметом, которому он приносит такие же жертвы и славословия, как религиозный человек своему богу. С религией --- и любовь есть религия --- обстоит точно так же. Религия почитает дерево, но не всякое без различия, а самое большое, высокое; не просто реку, по самую могучую, самую благодетельную, как, например, египтяне --- Нил, жители Индии --- Ганг; не каждый источник, но лишь источник, выделяющийся своими особыми свойствами: германцы, например, почитали в особенности соляные источники; религия почитает не светящиеся небесные тела вообще, а наиболее выдающиеся --- солнце, луну, планеты или другие выделяющиеся звезды. Или религия почитает человеческое существо, но не в каждом, однако, лице, а лишь в лицо прекрасного человека, --- так, например, греки обожествляли Филиппа Кротонского, хотя он и напал на их землю, потому что он был прекраснейший из мужей; или в лице своего деспота, как восточные народы, или в лице героя, имеющего заслуги перед отечеством, как греки и римляне; или существо человека вообще, дух, разум, потому что они считают его самым чудесным, прекрасным, наивысшим. Но как любовь, почет, которые я выказываю данной женщине, я могу выказывать и другой женщине, точно так же и почет, выказываемый мной данному дереву, я могу оказывать и другому: германцы, например, почитали дуб, а славяне --- липу; так могу я и почитание, которое я оказываю отвлеченному существу человека, духу, оказывать и действительному индивидуальному человеческому существу; почитание, которое я выражаю отвлеченному существу природы и, как причине ее, вымышленному существу богу, творцу, выражать и чувственному существу природы --- творению; ибо чувственное существо, создание, имеет за себя все чувства человека, нечувственное же существо имеет все чувства против себя и проявляет поэтому гораздо меньшую власть над человеком. 

Отсюда проистекает ревность религии, ревность ее предмета, бога. Я ревнивый бог, говорится об Иегове в Ветхом Завете. И это изречение евреи и христиане повторяли в тысячах вариаций. Но бог ревнив или представляется таковым, потому что чувства или аффекты и настроения преданности, любви, почитания, доверия, страха могут быть также легко перенесены и на другой предмет, на других богов, на другие существа, каковыми являются природные и человеческие существа, бог же один на них претендует. Лишь с появлением так называемых положительных, то есть произвольных, законов, возникает поэтому различие между идолопоклонством и богослужением. Вы не должны доверять людям, но мне; вы не должны бояться природных явлений, но лишь меня одного; вы не должны поклоняться звездам, как будто от них исходит ваше благосостояние и спасение, но мне, который поставил звезды на служение вам, --- так говорит бог Иегова, монотеистический бог вообще, своим слугам, чтобы предохранить их от идолопоклонства. Но ему не нужно было бы так говорить, приказывать, чтобы люди ему одному лишь доверяли и служили, если бы существовало особое религиозное чувство, особый религиозный орган. Как мне не нужно приказывать глазу: ты не должен слушать, служить звуку; или уху: ты не должно видеть, не должно обслуживать свет, так и предмету религии не нужно было бы говорить человеку: ты мне лишь должен служить, если бы был особый религиозный орган, ибо этот орган так же мало имел бы отношения к другому, не религиозному предмету, как мало ухо имеет отношения к свету, и глаз к предмету слуха. И так же мало, как глаз ревнует ухо, как мало он боится, чтобы ухо не отбило у него его предмета и не присвоило его себе, так же мало мог бы бог ревновать к природным и человеческим существам или мыслиться ревнивым, если бы существовал исключительно религиозный или божественный, только ему соответствующий орган. 

Орган религии есть чувство, есть сила воображения, есть потребность или стремление быть счастливым, но эти органы отнюдь не распространяют своей власти над особой категорией предметов --- предметов, обозначаемых как религиозные, точно существуют таковые, а не каждый предмет, каждая сила, каждое явление, как человеческое, так и природное, может быть предметом религии. Но предметом религии --- религии, по крайней мере в собственном смысле слова --- делается предмет фантазии, чувства, стремления к счастью лишь при особых условиях, при условиях, о которых я только что говорил, при наличности той точки зрения, когда человеку из-за недостаточности образования, учености, критики, различения между субъективным и объективным, предмет, или существо не является тотчас же тем, что оно само есть, чем оно является в действительности, что оно из себя представляет как объект разума и чувства, но лишь как существо чувства, фантазии, стремления к счастью. Правда, и для натуралиста природа есть предмет стремления к счастью, --- ибо кто может быть счастливым, например, в тюремной клетке, где нет ни простора, ни воздуха, ни света? --- предмет воображения, фантазии, предмет чувства, даже чувства зависимости, но натуралист не упускает из виду ее действительного предметного существования, и именно поэтому его не вводит в обман его стремление к счастью, его не одолевают его чувства, его не опережает его фантазия, и поэтому природа ему не кажется субъективным, то есть личным, произвольным, милостивым и немилостивым, карающим и награждающим существом, следовательно, существом, которое по необходимости, в силу своей природы, является предметом жертв и покаяний, хвалебных и благодарственных песен, почтительных просьб и коленопреклонений, то есть предметом религии. И натуралист или гуманист --- приведу еще пример --- еще почитает мертвых, но не религиозно, не как богов, потому что он не делает, как это делает религиозное воображение, существа, имеющиеся лишь в представлении, существами действительными, личными, потому что он не переносит ощущений, которые в нем вызывает мертвец, на самый предмет, не считает мертвецов ужасными, страшными существами, вообще существами, которые еще имеют волю, способность вредить или приносить пользу, которых еще следует почитать, бояться, просить и умилостивлять, как действительные существа. 

Вернемся, однако, назад, к нашей настоящей теме. Переход от язычества к христианству, от религии природы к религии духа или человека я объяснил как акт воображения. Сначала я показал, что бог есть образ, существо воображения, причем я одновременно показал различие между христианским, или монотеистическим, и языческим, или политеистическим, богом, а именно, что языческий бог есть материальный, телесный, единичный образ, христианский же бог есть духовный образ, есть слово, что поэтому, чтобы познать сущность христианского бога, необходимо лишь понять сущность слова. Этим, однако, я и ограничил мое выведение религии из воображения, я установил различие между произведениями религиозной силы воображения и простыми поэтическими вымыслами, или фикциями, показал, что религиозное воображение действует лишь в союзе с чувством зависимости, что боги никоим образом не являются только существами воображения, но и предметами сердечной потребности, --- предметами тех чувств, которые охватывают человека в важнейшие моменты жизни, в счастье и в несчастье; что боги именно потому, что человек стремится получить приятное, хорошее и устранить неприятное, плохое, являются и предметами стремления к счастью, потребности в нем. Этот пункт привел нас к различию между религией и образованием, молитвой и трудом; религия в том сходится с образованием, с культурой, с трудом, что она имеет культурные цели, но расходится в том, что она этих целей хочет достигнуть без культурных средств. После того, как я таким образом наметил это различие, я возвращаюсь к религии, как к предмету стремления к счастью. Я высказал по этому случаю смелое положение: боги суть превращенные в действительность или представленные, как действительные существа, желания людей; бог есть не что иное, как стремление человека к счастью, нашедшее свое удовлетворение в фантазии. Я заметил, однако, что боги столь же различны, как и желания людей или народов, ибо хотя все люди желают быть счастливыми, но один делает одно, другой другое объектом своего счастья. У язычников поэтому другие боги, чем у христиан, ибо у них другие желания. Или --- отличие христианского бога от языческого покоится на отличии христианских желаний от желаний язычников. <<Каково твое сердце, таков и твой бог>>  --- говорит Лютер. <<Все народы, --- говорит Мейнерс в указанном сочинении, --- просили богов вплоть до возникновения христианства лишь о временных благах и об устранении временных зол \hyperlink{25}{(25)}\hypertarget{b25}{}. Дикие племена рыбаков и охотников молились своим богам, чтобы они сделали удачными их рыбную ловлю и охоту, пастушеские народы --- чтобы боги благословили их пастбища и стада, земледельческие --- чтобы боги благословили их сады и поля. Все без исключения молили для себя и членов своего племени о здоровье и долголетии, о богатстве и благоприятствующей погоде и победе над врагами и противниками>>. То есть язычники имели ограниченные желания, чувственные, материальные, на языке христиан --- земные, плотские желания. Но именно поэтому у них были материальные, чувственные, ограниченные боги, и столько богов, сколько имеется чувственных желательных благ. Так, у них был бог богатства, бог здоровья, бог счастья, удачи и так далее, и так как желания людей сообразовались с их сословием, с их занятием, то каждое сословие у греков и римлян имело своих особых богов, пастух --- пастушеских богов, земледелец --- крестьянских богов, купец --- своего Меркурия, которого он умолял о прибыли \hyperlink{26}{(26)}\hypertarget{b26}{}. 

Различные предметы языческих желаний, впрочем, не <<безнравственны>>; не безнравственно желать здоровья; наоборот, это --- совершенно разумное желание; не безнравственно также желать быть богатым, --- ведь благодарят же благочестивые христиане своего бога, когда получают богатое наследство или натыкаются на счастливую находку; безнравственными или, вернее, бесчеловечными, потому что только бесчеловечное безнравственно, были тогда лишь пожелания или молитвы язычников о богатстве, когда они просили богов, чтобы те отправили на тот свет их родственников, их родителей, чтобы тем самым получить их имущество. Языческие желания были желаниями, не выходившими из рамок природы человека, не переступавшими границы этой жизни, этого действительно чувственного мира. Но именно поэтому и их боги не были такими неограниченными, супранатуралистическими, то есть сверхъестественными, существами, как христианский бог. Как желания язычников не были внемировыми и сверхмировыми желаниями, так не были таковыми и их боги; они скорее были едиными с миром, мировыми существами. Христианский бог делает с миром что хочет, он его создает из ничего, потому что мир для него ничто, потому что он сам был, когда мир был еще ничем; но языческий бог в своем творении и действии связан с веществом, с материей; даже те языческие философы, которые больше всего приближались к представлениям христианства, верили в вечность материи, основного вещества мира, уделяли своему богу лишь роль мирового формировщика, но не настоящего творца. Бог язычников был связан с материей, потому что языческие желания и мысли были связаны с веществом, с содержанием действительного мира. Язычник не отделялся от мира, от природы; он мог мыслить себя лишь как ее часть; у него не было поэтому отличного от мира и от него оторванного бога. Мир был для него божественным, чудесным существом или, вернее, самым высоким, самым прекрасным, что он мог себе представить. В одинаковом смысле употребляют поэтому языческие философы-теисты слова: бог, мир, природа. Каков человек, таков и его бог; языческий бог есть образ языческого человека или, как я выражаюсь в <<Сущности христианства>>  не что иное, как сущность человека-язычника, опредмеченное и изображенное как самостоятельное существо. Общее или одинаковое в различных богах и религиях есть лишь то, что обще человеческой природе. Как ни различны люди, но все они все-таки люди; одинаковость и единство человеческого рода, человеческой организации есть одинаковость богов; эфиоп рисует себе бога черным, как он сам, кавказец --- таким, каков цвет его кожи; но все они придают богам человеческий облик или представляют их себе как человеческие существа. 

Поверхностно, впрочем, не принимать во внимание различия богов; для язычника есть только языческий бог, тот бог, который един с его, язычника, отличием от других народов и людей, бог для христианина есть лишь христианский бог. Многие строгие христиане отрицали за язычниками даже самую веру в бога, ибо боги язычников --- не боги в понимании христиан, они уже в силу своего множества противоречат христианскому понятию божества. Христианский же бог есть не что иное, как существо человека-христианина, олицетворенное или опредмеченное, представленное силой воображения как самостоятельное существо. Христианин имеет сверхземные, сверхчувственные, сверхчеловеческие, сверхмировые желания. Христианин --- по крайней мере истинный христианин, который не вобрал в себя языческих элементов, как современные светские люди и ханжи, --- не желает себе ни богатства, ни почетных мест, ни долгой жизни, ни здоровья. Что такое здоровье в глазах христианина? <<Ведь вся эта жизнь есть не что иное, как болезнь, только в вечной жизни здоровье>>  --- говорит св. Августин. Что такое долгая жизнь в понимании христианина? В сравнении с вечностью, которую христианин имеет в своей голове, самая долгая жизнь есть мимолетное мгновение. Что такое земные блеск и слава? В сравнении с небесной славой то же самое, что блуждающий огонек в сравнении с небесным светом. Но именно благодаря этим своим желаниям христианин имеет и сверхземного, сверхчеловеческого и сверхмирового бога. Христианин не смотрит на себя, подобно язычнику, как на члена природы, как на часть мира. <<Настоящего града не имамы, --- говорится в Библии, --- но грядущего взыскуем>> (Не имеем здесь находящегося града, но ищем будущего царства небесного). <<Наша жизнь (то есть наше право уроженца и право гражданина) на небесах>>. <<Человек, --- говорит определенно отец церкви Лактанций, --- не есть продукт мира, не есть часть мира>>; <<Человек, --- говорит Амвросий, --- выше мира>>. <<Одна душа, --- говорит Лютер, --- лучше целого мира>>. 

У христианина есть свободная причина природы, господин природы, чьей воли, чьего слова природа слушается, есть бог, который не связан так называемой причинной связью, необходимостью, цепью, которая соединяет следствие с причиной и причину с причиной; тогда как языческий бог связан с природной необходимостью и даже своих любимцев не может избавить от рокового жребия смерти. 

Но у христианина есть свободная причина, потому что он в своих желаниях не связывает себя с общим порядком, с необходимостью природы. Христианин желает себе существования, --- и верит в него, --- жизни, где он был бы избавлен от всех потребностей, от всей природной необходимости вообще, где он жил бы, не имея надобности дышать, спать, есть, пить, производить и рожать, тогда как у язычников даже бог подвержен необходимости сна, любви, еды и питья, именно потому, что язычник не освободился от необходимости природы, не мог мыслить себе существования без естественных потребностей. Христианин осуществляет поэтому свои желания быть свободным от всех потребностей и необходимостей природы в существе, которое действительно свободно от природы, которое может упразднить и устранить и действительно устраняет все ограничения и препятствия природы, мешающие осуществлению этих христианских желаний. Ведь природа есть единственное ограничение человеческих желаний. Ограничением желания летать, как ангел, или в один миг оказываться в желательном отдаленном месте является тяжесть; ограничением желания постоянно заниматься религиозными созерцанием и чувствами является телесная потребность; ограничением желания быть безгрешным, или, что то же, праведным \hyperlink{27}{(27)}\hypertarget{b27}{} --- телесность и чувственность моего существования; ограничением желания жить вечно, то есть ограничением, противостоящим осуществлению этого желания, --- смерть, необходимость конечности, тленности. Таким образом, все эти желания христианин осуществляет или создает возможность их осуществления в существе, которое, согласно его воображению, стоит над природой и вне ее, против воли которого природа бессильна. 

Чем человек не является в действительности, но чем он хочет быть, тем он делает своего бога или --- это его бог. Христианин желает быть существом совершенным, безгрешным, нечувственным, не подчиненным телесным потребностям, блаженным, бессмертным, божественным, но он им не является; он поэтому представляет себе то, чем он сам хочет быть и чем он когда-нибудь надеется стать, как существо, от него отличное, которое он называет богом, которое, однако, в своем основании есть не что иное, как существо его собственных сверхъестественных желаний, как его собственное существо, выходящее за пределы природы. Вера в происхождение мира от свободного, внемирового, сверхъестественного существа теснейшим образом связана поэтому с верой в вечное небесное существо. Ведь порукой, что сверхъестественные желания христиан будут исполнены, является как раз то, что сама природа зависит от сверхъестественного существа, что она обязана своим существованием произволу этого существа. Если природа не от бога, если она от самой себя, если она необходима, то необходима и смерть, то неизменны, непреодолимы и вообще все законы, или естественная необходимость, которой подвержено человеческое существование. Где природа не имеет начала, там не имеет она и конца. Христианин же верит в конец природы, или мира, и желает его; он верит, что все жизненные отправления и естественная необходимость прекратятся, и желает этого, он должен поэтому верить и в начало и притом в духовное, произвольное начало природы, телесного существа и жизни. Необходимой предпосылкой конца является начало, необходимой предпосылкой веры в бессмертие является вера в божественное всемогущество, которое даже мертвых пробуждает, для которого нет ничего невозможного, для которого не существует естественного закона, не существует необходимости. При посредстве догмата сотворения из ничего, являющегося величайшим шедевром божественного всемогущества, человек внушает себе уверенность, говорю я в <<Сущности христианства>> в том, что мир есть ничто и бессилен против человека, а, выражаясь точнее, человек посредством этого догмата обретает утешительную веру. <<Мы имеем господа, --- говорит Лютер, --- который более велик, чем весь мир; мы имеем такого могущественного господа, что когда он говорит, то рождаются все вещи. Зачем же нам бояться, когда он к нам благосклонен>>. <<Кто верит, --- говорит он же в своем толковании Моисея, --- что бог есть творец, который из ничего делает все, тот по необходимости должен так умозаключать и говорить: поэтому бог может и мертвых пробуждать>>. Вера в чудеса тождественна с верой в бога --- по крайней мере, в христианском смысле этого слова --- стало быть, и с верой в христианского бога. 

\phantomsection
\addcontentsline{toc}{section}{Двадцать шестая лекция}
\section*{Двадцать шестая лекция}

Понятие чуда --- одно из важнейших для познания сущности религии, в особенности христианской. Мы должны поэтому немного на нем остановиться. Прежде всего мы должны поостеречься смешать чудеса религии с так называемыми чудесами природы, например с <<чудесами неба>>  как озаглавил один астроном свою астрономию, с <<чудесами геологии>>  или истории земли, --- как один англичанин окрестил свою геологию. Чудеса природы суть вещи, возбуждающие наши удивление и изумление, потому что они выходят из круга наших ограниченных понятий, наших ближайших, обыкновенных представлений и опыта. Так, мы дивимся, например, окаменелым скелетам животных пород, когда-то хозяйничавших на земле, динозавров, мегатериев, ихтиозавров и плезиозавров, этих чудовищных разновидностей ящеров, потому что величина их далеко превосходит те размеры, которые мы знаем у ныне живущих пород животных. Но религиозные чудеса не имеют ничего общего с мегатериями, с динозаврами, ихтиозаврами и плезиозаврами геологии. Так называемые чудеса природы чудеса для нас, но не чудеса сами по себе или для природы; они имеют свое основание в существе природы, все равно, откроем и поймем мы его или нет. Только теистические, религиозные чудеса превосходят силы природы; они не только не имеют своего основания в существе природы, но они ему противоречат; они суть доказательства, произведения существа, от природы отличного и сверхъестественного. <<Хотя, --- говорит, например, ученый Фоссий в своем сочинении о происхождении и развитии язычества, --- бог и предписал небесам их порядок, он тем не менее не отказался от своего права менять его, ибо он даже солнцу приказал остановиться. Так, вопреки естественному порядку, который называется необходимым, вопреки естественной необходимости, по его повелению родила дева, слепые стали зрячими, мертвые неоднократно воскрешались>>. Чтобы сделать религиозное чудо правдоподобным, постоянно ссылались, правда, на естественные чудеса, которые, однако, совсем не чудеса. Этот прием принадлежит к числу тех многочисленных благочестивых обманов, которые во все времена и во всех религиях допускались с целью одурачить людей и удержать их в религиозном рабстве. Однако различие между обоими видами чудес становится очевидным уже из одного того, что естественное чудо есть нечто для человека безразличное, в религиозном же чуде человек заинтересован, в нем участвует его эгоизм. 

Религиозное чудо имеет свое основание не во внешней природе, но в человеке. Религиозное чудо имеет своей предпосылкой какое-либо человеческое желание, человеческую потребность. Религиозные чудеса происходят тогда, когда есть нужда, когда человек хочет быть избавленным от какого-либо зла, от которого он, однако, пока дело идет естественным порядком, не может быть избавлен. В чудесах дает себя чувственно знать сущность религии. Как и религия, чудо есть дело не только чувства и фантазии, но и воли, стремления к счастью. Поэтому я в <<Сущности христианства>> определяю чудо как реализованное супранатуралистическое, то есть как осуществленное в действительности или как представленное осуществленным сверхъестественное желание. Я говорю сверхъестественное, потому что желания христианина по своему предмету и содержанию выходят за пределы природы и мира. Впрочем, желания вообще, по крайней мере по форме, по тому, как они хотят быть исполнены, супранатуралистичны. Я желаю, например, быть дома тогда, когда я нахожусь далеко от дома, на чужбине. Предмет этого желания не есть что-либо сверхъестественное; ибо я могу, ведь, достигнуть его естественным путем, стоит мне только отправиться домой. Но сущность желания такова, что я хотел бы сейчас же, не тратя времени, быть дома, что я хотел бы тотчас же оказаться в действительности там, где я нахожусь в мыслях. Если мы рассмотрим чудеса, то мы найдем, что в них опредмечивается, принимает чувственную форму, становится действительным не что иное, как сущность желания. Христос исцеляет больных; исцеление больных не чудо; сколько больных выздоровело естественным путем! Но он исцелял их так, как больной желал бы быть исцеленным: мгновенно, а не скучным, трудным и дорогим путем естественных целебных средств. <<Он говорит, --- замечает Лютер, --- будь здоров! и больные делаются здоровыми. Следовательно, ему не нужно никакого лекарства, он делает их здоровыми своим словом>>. Христос исцеляет больных даже на расстоянии; ему совсем не нужно телесно передвигаться с места на место, чтобы исцелять; больной ведь не может дождаться своего врача; это именно желание является чем-то таким, что хочет иметь человека около себя, перенести его волшебным образом; желание не связывает себя рамками пространства и времени; желание не ограничено, не связано, свободно, как бог. Христос исцеляет, однако, не только болезни, которые могли бы быть преодолены естественным путем; он исцеляет и неисцелимые болезни; он делает слепорожденных зрячими. Искусство излечивает и слепорожденных, но лишь в том случае, когда слепота излечима, когда, стало быть, исцеление не есть чудо. <<С тех пор, как мир стоит, ни разу не случилось, чтобы кто-нибудь слепорожденному раскрыл глаза. Если бы исцеливший не был от бога, он не мог бы этого совершить>>. 

Но и эта божественная чудотворная сила делает лишь чувственно доступным, наличным силу человеческих желаний. Для человеческого желания нет ничего невозможного, недостижимого. Христос пробуждает мертвецов, как, например, Лазаря, <<который уже четыре дня находился в гробу>>  <<который уже смердел>>. Но мы каждый день в наших желаниях, в нашей фантазии пробуждаем дорогих нам мертвецов. Правда, мы остаемся при одних лишь желаниях, при одной лишь фантазии. Но бог может выполнить то, что человек лишь желает, то есть религиозная фантазия воплощает в своих богах желания человека. Вера в бога и вера в чудеса поэтому --- одно и то же; чудо и бог отличаются друг от друга лишь так, как действие --- от действующего существа. Чудеса --- это доказательства того, что существо, творящее чудеса, есть существо всемогущее, то есть существо, которое может исполнить все желания человека и именно поэтому и характеризуется человеком и почитается им как существо божественное. Бог, который не творит больше чудес, а стало быть, и не внемлет молитвам, не исполняет желаний, --- за исключением тех желаний, удовлетворение которых уже заложено в естественном ходе вещей, естественно, возможно, которые, следовательно, и без него и без молитвы были бы исполнены, --- есть негодный, бесполезный бог. Нет ничего более поверхностного, произвольного, чем то, как современные христиане, так называемые рационалисты, обходятся с чудесами, как они их упраздняют и тем не менее все же хотят сохранить христианство, христианского бога, как они хотят естественным образом объяснить чудеса и, стало быть, уничтожить тот смысл, который имеет или должно иметь чудо, или как-нибудь иначе самым легкомысленным образом от них отделаться. Один современный, уже ранее упоминавшийся рационалист приводит даже в защиту своего поверхностного и легкомысленного толкования чуда следующее место из Лютера: <<Так как гораздо большее значение принадлежит слову, чем поступкам и делам Христа, то если бы нужно было выбирать, то лучше было бы лишиться дел и происшествий, чем слова и учения; поэтому надлежит превыше всего воздать хвалу тем книгам, которые больше всего трактуют об учении и слове господа Христа. Ибо если бы чудесных деяний Христа и не было, и мы бы о них ничего не знали, то нам все же достаточно было бы слова, без которого у нас не было бы жизни>>. Если Лютер порой и равнодушен к чуду, то он имеет в виду чудо, рассматриваемое без религиозного отношения, без веры, как некоторое историческое, то есть прошедшее, мертвое событие. Что толку другим людям от того, что тот или другой еврей был чудесно исцелен, тот или другой был чудесно накормлен? Чудо, как исторический факт, простирает свое влияние лишь на то время и на то место, когда и где оно произошло; око, правда, в этом смысле имеет, как говорят рационалисты, лишь относительную ценность, только для современников, у которых эти чудеса произошли на глазах и которым они пошли на пользу. Но это отнюдь не есть истинный религиозный смысл чуда. 

Чудо должно служить на деле доказательством того, что чудотворец есть всемогущее, сверхъестественное, божественное существо. Мы должны удивляться не чуду, а причине его, тому существу, которое это чудо совершает и подобные чудеса совершать может, если это потребует человеческая надобность. Слово, учение, разумеется, постольку больше значит, чем дело, поскольку дело идет на пользу только отдельным лицам, поскольку оно прикреплено к определенным времени и месту, тогда как слово проникает всюду, не теряет своего смысла и в наше время. И, однако, чудо говорит то же самое, если я его верно понимаю, что и слово, что и учение, с той лишь разницей, что учение говорит отвлеченно, словами то, что чудо выражает в чувственных примерах. Слово говорит: <<Я есмь воскресение и жизнь. Кто верит в меня, тот будет жить, хотя бы умер. И кто живет и верит в меня, тот никогда не умрет>>. Но что говорит воскрешение Лазаря из мертвых? Что говорит собственное воскресение Христа из гроба? Оно говорит то же, но примерами, и подтверждает чувственными, отдельными действиями то, что слово говорит в общей форме. Поэтому чудо есть также учение, слово --- только драматическое слово, спектакль. Чудо, говорю я в <<Сущности христианства>>  имеет всеобщее значение, значение примера. <<Эти чудеса написаны в присутствии нас, избранных>> --- говорит Лютер. <<Такое дело, как прохождение через Красное море, произошло для наглядности, для примера, чтобы нам возвестить, что и с нами так же случится>>  то есть что в подобных же случаях необходимости бог будет совершать подобные же чудеса. Если, стало быть, Лютер дает невысокую оценку чудесам, то это относится к чудесам постольку, поскольку они рассматриваются как мертвые, исторические, нас нисколько не касающиеся происшествия. Но с такой же низкой оценкой Лютер подходит и к другим предметам и даже ко всем учениям, ко всем догматам веры, если они рассматриваются только исторически, если они не поставлены в связь с настоящим временем, с живым человеком, и даже к самому богу, если он рассматривается как существо само по себе, а не как существо для человека. Достаточно для этого сравнить в особенности места, приведенные в <<Сущности веры в понимании Лютера>>. Если, стало быть, Лютер низко расценивает чудеса, то смысл его низкой оценки таков: что тебе от того, что ты веришь в воскрешение Лазаря Христом, если ты не веришь, что он и тебя, и твоего брата, и твоего ребенка может воскресить, если захочет? Что толку верить, что Христос насытил 5000 человек пятью ячменными хлебами, если ты не веришь, что он может и тебя и вообще всех голодных насытить при помощи таких же малых средств или даже без всяких средств, раз он этого только захочет? Силу творить чудеса Лютер приписывал поэтому также отнюдь не одним только первым временам христианства, когда, как обычно принималось, она якобы только и была необходима для того, чтобы распространить христианскую веру. Кстати сказать, смешное различие! Чудеса либо всегда необходимы, либо никогда. Так, например, никогда чудеса не были бы более необходимы, чем в наше время, когда так много и столь основательно неверующих, как, быть может, ни в какое другое время. Лютер, таким образом, приписывал чудотворчество отнюдь не одним только первым временам христианства: <<Мы еще обладаем, --- говорит он, силой творить эти знамения>>  разумеется лишь тогда, когда, как он это замечает в другом месте, они нужны. 

Нет поэтому ничего более произвольного, беззаконного, не соответствующего истине, чем желание отделять веру в бога от веры в чудеса, христианское учение --- от христианского чуда. Это все равно, что захотеть отделить основание от ее следствий, правило --- от его применения, учение --- от тех примеров, в которых оно только и находит свое подтверждение, --- захотеть отделить и сохранить особо. Если вы не хотите никакого чуда, так вы не хотите и бога. Но если вы, выходя из рамок мира, из рамок природы, приходите к признанию бога, то выходите же и из рамок действий природы. Если бог, то есть существо, от природы, от мира отличное, есть причина природы или мира, то по необходимости должны существовать и действия, отличные от действий природы, действия, которые и являются доказательствами, поступками этого отличного от природы существа. Эти действия и представляют собой чудеса. Нет других доказательств бытия бога, чем чудо. Бог есть не только от природы отличное, но и природе противоположное существо. Мир есть существо чувственное, телесное, плотское, бог же, согласно вере даже наших рационалистов, существо не чувственное, не телесное; но если есть такое существо, то должны быть по необходимости и действия этого существа, следовательно действия, которые противоположны действиям природы и им противоречат. Но этими противоречиями существу природы и являются чудеса. Если я отрицаю чудеса, то я должен остаться при одной природе, при одном мире, и хотя бы я мыслил себе тела, попадающие в сферу наших чувств, --- эти звезды, эту землю, эти растения, этих животных --- происшедшими, произведенными некоей причиной, я все же должен принять только причину, не отличающуюся по своему существу от природы, причину, к которой поэтому я, лишь злоупотребляя, могу прилагать имя бога; потому что бог всегда обозначает произвольное, духовное, фантастическое, от природы отличное существо. Чтобы выйти из рамок природы, то есть придти к богу, я должен скакнуть, сделать прыжок. Этот прыжок и есть чудо. 

Однако рационалист в области религии верит в бога; он верит, как указанный рационалист выражается, что <<то, что мы называем законом, мировым порядком, без всякого основания приписывается природе вещей, тогда как она не может дать закона, а может лишь получить его>>. Природа, правда, не издает законов, но она их и не получает. Законы издают лишь властители людей, и получают законы лишь люди --- подданные; но оба понятия не приложимы к природе по той простой причине, что солнце, луна, звезды и составляющие их вещества --- не люди. Законодатель не повелевал кислороду соединиться с другими веществами в данной определенной весовой пропорции, и кислород этого закона не принимал, но это заложено в его свойствах, которые составляют нечто единое с его природой и существованием. Рационализм же принимает существование бога, который, как король своим подданным, дает миру законы, не заложенные в природе мира, вещей; он должен, стало быть, если желает быть последовательным, принимать также и то, что имеются и доказательства существования такого законодателя, доказательства того, что то, что мы называем законом, мировым порядком, не лежит в природе вещей; но этими доказательствами и являются чудеса. Доказательство, например, того, что не существует необходимого закона, основанного на природе женщины, но что исключительно от воли бога зависит, чтобы женщина сделалась матерью только через посредство мужчины, --- это доказательство дается лишь тем, что женщина и без мужа стала беременной. Но в чудеса рационалист не верит, он их отрицает, то есть он отрицает попадающие в сферу чувств очевидные несообразные действия и следствия веры в бога; но причины этих несообразностей, так как они не бросаются в глаза, так как они выясняются лишь при помощи основательного мышления и исследования, --- а он для этого слишком ленив, слишком ограничен, слишком поверхностен, --- их он оставляет нетронутыми. И, однако, я должен, если хочу быть последовательным, вместе со следствиями устранить и причину, или же вместе с причиной признавать и следствия. Делать природу зависимой от бога --- значит мировой порядок, необходимость природы делать зависимыми от воли; значит во главе природы поставить князя, короля, властителя. Но точно так же, как князь лишь тем доказывает, что он действительный властитель, что он может издавать и отменять законы, точно так же и бог доказывает свою божественность лишь тем, что он может законы упразднить или приостановить, по крайней мере, когда этого требует данный момент. Доказательством, что он их дал, является лишь то, что он их отменяет. Но это доказательство доставляет <<чудо>>. <<Бог, говорит епископ Немезий в своем сочинении о природе человека, --- не только стоит вне всякой необходимости, но он также и господин этой необходимости; ибо так как он существо, которое все может, чего хочет, то он ничего не делает ни в силу необходимости природы, ни в силу веления закона; для него все возможно (случайно), даже необходимое. И дабы это было показано, он однажды задержал движение солнца и луны, которые в силу необходимости движутся и постоянно совершают это единообразно, задержал, чтобы показать, что для него нет ничего необходимого, но все по произволу возможно>>. 

Рационалист пытается избегнуть необходимости чуда отговоркой, заявляя: <<божественная воля есть самая совершенная; как таковая, она не может меняться, но должна неизменно действовать в направлении одной цели; божественная воля должна быть, поэтому, наиболее постоянной, она должна нам представляться в виде неизменного закона, постоянного правила, не допускающего никогда исключения>>. Как смешно! В том-то и дело, что воля, которая представляется не волей, которая представляется в виде неизменного закона, не есть воля, есть лишь пустая фраза и описательный оборот для естественной необходимости, есть лишь выражение рационалистической половинчатости и неосновательности, которые находятся слишком во власти теологии, чтобы признать природу в ее полной истинности, и слишком во власти познания природы, чтобы признавать все выводы теологии, которая поэтому ставит во главу мира портрет своей собственной нерешительности, волю, которая не есть воля, и необходимость, которая не есть необходимость. Воля, которая делает постоянно одно и то же, не есть воля. Мы потому только не признаем за природой воли, свободы, что она всегда и неизменно делает одно и то же. Мы говорим, что яблоня в силу необходимости, а не в силу свободной воли приносит яблоки, что она приносит одни только яблоки и притом яблоки одного и того же вида и качества; мы лишь потому не признаем за птицей свободы пения, что она постоянно поет одни и те же песни и, следовательно, других петь не может. Человек, однако, приносит не всегда, как дерево, одни и те же плоды; он не всегда, как птица, поет одни и те же песни; он поет то одну песню, то другую, то грустную, то веселую. Разнообразие, многогранность, изменчивость, непостоянство, незакономерность одни только представляют собой те явления, те действия, причиной которых мы мыслим себе свободное, обладающее волей существо. Так, христиане из неизменности и закономерности движения светил умозаключили, что они не божественные, свободные существа, как верили язычники, ибо если бы их движение было свободно, то они двигались бы то туда, то сюда. Они были правы: свободное существо дает себя знать свободными, непостоянными проявлениями. Река, источник, которые с шумом проносятся перед нашими зрением и слухом, производят на меня то более слабое, то более сильное впечатление, соответственно тому, увеличивается или уменьшается масса воды, но по существу это будет одно и то же впечатление. А пение человека! То оно меня настраивает на один лад, то на другой; то вызывает во мне одно ощущение, то совершенно другое, противоположное; то оно в одном тоне звучит, то в другом. Монотонное существо, существо, которое всегда одинаково себя проявляет, всегда однообразно действует, так же м\'{а}ло, поэтому, является свободным, как м\'{а}ло всегда себе равная в своей деятельности, неизменная вода является свободным человеческим существом. 

Когда верующий рационалист отрицает за богом чудотворную деятельность потому, что она является слишком человеческим представлением, то это свидетельствует лишь о его недомыслии. Если уничтожить человечность или человекоподобие бога, то тем самым уничтожается и само божество. Что бога отличает от людей, так это --- природа, заимствованные у природы свойства или силы, как, например, та сила, благодаря которой растет трава и формируется дитя во чреве матери. Если хотят поэтому иметь существо, ничего общего с человеком не представляющее, то пусть на место бога поставят природу; но если хотят иметь существо, у которого, как у человека, есть воля, разум, сознание, личность, то это значит, что хотят иметь совершенное, целое, человеческое существо, и пусть не отрицают тогда, что бог творит чудеса, что он в разное время и при различных обстоятельствах различное делает и имеет в виду, короче говоря, --- что его воля столь же переменчива, как и воля монарха, человека вообще, потому что только переменчивая воля и есть воля. Voluntashominis, говорят юристы, esfc ambu-latoria usque ad mortem, то есть воля человека переменчива до самой его смерти. Чего я постоянно, неизменно хочу, того мне не нужно желать; постоянство, неизменность есть уничтожение, смерть воли. Я хочу идти, потому что я до сих пор сидел или стоял; работать --- потому что я до сих пор отдыхал или ленился; отдыхать --- потому что я до сих пор работал. Воля имеется лишь там, где имеются противоположности, непостоянства, разрывы и перерывы. Но эта смена, этот разрыв вечного безразличия природы в сфере религиозного верования, ставящего во главе мира существо, обладающее волей, есть чудо. Поэтому чудо нельзя, не прибегая к величайшему произволу, отрывать от веры в бога. 

Но именно этот произвол, эта половинчатость, этот недостаток основательности есть сущность наших рационалистов в области веры, наших современных христиан вообще. Еще один пример, чтобы подтвердить это мое утверждение. Тот же самый, уже указанный рационалист --- впрочем, в противоречии с некоторыми другими рационалистами, объясняющими воскресение таким образом, что Христос на кресте на самом деле не умер, --- принимает, наоборот, воскресение и допускает его как исторический факт, но не только без тех необходимых выводов, которые с воскресением связаны, но и без тех ближайших обстоятельств, согласно священному писанию, якобы сопровождают этот факт. То, что, как единогласно рассказывают Марк, Матфей и Лука, в момент смерти Иисуса завеса в храме перед святая святых разорвалась сверху донизу на две части, что, как рассказывает Матфей, скалы треснули, гробы вскрылись, сама земля заколебалась --- и это было в момент смерти и в момент воскресения Христа, --- все это объясняет он прикрасами устной передачи. Однако если Христос в самом деле воскрес из мертвых, а не просто проснулся к жизни из мнимомертвого состояния, то это воскресение из мертвых было чудом, и притом чудовищно большим и важным чудом, ибо это, ведь, была победа над смертью, победа над естественной необходимостью, и притом над самой жестокой, над самой неумолимой, уничтожить которую не могли даже сами языческие боги. Как возможно поэтому, чтобы такое великое чудо было одиноко? Не должны ли были с этим чудом быть связаны еще и другие чудеса? Не естественно ли, не необходимо ли, признав это чудо, поверить тому, что вся природа была потрясена в тот момент, когда была насильственно разорвана цепь естественной необходимости, цепь, которая приковывает мертвых к смерти, к гробу? Поистине, наши верующие предшественники были гораздо более рационалистами, чем наши нынешние рационалисты; ибо их вера была связанным целым; они полагали, что если я верю в это, то я должен верить и в другое, хотя бы это другое мне и было не по душе; если я принимаю основание, то я должен принять и следствие, --- короче, если я говорю А, то я должен сказать и Б. 

\phantomsection
\addcontentsline{toc}{section}{Двадцать седьмая лекция}
\section*{Двадцать седьмая лекция}

В прошлой лекции я утверждал, что чудесные события во время смерти Иисуса находятся в теснейшей связи с его воскресением. Если Христос и в самом деле воскрес из мертвых, то его воскресение есть чудо, доказательство божественного всемогущества, перед которым смерть ничто; но такое чудо не может стоять особняком и нуждается для своего подтверждения в других чудесных и чрезвычайных событиях. Воскресение было бы совершенно бессмысленно, если бы оно не было подготовлено и подкреплено другими чудесами. Когда умирает существо, которое затем должно опять воскреснуть из мертвых и тем дать миру доказательство того, что смерти не существует, --- ибо в этом заключается смысл воскресения, --- не может происходить все в обычном и естественном порядке, как тогда, когда умирает обыкновенный человек. Поэтому, если я иду так далеко, как вышеназванный рационалист, и объявляю чудеса, стоящие в непосредственной связи с воскресением, легендами, поэтическими прикрасами, плодами фантазии, то я по необходимости должен сделать еще один шаг дальше и объявить и само воскресение продуктом религиозного воображения. 

То, чего человек желает, чего он необходимо должен желать, --- необходимо с той точки зрения, на которой он стоит, --- тому он верит. Желание есть потребность, чтобы что-нибудь было, чего нет; сила воображения, вера представляет это человеку как существующее. Так, христиане желали небесной жизни; они не имели земных желаний, как их имели язычники, не имели интереса ни к миру природы, ни к миру поэтическому. <<Определение, --- говорит, например, греческий отец церкви Феодорит, --- которое Платон дает истинному философу, а именно, что ему нет никакого дела до политики и до политической деятельности не подходит к языческим философам, а только к христианам, ибо величайший философ Сократ толкался по гимназиям и мастерским и служил даже в качестве солдата. Но те, кто усвоил себе христианскую, или евангельскую, философию, удалились от политической суеты и посвятили себя в уединенном месте религиозному созерцанию и ему соответствующему образу жизни, не отвлекаясь заботами о женах, детях и земных благах>>. Желания христиан направлялись к другой, лучшей жизни, и они верили поэтому, что таковая существует. Кто не желает другой жизни, тот и не верит в нее. Но бог, религия есть не что иное, как нашедшее себе удовлетворение в фантазии стремление к счастью, желание счастья, имеющееся у человека. Поэтому христианское желание небесной блаженной жизни, жизни без конца, без ограничения смертью, сила религиозного воображения представляла себе исполненным в воскресшем от смерти Христе; ибо ведь от его воскресения зависит и воскресение, бессмертие христианина; ибо он ведь прообраз его. Правда, исполнение этого желания или, вернее, вера в его исполнение, действительное воскресение Христа, --- независимо даже от того, что вообще вера в воскресение существовала уже задолго до христианства, уже была частью символа веры религии Зороастра, или персидской, --- могло иметь историческое основание, а именно в том, что Христос своими считался за мертвого, как умерший был оплакан и потому, когда вновь явился, был принят как действительно воскресший от смерти. Но было бы педантизмом и совершеннейшим непониманием религии сводить религиозные факты, существующие лишь в области веры, к фактам историческим, желать разыскать историческую истину, лежащую в их основе. В историческом нет ничего религиозного, а в религиозном --- ничего исторического: историческая личность, историческое событие перестают быть историческими, как только они делаются предметом религии и становятся созданиями мысли и фантазии. Так и Иисус, как я уже говорил в одной из предыдущих лекций, Иисус, который и в том виде, как он нам дан в Библии, не является исторической личностью, а религиозной; он нам представлен здесь, как чудесное, творящее чудеса, всемогущее существо, которое может исполнить и действительно исполняет все желания человека, то есть все желания, не стремящиеся ни к чему худому, ни к чему с точки зрения Христа безнравственному; он нам представлен как существо фантазии, воображения. 

Чтобы упразднить чудо, рационалист прибегает к тому доводу, что, как он говорит, <<понятие чуда, если оно должно дать научное доказательство откровения, должно быть определено как такой факт в чувственном мире, который не может быть объяснен естественным сочетанием действующих причин, и где, стало быть, нужно прийти к заключению, что рука божия непосредственно вмешалась и действовала. Но чтобы достоверно знать, что какой-нибудь факт не произошел в силу естественного порядка вещей, нужно было бы полностью знать всю природу и все ее законы. Но так как ни один человек не имеет таких знаний и не может иметь, то и суждение о том, что данный факт не мог просто явиться результатом сочетания данных явлений, но должен был быть следствием чрезвычайного вмешательства божественного всемогущества, не может быть поддерживаемо с полной очевидностью>>. Однако чудеса отличаются настолько очевидно и бесспорно от действий природы, что можно, не сомневаясь, утверждать, что они не могли возникнуть из связи естественных предметов и причин, потому что желания и фантазии человека, которые представляют нам чудеса в виде действительных фактов, находятся вне связи вещей и над ней, вне природной необходимости и над ней, точно так же, как и желание неизлечимо слепого видеть стоит вне всякой естественной связи с природой слепоты и даже в прямом противоречии с естественными условиями и законами исполнимости этого желания; поэтому определение древних теологов, что чудо не только стоит над естественным порядком вещей, но и противно ему, противно сущности природы, совершенно правильно; ибо оно характеризует нам природу желания. Конечно, можно так же аподиктически, как это говорят философы, то есть с безусловной уверенностью и решительностью, утверждать, что чудеса не могут быть объяснены из природы, то есть из внешней природы, что они произошли лишь благодаря чрезвычайному вмешательству божественного всемогущества; надо только заметить, что эта божественная сверхъестественная сила есть именно сила человеческих желаний и воображения. Короче говоря, сущность религии, сущность божества выявляет только природу желания и неразрывно с ними связанного воображения; ибо лишь воображение представляет как существо, имеющее свое бытие вне мышления, даже и бога рационалистов, бога мыслящих философов, бога, который есть не что иное, как его --- мыслителя собственный образ мыслей. Из приведенного объяснения явствует также, как нелеп вопрос или спор о возможности, действительности и необходимости чудес. Этот спор, этот вопрос может возникнуть лишь в том случае, если рассматривают чудо само по себе или держатся лишь за внешние проявления, не восходя к внутреннему психологическому, или человеческому, основанию, которому эти внешние проявления обязаны своим существованием. Психологическое, или человеческое, происхождение чуда очевиднейшим образом явствует уже из того, что чудеса происходят через посредство людей, или, как выражается религиозное верование, происходят от бога при посредстве людей. В этом заключается бросающееся в глаза различие, которого мы уже раньше коснулись, различие между так называемыми естественными и религиозными чудесами. Религиозные чудеса немыслимы без человека, ибо они имеют отношение лишь к человеку. Естественные чудеса, то есть вызывающие наше изумление явления природы, существуют и без того, чтобы был человек и им удивлялся. Чудеса геологии, мегатерии, динозавры, ихтиозавры существовали --- по крайней мере по представлению нашей современной геологии --- еще до того, как жили люди; но естественное движение солнца не прерывалось до появления Иисуса Навина. 

На первый поверхностный взгляд представляется парадоксальным, то есть поражающим и странным, выводить религию из желаний человека, и даже божество, предметное существо религии, отождествлять с желанием; потому что в религии, по крайней мере, в христианской, человек молится: <<Господи, не моя, а твоя да будет воля>>; ведь религия требует принесения в жертву человеческих желаний. Но христианин --- разумеется, настоящий, античный, не современный христианин --- приносит в жертву желание богатств, желание иметь детей, желание здоровья или долгой жизни, но не желание бессмертия, не желание достигнуть божественного совершенства и блаженства. Он подчиняет все эти, в его понимании, временные, земные, плотские, желания одному главному и основному желанию;и от этого желания, от представления, от фантазии вечной небесной жизни не отличается христианское божество; оно есть лишь это олицетворенное, представленное как действительное существо, желание. Поэтому божество и блаженство для истинного христианина одно и то же. Даже человек, который не имеет подобных преизбыточных, сверхземных желаний, как христианин, человек, который стоит со своими желаниями на почве действительности, на почве действительной жизни и человечества, и даже тот, у которого совершенно обыденные эгоистические желания, должен принести в жертву бесчисленные второстепенные желания своему главному, если он его хочет осуществить. \emph{Человек, у которого нет другого желания, как сделаться богатым или остаться здоровым, должен подавить бесчисленное множество желаний, если он действительно хочет сделаться богатым или если он хочет действительно остаться здоровым}. Как ни хочет он в данный момент доставить себе удовольствие, он должен от него отказаться, если он не хочет удовлетворением своего минутного стремления, побуждения или желания повредить своему основному желанию. Если поэтому в христианской религии говорится: <<не моя, а твоя да будет воля>>  --- то смысл этих слов лишь следующий: да будет не та воля, не то желание, которое хочет того или другого, и которое, когда оно исполнится, окажется, быть может, мне впоследствии на пагубу, не желание вообще так называемых временных благ; этим никоим образом не сказано, что должна быть воля, стремление к блаженству вообще, а не желание постоянного вечного счастья, небесного блаженства. Ведь христианин, когда он чего-нибудь желает или о чем-нибудь молится, заранее предполагает, что воля божия будет исполнена, что эта воля хочет лишь лучшего, лишь блага человека, по крайней мере, вечного блага и счастья \hyperlink{28}{(28)}\hypertarget{b28}{}. Таким образом сведению религии и богов к желаниям людей никоим образом не противоречит та резиньяция, тот отказ от удовлетворения тех или других отдельных желаний, которые предписывает религия. Не подлежит сомнению: там, где прекращается человек, прекращается и религия, но там, где прекращается желание, прекращается и человек. Нет религии, нет бога без желания; но нет и человека без желания. \emph{Различие между желаниями, без которых нет религии и божества, и желаниями, без которых нет человечества, без которых человек не есть человек, заключается лишь в том, что религия имеет желания, которые осуществляются лишь в воображении, в вере; человек же, как таковой, или человек, который на место религии ставит образование, разум, естественнонаучное мировоззрение, на место неба землю, --- этот человек имеет желания, не выходящие из границ природы и разума, лежащие в пределах естественной возможности и осуществимости.}

Видимое противоречие между сущностью желания и сущностью религии может быть также выражено следующим образом. Желания человека произвольны, беззаконны и безудержны; религия же дает законы, возлагает на человека обязанности, ограничения. \emph{Но обязанности --- это не что иное, как те основные стремления, основные склонности, основные желания человека, которые во времена некультурности, при отсутствии образования, делает законами религия или бог, в культурные же времена --- разум, собственная природа человека, законами, которым он должен подчинить те или иные особые вожделения, желания и страсти.} Все религии, но особенно те, которые имеют значение в истории человеческой культуры, не имели ничего другого в виду, как благо человека. Обязанности, ограничения, которые они возлагали на людей, они возлагали лишь потому, что полагали, что без них невозможно достигнуть и осуществить основной цели, основного желания человека --- быть счастливым. Есть, конечно, случаи в жизни, когда обязанность и стремление к счастью вступают в человеке друг с другом в конфликт, когда своему долгу приходится жертвовать даже своей жизнью, но такие случаи --- случаи трагические, несчастные или вообще особые, чрезвычайные. Их нельзя приводить, как довод, чтобы тем возвести противоречие между обязанностью, моралью и стремлением к счастью --- в закон, в норму, в принцип. Обязанность первоначально по своей природе не преследует иной цели, кроме блага и счастья человека. То, чего человек желает, больше всего другого желает, то он делает своим законом, своей обязанностью. Там, где существование, или, что то же, благосостояние народа, --- ибо что такое существование без блага, без счастья? --- и тем самым также и благосостояние отдельного человека было связано с земледелием; там, где человек без земледелия не мог быть счастливым, не мог быть человеком, --- ибо только счастливый человек есть человек цельный, свободный, истинный человек, чувствующий себя таковым, --- где, следовательно, главным желанием человека было процветание и успехи земледелия, там это последнее было и религиозной обязанностью и религиозным делом. И там, где человек не может осуществить своих человеческих желаний и целей, не уничтожив вредящих ему диких зверей, там это уничтожение есть религиозная обязанность, там даже животное, помогающее человеку в исполнении этих желаний, в осуществлении его счастья, в достижении его человеческих целей, является религиозным, священным, божественным животным, как, например, собака в древнеперсидской религии. Одним словом, противоположность между обязанностями и желаниями взята из особых случаев человеческой жизни и не имеет общей истинности, общей значимости. Наоборот: \emph{чего человек в глубине своего сердца желает, то и есть единственное правило и обязанность его жизни и деятельности}. Обязанность, закон превращает лишь в предмет воли и сознания то, чего хочет бессознательное влечение человека. Если это есть, --- я хочу показать это на примере с духовными особенностями и склонностями человека, --- если твое влечение сделаться художником есть твое обоснованное желание, то оно в таком случае также и твоя обязанность им сделаться и с этим сообразовать свой образ жизни. 

Но как же приходит человек к тому, чтобы превратить свои желания в богов, в существа, как, например, желание сделаться богатым --- в бога богатства, желание плодородия --- в божество, приносящее плодородие, желание сделаться блаженным --- в блаженного бога, желание не умереть --- в бессмертное существо, преодолевающее смерть? То, чего человек желает, и --- соответственно своей точке зрения --- необходимо, существенно желает, в то он, как было уже сказано, верит, то он считает --- в той обстановке, в которой имеет свои корни религия --- за нечто действительное или возможное; он не сомневается в том, что это может быть; порукой этой возможности служит ему его желание. Желание уже само по себе имеет в его глазах значение волшебной силы. В древнегерманском языке <<желать>> --- то же, что <<колдовать>>. В древнегерманском языке и религии величайший из богов именовался, между прочим, также и желанием, чем древний язык, как об этом говорит Яков Гримм, в своей <<Германской мифологии>>  выражал понятие счастья и блаженства, исполнения всех благ, и он полагает, что слово Wunsch, желание, происходит от Wunjo, означающего блаженство (Wonne), радость, или всякого рода совершенство. Гримм считает поэтому пережитком древнего языческого словоупотребления то обстоятельство, что некоторые из поэтов тринадцатого века олицетворяют желание, представляют его в виде могучего творческого существа, и при этом Гримм замечает, что у них в большинстве случаев на место слова <<желание>> можно поставить слово <<бог>>. Если он, однако, при этом отличает значение желания в позднейшем словоупотреблении, когда оно означает стремление к тем благам и совершенствам, которыми обладает бог, от значения первоначального, то не следует упускать из виду, что первоначально в понимании языка и религии желание и предмет желания отождествлялись. Что я хочу иметь, то я имею ведь в воображении; чем я желаю быть --- здоровым, богатым, совершенным, --- тем ведь я и являюсь в воображении; потому что, когда я желаю для себя здоровья, то я себя представляю здоровым. Именно поэтому желание, как таковое, есть божественное существо, сверхъестественная волшебная сила, ибо оно сыплет на меня из рога изобилия фантазии все силы и блага, какие только я себе пожелаю. С языческим желанием дело обстоит, как с христианским благословением. Благословлять --- это то же, что желать блага, благословение, стало быть, то же, что желание, но благословение означает также и предмет, то благо, которое желают себе и другим. <<Поэтому и в писании, --- говорит Лютер в своем толковании благословения, --- обычна форма речи: дай мне благословение, разве у тебя нет больше благословения? --- что значит: дай мне что-нибудь из добра, хлеба, платья. Ибо это ведь все божьи дары, и то, что мы имеем, мы имеем через благословение бога, и потому благословение означает и дар божий, который он дает нам через свое благословение>>. Различие между божественным желанием, или благословением, и человеческим желанием, или благословением, лишь то, что божественное желание, или благословение, есть человеческое желание, исполненное, осуществленное. Поэтому бог называется желанием на том же основании, на каком бог вообще может и должен быть охарактеризован как человеческое желание счастья, удовлетворенное в фантазии, --- может и должен по той же причине, по какой <<молитва>> называется <<всемогущей>>  по какой божественное всемогущество есть не что иное, как превращенное в конкретное существо или в качестве такого существа представленное всемогущество человеческой молитвы и человеческого желания. 

Религия, подобно поэзии, изображает как бы действительно, как бы чувственно существующим то, что существует лишь в представлении, она превращает желание, мысли, воображение, душевные настроения в действительные существа, отличные от человека. Вера в колдовство и волшебство имеет как раз своим источником то, что люди приписали желанию власть и силу, выходящую за человеческие пределы и действующую во вне, что они уверовали в то, что с человеком действительно приключится беда, если ему ее накликать. Римляне и греки представляли себе пожелания мстительного чувства, пожелания зла, проклятия в виде богов или больше богинь, то есть в виде существ, приводящих проклятия в исполнение, осуществляющих пожелания мести. У одних они назывались Диры (Dirae), у других Ары (Агае). Но то, что относится к проклятию, относится и к благословению. <<В священном писании, --- говорит Лютер в своем толковании Моисея, --- имеются действенные благословения, а не только благословения-желания, --- благословения, которые определенным образом влияют, которые фактически дают и с собою приносят то, что заключено в словах\dots Если, таким образом, я сказал: дай бог, чтобы тебе простились грехи, то это можно назвать благословением любви. Благословение же обетования и веры и приносимых даров гласит так: <<я отпускаю тебе твои грехи>>. Это именно значит, что вера, воображение превращают субъективное в объективное, представляемое в действительное, желаемое в осуществленное. Но так как человек, само собой разумеется, заключает свои желания --- все равно, добрые, или злые, благословения, или худые посулы, --- в определенные слова и формулы, то он именно этим формулам, словам, именам приписывает выходящие за человеческие пределы конкретные действия, то есть волшебные силы. Так, например, религиозные римляне верили в то, что можно известными формулами молитвы и волшебства производить дождь и непогоду и прогонять их, что можно заколдовать плоды на поле, оберечь дома от пожара, излечивать раны и болезни, приковывать к месту людей, которые собираются бежать. Так еще и поныне старобаварцы верят в то, что можно кого-нибудь <<замолить до смерти>>  то есть убить молитвами. Именно благодаря этой вере или этому суеверию и происходит то, что люди опасаются произнести слова или названия вещей, которых они боятся, потому что вместе с именем, как они воображают, они приворожат себе и предмет или взвалят его себе на шею. Североамериканские дикари до такой степени боятся мертвецов, что даже не произносят имени умершего, что оставшиеся в живых и носящие то же имя берут другое имя. Они верят, следовательно, что покойник --- как мертвец, как привидение существует до тех пор, пока его называют, и представляют себе, что он, напротив того, не существует более, если он для них больше не существует, что его нет, если они его себе не мыслят, не называют. Так, греки и римляне верили в то, что знамение, предзнаменование лишь тогда оказывает действие, когда на него обращают внимание, что, разумеется, совершенно верно, потому, что оно оказывает хорошее или дурное действие в том случае, если мы ему придаем радостное или печальное значение. Точно так же многие народы, большинство из них в детском или первобытном состоянии, верят, что если они видят во сне мертвецов, то мертвецы к ним действительно приходили; они вообще считают образ, представление о каком-либо существе, о каком-либо предмете за самое существо, за самый предмет. Некультурные народы верят даже в то, что душа во время сна выходит из тела, чтобы бродить, и направляется в такие места, куда фантазия переносит человека во сне; они считают, таким образом, подобные путешествия во сне за действительные путешествия, ту ложь и те сказки, какие преподносит им фантазия, --- за истину и факты. Гренландцы верят даже в то, что и в бодрственном состоянии душа отделяется от тела и предпринимает путешествия, потому, что ведь и во время бодрствования часто переносишься мысленно в далекие места, находишься духовно не там, где телесно. 

Мы имеем, впрочем, в этих представлениях лишь чувственные, грубые, бросающиеся в глаза примеры того, как вообще человек превращает субъективное в объективное, то есть делает чем-то существующим вне мышления, представления, воображения то, что существует только в его мышлении, представлении, воображении; он поступает так тогда, когда то, что он представляет себе, есть предмет, связанный с его стремлением к счастью, предмет, которого он желает, как блага, боится, как зла; потому что как страх, так и любовь, потребность, влечение к чему-нибудь делает человека слепым, так что он ничего другого не видит, кроме того, что он именно любит и чего желает, забывая из-за этого все остальное. Или, иначе выражаясь: человек превращает в существа не безразлично каждое представление, каждый объект воображения, каждую мысль и желание, но главным образом те, которые теснейшим образом связаны с его собственным существом, которые являются характерным выражением его существа, которые именно поэтому для него так же действительны, как его собственное существо, имеющее для него характер необходимости, потому что именно эти представления заложены в его существе. Так, язычники считали своих богов действительными существами, потому что они не могли себе мыслить других богов, потому что только эти боги соответствовали их языческому существу, отвечали их языческим потребностям и желаниям. Наоборот, христиане не сомневаются в том, что боги язычников --- лишь воображаемые существа, но только потому, что те блага, которые эти боги раздавали, те желания, которые эти боги исполняли, являются в представлении истинных христиан суетными, ничтожными желаниями. По представлению истинного христианина, нет необходимости быть здоровым, --- к чему, стало быть, бог здоровья? Не необходимо быть богатым, --- к чему, стало быть, бог богатства? По их представлению необходимо лишь то, что ведет к вечному небесному блаженству. Короче говоря: христианин считает лишь те мысли, представления, создания воображения за действительные существа, которые совпадают и связаны с его христианским существом, которые являются отражением его собственного существа, которые его собственное существо опредмечивают. Так, христианин не сомневается в истинности и действительности бессмертия, другой жизни после смерти, а между тем эта жизнь существует лишь в его представлении, в его воображении. И не сомневается он потому, что это воображение связано с христианским существом, уносящимся за пределы действительности. Именно потому, что человек верит лишь в бога, который выражает и отражает собственное существо человека, потому что он считает лишь то мысленное, представленное им себе или воображаемое существо действительным существом, которое находится в гармонии с его интимнейшими сердечнейшими желаниями. Именно поэтому я и высказал в <<Сущности христианства>> то положение, что вера в бога есть не что иное, как вера человека в самого себя, что он в своем боге ничего другого не почитает, ничего другого не любит, как свое собственное существо, что именно поэтому является ныне нашей задачей превратить это бессознательное, извращенное, фантастическое почитание в почитание сознательное, честное и разумное. 

\phantomsection
\addcontentsline{toc}{section}{Двадцать восьмая лекция}
\section*{Двадцать восьмая лекция}

Человек, таким образом, превращает свои чувства, желания, продукты своего воображения, представления и мысли в существа, то есть, чего он желает, что он представляет себе, что он мыслит, имеет для него значение вещи, находящейся за пределами его головы, хотя она только в голове его и сидит. <<Все предметы мысли, --- говорит Клейбер о религии Ормузда в своей ,,Зенд-Авесте`` (но то, что он говорит, имеет значение и для каждой другой религии, только предметы там иные), --- все предметы мысли (то есть в данном случае все различия мысли или существа мысли) являются одновременно и действительными существами и вместе с тем и предметами почитания>>. Отсюда и происходит то, что человек поставил вне себя ничто, которое является лишь мыслью, словом, и пришел к бессмысленному представлению о том, что раньше мира было ничто и что даже мир был создан из этого ничто. Но человек превращает главным образом в вещи, в существа, в богов, только те мысли и желания, которые имеют отношение к его личности. Так, например, --- дикарь превращает каждое болезненное ощущение в злое существо, терзающее человека, каждый образ своего воображения, нагоняющий на него страх и ужас, --- в дьявольское привидение. Гуманный человек превращает свои человеческие чувства в божественные существа. Из всех греков одни лишь афиняне, по словам Фоссия, воздвигали алтарь состраданию, сочувствию. Так общественный человек превращает в богов свои политические желания и идеалы. Так в Риме была богиня свободы, которой Гракх построил храм; так имело свой храм и единодушие; так же точно и общественное здравие, честь, словом, все, что представляется общественному человеку, имеющим особую важность. Наоборот, царство христиан не было царством мира сего; они смотрели на небо, как на свое отечество. Поэтому первые христиане не праздновали, подобно язычникам, день своего рождения, но праздновали день смерти человека, потому что они в смерти видели не только конец этой жизни, но в то же время и начало новой, небесной жизни. В этом их отличие от язычников, все существо которых растворялось в атмосфере естественного и гражданского мира. Поэтому христиане превращали в существа лишь те желания, мысли и представления, которые имели отношение к этому их отличию, к этому их существу. Язычники делали богом человека во всей его телесности, христиане делают богом только духовное и душевное существо человека. Христиане отделяют от своего бога все чувственные свойства, страсти и потребности, но только потому, что они их отбрасывают мыслью и от своего собственного существа, потому что они верят, что и их существо и их дух, как они выражаются, отделится от этой телесной скорлупы и оболочки, что они когда-нибудь будут существами, которые больше не едят и не пьют, которые будут чистыми духами. 

То, чем человек в действительности еще не является, но чем он --- как он надеется и верит --- когда-нибудь станет; то, что поэтому есть лишь предмет желания, страстного влечения, стремления и, следовательно, предмет не чувственного воззрения, а лишь фантазии, воображения, то называют идеалом, или же, иначе, прообразом. Бог человека или народа, по крайней мере того народа, который не остается постоянно на одном и том же месте, на уровне примитивности, который хочет идти вперед, который именно поэтому имеет историю, --- ибо история имеет свое основание в стремлении человека, в его тяге к совершенствованию, к тому, чтобы добиться достойного существования, бог такого народа есть не что иное, как идеал. <<Будьте совершенны, как совершенен отец ваш небесный>> --- говорится в Новом завете. А в Ветхом завете говорится: <<Я господь бог ваш: освящайтесь и будьте святы, ибо я свят>>. Если поэтому понимать под религией не что-либо иное, как культ идеала, то совершенно правы те, кто называет уничтожение религии бесчеловечным, ибо необходимо, чтобы человек ставил себе цель своего стремления, известный образец. Но идеал, каким он является в качестве предмета религии, и в частности христианской религии, не может быть нашим образцом. Бог, религиозный идеал, есть, правда, всегда человеческое существо; но при этом так, что множество свойств, принадлежащих действительному человеку, от него изъято; он не все человеческое существо; он только нечто от человека, нечто, выхваченное из всего целого, <<афоризм>> человеческой природы. Так, христиане вырывают у человека дух, душу из тела и делают этот вырванный, лишенный тела дух своим богом. Даже язычники, как, например, греки, которые делали человека, так сказать, во всей его телесности богом, даже они делали фигурой своих богов человеческую фигуру лишь постольку, поскольку она является предметом для глаза, но не предметом для телесного осязания. Хотя на практике, в жизни, при исполнении культа они и обращались со своими богами, как с действительными людьми, приносили им даже еду и напитки, но все же боги в их представлении, в их поэзии были существами отвлеченными, существами без плоти и крови. В еще большей степени это относится к христианскому богу. Но как может отвлеченное, не чувственное, бестелесное существо, существо без телесных потребностей, влечений и страстей от меня требовать, чтобы я, существо телесное, чувственное, действительное, был подобен ему, был на него похож? Как может оно быть законом, образцом моей жизни и деятельности? Как может оно вообще предписывать мне законы? Человек не понимает бога, говорит теология, но бог также не понимает человека, говорит антропология. Что может знать дух о чувственных влечениях, потребностях и страстях? 

Откуда же законы морали, кричат верующие, если бога не существует? Глупцы! Законы, соответствующие человеческой природе, имеют своим источником лишь человека. \emph{Закон, который я не могу выполнить, который превышает мои силы, не есть закон для меня, не есть человеческий закон; но человеческий закон имеет поэтому и человеческое происхождение}. Бог может все, что угодно, то есть все вообразимое, и поэтому он может и от человека потребовать также всего, что угодно. Точно так же, как он может сказать людям: вы должны быть совершенны и святы, как я, точно таким же образом он может сказать людям: вы не должны есть и пить, ибо я, господь бог ваш, также не ем и не пью. В глазах бога еда и питье нечто в высшей степени непристойное, не святое, животное. Законы, которые бог дает человеку, то есть законы, имеющие своим основанием и целью существо отвлеченное и именно поэтому существующее лишь в воображении, непригодны поэтому для человека, имеют своим следствием величайшее лицемерие, ибо я не могу быть человеком, не отрицая моего бога, или величайшую противоестественность, как это доказала история христианства и других подобных религий. Необходимым последствием духовного, то есть абстрактного, отвлеченного существа, или бога, которого человек делает законом своей жизни, является истязание, умерщвление плоти, самоумерщвление, самоубийство. Причиной материальной нищеты христианского мира является поэтому в конечном счете духовный бог, или идеал. Духовный бог печется лишь о спасении души, а не о телесном благополучии человека. И телесное благополучие находится даже в величайшем противоречии со спасением души, как то говорили уже благочестивейшие и лучшие христиане. Вместо религиозного содержания человек должен в настоящее время поставить себе поэтому другой идеал. Наш идеал --- не кастрированное, лишенное телесности, отвлеченное существо, наш идеал, это --- цельный, действительный, всесторонний, совершенный, образованный человек. К нашему идеалу должно относиться не только спасение души, не только духовное совершенство, но и совершенство телесное, телесное благополучие и здоровье. Греки и в этом отношении своим примером являются для нас светочами, указующими путь. Телесные игры и упражнения входили в состав их религиозных празднеств. 

С религиозным идеалом, далее, связываются всегда различные неразумные, даже суеверные представления. Дело в том, что религия представляет этот идеал в то же время и как существо, от воли которого зависит судьба человека, как существо личное, или, во всяком случае, самостоятельное, от человеческого существа отличное, которое человек должен почитать, любить и бояться, к которому, короче говоря, он должен обращаться со всеми теми своими ощущениями и умонастроениями, которые можно испытывать лишь по отношению к действительному, живому существу. Человек не имеет никакого представления, никакого предчувствия другой действительности, другого существования, кроме чувственного, физического. Религия представляет поэтому идеал, хотя он и является существом, живущим только в мыслях, или только моральным, --- существом одновременно и физическим. Она превращает существо, являющееся лишь в представлении человека высшим существом или образцом для подражания, в существо, которое есть первое и само по себе, в существо, от которого произошли все остальные чувственные телесные существа и от которого они зависят в своем бытии. В том и заключается бессмыслица религии, что она цель человека делает началом мира, принципом природы. Так как он себя чувствует и сознает зависимым от своего идеала, так как он чувствует, что без этой цели он ничто, что он вместе с нею теряет и смысл, и основу своего бытия, то он верит также и в то, что мир не может существовать без этого прообраза, что он без него --- ничто. Это человеческое тщеславие, которое дает себя знать не только в блестящем мундире государства, но и в смиренном монашеском или священническом облачении религии, это --- употребляя современное выражение --- романтика, отводящая своему религиозному идеалу первое место и даже приносящая в жертву все прочие вещи для того, чтобы выразить этим свое почитание, Как влюбленный, по крайней мере романтический влюбленный, видит, как перед ним исчезают добродетели и прелести всех других женщин по сравнению с его возлюбленной, как она в его глазах --- единственная, несравненная, чья красота не поддается описанию и есть образец и воплощение всех женских добродетелей и прелестей, так что на долю других женщин ничего не остается, то есть они лишены всех прелестей, которыми завладела эта единственная, точно так же относится человек и к идеалу своей религиозной любви. Все другие вещи и существа превращаются в ничто по сравнению с ним, ибо он есть воплощение всех добродетелей, всех совершенств. Бытие всех других вещей ему как таковое непонятно, потому что безразлично, как романтическому влюбленному бытие всех других женщин; но так как они все же существуют, несмотря на его религиозный идеал, который один достоин существовать, то ему приходится выдумывать какое-нибудь, хотя бы и самое плохое основание их бытия; и он его находит в их, правда, весьма отдаленном сходстве с его религиозным идеалом, находит лишь в том, что они все же имеют в себе нечто божественное, нечто, хотя и весьма мало совершенное, так и романтический влюбленный оказывает другим женщинам ту милость, что оставляет их жить рядом с его единственной, потому что все же они имеют некоторое сходство с нею. Ведь другие женщины --- тоже женщины, точно так же, как и другие существа суть так же существа, как и божественное существо. 

В силу этого только что изложенного основания, но, разумеется, не только в силу его одного, и выходит, стало быть, что человек своему религиозному идеалу уделяет первое место среди всех существ и все остальные существа заставляет возникнуть не только после него, но также и из него. После него заставляет он их возникнуть потому, что они по своему достоинству ниже его, потому что он первое по достоинству существо делает и первым по времени, потому что человек, а именно человек древнего мира, из которого религия вышла, рассматривает старейшее, более раннее, как нечто более высокое, чем то, что более молодо и ново; из него же, из идеала, он заставляет их произойти лишь в силу отрицательного, а потому и несущественного основания, лишь в силу своего невежества, лишь потому, что он не знает, откуда бы их произвести ему. Например, <<Древность стоит ближе всего к богам>> (Цицерон, О законах). Одна ошибка влечет за собой другую. \emph{Первой ошибкой религии является то, что она делает религиозный идеал первичным, первым существом, второй ошибкой --- то, что она заставляет остальные существа из него произойти; первая же ошибка необходимо влечет за собой вторую}. <<Окажи сопротивление началу!>> --- это положение применимо как в религии, так и в политике. Но насколько оно общепринято и восхвалено в медицине, морали и педагогике, настолько же оно ославлено в политике, в религии. Рационализм если взять пример из области религии, которая ведь является предметом нашего изложения, --- врачует и повсюду верно отмечает очевиднейшие ошибки религиозного верования, являющиеся, однако, ошибками только второго, подчиненного сорта, основные же ошибки, следствием которых являются все прочие, он оставляет нетронутыми как неприкосновенные святыни. Поэтому на вопрос рационалиста, обращенный к атеисту, --- что такое атеизм, --- ответ гласит: рационализм есть недопеченный, половинчатый, неосновательный атеизм. Или ответ таков: рационализм, это --- хирург; атеизм --- терапевт. Хирург лечит лишь от болезней, бросающихся в глаза, терапевт --- от болезней внутренних, неуловимых пальцами и щипцами. Однако от этого эпизода вернемся опять к нашей теме. 

Бог, религиозный идеал христианина, есть дух. Христианин отвергает свое чувственное существо; он ничего не хочет знать об обыкновенном <<животном>> стремлении к еде и питью, об обыкновенном <<животном>> влечении к половой любви и любви к детям; он смотрит на тело, как на позор, на постыдное пятно, лежащее со дня его рождения на его благородстве, на его чести, которая состоит в том, чтобы быть духовным существом, как на временно необходимое принижение и отрицание своего истинного существа, как на грязный дорожный костюм, как на вульгарное инкогнито своего небесного гражданства. Он хочет быть и стать только духом. Правда, древние христиане верили в воскресение плоти, и различие между верой христиан, по крайней мере древних, и верой языческих философов в том именно и состоит, что они верили не только в бессмертие духа, способности мышления, разума, но и в телесное бессмертие. <<Я не хочу пережить не только душу, но и тело. Я тело хочу иметь с собой>>  говорит Лютер. Но в том-то и дело, что это тело целиком небесное, духовное, то есть воображаемое тело, которое, как и вообще религиозные предметы, нам ничего другого не представляет и не конкретизирует, как сущность человеческих желаний и воображения. Духовное тело есть такое тело, которое, подобно фантазии, воображению человека, в один миг переносится в отдаленные места; которое, подобно представлению, проникает сквозь затворенные двери, потому что затворенные двери или стена не являются препятствием к тому, чтобы я мог в своем воображении нарисовать себе то, что происходит за стеной; это --- такое тело, которому нельзя нанести удар кулаком или ногой, которое нельзя ранить ударом или выстрелом, как нельзя нанести удар кулаком или ногой образу фантазии или сновидения; оно поэтому совершенно чудесное тело, осуществленное сверхъестественное желание человека иметь тело без болезней, без бед, без страданий, без возможности поранения и без смертности, а следовательно и без всяких потребностей; ибо ведь от разнообразия потребностей нашего тела и происходят как раз разнообразные болезни и страдания его, как, например, потребностью в воздухе объясняется как существование легких, так и их заболевания. Если бы мы не нуждались в воздухе и не имели легких, то имели бы одним источником и родом болезней меньше, чем имеем теперь. Но небесное, духовное тело не нуждается в воздухе, в еде, в напитках; это --- лишенное потребностей, божественное духовное тело; короче говоря, --- это вещь, не отличимая от продукта человеческого воображения и человеческих желаний, это тело, которое на самом деле не есть тело. Поэтому, несмотря на наличность небесного тела, мы можем рассматривать дух как идеал, как цель христианина и даже древнего. Различие между различными категориями христиан заключается лишь в том, что древние христиане, верившие в чудеса, имели своим идеалом или образцом главным образом дух воображения, дух, чреватый чувственными образами, созданными настроением; христианские, верующие в бога философы имели своим идеалом дух, создающий из образов общие понятия, мыслящий дух; рационалисты и моралисты --- дух, выражающийся в действиях, практический моральный дух. 

Таким образом, так как для христианина дух --- чувствующее, мыслящее, желающее существо есть его высшее существо, его идеал, то он делает его и первым существом, причиной мира; то есть он превращает свой дух в предметное, вне его существующее существо, от него отличное, из которого он поэтому производит или выводит вне его существующий овеществленный мир. По представлению христианина, бог, то есть овеществленный, вне человека представленный существующим дух, создал мир своей волей и разумом. Но этот дух, творящий мир, христианин отличает в качестве совершенного бесконечного духа от своего или вообще человеческого духа как духа несовершенного, ограниченного, конечного. Этот процесс различения, это умозаключение от <<конечного>> духа к духу бесконечному, это доказательство бытия бога, то есть в данном случае совершенного духа, есть доказательство психологическое. В то время, как так называемое космологическое доказательство исходит от мира вообще, физиологическое или телеологическое доказательство --- от порядка и связи, от целесообразности природы, психологическое доказательство, являющееся доказательством, характерным для сущности христианства, исходит от психики, или души, от духа человека. Языческий бог есть бог, выведенный из природы, происшедший из природы; христианский бог есть бог, выведенный из души, из духа, есть бог, возникший из души. Умозаключение, коротко говоря, следующее: человеческий дух существует; в его бытии мы не можем сомневаться; это нечто --- невидимое в нас, бестелесное, которое мыслит, желает и чувствует, но знание, воля и хотение человеческого духа недостаточны, ограничены чувственностью, зависят от тела; ограниченное же, конечное, несовершенное, зависимое предполагает существование чего-то неограниченного, бесконечного, совершенного; следовательно, конечный дух предполагает существование бесконечного духа как своего основания; следовательно, таковой существует, и он есть бог. Но следует ли отсюда самостоятельность, действительное существование подобного духа? Не есть ли бесконечный дух именно дух человека, желающий быть бесконечным, совершенным? Не принимается ли во внимание при возникновении этого бога также и желание человека? Не хочет ли человек быть свободным от границ своего тела, не желает ли он быть всеведущим, всемогущим, вездесущим? Не есть ли, следовательно, и этот бог, этот дух --- осуществленное желание человека быть бесконечным духом? Следовательно, не овеществили ли мы и в этом боге человеческое существо? Не умозаключают ли христиане и даже нынешние рационалисты в области веры от бесконечной жажды познания, имеющейся у человека, жажды, которая, однако, здесь не удовлетворяется и которая не может быть удовлетворена, от бесконечного стремления к счастью, не удовлетворенного никаким благом, никаким счастьем на земле, от потребности в совершенной, не запятнанной никакими чувственными влечениями нравственности, --- не умозаключают ли они от всего этого к необходимости и действительности жизни, не ограниченной временем пребывания на земле, не привязанной ни к телу, ни к смерти, --- к бесконечной жизни, бесконечному бытию человека? Не признают ли они тем самым, однако, хотя и не прямо, божественность человеческого существа? Не есть ли вечно длящееся, не имеющее конца, не привязанное ни к какому времени и месту, способное к всеведению, вообще к бесконечному совершенству существо --- бог или божественное существо? Не есть ли, следовательно, их бог, их бесконечный дух лишь символ и образец того, чем они сами когда-нибудь хотели бы стать, первообраз и отражение их собственного существа, развертывающегося в будущем? 

В чем же различие между божественным и человеческим духом? Единственно в совершенстве или бесконечности; свойства, определения существа в обоих одинаковы; дух, согласно христианской психологии, не имеет ничего общего с материей, с телом; он есть, как выражаются христиане, существо, абсолютно отличное от чувственного, телесного существа; но таков же и бог; бог невидим, неосязаем, но также и дух; дух мыслит, но мыслит и бог; христиане, даже рационалисты, усматривают ведь в вещах только осуществленные, сделавшиеся чувственными, принявшие телесность мысли бога; дух имеет сознание, волю, личность, или является ими, также и бог; различие заключается лишь в том, что то, что в человеке ограничено, в боге, разумеется, не ограничено, бесконечно. Что раскрывается, однако, в этой бесконечности божественных свойств? Бесконечность или неограниченность человеческих желаний, человеческого воображения и человеческой способности к абстракции, то есть способности выводить общее из единичного и отдельного, как я, например, вывожу из многих отдельных деревьев общее понятие дерева, оставляя в стороне все отличия, все особенности, которыми отдельные деревья отличаются в действительности. Бесконечный дух есть не что иное, как родовое понятие духа, которое, однако, воображением воплощается в самостоятельное существо по требованию человеческих желаний и влечения к счастью. <<Чем менее определенно, --- говорит Св. Фома Аквинский, --- чем общее, чем отвлеченнее какое-нибудь слово или определение, тем более оно подходит для бога, тем более оно ему соответствует>>. Мы уже видели это раньше, когда ставился вопрос о существовании, о существе бога вообще. Сейчас, когда мы имеем дело с сущностью христианства, сущность которого есть дух, нам надлежит это доказать на примере духовных определений бога. Бог, говорится, например, в Библии, есть любовь. Под этой любовью надо понимать любовь отвлеченную, любовь вообще, человеческая же любовь имеет различные виды, как-то: любовь к другу, любовь к отечеству, половая любовь, любовь к детям, любовь к родителям, любовь в смысле благожелательного отношения к людям вообще, в смысле дружелюбия; она базируется в человеке на его склонности, ощущении и чувственности. Любовь, отвлеченная от всех этих ее разновидностей, от всех чувственных и специальных определений, любовь, мыслимая исключительно как таковая, есть любовь, которая приписывается богу или мыслится как сам бог. Другой пример, это --- слово бога, или божественное слово. Древние христиане, которые мыслили гораздо последовательнее, чем современные, которые почти всю психологию и антропологию переносили в теологию, приписывали божественному духу также и божественное слово и при этом совершенно правильно. Дух высказывает себя самым духовным, самым ему соответствующим образом только в слове; ведь мысль и речь, хотя бы и не высказанная внешне губами, друг с другом неразрывны; со словом исчезает и мысль, с названием --- и вещь, которую название обозначает; люди поэтому начали думать тогда, когда они начали говорить, образовывать слова. Если поэтому богу приписывается дух, разум, если говорят о мыслях божьих, то нужно быть также последовательным и говорить о божьих словах. Если не стесняются объяснять возникновение чувственного мира мыслью и волей духа, не стесняются утверждать, что вещи не потому мыслятся, что существуют, но существуют потому, что мыслятся, то пусть не стесняются объяснять их возникновение и силой слова, а также утверждать, что не слова существуют потому, что есть вещи, но вещи существуют только благодаря словам. Дух, как таковой, действует лишь через слово; выступая из своих рамок, он вступает в мир и выявляется. 

Древние теология и религия производили поэтому мир методом, соответствующим существу бога как духа, заставляя мир возникать через слово божие, через божественный глагол. Это представление о происхождении мира через слово нисколько, впрочем, не свойственно одной лишь еврейской или христианской религии, оно уже встречается в древней персидской религии. Что по-гречески значит <<логос>>  то там есть honover, что и по новейшим исследованиям, как, например, Рета (<<Египетские и зороастровские вероучения>>), ничего другого не означает, как самое доподлинное слово. Но слово божие, по крайней мере в христианской теологии, есть не что иное, как понятие слова вообще; божественное слово есть не то или другое определенное слово, не латинское, немецкое, еврейское, греческое слово, не отдельное или особенное, теряющееся в воздухе, временное слово; но все эти и подобные свойства, которыми теологи наделяли слово божие, применимы и к понятию слова или к тому, что является словом вообще, словом самим по себе. Это родовое понятие, эту сущность слова, общую всем бесчисленным разнообразным словам, религиозная и теологическая фантазия овеществляет как особое, личное существо, в свою очередь отличающееся от слова и его сущности, как она представляет себе существо бога, --- хотя это существо первоначально и на самом деле является не чем иным, как сущностью мира, --- как особое существо, отличное от сущности мира. То же, что относится к слову, к любви, то относится и к духу вообще, к разуму, к воле, к сознанию, к личности, как она приписывается богу или обожествляется, как она представляется в виде бога. Это всегда лишь какая-нибудь человеческая сила, свойство, способность, которая обожествляется, но когда она обожествляется, то отделяется от всех особых определений, присущих этой силе, этой способности как чему-то действительному и человеческому, так что, когда это отделение производится до наивысших или крайних пределов, то в конце концов ничего не остается, кроме голого названия, --- названия воли, названия сознания; сущность же сознания и воли, то есть то, что, делает сознание и волю действительным сознанием и волей, исчезает. Таким образом, теология сводится в конце концов к пустой, но утешительной фразеологии. 

\phantomsection
\addcontentsline{toc}{section}{Двадцать девятая лекция}
\section*{Двадцать девятая лекция}

Смысл и суть моего изложения психологического доказательства бытия божия, доказательства, которое я определил как характеризующее сущность христианства, были те, что доказательство бытия бога или, вернее, бесконечного духа, --- потому что в качестве такового бог определяется в христианстве --- есть лишь косвенное, непрямое доказательство бесконечности человеческого духа, и, наоборот, --- доказательство бессмертия говорит непосредственным, прямым образом о бесконечности как свойстве человеческого духа. Дело в том, что христиане умозаключают, что должен быть бесконечный дух, раз существует конечный; --- совершенный, раз существует несовершенный; всеведущий, раз есть знающий кое-что; --- всемогущий, раз есть дух, кое на что способный. Но они точно так же заключают, что должна быть вечная, бесконечная жизнь человека, потому что духовные человеческие силы и способности в рамках этой жизни, этого тела не находят себе места, не могут развиться в меру своих желаний и возможностей; --- что человек должен когда-нибудь все знать, потому что у него неограниченная жажда познания; что человек, или человеческий дух, должен когда-нибудь сделаться вполне нравственным и счастливым или --- как хитроумно выражаются современные рационалисты --- если и не вполне совершенным, то, по крайней мере, все более совершенным, шаг за шагом, вплоть до бесконечности, потому что у него есть не только бесконечная способность к совершенствованию, но и бесконечное влечение к совершенствованию и счастью --- влечение, которое, однако, на этой маленькой земле, в этот короткий промежуток времени, в этой юдоли скорби остается неудовлетворенным. Мы отсюда видим, что умозаключение к богу и умозаключение к бессмертию --- по существу одно и то же умозаключение, что именно поэтому идея божества и идея бессмертия по существу, в основе своей одна и та же. Умозаключение к богу предшествует только умозаключению к бессмертию; божество есть предпосылка бессмертия, без бога нет бессмертия. Но лишь бессмертие является смыслом и целью бытия божьего или умозаключения к его бытию. Без бога вера в бессмертие не имеет точки опоры, не имеет ни начала, ни основы, ни принципа. Бессмертие есть сверхчувственное желание и необузданная мысль, противоречащие свидетельству чувств, которые удостоверяют реальность смерти. Как могу я верить в истинность этой мысли, в осуществимость этого желания, если не существует соответствующего этому желанию, этой мысли, противоречащего чувствам и сверхчувственного, чрезвычайного существа? Как могу я связать эту веру с природой, с миром? 

В природе не существует другого бессмертия, кроме продолжения рода, при котором данное существо продолжается в существах себе подобных, то есть место умершего индивидуума постоянно заступает новый. У низших животных, как, например, у бабочек, с актом оплодотворения связана даже непосредственно смерть. Бабочка умирает, как только она выпустила в мир других бабочек или, по крайней мере яички --- их зародыши. Если бы не было размножения, не было бы и смерти, потому что в воспроизведении данное существо исчерпывает свою жизненную силу; приумножением самого себя, или, выражаясь иначе, тем, что оно дает миру много существ себе подобных, своего вида, оно уничтожает неповторяемость и тем самым необходимость своего существования. Правда, человек на много переживает потерю своей потребности воспроизводства; но там, где способность воспроизводства человека исчерпана, там вместо с тем начинается и старость, там он приближается, хотя и медленно, к концу. Как могу я, стало быть, связывать с природой веру в бессмертие? Природа дает смерть, бессмертие же дает только бог. Правда, раз человек верит в бессмертие, то он находит также и в природе достаточно примеров и доказательств этой своей веры, то есть он истолковывает природу в своем духе, благоприятно для своей веры; поэтому то, в чем христиане усматривали, как, например, в смене времен года, в восходе и заходе солнца, доказательства или образы своего бессмертия, своего воскресения --- потому что они в них верили и смотрели на все сквозь очки этой веры, --- в том язычники, которые ни в какое бессмертие не верили, усматривали как раз доказательства или образы своей тленности, или смертности. Лед, --- говорит, например, Гораций, --- растапливают зефиры, весну прогоняет лето, которое исчезает, лишь только осень принесет свой урожай, и сейчас же вслед за тем возвращается опять безжизненная зима. Однако потери неба возмещаются новыми лунными оборотами; только мы, однажды спустившись туда, где находятся благочестивый Эней, богатый Туллий и Анк, становимся прахом и тенью. Как могу я, далее, верить тому, что после видимой, очевидной, бесспорной гибели тела еще остается так называемая душа, то есть сущность человека, если я не верю, что существует вообще душа, дух без тела, и что этот дух без тела есть наивысшая сущность --- та сущность, в сравнении с которой все чувственные и телесные существа ничтожны, бессильны? Вера в бессмертие предполагает поэтому веру в божество, то есть человек мыслит себе бога потому, что он не может мыслить себе бессмертия без бога. В представлении, в доктрине, в учении бессмертие есть лишь следствие веры в бога; но на практике или в действительности вера в бессмертие есть основа веры в бога. \emph{Человек не потому верит в бессмертие, что верит в бога, но он верит в бога потому, что верит в бессмертие, потому что без веры в бога нельзя обосновать веру в бессмертие. По видимости на первом месте стоит божество, на втором --- бессмертие; но на самом деле на первом --- бессмертие, на втором --- божество.} Божество есть первое лишь постольку, поскольку оно является средством, условием бессмертия или, иначе выражаясь: оно есть первое, потому что оно есть воплощенное блаженство и бессмертие, будущее существо человека, представленное, олицетворенное в виде наличного существа, так что вера в бессмертие и вера в божество не являются особыми символами или предметами веры, но вера в бога есть непосредственно вера в бессмертие, и, наоборот, вера в бессмертие есть вера в бога. 

Против этого утверждения, гласящего, что бог и бессмертие --- одно и то же, что они неразличимы, можно привести то возражение, что в бога можно верить, не веря в бессмертие, как это доказали не только многие индивидуумы, но и целые народы. Однако бог, с которым не связано представление или вера в бессмертие человека, не есть еще настоящий, подлинный бог, он есть лишь обожествленное существо природы; ибо божественность и вечность существа природы еще не включают в себя и бессмертия человека: природа не имеет сердца, она не чувствительна к человеческим желаниям, ей нет дела до человека. Если я мыслю себе, подобно древним персам и другим народам, солнце, луну и звезды как вечные существа, то что из этого для меня следует? Солнце, луна и звезды существовали прежде, чем их увидел человеческий глаз, они существуют не потому, что я их вижу, но я их вижу потому, что они существуют; хотя они существуют только для существа, обладающего зрением, но все же они на мой глаз оказывают то влияние, которое мы называем светом; короче говоря, мое видение их предполагает их бытие; они были прежде, чем я их видел, и они будут, когда я их больше не буду видеть, ибо я полагаю, что они не для того существуют, чтобы я мог их видеть. Таким образом, что вытекает отсюда для бессмертия моего глаза или моего существа вообще? Бог, из бытия которого не вытекает бессмертия, есть, стало быть, или некий предмет природы, или, хотя и человеческое, но аристократическое существо, как боги политеистов, в частности древних греков. У греков люди называются смертными, боги --- бессмертными. Бессмертие здесь совпадает с понятием божества; но оно есть привилегия божества; оно не переходит на человека, потому что боги --- аристократы, которые ничем из своих преимущественных прав не поступаются, потому что они существа ревнивые, эгоистичные, завистливые. Они, правда, насквозь человеческие существа, у них все пороки и страсти греков; но они образуют все же особый класс существ и поэтому не дают обычной человеческой твари принять участие в их блаженстве и бессмертии. <<Боги сделали уделом бедных людей страх и страдание; они же сами живут счастливо и беззаботно>> - говорится в <<Илиаде>>. Впрочем, по крайней мере для Гомера, отца или крестного отца греческих богов, бессмертие богов так же не слишком много значит, ибо, хотя они в действительности и не умирают, но могут все же умереть. Или же бог, с которым не связано бессмертие, есть национальный бог, как бог древних евреев. Евреи не верили в бессмертие, но лишь в продолжение рода через размножение, они желали себе лишь долгой жизни и потомства, как вообще древние, в том числе восточные народы, у которых считалось величайшим несчастьем --- и считается до сих пор --- покинуть этот мир, не оставив детей. Но бог Иегова, по крайней мере древний, по своему существу ничем не отличается от существа древних израильтян. Что израильтянин ненавидел, то ненавидел и его бог; что израильтянин охотно обонял, то и господу было приятным запахом. <<Ной при выходе из ковчега принес жертву, и господь обонял приятный запах>>. Яства, которые они сами вкушали, были и яствами их бога. С национальным богом может связываться лишь идея бесконечного распространения и длительности существования нации. <<Я хочу, --- сказал Иегова родоначальнику евреев, Аврааму, --- я хочу благословить твое семя и размножить, как звезды на небе и песок на морском берегу>>. 

Бог, который не внушает человеку сознания его бессмертия, который для человека не является порукой, что тот будет вечно жить, есть лишь бог по имени, но не на деле. Таким номинальным богом является, например, бог некоторых так называемых спекулятивных философов, отрицающих бессмертие и держащихся за бога; но они потому за него держатся, что они многого без него не могут мыслить и объяснить, потому что им нужно заполнить пробелы их системы, их ума; он поэтому --- только теоретическое, философское существо. Таким богом является, далее, бог некоторых рационалистических естествоиспытателей, который есть не что иное, как олицетворенная природа, или естественная необходимость, вселенная, с чем, разумеется, не согласуется представление о бессмертии, ибо в представлении о вселенной человек теряет себя из виду и видит, как он исчезает, --- или же этот бог не что другое, как первая причина природы, или мира; но первая причина мира еще далеко не есть бог. Первой причиной мира я могу мыслить себе простую силу природы. Бог есть по существу предмет почитания, любви, поклонения; но природную силу я не могу любить, религиозно почитать, ей поклоняться. Бог не есть естественное существо, естественная сила; бог есть сила абстракции, сила воображения и сердца. Бог существует только в сердце, бог, говорю я в предпоследнем параграфе своего трактата о <<Сущности религии>>  который лежит в основе этих лекций, бог не есть вещь, которую ты можешь найти телескопом на астрономическом небе или лупой в ботаническом саду, минералогическим молотком в рудниках геологии или анатомическим ножом во внутренностях животных и людей: ты его найдешь только в вере, только в способности воображения, только в сердце человека; потому что он сам не что иное, как сущность фантазии, не что иное, как сущность человеческого сердца. Поэтому по своей сущности бог есть существо, исполняющее желания человека. Но к желаниям человека, по крайней мере человека, который не ограничивает своих желаний естественной необходимостью, принадлежит прежде всего желание не умереть, жить вечно; и это желание есть последнее и высшее желание человека, желание всех желаний, потому что жизнь есть понятие, включающее в себя все блага; бог, который поэтому не выполняет этого желания, который не уничтожает смерти или не заменяет, по крайней мере, ее другой, новой жизнью, --- не есть бог, по крайней мере, не есть истинный бог, соответствующий понятию бога. Как вера в бессмертие лишена основания без веры в божество, так и вера в бога бессмысленна без веры в бессмертие. Бог есть по существу своему идеал, прообраз человека; но прообраз человека существует не для себя, а для человека; его значение, его смысл, его цель лишь те, чтобы человек сделался тем, что представляет собой его прообраз; прообраз есть лишь олицетворенная, представленная в виде отдельного существа будущая сущность человека. \emph{Бог есть поэтому по существу своему коммунистическое, а не аристократическое существо; всем, что он собой представляет и что он имеет, он делится с человеком; все его свойства становятся свойствами человека, и притом с полным правом: они ведь произошли от человека, они ведь взяты от человека, а потому в конце концов и возвращаются опять к человеку.} <<Бог блажен,- говорит, например, Лютер, --- но он не хочет быть блаженным в одиночестве>>. 

Религия представляет бога в виде самостоятельного, личного существа: она видит поэтому в бессмертии и других божественных свойствах, к которым приобщается человек, как бы дар божественной любви и доброты. Но истинная причина, почему человек при том вырождении религии, при котором мы сейчас присутствуем, становится в учении о последних вещах божественным существом, --- та, что бог, по крайней мере христианский, есть не что иное, как существо человека. Но если существо человека есть божественное существо, то необходимым следствием является то, что и индивидуумы, отдельные люди суть боги или становятся ими. Идеалом или образцом и в то же время порукой божественности и бессмертия не только отвлеченной сущности человека, которая есть дух, разум, воля, сознание и которая обожествляется в лице невидимого, неосязаемого бога, так называемого бога-отца, но также и отдельного, то есть действительного, человека, индивидуума, является в христианстве богочеловек, Христос, в котором поэтому явственно дает себя знать и обнаруживается то, что божественное существо есть существо, не отличающееся от человека. Современный рационалист в своей половинчатости, бестактности и поверхностности отказался от богочеловека, но сохранил бога, отказался от вывода, от необходимого следствия веры в бога, но основание ее оставил; он, как я уже показал по другому поводу, сохранил учение, но отверг применение, пример, индивидуальный частный случай, подтверждающий данное учение. Рационалист сохранил дух: бог есть дух, говорит подобно правоверному христианину рационалист, но, несмотря на свой дух и свою разумную веру, он потерял голову; у него дух лишен головы, тогда как правоверный христианин совершенно разумным и естественным образом придал божественному духу в лице своего богочеловека голову, как необходимый орган и символ духа. Рационалист признает божественную волю; но без необходимых условий и внешних средств для этой воли, без двигательных нервов и мускулов, одним словом, без тех орудий, при посредстве которых христианский бог чудесами богочеловека подтверждает и доказывает, что он имеет действительную волю; рационалист говорит о божественной благости и провидении, но он устраняет человеческое сердце богочеловека, сердце, без которого благость и провидение --- лишь пустые слова без истинного содержания; рационалист основывает бессмертие на идее бога --- правда, не целиком на ней, но отчасти; он говорит о божественных свойствах как ручательстве бессмертия: <<как верно то, что есть бог, так верно и то, что мы бессмертны>>; тем не менее он отвергает свидетельство неразрывности, единства божества и бессмертия, ибо отвергает единство божественного и человеческого существа в богочеловеке как идолопоклонническое суеверие. Заключение: <<как истинно то, что есть бог, так истинно то, что мы бессмертны>>  оправдывается лишь в том случае, если оно имеет своей предпосылкой положение или может быть положением, гласящим: <<как истинно то, что бог есть человек, так истинно и то, что человек есть бог, а стало быть, истинно также и то, что свойство бога не подлежать смерти и не подвергаться уничтожению есть свойство человека>>. 

Вывод о бессмертии человека, следующий из понятия бытия божьего, основывается только на единстве, то есть на отсутствии различия между божественным и человеческим существом. Даже религиозная вера, хотя она и представляет бессмертие как следствие благости божией, как дело милосердия, свободной воли бога, основывается в то же время на родственности божественного и человеческого существа и духа. Но родственность предполагает единство и одинаковость существа или природы, или, лучше, она есть лишь чувственное выражение для единства и одинаковости. Поэтому --- и этим я ограничиваю и поправляю ранее высказанное положение --- вера человека в бессмертие, разумеется не в смысле христианского бессмертия, может быть связана с безразличным и не имеющим никакого отношения к человеку предметом, существом природы, с солнцем и другими небесными телами; но лишь при условии, что человек рассматривает себя как существо, родственное этим небесным телам, что он верит, что его существо и существо этих небесных тел имеют одну и ту же природу. Если я небесного происхождения, имею небесное существо, то я, разумеется, так же мало могу умереть, как и эти небесные тела, если я их мыслю себе бессмертными. Их бессмертие ручается за мое собственное, ибо как мог бы отец допустить гибель и смерть своих собственных детей? Он бы тем самым выступил против своих собственных плоти и крови, против своего собственного существа. Как небесное существо производит лишь небесные, так и бессмертное существо производит в свою очередь лишь бессмертных детей или существ. Поэтому человек ведет свое бытие от бога, дабы тем самым заверить свое божественное происхождение и через него божественность, то есть бессмертие своего существа. Кто хочет выйти из рамок смерти, этого следствия естественной необходимости, тот должен выйти и из рамок ее основания, из рамок самой природы. Кто не хочет кончить в рамках природы, тот не может и начинать в рамках природы, а только лишь вместе с богом. Не природа, нет! сверхъестественное, божественное существо есть творец, есть моя причина, то есть, попросту говоря: я сверхъестественное, божественное существо; но основанием моей сверхъестественности и божественности является не мое происхождение от сверхъестественного существа; наоборот, я произвожу себя от такового потому, что я в глубине моего существа до такого выведения уже рассматриваю себя как таковое, и поэтому не могу себя мыслить происшедшим от природы, от мира. <<Мы видим, --- говорит Лютер в своем толковании 1-й книги Моисея, --- что человек есть особое творение, что он создан потому, что он причастен к божеству и бессмертию, ибо человек есть более совершенное творение, чем небо и земля со всем, что на них имеется>>. <<Я есмь человек, --- говорит он же в другом месте, уже приведенном в одном из моих более ранних сочинений, --- и это значит больше, чем быть князем. Причина же та, что князя бог не сотворил, но сотворил человека, что я есмь человек, это сделал один только бог>>. Точно так же говорит языческий философ Эпиктет, в своих учениях и представлениях в высшей степени приближающийся, однако, к христианству: <<Если бы кто твердо усвоил себе, что мы все имеем главной причиной бога, что бог есть отец людей (и богов), то он наверное, никогда не подумал бы о себе неуважительно или низко. Если бы император тебя усыновил, то никто не мог бы вынести твоей гордости. Не должна ли, стало быть, мысль о том, что ты сын божий, тебя подымать, делать тебя гордым?>>. Но не есть ли и каждая вещь, и каждое существо также создание божие? Согласно религии, не сотворил ли бог все? Да, но он не есть в том же смысле творец животных, растений, камней, как он есть творец людей; в отношении к людям он отец, но он не отец животных, иначе бы христиане заключили братство с животными, как они из того, что бог есть отец людей, выводят, что все люди --- братья и должны быть братьями. <<Он (бог), говорит, например, Лютер, в своем собрании церковных проповедей, --- он ведь ваш отец, и только ваш отец, а не отец птиц, гусей или уток (а также и не безбожных язычников)>>. Точно так же и платоники, у которых была почти та же теология, как и у христиан, но не было христологии, различали между богом как строителем, демиургом и богом как отцом; бога, как создателя духовных существ --- людей, они называли отцом, бога, как творца неодушевленных существ и животных, они называли строителем, демиургом (см. Плутарх, Платоновские вопросы). Смысл учения, что бог есть отец людей, или что люди --- дети бога, заключается, стало быть, в том, что человек --- божественного происхождения, сущность его божественна, а стало быть, и бессмертна. Бог как общий отец людей есть не что иное, как олицетворенное единство и одинаковость человеческого рода, понятие рода, в котором все различия людей уничтожены, устранены, но при этом родовое понятие в отличие от действительных людей представляется в виде самостоятельного существа. 

Поэтому совершенно естественно и необходимо, чтобы божественные свойства стали свойствами человека; ибо то, что относится к роду, относится и к индивидуумам. Род есть лишь то, что охватывает всех индивидуумов, то, что обще всем. Поэтому там, где верят в бога без веры в бессмертие, там либо истинный смысл и понятие божества еще не найдены, либо опять утрачены. А этот смысл заключается в том, что бог есть олицетворенное родовое понятие человека, олицетворенная божественность и бессмертие человека. Поэтому вера человека в бога, --- разумеется, в бога, поскольку он не выражает собой существа природы, --- есть, как я это говорил в <<Сущности христианства>>  лишь вера человека в свое собственное существо. Бог есть лишь существо, исполняющее, осуществляющее желания человека, но как могу я верить в существо, осуществляющее мои желания, если я заранее или одновременно не верю в святость, в существенность и правомерность, в безусловную значимость моих желаний? Но как могу я верить в необходимость исполнения моих желаний, в необходимость, которая является основой необходимости существования исполнителя желаний --- бога, если я не верю в себя, в истинность и святость моего существа? То, чего я желаю, --- это мое сердце, мое существо. Как могу я отличить мое существо от моих желаний? Поэтому вера в бога зависит лишь от веры человека в сверхъестественное величие его существа. Или --- в божественном существе опредмечивает он свое собственное существо. В божественном всеведении находит он --- чтобы еще раз коротко сформулировать сказанное --- исполнение своего желания все знать, или овеществляет способность человеческого духа в своем знании не ограничиваться тем или другим предметом, но охватывать все; в божественном вездесущии или повсеместности осуществляет он лишь желание не быть связанным ни с каким местом, или опредмечивает способность человеческого духа в своих мыслях быть всюду; в божественной вечности, или принадлежности ко всем временам, осуществляет он желание не быть связанным ни с каким временем, не иметь конца, или опредмечивает бесконечность и (по крайней мере, если он последовательно мыслит) безначальность человеческого существа, человеческой души, ибо если душа не может умереть, прекратить свое существование, то не может она возникнуть, иметь начало, как многие в это совершенно последовательно верили; в божественном всемогуществе осуществляет он лишь желание все мочь, желание, связанное с желанием все знать или являющееся лишь его следствием; ибо человек что-нибудь может лишь в той мере, в какой, как говорит англичанин Бэкон, он знает, потому что кто не знает, как делают какую-нибудь вещь, тот и не может ее сделать; способность делать предполагает знание; поэтому кто хочет все знать, тот хочет и все мочь; или --- в божественном всемогуществе человек опредмечивает и обожествляет лишь свое всемогущество, свою неограниченную способность ко всему. Животное, --- говорит один христианский мыслитель, который сам писал об истинности христианской религии, Гуго Гроций, --- может только либо то, либо это; но могущество, способность человека неограниченны. В божественном блаженстве и совершенстве человек осуществляет лишь желание самому быть блаженным и совершенным, а стало быть, и морально совершенным; ибо без морального совершенства нет блаженства: кто может быть блажен при наличии у него ревности, зависти и недоброжелательства, злобности и мстительности, жадности и пьянства? Божественное существо есть, стало быть, существо человека, но не то, которое имеется в прозаической действительности, а то, которое соответствует поэтическим требованиям, желаниям и представлениям человека, которое, вернее, должно быть и когда-нибудь будет. Но самое горячее, самое искреннее, самое святое желание и мысль человека есть или, по крайней мере было некогда, желание бессмертной жизни и стремление стать бессмертным существом. Поэтому существо человека, поскольку он желает и мыслит бессмертие, есть божественное существо. Иначе бог есть не что иное, как будущее бессмертное существо человека, мыслимое --- в отличие от телесного, чувственно существующего человека --- самодовлеющим, духовным существом. Бог есть нечеловеческое, сверхчеловеческое существо; но будущий бессмертный человек есть также существо, стоящее над нынешним, действительным, смертным человеком. Как бог отличен от человека, так же отличен и составляющий предмет веры будущий или бессмертный человек от действительного, нынешнего, или смертного, человека. Короче говоря, единство, неразличимость божества и бессмертия, а следовательно, божества и человечества, есть разрешенная загадка религии, в частности христианской. \emph{Как, стало быть, природа как предмет и содержание человеческих желаний и воображения, есть сущность естественной религии, так и человек как предмет и содержание человеческих желаний, человеческого воображения и абстракции есть сущность религии духа, христианской религии}. 

\phantomsection
\addcontentsline{toc}{section}{Тридцатая лекция}
\section*{Тридцатая лекция}

Доказав в предыдущих лекциях, что лишь в бессмертии находится и достигается смысл и цель божества, что божество и бессмертие --- одно и то же, что божество из самостоятельного существа, каким оно является первоначально, становится, в конце концов в виде бессмертия свойством человека, я пришел к цели моей задачи и тем самым к концу моих лекций. Я хотел доказать, что бог естественной религии есть природа, бог духовной религии, христианства, дух, то есть сущность человека. Я хотел это доказать для того, чтобы человек впредь искал и находил определяющий мотив своего поведения, цель своего мышления, источник исцеления своих бед и страданий в себе самом, а не вне себя, как язычник, или над собой, как христианин. Я не мог, разумеется, это доказательство, в отношении к наиболее интересующему нас христианству, провести через все отдельные учения и представления христианства; я еще меньше мог его распространить, как я первоначально имел в виду, на историю христианской философии. Но этого и не нужно, по крайней мере когда дело идет о предмете, каким был предмет этих лекций, не нужно проводить свою тему вплоть до единичных и особых случаев. Главное, --- это основные положения, основные тезисы, из которых частные выводы вытекают как простые следствия. И эти основные положения моего учения я представил, и притом представил наивозможно ясней. Разумеется, я мог бы быть короче в моих первых лекциях. Но меня извиняет то обстоятельство, что я --- не академический доцент, что у меня нет навыка к лекциям, что я не имел перед собой готовой тетрадки, и я поэтому не сумел мой материал соразмерить и соответственно подразделить по мерке академического времени. Но в моих лекциях остался бы пробел, я закончил бы их фальшивой нотой, если б я вздумал оборвать их на том доказательстве, которое я развивал в последний раз, потому что я оставил нетронутыми те предпосылки, исходные положения или гипотезы, от которых христианин умозаключает к божеству и бессмертию. 

Бог, говорил я, есть осуществитель или действительность человеческих желаний счастья, совершенства, бессмертия. Кто, стало быть, --- можно из этого заключить, --- отнимет у человека бога, тот вырвет у него сердце из груди. Однако я оспариваю те предпосылки, на основании которых религия и теология доказывают необходимость и бытие божества или, что то же, бессмертия. Я утверждаю, что желания, которые исполняются лишь в воображении, или исходя из которых доказывают бытие воображаемого существа, являются тоже лишь воображаемыми, а не действительными, не истинными желаниями человеческого сердца; я утверждаю, что те пределы, которые религиозное воображение преодолевает в божестве или бессмертии, являются необходимыми определениями человеческого существа, неотделимыми от него, а следовательно, являются пределами только в воображении человека. Так, например, для человека, отнюдь не является ограничением то, что он связан с местом и временем, что <<его тело приковывает его к земле>>  как говорит рационалист, <<и мешает ему поэтому знать, что делается на Луне, на Венере>>. Тяготение, привязывающее меня к земле, есть не что иное, как выражение моей связи с землей, моей неотделимости от нее; что я такое, если я порву эту связь с землей? Фантом, потому что я по существу --- земное существо. Мое желание перенестись на какое-нибудь другое мировое тело есть поэтому лишь воображаемое желание. Если бы я мог осуществить это желание, то я бы убедился, что это лишь фантастическое, глупое желание, потому что я почувствовал бы себя в высшей степени неладно на чужом мировом теле и поэтому --- к сожалению, слишком поздно --- понял бы, что было бы лучше и разумнее оставаться на земле. 

Есть много человеческих желаний, которых не понимают, полагая, что они стремятся быть осуществленными действительно. Они хотят оставаться лишь желаниями, они имеют цену лишь в воображении; их исполнение было бы горчайшим разочарованием для человека. Таким желанием является и желание вечной жизни. Если бы это желание исполнилось, то люди пресытились бы вечной жизнью и начали бы желать смерти. В действительности человек не желает только преждевременной, насильственной, страшной смерти. Все имеет свою меру, --- говорит один языческий философ, --- всем под конец пресыщаешься, даже и жизнью, и человек желает поэтому в конце концов смерти. Нормальная, сообразная с природой смерть, смерть законченного человека, который изжил себя, не имеет поэтому ничего страшного. Старцы поэтому часто жаждут смерти. Немецкий философ Кант в своем нетерпении не мог дождаться смерти, до такой степени он ее жаждал, но не для того, чтобы вновь воскреснуть, а из потребности конца. Только неестественный, трагический случай смерти --- смерть ребенка, юноши, мужа в цвете сил --- возмущает нас против смерти и порождает желание новой жизни. Но как ни страшны, ни горестны подобные случаи для тех, кто остается в живых, они уже по тому одному не оправдывают принятия нами потустороннего мира, что эти отклоняющиеся от нормы случаи --- хотя бы они чаще встречались, чем естественная смерть, --- имеют своим выводом тоже отклоняющийся от нормы потусторонний мир, --- мир лишь для насильственно или преждевременно умерших; но подобный странный потусторонний мир был бы чем-то невероятным и противным рассудку. 

Но как желание вечной жизни, так и желание всеведения и безграничного совершенства есть лишь воображаемое желание, и влечение человека к безграничному знанию и совершенствованию, якобы лежащее в основе этого желания, лишь присочинено человеку, как доказывают ежедневный опыт и история. Человек не хочет знать всего, он хочет знать лишь то, к чему он питает особую любовь и склонность. Даже и человек с универсальной жаждой знания, что представляет собой редкое явление, никоим образом не хочет знать всего без различия; он не хочет знать всех камней, как минералог, или всех растений, как ботаник; он довольствуется обобщениями, потому что они удовлетворяют его склонность к общему, и точно так же человек не хочет все мочь, а только то, к чему он чувствует в себе особое влечение; он не стремится к безграничному, неопределенному совершенству, которое осуществляется лишь в боге или бесконечном потустороннем мире, а только к совершенству определенному, ограниченному, к совершенству в границах определенной области. Поэтому мы видим, что не только отдельные люди останавливаются на одном месте, однажды достигнув определенного миропонимания, определенной высоты развития и совершенствования их наклонностей, но останавливаются даже и целые народы, в течение веков не трогаясь с места, сохраняя один уровень. Так, китайцы и индийцы еще и поныне находятся на том же уровне, на котором они стояли уже за тысячелетия до того. Как согласовать эти явления с тем безграничным стремлением к совершенствованию, которое рационалист приписывает человеку и которому он пытается поэтому уделить место в бесконечном потустороннем мире? Человек, наоборот, имеет не только стремление к тому, чтобы идти вперед, но также и стремление к отдыху, к остановке на уровне, однажды достигнутом и отвечающем определенности его существа. Из этих противоположных стремлений рождается историческая борьба, борьба и нашего времени. Прогрессисты, или так называемые революционеры, стремятся вперед, консерваторы хотят оставить все по-старому, хотя они, --- будучи большею частью также и верующими, --- отнюдь не принадлежат к числу сторонников застоя, когда речь идет о смерти, но, наоборот, готовы допустить в потустороннем мире самые радикальные изменения, самые революционные преобразования своего существа, чтобы там иметь возможность продолжать свое драгоценное существование. Но и революционеры не хотят идти вперед до бесконечности, у них есть определенные цели, достигнув которых, они останавливаются и сами делаются сторонниками застоя. Поэтому за продолжение нити истории, которую обрывают старые прогрессисты, как только они достигли цели своих желаний и с тем вместе пределов своего существа и разума, всегда берется новое, молодое поколение. 

И так же мало, как у человека имеется неограниченное влечение к знанию и совершенствованию, так же мало у него и неограниченного, ненасытного стремления к счастью, не могущему быть удовлетворенным никакими благами земли. Наоборот, человек, даже верящий в бессмертие, вполне довольствуется земною жизнью до тех пор, по крайней мере, пока ему хорошо живется, пока он не нуждается в самом необходимом, пока его не постигает какое-либо особое, тяжелое несчастье. Человек хочет только устранить беды этой жизни, но не хочет существенно иной жизни. <<Гренландцы, например, помещают своих блаженных под море, потому что из моря они получают большую часть своей пищи. Там, --- говорят они, --- есть хорошая вода и изобилие птиц, рыб, тюленей и северных оленей, которых можно без труда поймать или даже найти заживо сваренными в большом котле>>. Перед нами здесь пример, образчик человеческого стремления к счастью. Желания гренландца не выходят за пределы его страны, его природы: он не хочет по существу других вещей, чем те, которые ему дает его страна; он лишь хочет, чтобы то, что она ему дает, было хорошего качества и в изобилии. Он и на том свете хочет продолжать ловить рыбу и тюленей. --- То, что он собой представляет, не является для него ограничением, не является для него тяжестью; он не хочет выходить за пределы своего рода, из рамок своего положения и жизненного призвания, --- он хочет лишь на том свете устроить себе более удобную и легкую ловлю. Какое скромное желание! Культурный человек, дух и жизнь которого не привязаны к ограниченному месту, как дух и жизнь дикаря, не знающего ничего другого, кроме своей страны, и разум которого не простирается дальше нескольких географических миль, имеет, разумеется, не столь ограниченные желания. Он желает --- если оставаться при том же примере --- потреблять животных и растения не только своей страны; он хочет удовлетворять свои потребности и продуктами самых отдаленных стран; его потребности и желания, по сравнению с потребностями и желаниями дикаря, бесконечны; и, тем не менее, они все же не выходят ни из границ природы земли, ни из границ природы человека. Но по существу своему желания культурного человека совпадают с желаниями дикаря; он не хочет небесных яств, о которых он ничего не знает; он хочет лишь произведений земли; он не хочет упразднить еду вообще, но только потребление грубых произведений, всецело ограниченных данным местом. Одним словом: разумное и сообразное с природой стремление к счастью не выходит из границ существа человека, из границ существа этой жизни, этой земли; оно хочет уничтожить лишь те беды, лишь те ограничения, которые действительно можно уничтожить, которые не необходимы и не принадлежат к сущности жизни. 

Поэтому желания, выходящие из границ человеческой природы, или рода, как например, желание совсем не есть, вообще не зависеть от телесных потребностей, --- суть воображаемые, фантастические желания, а стало быть, и существо, удовлетворяющее эти желания, и жизнь, в которой они находят себе удовлетворение, являются существом и жизнью воображаемыми и фантастическими. Наоборот, желания, не выходящие из границ человеческого рода, или природы, имеющие свое основание не в одном только беспочвенном воображении и неестественной взвинченности чувств, но и в действительной потребности и влечении человеческой природы, находят свое удовлетворение в границах человеческого рода, в ходе человеческой истории. Заключение о религиозном, или теологическом, потустороннем мире, о будущей жизни, в целях усовершенствования людей находило бы поэтому себе оправдание лишь в том случае, если бы человечество оставалось постоянно на том же месте, если бы не было истории, совершенствования, улучшения человеческого рода на земле, хотя и в этом случае подобное умозаключение, если бы даже и имело свое оправдание, не было бы все же на этом основании еще истинным. 

Существует, однако, культурная история человечества: ведь даже животные и растения с течением времени видоизменяются и культивируются до такой степени, что мы не можем даже найти и установить их прародителей в природе! Бесчисленное множество вещей, которых не умели и не знали наши предшественники, мы умеем и знаем в настоящее время. Коперник, на пример которого я уже ссылался, исходя из точки зрения антропологии, в моем <<Вопросе о бессмертии>>  но который я не могу не повторить тут, ибо считаю его весьма поучительным, --- Коперник еще на смертном одре скорбел о том, что ему в течение всей его жизни не пришлось ни разу увидеть планеты Меркурия, как он того ни желал и каких усилий к этому ни делал. Теперь астрономы видят в ясный день эту планету в свои превосходные телескопы. Так, в ходе истории, исполняются те желания человека, которые не являются воображаемыми, фантастическими. Так осуществится когда-нибудь и то, что является теперь только желанием, станет действительностью бесчисленное множество вещей, кажущихся невозможными исполненным предрассудков защитникам и сторонникам существующих верований и религиозных учреждений, существующих социальных и политических отношений; бесчисленное множество вещей, которых мы теперь не знаем, но узнать хотим, узнают наши потомки. Поэтому на место божества, в котором находят свое осуществление беспочвенные и безмерные желания человека, нам надлежит поставить человеческий род или человеческую природу, на место религии --- образование, на место потустороннего мира над нашей могилой где-то там, в небесах, --- потусторонний мир над нашей могилой тут, на земле, историческую будущность, грядущее человечества. 

Христианство поставило себе целью осуществление неосуществимых желаний человека, но именно поэтому оно оставило без внимания достижимые человеческие желания; посулив людям вечную жизнь, оно лишило их жизни временной; внушив доверие к помощи бога, оно лишило их доверия к собственным силам; внушив веру в лучшую жизнь на небе, оно отняло веру в лучшую жизнь на земле и стремление осуществить подобную жизнь. Христианство дало человеку то, чего он желает в своем воображении, но именно поэтому не дало ему того, в чем он по-настоящему, в действительности нуждается и чего он желает. В своем воображении он стремится к небесному, преизбыточному счастью, в действительности --- к земному, умеренному. К земному счастью принадлежат, разумеется, не богатство, роскошь, пышность, великолепие, блеск и прочая мишура, но только самое необходимое, только то, без чего человек не может по-человечески существовать. Но какое бесчисленное множество людей нуждается в самом необходимом! На этом основании христиане признают святотатственным или бесчеловечным отрицать потусторонний мир и тем лишать несчастных, горемычных этой земли единственного утешения, надежды на лучший мир. Именно в этом видят они еще и ныне нравственное значение потустороннего мира, его тождественность с божеством; ибо без потустороннего мира нет воздаяния, нет справедливости, которая, по крайней мере, вознаградила бы на небесах за их горести тех, которые здесь страдают и несчастны. Однако этот аргумент в защиту потустороннего мира есть лишь предлог, потому что отсюда вытекали бы потусторонний мир, бессмертие только для несчастных, но не для тех, которые уже на земле были столь счастливы, что нашли необходимые средства для удовлетворения и развития своих человеческих потребностей и наклонностей. Для последних, из указанного соображения, по необходимости вытекает лишь то, что они либо после смерти перестают существовать, потому что достигли цели человеческих желаний, или что им на том свете будет хуже, чем на этом, что они на небесах займут то место, которое занимали когда-то на земле их братья. Так, камчадалы и в самом деле верят, что те, которые здесь были бедны, на том свете будут богаты, богатые же, наоборот, будут бедны, дабы установилось известное равновесие между состоянием на том и на этом свете. Но этого не хотят и в это не верят те господа из христиан, которые вышеуказанными соображениями защищают потусторонний мир; они хотят там так же хорошо жить, как и несчастные, как и бедняки. 

С этим соображением в защиту потустороннего мира обстоит так же, как и с тем соображением в защиту веры в бога, которое у многих ученых на устах, говорящих, что хотя атеизм и верен, и хотя сами они атеисты, но атеизм есть лишь дело ученых господ, а не людей вообще, что он не есть принадлежность толпы вообще, народа; поэтому неудобно, непрактично и даже святотатственно открыто проповедовать атеизм. Однако господа, так говорящие, прячут за неопределенными, имеющими различные значения словами <<народ>> или <<толпа>> свою собственную нерешительность, неясность и неуверенность; народ для них лишь предлог. То, в чем человек истинно убежден, того он не только не боится высказать, но и должен высказать открыто. То, что не имеет мужества выступить на свет общественности, то не имеет и силы вынести света. Атеизм, страдающий светобоязнью, есть поэтому недостойный, пустой атеизм. Он не доверяет себе, поэтому ему нечего сказать, нечего высказать. Тот, кто является атеистом лишь приватно или скрытно атеистом, тот говорят или думает лишь про себя, что нет бога, его атеизм выражается лишь в этом отрицательном положении, и это положение сверх того стоит у него одиноко, так что, несмотря на его атеизм, у него все остается по-старому. И, на самом деле, если бы атеизм был не чем иным, как отрицанием, простым непризнанием без содержания, то он бы не годился для народа, то есть не годился бы для людей, для общественной жизни; да и внутренняя ценность его была бы ничтожна. Однако атеизм истинный, не боящийся света, вместе с тем и положителен; атеизм отрицает существо, отвлеченное от человека, которое называется богом, чтобы на его место поставить в качестве истинного действительное существо человека. 

Теизм, вера в бога, напротив того, отрицателен; он отрицает природу, мир и человечество: перед богом мир а человек ничто, бог есть и был прежде, чем существовали мир и люди; он может быть без них, он есть олицетворенное ничто мира и человека; бог может --- так, по крайней мере, думает строго придерживающийся веры в бога --- каждый миг превратить мир в ничто; для истинного теиста нет власти и красоты природы, нет добродетели человека; все отнимает верующий в бога человек у человека и природы, чтобы только своего бога разукрасить и прославить. <<Только бога следует любить, говорит, например, св. Августин, --- весь же этот мир, то есть все чувственное, надлежит презирать>>. <<Бог, --- говорит Лютер в одном латинском письме, --- хочет быть единственным другом или никаким>>. <<К богу одному, --- говорит он в другом письме, --- подобает обращать веру, надежду, любовь, поэтому и называются они теологическими добродетелями>>. Поэтому теизм <<отрицателен и разрушителен>>; только на ничтожестве мира и человека, то есть действительного человека, строит он свою веру. Но ведь бог не что иное, как отвлеченное, фантастическое, превращенное воображением в самостоятельное существо человека и природы; теизм приносит поэтому в жертву действительную жизнь и существо вещей и людей простому существу, имеющему свое бытие лишь в мыслях и фантазии. Атеизм, наоборот, приносит в жертву действительной жизни и действительному существу мысленное, фантастическое существо. Атеизм поэтому положителен, утвердителен; он возвращает природе и человечеству то значение, то достоинство, которое отнял у них теизм; он оживляет природу и человечество, из которых теизм высосал лучшие силы. Бог ревнует к природе, к человеку, как мы раньше уже видели; он один хочет быть почитаем, любим, он хочет, чтобы ему одному служили; он один хочет быть чем-то, все остальное должно быть ничем, то есть теизм завистлив по отношению к человеку и миру; он не предоставляет им ничего хорошего. Зависть, недоброжелательство, ревность разрушительные, отрицающие страсти. Атеизм же либерален, щедр, свободомыслящ; он предоставляет каждому существу иметь свою волю, свой талант; он от сердца радуется красоте природы и добродетели человека: радость, любовь не разрушают, а оживляют, утверждают. 

Но точно так же, как с атеизмом, обстоит дело и с неразрывно с ним связанным упразднением потустороннего мира. Если бы это упразднение было всего только пустым, бессодержательным и безрезультатным отрицанием, то было бы лучше или, по крайней мере, безразлично, поддерживать ли его или от него отказаться. Но отрицание того света имеет своим следствием утверждение этого; упразднение лучшей жизни на небесах заключает в себе требование: необходимо должно стать лучше на земле; оно превращает лучшее будущее из предмета праздной, бездейственной веры в предмет обязанности, в предмет человеческой самодеятельности. Конечно, вопиюще несправедливо, что в то время как у одних людей есть все, у других нет ничего; в то время как одни утопают во всех наслаждениях жизни, искусства и науки, другие лишены самого необходимого. Однако неразумно на этом основании базировать необходимость другой жизни, где люди получают вознаграждение за страдания и лишения на земле, так же неразумно, как если бы я из недостатков тайной юстиции, которая до сих пор у нас существовала, вывел заключение о необходимости публичного и устного судопроизводства только на небесах. Необходимым выводом из существующих несправедливостей и бедствий человеческой жизни является единственно лишь стремление их устранить, а отнюдь не вера в потусторонний мир, вера, которая складывает руки на груди и предоставляет злу беспрепятственно существовать. Но на это можно возразить, что бедствия нашего гражданского и политического мира могут быть устранены; однако что из того тем, которые от этих бедствий пострадали и уже умерли? Что отошедшим вообще до лучшего будущего? Им, разумеется, нет пользы от него, но они не имеют пользы и от потустороннего мира. Потусторонний мир всегда приходит со своими снадобьями слишком поздно; он ценит бедствия после того, как все уже прошло, лишь со смертью или после смерти, следовательно, тогда, когда человек уже не имеет чувства страдания, а следовательно, и потребности исцеления; ибо смерть имеет для нас то плохое, --- по крайней мере, до тех пор, пока живем и ее себе представляем, --- что она у нас отнимает вместе с жизнью и ощущение, сознание хорошего, прекрасного, приятного, но зато и то хорошее, что она вместе с ощущением, сознанием избавляет нас от всех бедствий, страданий и горестных чувств. Любовь, которую создал потусторонний мир, которая страждущих утешает тем светом, --- есть любовь, которая исцеляет больного после того, как он умер, утоляет жаждущего после того, как он погиб от отсутствия питья, насыщает голодного после того, как он умер от голода. 

Оставим поэтому мертвых почивать в мире! Будем в этом отношении следовать примеру язычников! <<Язычники, --- говорю я в своем <<Вопросе о бессмертии>>  --- напутствовали своих мертвых в могилу словами: <<Пусть мирно почиют твои кости!>> --- или: <<Спи с миром!>>. Тогда как христиане, как рационалисты, кричат в уши умирающему веселое <<vivas et crescas in infinitum (живи и преуспевай во веки веков)>> --- или в качестве поэтических врачевателей душ как доктор Энзенбарт вколачивают ему вместо страха смерти страх перед богом, как залог его небесного блаженства. Оставим, таким образом, мертвых и позаботимся лишь о живых! Если мы в лучшую жизнь больше не верим, но ее хотим, --- хотим не в одиночку, а соединенными силами, --- то мы и создадим лучшую жизнь, то мы и устраним, по крайней мере, самые грубые, самые вопиющие и терзающие несправедливости и бедствия, от которых до сих пор страдало человечество. Но чтобы этого хотеть и это осуществить, мы должны на место любви к богу поставить любовь к человеку, как единственную истинную религию, на место веры в бога --- веру человека в самого себя, в свою собственную силу, веру в то, что судьба человечества зависит не от существа, вне его или над ним стоящего, а от него самого, что единственным дьяволом человека является человек грубый, суеверный, своекорыстный, злой, но также единственным богом человека является человек. 

Этими словами, господа, я заключаю свои лекции и желаю лишь, чтобы мне удалась та задача, которую я себе поставил в этих лекциях и которую я изложил еще в самом начале курса, а именно --- превратить вас из друзей бога в друзей человека, из верующих --- в мыслителей, из молельщиков --- в работников, из кандидатов потустороннего мира --- в исследователей этого мира, из христиан, которые, согласно их собственному признанию и сознанию, являются <<наполовину животными, наполовину ангелами>>  --- в людей, в цельных людей. 

\phantomsection
\addcontentsline{toc}{section}{Приложения и примечания Людвига Фейербаха}
\section*{Приложения и примечания Людвига Фейербаха}

\phantomsection
\addcontentsline{toc}{subsection}{К лекции четвертой}
\subsection*{К лекции четвертой}

\hypertarget{1}{(1)} \hyperlink{b1}{Объясняя религию из страха, следует, как я указываю в одной из позднейших лекций, иметь в виду не один лишь низший вид страха, страх перед тем или иным явлением природы, страх, который начинается и оканчивается вместе с морскою бурей, грозой, землетрясением, стало быть, не временный и местный страх, а тот --- не ограниченный каким-либо определенным предметом, охватывающий в своем представлении все, какие только возможны, несчастные случаи, вездесущий, непрестанный, то есть бесконечный, страх человеческого духа. <<Человек, --- говорит Лютер в своем утешающем послании курфюрсту Фридриху от 1520 г., --- легче, спокойнее перенесет все настоящие бедствия и злые явления как менее значительные, если он обратит свой дух к будущим бедствиям или злым явлениям, которых так много и которые столь велики, что против них дано лишь одно великое движение духа, принадлежащее к числу важнейших, именуемое страхом\dots Так и святой Павел говорит римлянам: ты не должен быть самомнителен, но страшиться или пребывать в страхе. И бедствие это тем значительнее, чем менее известно, в какой мере и как оно будет велико\dots Так что каждое настоящее бедствие или обременение есть не что иное, как напоминание о великой милости, которой бог нас почтил, не давая нам погибнуть под великим изобилием бедствий, отягощений и неприятностей, в которых мы находимся. Ибо что это за чудо, когда кто-нибудь подвергается бесконечным и бесчисленным ударам и когда тот же человек, в конце концов, поражается одним ударом? Ведь это милость, что не все удары попали в цель>>. <<Каких только бесчисленных несчастий, --- говорит Августин в ,,Граде божьем``, ---  не надлежит человеку опасаться извне: от жара и холода, бурь, ливней, наводнений, метеоров, зарниц, грома, грозы, молнии, землетрясения и обвалов, от толчков и испуга или злобы рабочего скота, от многочисленных ядовитых трав, от вод, воздушных течений и животных, от опасных или даже смертельных укусов хищных зверей, от бешеных собак? Какие беды приходится испытывать во время морского путешествия, какие беды на сухопутье? Можно ли сделать хоть один шаг, не подвергаясь неожиданным несчастным случаям? Человек идет с базара домой, он падает, имея здоровые ноги, ломает себе ногу и умирает от этой раны. Что может быть безопаснее, чем сидеть на месте? И все же священник Элий упал со стула и от этого умер>>. <<Бесчисленны, --- говорит Кальвин в своем ,,Установлении христианской религии``,  --- те бедствия, которые окружают человека и угрожают ему несчастными смертными случаями. Взойди на корабль --- ты окажешься лишь на шаг от смерти. Сядь на лошадь, --- она оступилась, и твоя жизнь в опасности. Пройдись по улицам города, --- сколько черепиц на крышах, стольким же смертельным опасностям ты подвергаешь себя. Возьми нож в руки --- перед тобой явная смерть. Взгляни на диких зверей --- они все тебе на погибель снабжены оружием. Что может быть, следовательно, более жалким, чем жизнь человеческая?>>. И христианский поэт г-н Д. В. Триллер в своих <<Поэтических размышлениях, противопоставленных размышлениям атеистов и натуралистов>> высказывается по этому поводу следующим образом: 
}

\begin{quote}

Куда б ни обратил ты взоры, 

Повсюду смерть свои дозоры 

Расставила --- о человек! 

Чтоб сократить земной твой век. 

Огонь, и град, и ветр, и воды, 

Стихии грозные природы, 

Свинец и порох, дым и яд: 

Все гибелью тебе грозят. 

Крючок и сталь, топор и стрелы, 

Колеса, нож и плод незрелый, 

Смолу, известку, грязь, песок 

Использовать все может рок. 

Яйцо иль косточка от сливы, 

Кусочек яблока червивый 

Все --- смерти на руку; ей грош, 

Чтоб погубить тебя, хорош. 

Кирпич, упавший ненароком, 

Использован быть может роком: 

Щиты ничтожных черепах 

Тебя повергнуть могут в прах. 

И нет таких на свете тварей 

В сравненьи с нами жалких парий 

Чтоб им при случае злой рок 

Нас уничтожить не помог. 

И черви --- пасынки природы 

Нам часто сокращают годы; 

К нам через уши или рот 

Червяк легко находит вход. 

Итак, смирись, сын праха бренный, 

И помни, что во всей вселенной 

Тебя повсюду и всегда 

Готова подстеречь беда. 

Нам в бытие путей немного: 

К нему одна ведет дорога; 

Когда же к смерти нам идти, 

Тогда бесчисленны пути. 

\end{quote}

\hyperlink{b1}{Однако довольно с нас духовного триллера, хотя он и далеко не кончился! Но каким же образом из человеческого страха рождаются боги? Различным путем. Если, например, человек менее чувствителен к добру, чем к злу, если он слишком легкомыслен или безрассуден, чтобы обращать внимание на жизненное благо, то у него имеются лишь злые боги; если представление о зле и ощущение его находятся в равновесии с представлением о добре и с его ощущением, то у человека одинаково имеются как добрые, так и злые боги; если же представление о добре и ощущение его преобладают над представлением и ощущением зла, то его бог --- бог добрый, побеждающий власть злого. Или: страх есть либо положительное, либо отрицательное основание религии или божества; то есть религия происходит либо из раболепства, либо из оппозиции к страху. В первом случае создаются страшные, злые боги, во втором --- добрые. Страх есть зло, и отношение к злу либо страдательное, либо активное; либо я предоставляю ему действовать, отдаюсь ему, хотя и против воли, либо сопротивляюсь, на него реагирую. Так из реакции против страха перед бесчисленным множеством возможных бед и смертельных опасностей, постоянно представляющихся неясно фантазии боязливого ума человека в качестве злых духов, возникает представление о бесконечно благом существе, о всемогущей любви, которая столь же много может сделать добра, как страх --- сделать зла, которая может предохранить от всяких бед, и в воображении действительно предохраняет. Именно отсюда, из этого вездесущего страха, и проистекает то, что политеистическая вера или суеверие населяет всякое место, всякий угол, всякую точку в пространстве богами-охранителями или охранителями-духами. Так, например, Пруденций говорит, полемизируя против Симмаха: <<Вы имеете обыкновение врата, дома, термы, конюшни наделять особыми гениями, вы присочиняете многие тысячи гениев ко всем площадям и частям города, чтобы каждый угол имел свое привидение>>. Если поэтому ученые господа, несмотря на бесчисленные алтари, воздвигнутые людьми в честь страха, все же не считают его божеством и притом первым божеством, то происходит это лишь потому, что они из-за деревьев не видят леса. Дело в том, что божественная любовь не простирается дальше, чем человеческий страх, ибо она может делать добро лишь в тех пределах, в каких страх творит зло; вечно небо любви, но вечен и ад страха; бесчисленны толпы ангелов, которых посылает в мир любовь, но бесчисленны также и толпы дьяволов, которых посылает страх; любовь доходит до самого начала мира, страх же --- до самого конца его; любовь создала первый день мира, страх же его последний, судный день. Короче говоря, там, где творческое всемогущество человеческого страха прекращается, там прекращается и всемогущество божественной любви. Близкий нам пример происхождения религии из страха и реакции против него мы имели в происхождении протестантизма, а именно: лютеранства, которое возникло из ужаса, из страха перед бесчеловечным, гневным, ревнивым богом, который в Ветхом завете сам называется ужасом или страхом Израиля, который, не считаясь с человеческой природой и не питая к ней какого-либо чувства, требует от человека, чтобы тот был подобен ему, то есть чтобы человек был не человеком, живым существом, а моральным привидением, воплощенным законом. Но Лютер был, несмотря на свое первоначальное монашеское и священническое звание, слишком практичной и чувственно крепкой натурой, чтобы он мог молитвами, постом, самоумерщвлением принести себя в жертву тому богу, чье одно уже имя --- Шаддай --- ведет свое происхождение от опустошения, истребления. Лютер хотел быть не ангелом, а человеком; он был в теологии теологом, борющимся против теологии; он желал действительного средства против злого существа теологии, которая, под предлогом примирения с богом, приводит человека в противоречие с его собственным существом, которая отравляет человеку кровь в сердце желчью божественной ревности, которая сжигает в голове его мозг адским огнем божественного гнева, которая уже за одно простое стремление человека быть человеком осуждает его на вечную смерть. Враждебное человеку злое существо христианской теологии с классической резкостью нашло свое выражение в особенности в Кальвине: <<Все вожделения плоти, --- как будто и вожделение вечной жизни не есть плотское вожделение, --- являются грехами>>; <<всякий грех есть смертный грех>>; <<закон, --- говорит Павел, --- духовен; этим он указывает, что закон требует не только послушания души, духа, воли, но и ангельской чистоты, которая, будучи свободна от всякой плотской грязи, стремится лишь к духу>>. Какая дьявольская бессмыслица под ангельской маскою! Но так как он средства против устрашающих образов религии или теологии искал в самой теологии или религии, то есть искал средства против злого существа, бесчеловечного бога в человечном боге, подобно тому, как человек, исповедующий естественную религию, в человеческой природе ищет средства против природы бесчеловечной, --- тунгус, например, в религиозной человеческой эпидемии ищет целительного средства против природной нечеловеческой эпидемии, --- то понятно само собой, что лечение не было и не могло быть радикальным. Это доказывают письма Лютера, представляющие большой психологический интерес, потому что они показывают нам различие между общественной личностью Лютера и его частной личностью, между силою веры на кафедре и силою или, вернее, бессилием ее у домашнего очага, --- показывают, как мало он на собственной личности испытал влияния веры, внушаемой им другим, как приносящей блаженство, как постоянно его преследовали устрашающие образы его собственного религиозного воображения. К счастью, Лютер, несмотря на свои теологические предрассудки, находил еще рядом и вне религии или теологии целительные средства против силы греха, ада, дьявола, или, что то же, гнева божия. Так, в одном латинском письме к Л. Зенфелю он пишет, что и музыка дает человеку то, что, вообще говоря, дает человеку лишь теология, а именно: веселие и спокойствие, что дьявол, виновник всех забот и нарушений мира, почти так же бежит от голоса музыки, как и от слова теологии. И даже в одном письме к Г. Веллеру он пишет, что порой следует пить, играть, шутить и даже грешить, наперекор дьяволу и в насмешку над ним, чтобы не давать ему повода для упреков совести по поводу мелочей. Воистину, хотя и весьма антитеологическое, но именно поэтому в высокой степени испытанное антропологическое целительное средство!}

\phantomsection
\addcontentsline{toc}{subsection}{К лекции пятой}
\subsection*{К лекции пятой}

\hypertarget{2}{(2)} \hyperlink{b2}{Есть ли чувство зависимости или сознание зависимости, и то и другое неразделимо в человеке, <<чего я не знаю, то оставляет меня холодным>>  --- верное универсальное понятие или выражение для субъективного, то есть человеческого (и притом практического, а не теоретического) основания религии? Под этим номером я собрал ряд вопросов, которые составляют элементы или фрагменты самостоятельного сочинения, которое, однако, я при ненадежности всех предприятий в результате нашей ужасной, безотрадной политики тотчас же присоединил к этим лекциям, и поэтому прошу благосклонного читателя ознакомиться с ними лишь после окончания лекций. Хотя я привел уже достаточно доказательств для утвердительного ответа на этот вопрос, я все же хочу привести еще несколько, но уже из мира классических язычников, а не христиан, и не только потому, что у христиан зависимость <<твари>> от <<независимой причины>> сделалась даже техническим термином их теологии и метафизики, но также и потому, что древние классические народы в противоположность христианам не подавляли и не скрывали первоначальных естественных чувств и настроений человека, --- ибо тезис Плиния: у греков природа обнажена, применим и здесь, --- не жертвовали ими в угоду условному, догматическому понятию бога, и потому явили нам как в политике, таи и в религии поучительнейшие интереснейшие примеры того, как возникало понятие бога. <<Все люди, --- говорит Гомер в Одиссее, --- имеют потребность в богах>>. Но что такое потребность, как не патологическое выражение зависимости? По этому случаю я должен заметить, что источник противоположности между человеческим и божественным как в <<Сущности веры>>  так и в <<Сущности христианства>> и источник чувства зависимости в <<Сущности религии>> сводятся к одному с тою лишь разницею, что первая противоположность обязана своим существованием больше рефлексии, размышлению над чувством зависимости. Если люди нуждаются в богах, то это ведь необходимое следствие того, что боги имеют то, чего недостает людям, что, стало быть, отсутствие потребностей у божества составляет противоположность человеческой нужде, противоположность, которую позднейшая греческая мысль или философия и выразила определенным образом, хотя уже и у Гомера божественное существо, как эфирное, блаженное, бессмертное, всемогущее, противополагается обремененному, жалкому, смертному, немощному существу человека, но, разумеется, противополагается на чрезвычайно добродушный или поэтический лад, так что противоположность между бескровными богами и полнокровными людьми уничтожается в прозрачной влаге, текущей в жилах богов. Однако вернемся опять к Одиссее. <<От бога, --- говорит Гомер, --- идет разное разным, добро и зло идет от Зевса, ибо он царит со всемогуществом>> (дословно: ибо он все может). <<Невозможно, чтобы смертные оставались постоянно без сна, ибо боги предписали людям мору и цель каждой вещи>>. Зависимость человека ото сна, необходимость сна есть, стало быть, Мойра --- божественный рок или судьба. И сам сон есть божественное существо, <<властитель смертных людей и бессмертных богов>>. <<Так меняется понимание смертных обитателей земли по мере того, как отец, который господствует, приносит иные дни>>. В счастливые дни он заносчив, в несчастные --- малодушен, но эти дни зависят от отца богов и людей. <<Там, на небесах, --- говорится в Илиаде, --- исход битвы находится в руках бессмертных богов>>. Когда Одиссей и Аякс состязались в беге и были уже близки к цели, то Афина-Паллада, по просьбе Одиссея, поставила препятствие на пути Аякса: он споткнулся о бычачий помет, и Одиссей выиграл первый приз. Таким образом, победит ли человек или окажется побежденным, достигнет ли он беспрепятственно цели или по дороге поскользнется, это зависит от богов. <<Если, --- говорит Гезиод, --- корабль пустится в путь в надлежащее время, то корабль у тебя не будет изломан, и море не уничтожит людей, если только колебатель земли, Посейдон, или Зевс, царь бессмертных, не предрешили заранее гибели, ибо в их власти как добро, так и зло>>. <<От тебя, досточтимая! --- говорится в гомеровском гимне в честь праматери Земли, --- от тебя исходит изобилие детей и изобилие плодов, от тебя зависит, дать жизнь или взять ее у смертных людей; счастлив тот, кого ты благосклонно чтишь в своем сердце, ибо у него все имеется в избытке>>. <<Молись богам, --- говорит Феогнид, --- ибо велика их сила и ничто не случается без участия богов, ни доброе, ни злое>>. <<Суетны наши мысли; мы, люди, ничего не знаем, всем руководят, согласно своему пониманию, боги>>. <<Никто не виновник того вреда и той пользы, которые он получает: то и другое дают боги. И никто из людей не поступает, предвидя в своем уме, какой будет исход, хороший или дурной>>; но если все зависит от богов, хорошее и дурное, жизнь и смерть, здоровье и болезнь, счастье и несчастье, богатство и бедность, победа и поражение, то очевидно же, что чувство зависимости есть основа религии основа того, что человек превращает свою деятельность в страдание, свои желания, намерения --- в молитвы, свои добродетели --- в дары, свои недостатки в наказания, короче говоря --- превращает свое благополучие из предмета своей самодеятельности в предмет религии. Но приведем еще более специальные доказательства. <<Все люди нуждаются в богах, но не все нуждаются во всех богах>>  --- говорит Плутарх. <<В качестве крестьянина, например, я взываю, говорит Варрон в своем сочинении о сельском хозяйстве, --- не как Гомер и Энний, к музам, а к двенадцати важнейшим богам, но не к городским, чьи позолоченные статуи находятся на форуме, а к тем двенадцати богам, которые являются главным образом вождями (или господами) крестьян, стало быть, прежде всего к Юпитеру и Земле, ибо небо и земля включают в себя все плоды земледелия, во-вторых, я взываю к солнцу и луне, с чьим движением приходится сообразоваться, когда что-нибудь сеешь и сажаешь в Землю; далее, к Церере и Вакху, потому что их плоды являются в числе необходимейших для поддержания жизни, ибо от них идут ведь еда и питье, затем к палящему зною и флоре, потому что если они благоприятны, то не гибнут от жара хлеб и деревья и зацветают в надлежащее время; далее, я почитаю также Минерву и Венеру, ибо в ведении одной находится оливковое дерево, а в ведении другой --- сады. Наконец, молюсь я и воде и удаче, ибо без воды обработка земли суха и жалка, без успеха же и хорошего исхода лишь даром затрачиваются усилия. Как пастух или скотовод я обращаюсь в особенности к божеству Палее и прошу его, как это значится в овидиевых Фастах, чтобы оно прогнало болезни, сохранило здоровыми людей, стада и собак, не допускало бы голод, дало бы зелень и овощи, воду для питья и купанья, молоко и сыр, и ягнят, и шерсть; как купец же я обращаюсь к Меркурию и прошу его о прибыли в торговле>>. Таким образом, люди нуждаются в богах, но лишь в тех, от которых именно зависит их существование, --- все равно, в естественном или гражданском мире, и именно эта нужда, эта зависимость их существования, их судьбы от богов есть основа религии, основание, почему они рассматриваются и почитаются как боги. Поэтому первое, из практики, из жизни почерпнутое определение бога состоит в том, что бог есть то, в чем человек нуждается для своего существования и притом для своего физического существования, ибо это физическое существование есть ведь основа его существования духовного, что, стало быть, бог есть существо физическое; или, если субъективно выразиться, первый бог человека есть потребность и притом физическая потребность, ибо лишь от силы и власти, которые проявляет надо мною какая-либо потребность, зависит ведь то, что я почитаю как бога предмет, удовлетворяющий эту потребность. <<Мы имеем, --- говорит святой Августин в своем ,,Граде божьем``,  --- образ божественной троицы в нас самих; мы существуем и знаем, что существуем, и любим это наше бытие и знание; отсюда и разделение науки философами на естествознание, логику и этику, или мораль. Святой дух есть доброта, любовь или источник ее; второе лицо есть слово, разум или источник мудрости; первое лицо, бог-отец, есть бытие или творец бытия>>. То есть старейший первый бог --- бог, предшествующий моральному и духовному богу, есть физический бог: ибо как святой дух есть не что иное, как обожествленное существо морали, а бог-сын не что иное, как обожествленное существо логики, так и бог-отец не что иное, как обожествленная сущность физики, природы, и из нее одной человек вывел абстрактное понятие и выражение бытия. <<В силу известной естественной необходимости, --- говорит по этому случаю Августин, --- бытие уже как таковое (или голое бытие) приятно, так что из-за него одного несчастные не хотят погибнуть; ибо почему бы иначе им бояться смерти и предпочитать даже бедственную жизнь смерти, как не потому, что природа боится небытия? Отсюда проистекает и то, что и неразумные животные хотят быть и всеми возможными способами избегают гибели, что даже и бесчувственные растения и даже и совсем безжизненные тела стремятся сохранить и утвердить свое бытие>>. Таким образом, мы видим, что отвлеченное понятие бытия имеет лишь в природе плоть и кровь, истинность и действительность, что, следовательно, подобно тому, как бытие есть предпосылка мудрости и доброты, так же точно и физический бог предшествует духовному и моральному; мы видим в то же время, что связь любви, при помощи которой человек прикреплен к себе самому и к жизни, есть цепь, на которой держались все боги, что лишь потому Юпитер есть наивысший и самый могущественный бог, что потребность бытия, жизни есть наивысшая и самая могучая потребность человека, удовлетворение же этой потребности, то есть жизнь, в последнем счете зависит лишь от Юпитера; что, стало быть, то почтение, которое Юпитер внушает своим громовым шумом, есть лишь результат человеческой любви к жизни и боязни смерти. Таким образом, лишь из <<гневного огня>>  из тьмы человеческих вожделений, из хаоса человеческих потребностей могли возникнуть греческие и христианские боги. И как мог бы, в самом деле, человек провозглашать хлеб священным, как мог бы он прославить Цереру, как божественную благодетельницу, если бы он не испытывал голода как <<ужасного тирана>>? Нет, где нет дьявола, там нет и бога, где нет голода, нет и Цереры, где нет жажды, нет и Вакха. Нет поэтому ничего более забавного, чем когда ученые господа, --- в силу того, что для них религия именно древних народов представляет еще только теоретический или эстетический интерес, --- выводят и самую религию лишь из теоретических или идеальных мотивов, когда они из-за мифологических фигур и завитков, которыми воображение разукрасило геркулесовский щит религии, забывают, что ведь, несмотря на это художественное снаряжение и на роскошь, над которыми они и до сих пор ломают себе головы, щит преследует не иную какую-либо цель, как охрану жизни человека.} 

\hyperlink{b2}{Так как все зависит от богов, боги же являются субъективными, то есть личными, эгоистическими существами, --- существами, которые так же мыслят и чувствуют, как и человек: --- <<Я ревнивый бог>>  говорит Иегова в Ветхом завете; <<боги, --- говорит Венера у Еврипида, --- находят удовольствие в том, что их почитают люди>>; <<мы, --- говорят боги в овидиевых Фастах, --- народ честолюбивый>>; так как, стало быть, все зависит от милости или немилости, любви или гнева богов, то естественно, что они почитаются не только по мотивам человеческого, но также и божественного эгоизма; почитаются не только потому, что они делают человеку добро, но и потому, что они хотят быть почитаемы, короче говоря, они почитаются не только ради человека, но и ради самих себя. Субъективное или личное существо можно лишь тем почтить, что ему делают то, что отвечает его пониманию, что соответствует его существу, стало быть, устраняют все, что ему не нравится. В честь знатного гостя устраняют домашний сор и грязь, скорбь и печаль, ссоры и перекоры, убирают с глаз долой все то, что могло бы на него произвести неэстетическое, неприятное впечатление. Точно так же поступает человек и в праздничные дни, посвященные почитанию богов; он воздерживается от всяких занятий, поступков и удовольствий, противоречащих существу богов; он забывает собственные радости и горести, чтобы испытать радости и горести богов, как, например, в праздник Деметры. Но как раз это почитание богов в их духе и соответственно их интересам есть в то же время и почитание в духе человека и соответственно его интересам; ибо лишь этим целомудренным, бескорыстным почитанием приобретаю я их благоволение; но раз я имею их благоволение, я имею все, что мне желательно, я нахожусь у источника всех благ. Точно так же обстоит дело с укрощением гнева, с примирением богов с людьми. Безразлично поэтому, беру ли я их как средство или как цель, потому что раз устранен гнев бога, то устранено и всякое зло, раз уничтожена причина, то вместе с ней отпадает и следствие. <<Моя величайшая вина заключается в том, --- говорит Овидий в своих элегиях из Том, куда его сослал гнев земного Юпитера, императора Августа, --- что я его, императора Августа, оскорбил>>. <<Если бы даже, помимо гнева императора, меня и не удручало никакое зло, то не является ли гнев императора сам по себе достаточным злом?>>. <<Ведь немилость императора несет с собой всяческое зло>>. То же самое относится и к небесным богам. Утишить их гнев значит уничтожить источник всякого зла.} 

\hyperlink{b2}{Так как боги распоряжаются жизнью и смертью, счастьем и несчастьем, то с ними и с их почитанием связана мораль, теоретическое, и практическое различение между добром и злом, правдой и неправдой. Я говорю: связана, потому что сами по себе и первоначально религия и мораль, --- по крайней мере, в том смысле, в каком мы теперь эту мораль понимаем, --- не имеют между собой ничего общего, и притом по той простой и очевидной причине, что в морали человек устанавливает отношение к себе и к своим ближним, в религии же --- к другому, отличному от человека существу. <<Все священное писание, --- говорит Боден в своей ,,Демономании``,  --- полно свидетельств, что бог питает величайшее отвращение к колдунам (то есть к тем, которые отказываются от бога и соединяются с дьяволом), ибо они гораздо более заслуживают проклятия, чем отцеубийцы, кровосмесители и содомиты>>. <<Если бы волшебник, --- говорит он далее, --- и не причинил никакого вреда, не сделал ничего злого людям и скоту, то все же он заслуживал бы того, чтобы быть заживо сожженным, уже потому, что он отказался от бога, соединился с дьяволом и, следовательно, оскорбил величество бога>>. <<Намерение убить, говорит Лютер, --- не такой большой грех, как неверие, ибо причинение смерти есть грех против пятой заповеди, неверие же --- грех против первой и величайшей заповеди>>. <<Известно, --- говорит Кальвин, --- что в законе и пророках вера и то, что относится к богослужению, занимают первое место, любовь же поставлена ниже веры>>. Католическая церковь совершенно определенно отвергла, как еретическое, положение, гласящее, что тот не христианин, кто имеет веру без любви, и, стало быть, санкционировала то положение, что можно исповедывать христианство, иметь веру, религию --- без любви, то есть без морали. А благочестивый русский, эта последняя надежда наших отъявленных религиозных и политических абсолютистов, так строго блюдет свой пост, что скорее простит себе совершение кражи или убийства, чем нарушение поста (Штейдлин, Сборник по истории религий). <<Армянские священники также скорее отпускают грех совершения убийства и какого-либо другого грубого преступления, чем нарушение поста. Самые дурные люди из числа греков-христиан соблюдают посты не менее строго, чем и самые добродетельные>> (Мейнерс, цитируемое сочинение). Криминалист Карпцов был так набожен, был таким библейским человеком, таким христианином, что каждый месяц ходил к святому причастию и не меньше 53 раз --- подумайте: пятидесяти трех --- прочел Библию и тем не менее, или, быть может, именно поэтому, этот благочестивый человек приговорил к смерти не меньше 20.000 подумайте: двадцати тысяч --- злоумышленников, то есть бедных грешников (Штейн, История уголовного права). <<Коннетабль Анн де Монморанси\dots быть может, единственный глава католической партии, любивший религию ради нее самой\dots Если верить Брантому, он отдавал распоряжения пытать, убивать, поджигать, читая свой молитвенник и нисколько не обрывая своих молитв, до такой степени он был благочестив>> (Всеобщий словарь Ролине, ст. Лига). О том противоречии между моральным достоинством и духовным, между гуманностью и религиозностью, между нравственностью и церковностью, которое являют собою в жизни наши протестантские и католические духовные лица, умалчиваю потому, что считаю ненужным и недостойным писать о вещах, поражающих даже притупленное внимание наших крестьян. Что же общего у веры с любовью, у религии с моралью? Ничего или так же мало, как мало общего друг с другом у бога, к которому имеет отношение вера, и у человека, к которому имеет отношение любовь; ибо, согласно вере, человек и бог самым резким образом противопоставлены один другому: бог есть нечувственное, человек --- чувственное существо; бог существо совершенное, человек --- существо несчастное, жалкое, ничего не стоящее. Как же, таким образом, может из веры вытекать любовь? Так же мало может вытекать, как из совершенства --- жалкое состояние, из изобилия недостаток. Больше того! Мораль и религия, вера и любовь прямо противоречат друг другу. Тот, кто любит бога, не может более любить человека, он потерял понимание человеческого; но и наоборот: если кто любит человека, поистине от всего сердца любит, тот не может более любить бога, тот не может дать понапрасну испаряться горячей человеческой крови в пустом пространстве бесконечной беспредметности и недействительности. Религия, говорят, предохраняет от грехов своим представлением о всеведущем существе; однако уже древние говорили, что нужно так молиться богу, как если бы молитву слушали люди, и что <<тот, кто не боится людей, тот и бога самого обманывает>>; религия, говорят, наказывает грешников; но у нее также достаточно способов --- при помощи ли заслуг Христа, разрешительных грамот, коровьего навоза или воды для умывания --- освободить человека от греха или, по крайней мере, --- так как против самого греха вера совсем или почти бессильна, как в том признавались сами верующие, которые были честны, и как это доказывали их характер и жизнь, --- извинить грешника, домыть даже мавра до бела. Уже языческий поэт Овидий, правда, принадлежащий эпохе просвещения, а поэтому также и безверия, не может в своих Фастах, которые, впрочем, сами обязаны своим происхождением лишь антикварному вдохновению, удержаться от удивления тому, что его благочестивые предки могли верить, будто все проступки, даже такое ужасное преступление, как смертоубийство, могут быть погашены водою реки. Но как ни противоречат друг другу вера и любовь, религия и мораль, все же мораль не только имеет связь с религией, как я вначале уже указывал, но и действительно на нее опирается, по совсем, однако, другим основаниям, чем это обычно принимается. Религия всемогуща: она повелевает небом и землей, движением и остановкой солнца, громом и молнией, дождем и солнечным светом, --- словом, всем, что человек любит и чего он боится, счастьем и несчастьем, жизнью и смертью; она делает поэтому заповеди любви и морали предметами человеческого себялюбия, стремления к счастью, награждая исполнение этих заповедей возможными благами, невыполнение их наказывая самыми страшными бедами. <<Если ты не будешь повиноваться, --- говорит Иегова, --- голосу господа, твоего бога, и исполнять его заповеди и законы, которые я тебе ныне предписываю, то все эти проклятия падут на твою голову. Проклят будешь ты в городе, проклят и на поле и так далее Господь пошлет тебе неудачи, беды и несчастья во всем, за что ты только ни возьмешься, до тех пор, пока ты не будешь уничтожен. Господь поразит тебя опухолью, лихорадкою, жарой, пожаром, засухой, ядовитым воздухом и желтухой и будет преследовать тебя, доколе тебя не погубит. Господь поразит тебя египетскими желваками, наростами, паршами и чесоткой так, чтобы тебе не оправиться. Господь поразит тебя слепотой, безумием и бешенством сердца, и ты будешь ходить ощупью в полдень, как ходит ощупью слепец, и на своем пути не найдешь счастья>> и так далее. <<Видишь, я только что показал тебе жизнь и добро, смерть и зло, дабы ты господа бога твоего любил и шел его путями, и соблюдал его заповеди, законы и веления, и мог жить, и был приумножен. Господь бог твой даст тебе счастье во всех делах твоих рук, в плодах твоего тела, в приплоде твоего скота, в урожае страны твоей, чтобы все это шло тебе на пользу>>. Мы из этого классического места видим, каким образом религия делает любовь к добродетели любовью к долгой и счастливой жизни, страх перед нарушением велений морали * --- страхом перед египетскими желваками, наростами, паршой и чесоткой, короче говоря, перед всевозможными бедами и несчастиями, мы видим утверждение, что мораль опирается или должна опираться на религию --- не имеет другого смысла, как тот, что мораль должна опираться на эгоизм, на любовь к себе, на стремление к счастью, иначе она не имеет основы. Различие между иудейством и христианством заключается лишь в том, что в иудействе мораль опирается на любовь к временной и земной жизни, в христианстве же --- к вечной, небесной. Причина, почему не сознают того, что тайна веры в отличие от любви, тайна религии в отличие от морали заключается лишь в эгоизме, лежит только в том, что религиозный эгоизм не имеет видимости эгоизма, что человек в религии в форме самоотрицания утверждает себя, что он выявляет свое <<Я>> не в виде первого лица, свою волю не в повелительной форме, а в просящей, не в активной, а в страдательной, что он не сам себя любит, а смиренно предоставляет себя любить. Так содержание лютеровской веры в отличие от любви или морали не что иное, как содержание себялюбия в страдательной форме, содержание, гласящее: бог любит меня или я любим богом; но так как я любим богом --- такова связь веры с моралью, --- то я люблю людей; так как мой эгоизм удовлетворяется в религии, то мне не нужно удовлетворять его в морали; то, что я трачу и теряю в морали, то я получаю обратно или уже сторицей имею в вере, в уверенности, что меня любит всемогущее существо, в распоряжении которого находятся все сокровища и блага. Однако вернемся к нашей цитате из Ветхого завета! Что принадлежит религии, что морали. * В этом месте идет речь не только о моральных велениях, но и о религиозных заповедях. Но так как здесь дело идет именно о различии между моралью и религией, то мы должны, разумеется, подчеркивать лишь первую, что богу, что человеку? К человеку относится запрещение убийства, запрещение нарушения брачного союза, запрещение воровства, запрещение лжесвидетельства, запрещение желать жену, дом, поля и так далее ближнего, --- потому что хотя запрещение воровства, например, представляется вору бесчеловечным и находится в величайшем противоречии с его эгоизмом, но зато оно находится в величайшем созвучии с эгоизмом собственника. Мораль и право покоятся вообще на совершенно простом положении: <<чего ты не хочешь, чтобы люди тебе делали, того не делай и ты им>>. Но ни один человек не пожелает, чтобы у него отняли его жизнь, его жену, его поле, его доброе имя; поэтому весьма естественно, что это воля каждого, --- ибо даже вор не хочет, чтобы украденное было у него, в свою очередь, украдено, и убийца не хочет, чтобы у него была взята жизнь, --- эта воля делается определенно всеобщим законом и что противящийся этому наказуется. Что же принадлежит богу или религии? С одной, стороны, египетские желваки, парши, чесотка, опухоли желез и другие беды, которые религия предназначает злым, с другой стороны --- долгая жизнь, многодетность человека, плодовитость скота, урожай поля, которые она сулит добрым, ибо как эти блага, так и эти беды не находятся во власти человека. Хотя боги являются моральными силами постольку, поскольку они наказывают неправду, грех, награждают правоту, добродетель, --- тем не менее то, что является им свойственным, что составляет их существо, есть не мораль, а только лишь сила, дающая им возможность наказывать и награждать. <<Бог требует от вас не только христианской веры, он требует также, чтобы вы буди добры, благожелательны к людям, исполнены любви к своему ближнему>>. Это --- неправда: бог требует от вас лишь веры, человек же требует, чтобы вы были добры, благожелательны к людям и исполнены к ним любви, ибо бог заинтересован только в вере, в морали же заинтересован человек. Чему ты веришь, это мне совершенно безразлично, но не безразлично, что ты собой представляешь, что ты делаешь. Для <<Я>> рубашка веры, разумеется, ближе, чем пиджак морали, но для <<Ты>> пиджак ближе, чем рубашка; для <<Ты>> существует даже только мой пиджак, но не моя рубашка. Но и те, и другие являются предметом стремления к счастью, одни --- утверждая, другие --- отрицая, одни --- как предметы любви, желания, другие --- как предметы страха, отвращения. Что же является специфически, характерно присущим религии? Только стремление к счастью, только эгоизм, и притом тот эгоизм, удовлетворение которого не находится во власти человека. Себе, своей жене, своему полю, своему скоту я желаю всяких благ, какие только можно измыслить; тому же, кто нападает на мою жену, на мой скот, на мою жизнь и наносит им ущерб, на голову тому я с проклятиями накликаю всевозможные беды, и тогда именно, когда он не находится в моей власти, --- а он не всегда находится в ней; но оба желания --- как благословение, так и проклятие --- осуществляет или может осуществить лишь всемогущество божие или всемогущество веры. Таким образом, религия --- благодаря тому, что она повелевает жизнью и смертью, небом и адом, что она превращает законы в заповеди всемогущего существа, то есть того понятия, которое включает в себя все человеческие желания и страхи, --- имеет своей основой эгоизм, благодаря чему она проявляет страшную власть над человеком, особенно первобытным,власть, перед которой сходит на нет власть морали, в особенности абстрактной, философской, и потеря ее представляется поэтому невознаградимой. Однако не следует упускать из виду, что религия проявляет эту власть только при помощи силы воображения или что ее власть заключается лишь в силе воображения; потому что, если бы ее власть была больше, чем воображаемая, если бы религия была и в самом деле положительной основой и опорой права и морали, то религиозных посулов и наказаний должно было бы быть достаточно для основания и сохранения государств, и людям не пришло бы в голову применять столь многочисленные, столь изощренные, столь жестокие наказания для предупреждения преступлений. Или так: можно принять положение, что религия есть основа государств, но при этом надо прибавить: лишь в воображении, в вере, в суждении, потому что в действительности государства, даже христианские, опираются не на силу религии, --- хотя они ее (или, вернее, веру, слабую сторону человека) и употребляют как средство для своих целей, а на силу штыков и других орудий пыток. В действительности люди вообще действуют по совсем другим мотивам, чем они это себе представляют в своем религиозном воображении. Благочестивый Ф. де-Коммин в своей хронике короля Людовика одиннадцатого говорит: <<Все злое или все проступки происходят от недостатка веры; если бы люди твердо верили в то, что бог и церковь говорят нам о вечных и страшных наказаниях ада, то они не делали бы того, что они делают>>. Но откуда же происходит эта слабость веры? Она происходит оттого, что сила веры есть не что иное, как сила воображения, и как ни сильна власть силы воображения, все же власть действительности есть власть бесконечно более значительная и прямо противоречащая существу воображения. Вера, как и сила воображения, гиперболична, она живет лишь крайностями, преувеличениями; она знает лишь о небе и аде, ангелах и дьяволах; она хочет сделать из человека нечто большее, чем он должен быть, и именно поэтому делает из него нечто меньшее, чем он мог бы быть; она хочет сделать его ангелом, и делает его за то при благоприятных обстоятельствах настоящим дьяволом. Так гиперболическое и фантастическое существо веры, под влиянием противодействия прозаической действительности, превращается в свою собственную противоположность! Поэтому плохо обстояло бы дело с человеческой жизнью, если бы право и мораль не имели других оснований, кроме религиозной веры, столь легко превращающейся в свою противоположность, ибо она, как в том признавались величайшие герои веры, является прямой насмешкой над свидетельством чувств, над естественным чувством и врожденным человеку влечением к неверию. Но как может явиться надежным и прочным основанием нечто вынужденное, построенное на насильственном подавлении вполне обоснованного влечения, в каждый данный момент подверженное сомнениям разума и противоречиям опыта? Вера в то, что государство, --- я, разумеется, имею в виду государство вообще, а не наше искусственное, сверхнатуральное государственное здание, --- что это государство не может существовать без религии, подобна вере в то, что натуральные ноги недостаточны, чтобы стоять и ходить, что человек может стоять и ходить только на ходулях. Но натуральные ноги, на которых базируются мораль и право, это --- любовь к жизни, интерес, эгоизм. Нет поэтому ничего более неосновательного, чем представление о том и страх перед тем, что вместе с богами уничтожается и различие между правдой и неправдой, между добром и злом. Это различие существует и будет существовать до тех пор, пока будет существовать различие между <<Я>> и <<Ты>>, ибо лишь это различие есть источник морали и права. Если мой эгоизм и позволяет мне воровство, то эгоизм другого человека ему самым решительным образом воспротивится; если я сам по себе ничего не знаю и знать не хочу о бескорыстии, своекорыстие других мне обязательно внушит добродетель бескорыстия; если мой мужской эгоизм имеет тенденцию к полигамии, то женский эгоизм всегда воспротивится этой тенденции и будет отстаивать моногамию; если я и не вижу и не чувствую бревна в моем глазу, то ведь каждая пылинка в нем будет как бельмо в глазах тех, кто жаждет меня осуждать; короче говоря, если мне и безразлично, хорош ли я или дурен, то это никогда не будет безразлично для эгоизма других. Ведь кто был до сих пор правителем государств? Бог? Ах! Боги управляют лишь на небесах фантазии, но не на грешной земле действительности. Кто же, стало быть? Только эгоизм, но, разумеется, не простодушный эгоизм, а эгоизм дуалистический, который для себя изобрел небо, а для других ад, для себя материализм, для других идеализм, для себя свободу, для других рабство, для себя наслаждения, для других отказ от них. Тот эгоизм, который в лице правительств наказывает своих подданных за преступления, ими же, правительствами, совершенные, в лице отцов наказывает их же собственные, ими же взращенные грехи детей, в лице мужей их же собственные, в которых они же, мужья, повинны, слабости жен; эгоизм, который вообще все прощает себе и дает проявиться своему <<Я>> по всем направлениям, но который от других требует, чтобы они не имели своего <<Я>>  чтобы они жили одним лишь воздухом, чтобы они были совершенны и нематериальны как ангелы; я повторяю, не тот ограниченный эгоизм, к которому одному обыкновенно прилагают это имя, но который является лишь одною, хотя и самой вульгарной его разновидностью, но тот эгоизм, который включает в себя столько же видов и родов, сколько вообще существует видов и родов человеческого существа, ибо имеется не только одиночный или индивидуальный эгоизм, но также и эгоизм социальный, эгоизм семейный, корпоративный, общинный, патриотический. Конечно, эгоизм есть причина всех зол, но также и причина всех благ, ибо кто иной, как не эгоизм, вызвал к жизни земледелие, торговлю, искусства и науки? Конечно, он --- причина всех пороков, но также и причина всех добродетелей, ибо кто создал добродетель честности? Эгоизм запрещением воровства; кто создал добродетель целомудрия? Эгоизм запрещением прелюбодеяния, эгоизм, который не желает делиться предметом своей любви с другими. Кто создал добродетель правдивости? Эгоизм запрещением лжи, эгоизм, не желающий быть оболганным и обманутым. Таким образом, эгоизм есть первый законодатель и виновник добродетелей, хотя бы только из вражды к пороку, только из эгоизма, только потому, что для него является злом то, что для меня является пороком, как и наоборот: что для меня есть отрицание моего эгоизма, то для другого есть утверждение, что для меня есть добродетель, то для него есть благодеяние. \emph{Впрочем, для сохранения государств, по крайней мере наших проклятых государств, враждебных природе и человеку, пороки так же, если даже еще не более, нужны, как людские добродетели.} Если бы, например, --- я привожу пример из близкой мне области, ибо я пишу на баварской почве, хотя, разумеется, не в баварском и уже, разумеется, не в прусском или австрийском духе, --- если бы христианство было у нас чем-то большим, чем простая церковная фраза, если бы дух христианского аскетизма и отрицания чувственности охватил баварский народ и этот народ отказался от употребления пива, хотя бы в неумеренных размерах, то что стало бы с баварским государством? Русское государство имеет ведь, даже несмотря на свое <<субстанциальное правоверие>>  свой главный финансовый источник жизни в таком яде, как водка. Таким образом, без пива нет Баварии, без водки нет России --- и нет даже Боруссии (Пруссии). И перед лицом этих и бесчисленных других столь же распространенных фактов осмеливаются втирать народу очки, что религия есть связующее звено наших государств, скрепленных лишь каторжными цепями преступления против природы человека! Оставим, однако, ужасы политики! Мораль, говорят нам, должна основываться на религии, на божественном, а не на человеческом существе, иначе потеряет она всякий авторитет и всякую прочность.} 

\hyperlink{b2}{Что может быть более относительного, изменчивого, ненадежного, чем существо человека? Как же может опираться на него моральный закон? Но не значит ли это попадать из огня да в полымя --- от человеческого существа искать прибежища в существе бога? Не есть ли существо человека, при всем бесконечном разнообразии его основных влечений, нечто себе равное, надежное, даже чувственно достоверное? Не говорится ли даже в пословице: <<весь свет имеет одно лишь желание, --- чтобы ему было хорошо>>. Но существует ли что-нибудь более неизвестное, сомнительное, противоречивое, колеблющееся, неопределенное, относительное, чем существо бога? Не является ли оно по крайней мере столь же переменчивым и различным, как переменчивы и различны времена и люди? Не есть ли причина того, почему бог в известное время дает людям те, а не иные законы, те, а не иные откровения, не есть ли причина этого существо этих людей, которым соответствуют только данные законы и откровения? Но не есть ли, --- в том случае, если законодатель дает мне закон, соответствующий моему существу (а только такой закон есть истинный и действительный), --- не есть ли мое существо закон закона, то, что является его предпосылкой? Каково, стало быть, различие между человеческим и божественным существом как основой морали? Это есть различие между простою правдой и религиозной иллюзией или фантазией, которая превращает в личное существо второе <<Я>> человека в отвлечении от его воли и разума. <<Что-либо злое, --- говорит, например, ортодоксальный полигистор восемнадцатого века (Гундлинг), --- бог не может приказать, ибо он в высшей степени добр и мудр: следовательно, он приказывает доброе. Добро творится под знаком равенства, заповедь идет вслед; таким образом, он повелевает человеку то, что для него является благом, и запрещает то, что ему вредно. Finis Dei noster quoque finis sit oportet, --- божья цель должна быть и нашей целью>>. Разумеется; ибо наша цель есть и божья цель; чего мы не хотим, противоречащего нашей природе, злого, вредного, того не хочет и бог. Но хотя в действительности закон и существо божества имеют своей предпосылкой и основой человеческое существо, религиозная фантазия все это переворачивает вверх дном. Тот же теист по этому поводу замечает, что хотя атеист <<и может понимать моральные истины, имеющие связь с человеческой природой>>  но <<для их практического осуществления, так как эти истины противоречат нашим вожделениям и аффектам>>  только теизм может дать надлежащие средства. <<С противоположной точки зрения, --- говорит он же в согласии со всеми теистами, --- не остается ничего другого, как полезность, которая и должна меня удержать от воровства, от убийства, от оскорбления другого>>. <<Но предположим, --- продолжает он, --- ты встретишь своего заклятого врага в уединенном месте, как Саул Давида в пещере, и сможешь не опасаться, что будет раскрыт твой акт мести и что ты не будешь наказан. Бога ты не боишься\dots Ты --- атеист. Что же должно удержать тебя от умерщвления твоего врага?>> То же самое, что и тебя, хвастливый теист! Потому что в таких казуистических случаях решает лишь то, что ты собой представляешь, а не то, во что ты веришь или что ты думаешь; если ты злой, жаждущий мщения человек, то ты совершишь свое постыдное дело, несмотря на твою веру в бога и на твой страх перед богом, ибо благоприятствующий этому момент и страсть увлекут тебя; но если ты являешься противоположностью этому, если ты не подлая, а благородная натура, если ты и в самом деле человек, а не зверь, то ты и без страха перед богом и людьми найдешь в себе достаточно мотивов, которые удержат тебя от постыдного дела. Я назову прежде всего чувство чести, чувство, которое избегает делать тайным то, что стыдно делать при других, --- чувство, которого, к сожалению, однако, христианство совершенно не культивировало из-за своей веры в бога, --- чувство, которое не хочет обманывать других и согласно которому человек хочет также и быть тем, чем он представляется другим, применительно же к указанному индивидуальному случаю это --- то чувство, которое делает человека, если он не совсем зверь, как раз в тот момент, когда он может поступить как угодно с объектом своих вожделений, --- победителем над своими страстями, --- чувство, являющееся торжеством наибольшей власти над жизнью и смертью, но именно поэтому не унижающее себя до презренного ремесла палача. Поэтому как в физике, так и в этике, или морали, лишь по невежеству обратились к помощи теологии, но именно по этой же причине не позаботились культивированием заложенных в самом человеке основ и элементов добродетели и оставили народ и по сей день пребывающим в состоянии величайшей нравственной грубости. Что касается вышеприведенного положения, гласящего, что при атеизме нравственность зависит исключительно от пользы или вреда, --- положение, которое и до настоящего дня, хотя и другими словами и в других выражениях, отстаивается теологами и их последователями, умозрительными лакеями теологии, --- то следует заметить, что это противоположение ложно даже с точки зрения теизма. В этом противоположении речь идет не о вреде и не вреде, пользе и не пользе, --- в этом обе стороны сходятся, --- а об известном и неизвестном вреде, известной и неизвестной пользе. Вред атеиста неизвестен, вред теиста, предмет его страха, --- гнев, наказание божие, --- известны; но и, обратно, польза атеиста неизвестна, польза же теиста --- любовь, награда божия известна. Или лучше: противоположность между теизмом и атеизмом есть противоположность между бесконечным и конечным эгоизмом. В страхе перед богом исчезает, правда, из поля зрения эгоизм, ибо страх есть трепетание <<Я>> перед силой, уничтожающей или могущей уничтожить <<Я>>, но в известной и бесконечной награде божией явственным образом выступает бесконечный эгоизм. Атеист поэтому имеет, правда, ту моральную невыгоду по сравнению с теистом, что у него нет страха перед богом, но у него есть и то преимущество перед ним, что он не имеет в виду никакой божьей награды. Впрочем, я этими словами не хочу защищать ограниченный, поверхностный атеизм прежнего времени, в частности французский атеизм. Как истинная республика далека от республики французов, так далек и истинный атеизм от атеизма французов. В основе веры в божественное правосудие лежит, как я это показал уже в другом месте, и вера в Немезиду, в гибель зла и в победу добра, вера, составляющая основу всех исторических действий. \emph{Но эта вера есть вера, независимая от теизма, от веры в бога, ибо добро заложено в человеческой природе, заложено даже в человеческом эгоизме; добро есть не что иное, как то, что отвечает эгоизму всех людей, зло --- не что иное, как то, что отвечает и что выгодно эгоизму отдельных человеческих классов и что, стало быть, идет за счет других, но эгоизм всех или даже только большинства всегда сильнее, чем эгоизм меньшинства.} Достаточно, в самом деле, для этого бросить взгляд на историю! Когда начинается в истории новая эпоха? Всюду тогда, когда против исключительного эгоизма нации или касты заявляет свой вполне законный эгоизм угнетенная масса или большинство, когда классы или целые нации, одержав победу над высокомерными претензиями господствующего меньшинства, выходят из жалкого и угнетенного состояния пролетариата на свет исторической и славной деятельности. Так и эгоизм ныне угнетенного большинства человечества должен осуществить и осуществит свое право и начнет новую эпоху истории. Упразднению подлежит не аристократия образования, духа --- нет! \emph{Но недопустимо, чтобы немногие были благородны, а прочие --- чернью; все должны, --- по крайне мере должны, --- быть образованы. Не должна быть отменена собственность вообще --- нет! Но недопустимо, чтобы немногие имели собственность, а другие --- ничего не имели; собственность должна быть у всех.}} 

\hyperlink{b2}{Первоначальный предмет религии есть нечто, от человека отличное и независимое, от чего, однако, он сам зависит. Это есть не что иное, как природа. Весьма поучительны относительно этого пункта классики. Вот несколько примеров. <<Пусть боги, --- говорит Овидий Германику в своих письмах с Понта, --- дадут тебе только годы, то есть долгую жизнь, остальное возьмешь ты уже у самого себя>>. <<Молодой Цезон Квинкций, --- говорит Ливий, --- был благородного происхождения, крупного и крепкого телосложения. К этим дарам богов он присоединил еще сам блестящие доказательства храбрости на войне и красноречия на форуме>>. Сейчас же вслед за тем Ливий называет его юношей, наделенным или вооруженным всеми дарами или благами природы и счастья. Перед сражением с Ганнибалом, после перехода его через Альпы, Корнелий Сципион говорит --- согласно Ливию --- солдатам, между прочим, следующее: <<Я ничего больше не боюсь, как того, чтобы могло показаться, что не вы, а Альпы победили Ганнибала, хотя и вполне в порядке вещей, чтобы боги сами без человеческой помощи сражались с предводителем войска, ворвавшегося в страну, и победили его>>. <<Год, оскверненный столькими (человеческими) зверствами (каким был 66 год при Нероне), и сами боги отметили бурями и болезнями>>  говорит Тацит в своей летописи. Лукулл в биографии Плутарха отгоняет Митридата от моря при содействии богов тем, что буря уничтожает флот Митридата; у Флора же это поражение Митридата вызывают лишь волны и бури, как бы находящиеся с Лукуллом в союзе. Но ведь безразлично, сказать ли: природа или боги, потому что боги сами лишь поэтические существа природы. <<Все люди, --- говорит Котта в цицероновском сочинении ,,О сущности богов``, ---  полагают, что все внешние приятные вещи, --- виноградники, поля, оливковые сады, изобилие плодов полевых и древесных, словом, все, что относится к приятной и счастливой жизни, --- что все это они имеют от богов. Разве когда-нибудь кто благодарил богов за то, что он добродетельный муж? Нет! Он благодарил богов лишь за то, что он богат, что он почитаем, что он здоров>>. Короче говоря, мнение всех людей --- что счастье надлежит выпросить у богов, но что мудрость надо взять у самого себя. <<Пусть Юпитер, --- говорит Гораций в своих письмах, --- даст мне только жизнь, только имущество; веселый дух, то есть спокойное душевное расположение, я себе сам добуду>>  а цензор Метелл Нумидик говорит у Геллия: <<Боги должны награждать добродетель, но не наделять ею людей>>. Так, и персидский поэт Саади говорит: <<богатство и власть мы приобретаем не при помощи нашего искусства, --- только всемогущество божие уделяет их нам>>. <<Кто может в том сомневаться, --- говорит Сенека в своих письмах, --- что то, что мы живем, есть дар бессмертных богов, но то, что мы живем хорошо (праведно, морально), есть дар философии>>. Как отчетливо, как ясно высказано здесь, что божество или боги не означают чего-либо другого, как именно природу! Что вне власти человека, что не является действием человеческой самодеятельности, как, например, жизнь, то есть действие бога, то есть в действительности природы.} 

\hyperlink{b2}{Природа есть бог человека; но природа находится в постоянном движении и изменении, и эти изменения природы или явления в ней обрекают на неудачу или благоприятствуют, препятствуют или поощряют человеческие желания и намерения; они-то поэтому главным образом и возбуждают религиозное чувство и делают природу предметом религии. Подымается попутный ветер и приносит меня к желанной стране: я прибыл туда <<с помощью божьей>>; бурный ветер вздымает моим врагам пыль в лицо: бог поразил их слепотой; дождь освежает меня внезапно во время великой засухи: послали его боги; появляется повальная болезнь среди людей или скота: болезнь есть <<рука божия>>  или его власть. Но что данные явления природы как раз соответствуют или противоречат тем или другим человеческим желаниям, что они благоприятны или неблагоприятны для людей, это --- в большинстве случаев простая случайность. Случайность --- порой счастливая --- есть поэтому главный предмет религии. Кажется противоречием, что то, что, по выражению Плиния Старшего, как раз заставляет людей сомневаться в существовании бога, само принимается за бога. Однако случайность имеет в себе существенный и первоначальный отличительный признак божества, а именно: то, что она является чем-то непреднамеренным и непроизвольным, чем-то независимым от человеческого знания и воли, но в то же время таким, от чего зависит судьба человека. То, что язычники приписывали фортуне или фатуму, судьбе, то христиане приписывают богу, и, тем не менее, они так же обожествляют случайность, как и язычники, с той лишь разницей, что они не представляют ее себе в виде особого божества. Общее слово бог есть мешок, в который можно поместить все, что угодно; но существо, заключенное в мешок, от этого не перестает быть тем, чем оно является и вне мешка; только для меня оно утрачивает свои видимые свойства. Поэтому по существу, для содержания, совершенно безразлично, скажу ли я, что бог этого хотел или что этого хотел случай; безразлично, скажу ли я, что мне на долю выпал богатый урожай или что бог дал мне его; точно так же нет по смыслу никакой разницы между выражениями: <<если бог захочет, то зацветет метла>> и <<кому везет, у того и бык отелится>>  между выражениями: <<бог печется о глупцах>> и <<счастье благоприятствует дуракам>>; между выражениями: <<бог даст>> и <<не всегда дует тот же самый ветер>>; между выражениями: <<все идет, как угодно богу>> и <<все идет, как угодно судьбе>>; между <<бог обо всем заботится>> и <<позаботились о том, чтобы деревья не дорастали до неба>>; между <<кого бог промочил, того он и высушит>> и <<за дождем идет солнышко>>. Бог есть <<оно>>  превращенное в личное <<он>>. <<Он>> уютнее, утешительнее, чем безличное <<оно>> счастья или несчастья, но в этом и единственное различие. Несчастный случай остается тем же, лишил ли меня зрения упавший ласточкин помет или умышленный удар кулаком; сваливает ли меня с крыши случайное <<оно>>  или же капризный <<он>>,  мой всемилостивейший государь, подстреливает меня для собственного удовольствия. Поэтому нет ничего удивительного, что уже у греков слово <<theos>> (бог) имеет значение счастья, случая tyche и что даже благочестивое простодушие наших христианских предков отгадало и раскрыло ту тождественность естественного и божественного случая, которая, будучи отмечена в <<Сущности христианства>>  так возмутила современных христиан. Вместо нашего <<с богом>> греки начинали свои официальные документы и решения словами: <<с удачей>>. И римляне говорят то бог вместо счастья или случая, то случай вместо бога. <<Разве только поможет бог или какой-либо случай>>  --- пишет, например, Цицерон Тирону. Фортуна имела в Риме не менее 26 храмов. --- Как у нас <<оно>> и <<бог>> эквивалентны друг другу, так и у римлян одинаково употребляется: <<пусть бог обратит это во благо>> или <<пусть это кончится хорошо для меня и для вас!>>. Так, наивно благочестивый Авентин говорит: <<Бог, природа и счастье решили иначе, тогда как нам показалось, что они уже выиграли битву>>. И по другому поводу, когда <<венгерцы были обращены ветром и непогодой в бегство>>  он пишет: <<и тогда, быть может, по милости божией или по чему другому, солнце померкло>> и так далее.}

\hyperlink{b2}{Предметом религии является самостоятельное, от человека отличное и независимое, существо, которое есть не только внешняя природа, но и внутренняя природа человека как нечто независимое и отличное от его воли и разума. Вместе с этим положением мы подошли к самому важному пункту --- к происхождению религии, к тому подлинному месту, которое она занимает. Тайна религии есть, в конце концов, лишь тайна сочетания сознания с бессознательным, воли с непроизвольным в одном и том же существе. Человек имеет волю, и все же он имеет волю помимо своей воли --- как часто завидует он безвольным существам! --- он сознателен и, однако, он бессознательно пришел к сознанию, --- как часто губит он сам свое сознание! --- и как охотно в конце рабочего дня погружается он в бессознательное состояние! --- он живет и, однако, не имеет в своей власти ни начала, ни конца своей жизни; он сформировался и, однако, когда он закончил свое формирование, ему кажется, что он произошел через какое-то первичное зарождение, что он появился на свет внезапно, как за ночь появляется гриб; у него есть тело, он его ощущает своим при каждом наслаждении, при каждой боли, и все же он чужак в своем собственном доме; всякая радость приносит ему награду, которой он не заслужил, но зато и всякое страдание составляет для него наказание, которого он тоже не заслужил; в счастливые минуты он ощущает жизнь как подарок, которого он не просил, а в несчастливые --- как бремя, которое взвалено на него против его воли; он испытывает муку потребностей, и все же он удовлетворяет их, не зная, делает ли он это по собственному своему побуждению или по чьему-либо чужому, удовлетворяет ли он тем самым себя или чужое существо. Человек со своим <<Я>> или сознанием стоит на краю бездонной пропасти, являющейся, однако, не чем иным, как его собственным бессознательным существом, представляющимся ему чужим. Чувство, охватывающее человека перед лицом этой пропасти и изливающееся в словах изумления: <<что я такое? откуда? зачем?>> есть религиозное чувство, чувство того, что <<Я>> ничто без некоего <<Не-я>>, правда, от меня отличного, но, тем не менее, самым интимным образом со мною связанного, иного и, однако, составляющего мое собственное существо. Но что же во мне есть <<Я>> и что <<Не-я>>? Пусть <<Не-я>> есть голод как таковой или причина голода, но мучительное ощущение или сознание голода, которое меня побуждает направлять все мои орудия движения по направлению к предмету, который бы эту муку утишил, это есть <<Я>>. Элементами <<Я>>, или человека, человека в собственном смысле этого слова, являются, таким образом, сознание, ощущение, произвольное движение --- произвольное, потому что непроизвольное движение находится уже по ту сторону <<Я>>, в сфере божественного <<Не-я>> --- потому-то в болезнях, как, например, в эпилепсии, и в состоянии экстаза, безумия, сумасшествия и усматривали откровения бога или божественные явления. То, что мы только что показали на примере голода, относится и к высшим, духовным влечениям. <<Я>> испытываю, например, влечение только к поэтическому творчеству и удовлетворению его своей произвольной деятельностью, но самое влечение, склоняющее меня к этому удовлетворению, есть <<Не-я>>; хотя, --- что, впрочем, сюда не относится, --- <<Я>> и <<Не-я>> так переплелись между собою, что одно может быть поставлено на место другого, при чем <<Не-я>> так же мало существует без <<Я>>, как и <<Я>> без <<Не-я>>, и это единство <<Я>> и <<Не-я>> и составляет тайну, существо индивидуальности. Каково <<Не-я>>, таково и <<Я>>. Так, например, где влечение к еде является преобладающим <<Не-я>>, там и <<Я>>, или индивидуальность, характеризуется преобладающим развитием орудий еды. Данному <<Не-я>> соответствует лишь данное <<Я>> и наоборот. Если бы дело обстояло иначе, если само <<Не-я>> не было уже индивидуализировано, то явление или существование <<Я>> было бы столь же необъяснимо, чудесно и чудовищно, как и воплощение бога или соединение человека и бога в теологии. То, что является основой индивидуальности, является и основой религии: соединение или единство <<Я>> и <<Не-я>>. Если бы человек был одним только <<Я>>, то у него не было бы религии, ибо он сам был бы богом; но ее не было бы и в том случае, если бы он был <<Не-я>> или <<Я>>, не отличающееся от своего <<Не-я>>, ибо он был бы тогда растением или животным. Человек, однако, именно благодаря тому есть человек, что его <<Не-я>>, как и внешняя природа, является предметом его сознания, предметом даже его удивления, предметом чувства зависимости, предметом религии. Что я такое без наличности чувств, без воображения, без разума? В чем преимущество внешнего счастливого случая перед счастливой выдумкой, спасающей меня из нужды? Может ли помочь мне солнце на небе, если глаз не бдит над моими шагами? И что такое блеск солнца по сравнению с волшебным светом фантазии? Что такое чудо внешней природы вообще по сравнению с чудом внутренней природы, духа? Есть ли, однако, глаз продукт моих рук, фантазия продукт моей воли, разум --- изобретение, мною сделанное? Разве я сам <<дал>> себе все эти чудесные силы и таланты, которые составляют основу моего существа и от которых зависит мое существование? Таким образом, является ли моей заслугой, моим делом, что я --- человек? Нет! я смиренно признаю, --- в этом я совершенно одного мнения с религией, --- что я сам не создал ни глаза, ни другого нашего органа или таланта; но все человеческие способности не получены мною готовыми, как утверждает религия. Нет! --- здесь уже я прихожу в столкновение с религией --- они развились, и притом одновременно со мной, из недр природы. То, что не является продуктом человеческого произвола, религия делает продуктом божественного произвола; то, что не является заслугой, делом рук человека, она делает заслугой, даром, делом рук бога. Религия не знает другой созидающей деятельности, кроме произвольной деятельности человеческой руки, она вообще не знает другого существа, кроме человеческого (субъективного); человеческое существо для нее --- и притом более, чем все боги --- есть абсолютное, единственное существо, которое имеет бытие; тем не менее, однако, к своему величайшему изумлению, она наталкивается в самом человеке на <<Не-я>>; она делает поэтому нечеловеческое существо в человеке, в свою очередь, человеческим, <<Не-я>> превращает в <<Я>>, которое так же имеет руки (вообще орудия или силы для произвольной деятельности), как и человек, с тою лишь разницей, что божественные руки делают то, чего человеческие руки делать не могут. Нам следует, таким образом, отметить две черты в религии. Одна, это --- то смирение, с которым человек признает, что все, чем он является и что он имеет, он имеет не от себя; даже свою собственную жизнь и свое тело он имеет в аренде, но не в собственности и поэтому в любой момент может быть их лишен, --- кто поручится, что я не потеряю свой рассудок? --- что, стало быть, он не имеет ни малейшего основания для чванства и высокомерия. Понятие <<Я>>, то есть того вообще, что человек себе приписывает, весьма неопределенно и относительно, и в той мере, в какой он расширяет это понятие или его суживает, суживается или расширяется и понятие или представление о божественной деятельности. Человек может, --- правда, часто из простой религиозной галантности и лести по отношению к богам, --- зайти даже так далеко, что он станет во всем себе отказывать; потому что то обстоятельство, что я ощущаю, что я сознателен, что я есмь <<Я>>, это, ведь, в конце концов, тоже результат предпосылок, находящихся вне <<Я>>, дело природы или бога. В самом деле: чем больше человек уходит в себя, тем больше видит он, как исчезает различие между природой и человеком, или <<Я>>, тем более познает он, что он есть сознательное бессознательное или одно из сознательных бессознательных, что он есть <<Не-я>>, являющееся как <<Я>>, или одно из них. Поэтому человек есть самое глубокое и самое глубокомысленное существо. Но человек не понимает и не выносит своей собственной глубины и раскалывает поэтому спое существо на <<Я>> без <<Не-я>>, которое он называет богом, и <<Не-я>> без <<Я>>, которое он называет природой.} 

\hyperlink{b2}{<<Муж, --- говорит Софокл в Аяксе-биченосце, --- даже если он и имеет могучее тело, должен постоянно думать о том и бояться того, что он и от малейшего несчастного случая может погибнуть>>. <<Мы, люди, не что иное, говорит он там же, --- как легкие тени, лишенные существа. Если ты об этом подумаешь, то никогда не произнесешь дерзкого слова против богов и никогда не будешь кичиться тем, что сильнее или богаче других, ибо один единственный день может отнять у тебя все, что ты имеешь>>. Когда Аякс покинул отчий дом, то отец сказал ему: <<Мой сын, стремись к победе, но пусть победой бог тебя дарит>>. Аякс, однако, дал на это глупый и дерзкий ответ: <<Отец! При помощи богов и слабый врага осилит: я же и без них стяжать надеюсь доблести венец>>. Эта речь храброго Аякса была, без сомнения, не только не религиозна, но и необдуманна, потому что и у самого храброго, и у самого сильного мужа может за ночь парализовать руку простой ревматический припадок или вообще какой-нибудь несчастный случай. Таким образом, если Аякс и не хотел иметь никакого дела с богами, то все же ему бы следовало включить в свою речь скромное <<если>> и сказать: <<если со мной не случится какого-нибудь несчастья, я одержу победу>>. Религиозность есть поэтому не что иное, как добродетель скромности, добродетель умеренности, в смысле греческого благоразумия; бог любит, говорит Софокл, благоразумных, то есть ту добродетель, благодаря которой человек не переступает границ своей природы, не превосходит в своих мыслях и требованиях меру и возможности человеческого существа, не притязает на то, что несвойственно человеку, ту добродетель, благодаря которой он отказывается от гордого звания творца, а произведения, созидаемые им, даже произведения кузнечного и ткацкого ремесел, не зачисляет себе в заслугу, потому что способности к данному искусству и приемы его он имеет от природы, а не от себя. Будь религиозен! значит: подумай, что ты такое: --- человек, смертный! Не так называемое сознание бога, а сознание человека есть первоначально или само по себе сущность религии (в ее неизменном, положительном смысле) --- сознание или чувство, что я человек, но не причина человека, что я живу, но не являюсь причиною жизни, что я вижу, но не являюсь причиною зрения. Хотеть упразднить религию в этом смысле было бы так же бессмысленно, как бессмысленно было бы хотеть сделаться художником, не имея таланта, а обладая только волей и старанием. Начинать какую-нибудь работу без таланта и, стало быть, без призвания значит начинать ее без бога; начинать же ее, имея талант, значит начинать с успехом, значит начинать ее с богом. <<В нас, --- говорит Овидий в своих Фастах, --- живет некий бог, мы воспламеняемся, когда он нас возбуждает>>. Но что такое этот бог поэта? Олицетворенное поэтическое искусство, поэтический талант, опредмеченный в виде божественного существа. <<Все попытки, превосходно говорит Гете, --- ввести какое-нибудь иностранное новшество, потребность в котором не коренится в глубинах самой нации, неразумны, и все такого рода надуманные революции не имеют успеха, ибо они не имеют бога, который воздерживается от подобного рода негодных поступков. Но если существует в народе действительная потребность в большой реформе, то народ имеет бога на своей стороне, и реформа удается>>. Это значит: что творится без надобности, а стало быть, без права, ибо право нужды есть первичное право, то творится без бога. Где нет необходимости в революции, там не достает у революции и настоящего побуждения, таланта, головы, и она по необходимости должна потерпеть крушение. Предприятие без бога, или, что то же, без успеха, есть предприятие безголовое и бесполезное. Другая черта религии, которую следует отметить и которую мы уже отметили, есть то высокомерие, с которым человек, преисполненный лишь сознания собственного значения, все превращает в себе подобное, очеловечивает и таким образом делает также отличную от человека сущность его личным существом, следовательно, существом, которое является предметом молитв, благодарений и почитания. Религия благодаря тому, что она делает произвольное чем-то непроизвольным, силы и произведения природы --- дарами и благодеяниями, обязывающими человека благодарностью и почитанием по отношению к их творцам-богам, --- религия поэтому имеет за собой видимость глубокой гуманности и культуры, между тем как противоположная точка зрения, рассматривающая и принимающая блага жизни как непроизвольные произведения природы, имеет против себя видимость бесчувственности и грубости. Уже Сенека говорит в своем сочинении о благодеяниях: <<Ты говоришь: все эти блага идут от природы. Но разве ты не сознаешь, говоря это, что ты употребляешь другое название для бога? Ибо что другое природа, как не бог? Стало быть, ты ничего не говоришь, неблагодарнейший из смертных, когда говоришь, что ничем не обязан богу, а лишь природе, ибо нет ни природы без бога, ни бога без природы, но оба они --- одно и то же>>. Однако мы не должны дать себя ослеплять этим религиозным ореолом святости, но признать, что влечение человека выводить все действия природы из одной личной причины, добрые --- из доброй воли или доброго существа, злые --- из злой воли или злого существа, имеет свое основание в грубейшем эгоизме, что лишь из этого влечения произошли религиозные человеческие жертвоприношения и другие ужасы человеческой истории; ибо то же влечение, которое для блага, им получаемого, нуждается в личном существе для благодарности и любви, нуждается и для зла, с ним приключающегося, для того, чтобы ненавидеть и уничтожать, также в личном существе, будь то еврей или еретик, волшебник или ведьма. Один и тот же огонь вздымался к небу в знак благодарности за блага природы и сжигал в виде наказания за зло природы еретиков, волшебников и ведьм. Если поэтому признаком образования и гуманности является благодарение господа бога за благодетельный дождь, то таким же признаком образования и гуманности является и навязывание в качестве вины дьяволу и его товарищам вредоносного града. Там, где все доброе исходит от божественной доброты, там по необходимости все злое исходит от дьявольской злобы. Одно не отделимо от другого. Но совершенно ведь очевидно, что приписывание человеком злой воли действиям природы, противоречащим его эгоизму, свидетельствует о величайшей неразвитости. Нам не нужно для того, чтобы в этом убедиться, восходить до Ксеркса, который, согласно Геродоту, наказал 300 ударами кнутом Геллеспонт, в досаде на то, что через море нет моста; не нужно переселяться на остров Мадагаскар, где умерщвляют детей, которые во время беременности и родов причиняют своим матерям тягость и боли, будто бы из-за того, что они злы, мы ведь и сами видим, как наши некультурные, невежественные правительства приписывают все им неприятные, исторически необходимые явления в развитии человечества злой воле отдельных лиц, как некультурный человек лишь потому дурно обращается со своим скотом, со своими детьми, со своими больными, что он рассматривает ошибки или особенности природы, как действия преднамеренной закоснелости, как и вообще чернь со злорадством приписывает воле человека то, что от него не зависит, что он имеет от природы. Стало быть, таким же признаком некультурности, грубости, эгоизма, предубежденности является то, что человек противоположные, благодетельные явления природы приписывает доброй или божественной воле. Различение: <<я --- не ты, ты --- не я>> --- есть основное условие, основной принцип всякого образования, всякой гуманности. Тот же, кто приписывает действия природы воле, тот не различает между собой и природой, между своим и ее существом, тот поэтому и не относится к ней так, как должен относиться. Истинное отношение к предмету должно соответствовать его отличию от меня, от моего существа; это отношение, правда, не есть религиозное, но и не up религиозное, как это представляет себе обыкновенная и ученая толпа, которая знает лишь противоположность между верой и неверием, религией и нерелигиозностью, но не знает третьего, высшего над ними обоими. Будь так добра, милая земля, --- говорит религиозный человек, --- и дай мне хорошую жатву. <<Хочет земля или не хочет, она должна мне принести плоды>>  --- говорит нерелигиозный человек, Полифем; земля мне принесет, --- говорит настоящий, чуждый как религиозности, так и нерелигиозности, человек, --- если я ей дам то, что подобает ее существу; она ни не хочет дать, ни не должна дать --- дать вынужденно, против своей воли она просто только даст, если я с своей стороны исполню все условия, при которых она может что-либо дать или, вернее, произвести; ибо природа не дает мне ничего, я должен все, --- что, по крайней мере, не имеет непосредственно ко мне отношения, --- сам себе присвоить, и притом самым насильственным образом. Мы между собою, в силу разумных и эгоистических соображений, запрещаем убийство и воровство, но по отношению к другим существам, по отношению к природе, мы все убийцы и воры. Кто дает мне право на зайца? Лиса и коршун имеют такой же голод и такое же право существовать, как и я. Кто дает мне право на грушу? Она столько же принадлежит муравью, гусенице, птице, четвероногому. Итак, кому же она собственно принадлежит? Тому, кто ее берет. И за то, что я живу лишь убийством и кражей, я должен еще благодарить богов? Как глупо! Я обязан по отношению к богам благодарностью лишь в том случае, если они мне докажут, что я им и в самом деле обязан своей жизнью, и это они мне докажут лишь тогда, когда жареные голуби прямо с неба прилетят мне в рот. Я говорю: жареные? О! Это слишком мало, я должен сказать: разжеванные и переваренные, ибо для богов и их даров не приличествуют длительные и неэстетичные операции разжевывания и переваривания. Как может бог, который создает мир в один миг из ничего, тратить так много времени и стараний на то, чтобы образовать немного пищевой кашицы! И при этом мы опять убеждаемся в том, что божество состоит, так сказать, из двух составных частей, из которых одна принадлежит фантазии человека, другая --- природе. Молись! Говорит одна часть, то есть бог, отличный от природы. Работай! Говорит другая часть, то есть бог, не отличающийся от природы, а лишь выражающий ее существо. Ибо природа --- рабочая пчела, боги же --- трутни. Как же могу я от трутней заимствовать пример и заповедь трудолюбия? Кто выводит природу и мир из бога, тот утверждает, что голод происходит от сытости, нужда --- от избытка, богатство мыслей --- от их легковесности, работа --- от ничегонеделания, тот хочет из амброзии спечь себе черный хлеб, из нектара богов сварить себе пиво.} 

\hyperlink{b2}{Природа есть первичный бог, первичный предмет религии; но она для религии является предметом не как природа, а как человеческое существо, как существо, порожденное душевными настроениями, фантазией, мыслью. Тайна религии есть тождественность <<субъективного и объективного>> то есть единство человеческого и природного существа, но при этом отличающегося от действительного существа природы и человечества. Разнообразны те способы, которыми человек очеловечивает существо природы и, наоборот, --- так как оба явления неотделимы друг от друга, --- опредмечивает, отчуждает свое существо; мы, однако, ограничиваемся лишь двумя --- метафизической и практически-поэтической --- формами монотеизма. Последняя особенно отличает Ветхий завет и Коран. Бог Корана есть также, как и бог Ветхого завета природа или мир, действительное живое существо, в противоположность искусственному, мертвому, деланному существу идола, не кусок мира, кусок природы, как, например, камень, которому поклонялись арабы до Магомета, но вся нераздельная великая природа. Магомет, --- рассказывает Джелаледдин, послал одного ревностного магометанина обратить в ислам одного неверующего. <<Каков твой бог? --- спросил его неверующий. --- Он из золота, серебра или меди?>> Молния поразила безбожника, и он упал мертвым. Это --- весьма грубый, но убедительный урок, как различаются живой и искусственный боги. <<Скажи, --- говорится, например, в Коране, в 10-й суре, --- кто снабжает вас небесной и земной едой? Или кто имеет власть над слухом и зрением? Кто производит жизнь из смерти или смерть из жизни? Кто господин всех вещей? Конечно, вы ответите: бог. Так скажите, разве вы не хотите его бояться?>>. <<Бог дает прорасти, --- говорится в 6-й суре, --- семени и финиковой косточке\dots Он вызывает утреннюю зарю и дает ночь для отдыха, и солнце и луну --- для исчисления времени. Это устройство идет от всемогущего и всемудрого. Это он посылает воду с неба, благодаря ей мы имеем семена всех предметов и всю зелень, и хлеб, растущий рядами, и пальмовые деревья, с ветвей которых свешиваются финики, и сады с виноградом, оливками и гранатами всех родов. Посмотрите на их плоды, когда они подрастут и созреют. Воистину, здесь много знамений для верующих людей>>. <<Это бог, --- говорится в 13-й суре, --- поднял небеса на высоту, без того, чтобы подпирать их видимыми столбами\dots Это он же раздвинул землю на большое пространство и установил на ней неизменные горы и создал реки и каждый вид плода разделил на два пола. Он делает так, что ночь покрывает день\dots И он же вам показывает молнию, которую вы созерцаете в страхе и надежде, и напитывает тучи дождем. Гром возвещает ему хвалу, и ангелы превозносят его с трепетом. Он посылает свои молнии и уничтожает, кого хочет, и все-таки люди спорят о нем, о всемогущем>>. Отличительные признаки или проявления истинного бога, бога-подлинника, в противоположность богу-копии, то есть идолу, суть проявления природы. Изображение идола не может произвести ни живых существ, ни вкусных плодов, ни плодоносящего дождя, ни устрашающей грозы. Это может лишь бог, который является богом от природы, а не сделан богом лишь людьми, и который поэтому имеет не только видимость, но и сущность живого, действительного существа. Бог же, чьи проявления и отличительные признаки суть проявления природы, есть не что иное, как природа, однако, как уже говорилось, не кусок природы, который имеется здесь, но не имеется там, который существует сегодня, но не существует завтра и именно поэтому представляется и увековечивается в виде изображения, а совокупность природы. <<Когда темнота ночи, --- говорится в 6-й суре, --- его (Авраама) прикрыла своей тенью, он увидел звезду и сказал: это мой господин. Но когда она закатилась, то он сказал: я не люблю заходящих. И когда он увидел восходящим месяц, то он сказал: воистину, это мой господин. Но когда и месяц зашел, то сказал он: если мой господин мной не руководит, то я таков же, как и этот блуждающий народ. Но когда он увидел солнце восходящим, то он сказал: вот это мой бог, ибо это величайшее существо. Когда же и солнце закатилось, то сказал он: о, народ мой, я больше не участвую в вашем идолослужении, я обращаю лицо свое к тому, кто создал небо и землю>>. Быть всегда и везде, быть вездесущим, --- вот отличительный признак истинного бога, но ведь и природа существует везде. Где нет природы, там нет и меня, а там, где я, там есть и природа. <<Куда уйти мне>> от тебя, природа? И <<куда бежать>> от твоего существа? <<Взойду ли на небо, и там природа. Лягу ли в преисподнюю, --- глядь, и там природа>>. Где жизнь, там и природа, и, где нет жизни, там тоже природа; все полно природой; как же хочешь ты бежать от нее? Но бог в Коране, как и в Ветхом завете, есть природа и в то же время не природа, отрицание ее, а потому субъективное, то есть личное, --- как человек, знающее и мыслящее, как человек, желающее и действующее, --- существо. Проявления природы, составляющие предмет религии, являются одновременно и проявлениями человеческого невежества и воображения; сущность или причина этих проявлений природы является одновременно и сущностью человеческого невежества или воображения. Человек отделен от природы пропастью невежества; он не знает, как растет трава, как формируется ребенок в чреве матери, каково происхождение дождя, молнии или грома. <<Обозрел ли ты, --- говорится в книге Иова, --- широту земли? Скажи, если знаешь все это. Видел ли ты, откуда берется град? Кто отец дождя? Знаешь ли ты законы неба?>> Действия природы поэтому в качестве явлений, --- ни причины, ни материи, ни естественных условий которых человек не знает, --- являются для него действиями силы, не имеющей вообще границ и не обусловленной, для которой нет ничего невозможного, которая произвела даже мир из ничего, потому что она и по сей час еще производит действия природы из ничего --- этой бездны человеческого невежества. Бездонно человеческое невежество, и безгранична человеческая сила воображения; сила природы, лишенная, по невежеству, своего основания и благодаря фантазии --- своих границ, есть божественное всемогущество. Действия природы как творения божественного всемогущества не отличаются уже более от сверхъестественных действий; от чудес, от предметов веры; это та же сила, которая производит естественную смерть и сверхъестественное воскресение из мертвых, составляющее лишь предмет веры, та же сила, которая естественным путем производит на свет человека и которая, если захочет, создаст его из камней или из ничего. <<Подобно тому, как мы, --- говорится, например, в 50-й суре Корана, --- благодаря ему (дождю) вновь возвращаем к жизни мертвую местность, так же точно произойдет когда-нибудь и воскресение\dots Разве мы утомились от первого творения (разве создание вселенной --- значится во французском переводе Савари --- стоило нам хотя бы малейших усилии)? И все же они сомневаются в возможности нового творения, то есть воскресения>>. <<После зимы, --- говорит Лютер в своем кратком комментарии 147 псалма, --- он дает наступить лету, ибо иначе была бы сплошь зима, и мы все погибли бы от холода. Но как или каким образом дает он нам лето?>> <<Он говорит, и все растаивает>>. <<Он творит все своим словом, ему не нужно ничего больше, как слово; это значит-быть господом>>. Иными словами: божественное всемогущество есть отождествленная с человеческим воображением и слитая с нею воедино сила природы --- сила природы, которая, взятая обособленно от самой природы, выражает вместе с тем только существо человеческого воображения. Но подобно тому, как человек очеловечивает природу во всемогущее существо, поскольку она вообще производит на него впечатление импонирующей силы, точно так же очеловечивает он природу и во всеблагое существо, поскольку она удовлетворяет его бесчисленным потребностям, поскольку она вообще, как совокупность всех жизненных благ, производит на него впечатление высшего блага; он очеловечивает природу и в существо в высшей степени мудрое или всезнающее, поскольку она все вышеуказанное производит, вызывая в человеческом уме величайшее изумление. Одним словом, объективная сущность, как субъективная, сущность природы, как отличная от природы, как человеческая сущность --- вот что такое божественное существо, что такое существо религии, что такое тайна мистики и спекуляции, что такое великое thauma, чудо из всех чудес, по поводу которого человек погружается в величайшее изумление и восхищение. Это слияние воедино природного и человеческого существа, которое именно потому носит название высшего, что оно является высшей ступенью силы воображения, непроизвольно, как это само собой разумеется. Непроизвольности этого слияния обязан и <<инстинкт религии или божества>> своим основанием и именем. Бог имеет волю, как и человек, но что такое воля человека по сравнению с волею бога! по сравнению с той волей, которая является причиной великих проявлений природы, которая вызывает землетрясения, которая нагромождает горы, которая движет солнцем, которая повелевает бушующему морю: досюда и не дальше! Что невозможно для этой воли? <<Бог творит, что захочет>>  --- говорится в Коране и в Псалмах. У бога речь, как и у человека, но что такое слово человека в сравнении со словом бога! <<Хочет он, --- говорится в Коране (по Савари), --- чтобы что-нибудь было? --- он говорит: да будет! и оно есть>>. <<Если он хочет дать бытие существам, он говорит: будьте! и они существуют>>. Бог имеет ум, как и человек, но что такое знание человека по сравнению со знанием бога! Оно охватывает все, охватывает бесконечную вселенную. <<Он знает, --- говорится в Коране, --- что есть на земле и что на дне моря. Ни один листик не упадет без его ведома. В земле нет ни одного зернышка, которое бы не было отмечено в книге очевидности>>. Божественное существо есть человеческое существо, но такое человеческое существо, которое своей фантазией охватывает вселенную и имеет своим содержанием природу; то же существо есть совершенно другое, столь же от нас удаленное, как солнце от глаза, как небо от земли, столь же от нас отличное, как природа, совсем другое и в то же время то же самое существо отсюда захватывающее мистическое впечатление этого существа, отсюда возвышенность Корана и Псалмов. Различие между магометанским и иудейским монотеизмом, с одной стороны, и христианским --- с другой, заключается лишь в том, что в первом случае религиозная сила воображения или фантазия смотрит вовне, имеет глаза открытыми, непосредственно примыкает к чувственному воззрению на природу, тогда как в христианстве она закрывает глаза, совершенно отделяет очеловеченное существо природы, от почвы чувственного воззрения и таким образом делает из первоначально чувственного или духовно чувственного существа --- существо абстрактное, метафизическое. Бог в Коране и в Ветхом завете еще полон естественных сил, еще влажен от океана вселенной, из которого он произошел, но бог христианского монотеизма есть насквозь засохший бог, бог, из которого вытравлены все следы его происхождения из природы; он стоит перед нами, как создание из ничего; он запрещает, даже под угрозою розог, неотвратимый вопрос: что делал бог, прежде чем он создал мир? Или вернее: чем он был до возникновения природы? То есть он утаивает, он скрывает свое физическое происхождение за отвлеченной сущностью метафизики. Если природный бог происходит из смешения женской силы мысли и воображения с мужской силой материального чувства, то метафизический бог, наоборот, происходит лишь из соединения силы мысли, силы абстракции с силою воображения. Человек отделяет в мышлении прилагательное от существительного, свойство от существа, форму от материи, как выражались древние; ибо самый субъект, материю, существо он не может воспринять; он их оставляет вовне, на свободе. И метафизический бог есть не что иное, как краткий перечень или совокупность наиболее общих свойств, извлеченных из природы, --- совокупность, которую, однако, человек, отделенную к тому же от чувственного существа, от материи природы, превращает при помощи силы воображения опять в самостоятельного субъекта или существо. Но самые общие свойства всех вещей суть те, что каждая вещь есть и что есть что-то или нечто. Бытие как таковое, бытие в отличие от того, что имеет бытие, но само опять-таки в качестве чего-то, имеющего бытие, существо как таковое, в отличие от существа природы, но само опять-таки представленное или олицетворенное в виде существа, --- вот первая и вторая части божественной метафизики или учения о сущности. Но человек имеет не только сущность и бытие, общие со всеми другими вещами и существами природы; он имеет и отличающееся существо; он имеет разум, дух. К двум первым частям божественной метафизики присоединяется, таким образом, еще и третья: логика, то есть в голове человека с сущностью, вообще отвлеченной от природы, соединяется еще и сущность, в частности отвлеченная от человека. Бог поэтому имеет столько же существования или реальности, сколько и бытие, сущность, дух вообще, следовательно, имеет существование субъективное, логическое, метафизическое; но как неразумно желать превратить метафизическое существование в физическое, субъективное существование в объективное, логическое или абстрактное существование опять в существование нелогичное, действительное! Но, конечно, как удобно, как привлекательно мыслимую, отвлеченную сущность, с которой постоянно мысленно носишься и с которой можно делать, что угодно, считать за истинное существо и, таким образом, иметь возможность взирать даже с презрением на действительное существо, которое недоступно и непослушно! Разумеется, <<мыслимое существует>>  но только как мыслимое; мыслимое есть мыслимое и остается мыслимым, сущее --- сущим: ты не можешь одно подменить другим. \emph{<<Стало быть, имеется вечный разрыв и противоречие между бытием и мышлением?>> Да, лишь в голове; в действительности это противоречие давно уже разрешено, хотя и способом, отвечающим действительности, а не твоим школьным понятиям, а именно --- разрешено посредством не менее, чем пяти чувств.}} 

\phantomsection
\addcontentsline{toc}{subsection}{К лекции шестой}
\subsection*{К лекции шестой}

\hypertarget{3}{(3)} \hyperlink{b3}{Пролетает, например, птица; я следую за ней и прихожу к превосходному источнику; следовательно, эта птица возвещает счастье; кошка перебегает мне через дорогу, когда я только что начинаю свой путь; путешествие не удается; стало быть, кошка пророчит несчастье. Сфера религиозного суеверия, поистине, безгранична и бесконечна, ибо роль причинной связи играет в ней простой случай. Потому-то животное или другое какое-нибудь существо природы и может сделаться предметом религиозной веры, или суеверия, без того, чтобы было налицо или могло быть указано какое-либо объективное основание для него. Суеверие, без сомнения, связывается с каким-нибудь бросающимся в глаза свойством предмета или его особенностью, но смысл, значение, которые суеверие в пего вкладывает, часто произвольны или субъективны. Паув в своих <<Философских разысканиях о египтянах и китайцах>> (1774 г.) рассказывает, говоря о культе животных, что несколько лет тому назад французские крестьяне оказывали род религиозного культа куколкам гусениц, живущих на крапиве, потому что они усматривали в них явственные следы божества. Эти знаки божества были, очевидно, не что иное, как те блестящие золотые точки, которые имеются на куколке. По справедливости поэтому Паув предпосылает этому рассказу слова: <<Незначительные явления могут произвести очень сильное впечатление на умы простонародья>>. Но это простонародье в человеке есть так называемое религиозное чувство, то есть душевное настроение, позволяющее обворожить себя и мистифицировать, попросту же говоря: одурачить даже блеском золотых пятнышек куколки. Но этим не уничтожается указанное основание культа животных, ибо то, чего какая-нибудь вещь не имеет и что она собою не представляет, то она имеет и то она собою представляет в сфере веры. Ядовит ли паук? Нет; но вера сделала его ядовитым. Является ли аптечная очанка средством для лечения глаз? Нет; но вера сделала ее <<утешением глаз>>. Приносит ли ласточка счастье в дом? Нет; но вера кладет свои кукушкины яйца даже в гнезда ласточек. Если бы поэтому захотели отвергнуть указанный принцип почитания животных на том основании, что, мол, люди почитают животных, не приносящих ни пользы, ни вреда, то это равносильно было бы тому, как если бы на том основании, что абракадабра и другие слова амулетов являются словами бессмысленными и поэтому, собственно говоря, совсем не являются словами, стали бы отрицать, что люди могут приписывать таким словам силу и влияние. Сверхчувственность, то есть бессмыслица, сверхразумность, то есть неразумие, являются, ведь, как раз сущностью религиозной веры или суеверия. Впрочем, и в культе животных дают себя знать другие указанные моменты религии. Мы, ведь, уже видели, как религиозная любовь к животным приносит человека в жертву даже клопам, блохам, вшам. --- Банкрофт в своей истории Соединенных Штатов Америки очень хорошо и верно говорит о культе природы и животных у индейцев: <<Птица, которая таинственно прорезывает воздух, в который человек не в состоянии подняться, рыба, которая прячется в глубинах ясных прохладных озер, в глубинах, которых человек не в силах измерить, лесные животные, верный инстинкт которых ему кажется гораздо более надежным откровением, чем разум, --- таковы те внешние признаки божества, которым он поклоняется>>. Но когда он перед тем говорит: <<Его боги не являются плодом страха\dots Индеец почитает то, что вызывает его изумление и что тревожит его фантазию>>  то на это следует заметить, что одно простое изумление, одна простая сила воображения не порождает еще молитв и жертв. Он сам говорит дальше: <<Набожность дикаря была не простым чувством пассивной преданности, --- он пытался склонить в свою пользу неизвестные силы и отвратить их гнев\dots всюду среди краснокожих были в ходу особого рода жертвы и молитвы. Если жатва оказывалась изобильной, если охота приносила удачу, то они видели в этом влияние одного из Манито и самый обыкновенный несчастный случай приписывали гневу бога. <<О, Манито, --- воскликнул один индеец при наступлении дня, оплакивая вместе со своим семейством потерю ребенка, --- ты разгневан на меня; отврати свой гнев от меня и пощади остальных моих детей>>. Лишь это является зерном религии. Человек не теоретическое, а практическое существо, не существо эфирной силы воображения, но существо полной жизненных сил, голодной и печальной действительности. Не удивительно поэтому, что, как сообщает Лоскиль, индейцы устраивают жертвоприношения в честь некоего бога жратвы, который, по их мнению, никогда не может насытиться. Ведь увековечил же в песне <<счастливый улов сельдей, избавивший его от нужды>> даже <<крупнейший человек языческого севера>> Эйвинд Скальдаспиллир! Воистину смешно, впрочем, когда теисты вкладывают в уста дикарям дипломатическое теологическое различение, когда они заставляют их говорить, что они <<почитают не самих животных, а, собственно говоря, того бога, который в них заключается>>. Что же другое можно почитать в животных, как не их животную природу или сущность? Плутарх в своем сочинении <<Об Изиде и Озирисе>> говорит по поводу почитания египтянами животных: <<Если лучшие философы даже в бездушных вещах усматривали образы божества, то насколько больше можно их найти в существах чувствующих и живых. Но хвалить можно только тех, которые почитают не сами эти существа и не сами эти вещи, но через них или через их посредство божество. Легко понять, что нет ничего бездушного, что бы было лучше имеющего душу, нет ничего бесчувственного, которое было бы превосходнее имеющего чувство; божественная природа заключается не в цветах, не в фигурах или плоскостях, ибо самое безжизненное есть и самое худое. Но то, что живет, видит, движется и различает полезное от вредного, имеет в себе часть провидения, управляющего вселенной,- как говорит Гераклит>>. Не находится ли, таким образом, основа почитания животных все же в них самих? Если божественное существо существенно отличается от животной природы, то я не могу почитать его в ней или через ее посредство, ибо не нахожу в ней образов божества, не нахожу ничего схожего с божеством; но если верно противоположное, то безразлично и сделанное различение. Кто представляет себе богов в животном виде и таковыми их изображает, тот бессознательно почитает самих животных, хотя бы он и отрицал это перед своим сознанием и рассудком.} 

\hypertarget{4}{(4)} \hyperlink{b4}{Прекрасно также славословие Плиния в честь солнца в его <<Естественной истории>>. <<Среди так называемых блуждающих звезд движется солнце чудовищной величины и мощи, солнце, правящее не только временами и странами, но даже и звездами, и небом. Мы должны это солнце, если примем во внимание его действие на душу, а еще более --- на дух всего мира, рассматривать, как превосходного правителя и божество природы. Оно дает свет миру и устраняет тьму; оно затмевает прочие светила, оно устанавливает порядок времен и постоянно себя воспроизводящего года ко благу природы; оно проясняет пасмурное небо и прогоняет также и облака человеческого духа. Оно дает свет и прочим светилам, всех превосходя своим светом и среди всех выделяясь, все видя и все слыша, как это значится у Гомера>>. Перед нами здесь в сжатом виде все моменты религии.} 

\phantomsection
\addcontentsline{toc}{subsection}{К лекции седьмой}
\subsection*{К лекции седьмой} 

\hypertarget{5}{(5)} \hyperlink{b5}{Положение, гласящее, что у греков лишь греческие боги считались богами, что язычество, как я раньше утверждал, есть патриотизм, христианство же --- космополитизм, нуждается в разъяснении, ибо оно кажется прямо противоречащим признанной терпимости и либеральной восприимчивости политеизма. Ученый Бардт говорит даже в своем сочинении <<Древнегерманская религия, или Герта>> (2-е изд.): <<Если каждая религия и воспринимает кое-что от национальной окраски, точно так же, как и каждая нация кое-что от окраски религиозной, то все we религии не разделены, как народы и союзы государств, и, как в настоящее время мы не имеем испанской, шведской, русской религии, а имеем христианскую, так и в прежнее время среди сект не было делений подобного рода>>. Если, однако, из того обстоятельства, что современные народы являются все христианскими или называются таковыми, нужно бы было сделать заключение об единстве религии прошлого времени, то с этим единством дело обстояло бы плохо, ибо хотя мы и не говорим о немецкой или русской религии, то все же на деле существует такое же большое различие между немецкой и русской религией, какое существует между немецким и русским существом вообще. И ответ на этот вопрос будет до тех пор разно гласить, пока сами люди будут различны и различно будут мыслить: одни будут усматривать и выдвигать одинаковое и общее, другие --- отличное и индивидуальное. Но что касается нашего специального вопроса, то приходится сказать, что у римлян и греков политическое и религиозное было так тесно сплетено Друг с другом, что если их богов вырвать из этого сочетания, то от них так же много или так же мало останется, как мало останется, если я захочу вырвать из римлянина римлянина, из грека --- грека и оставить лишь человека. <<Юпитер, который по своей всеобщей природе есть бог для всякого состояния, представлял собой все виды родства и гражданских отношений, так что можно вместе с Крейцером сказать, что понятие, вложенное в Юпитера, выработалось в идеальный правовой институт. Он PoUeus (охранитель города), Metoikios, Phrafcrios (охранитель братства), Herkeios и так далее>> (Э. Платнер, Очерки по изучению аттического права). Но что же останется мне от Юпитера, если я устраню этот свод законов, эти политические эпитеты или правовые титулы? Ничего или столько же, сколько останется мне, если у меня как афинянина, отнимут все права, основывающиеся именно на этих качествах, если меня укоротят на голову. Как духовные Афины связаны с местными Афинами, как духовный Рим- с местным Римом --- с не могущей быть перенесенной фортуной места, как выражается у Ливия Камилл в речи, в которой он увещевает римлян не покидать Рима, --- так же точно и римские и греческие боги были по необходимости богами территориальными или местными. Юпитер Капитолийский, правда, находится в голове у каждого римлянина и вне Рима, но действительно существует он, действительно имеет свое <<местопребывание>> лишь в Капитолии, в Риме. Все площади в этом городе, говорит Камилл в упомянутой речи, полны богов и богослужебных обрядов (религиозных отношений). И всех этих богов вы хотите покинуть? Здесь находится Капитолий, где однажды была найдена человеческая голова и получен был ответ, что на этом месте будет глава мирового господства. Здесь, когда площадь Капитолия очищалась и многие прежние алтари были убраны, Юность и пограничный бог не дали себя сдвинуть с места, к величайшей радости наших отцов. Здесь --- огонь Весты, здесь --- щиты, упавшие с неба, здесь --- все боги, расположенные к вам, если вы останетесь. Поэтому, когда солдаты Вителлия подожгли Капитолий, то совершенно в согласии с римскими и вообще языческими представлениями распространилась, --- как рассказывает Тацит в своих <<Историях>>  --- среди галлов и германцев вера в то, что наступил конец Римской империи. Некогда город был занят галлами, но господство осталось за Римом, потому что местонахождение Юпитера не было тронуто. Нынешний же роковой пожар есть знамение божественного гнева и возвещает народам по ту сторону Альп господство над миром. Когда римляне хотели занять какой-либо город, они, как известно, предварительно своими волшебными формулами выкликали из города его богов-охранителей, поэтому-то они, --- как рассказывает Макробий в своих <<Сатурналиях>>  --- и держали втайне того бога, под охраной которого находился Рим, равно как и латинское имя города Рима. Они верили, таким образом, что защитная сила богов связана с местом, что они действуют только там, где они физически, телесно живут. Неудивительно поэтому, что политеизм, не находя защиты у своих домашних, отечественных богов, протягивает руки к чужим и их охотно к себе принимает, чтобы испытать их целительную и охранительную силу. Еще Цицерон в своем сочинении <<О законах>> хвалит греков и римлян за то, что они не делают, подобно персам, весь этот мир храмом и жильем для богов, но верят в то, что боги населяют те же города, что и они, --- верят и хотят этого.}

\phantomsection
\addcontentsline{toc}{subsection}{К лекции восьмой}
\subsection*{К лекции восьмой}

\hypertarget{6}{(6)} \hyperlink{b6}{У Геродота говорится, правда, лишь то, что козел публично соединился с женщиной, так что из этих слов остается неясным, добровольной или недобровольной жертвой животного сладострастия сделалась женщина. Но если к этому добавить, что случилось это в Мондесе, где козы и в особенности козлы почитались, где бог Пан изображался с лицом козы и с ногами козла и сам назывался Мендесом, то есть козлом, если к этому, далее, добавить, что это сочетание козла с женщиной считалось счастливым предзнаменованием --- так, по крайней мере, переводят и объясняют многие и в самом деле неопределенное геродотовское выражение, --- то не подлежит сомнению, что женщина исключительно из религиозного энтузиазма, то есть сверхгуманизма и сверхнатурализма, преодолела в себе эгоистическое и исключительное влечение женщины к соединению только с человеческой особью мужского пола; следовательно, из тех же мотивов, из каких Христос принес в жертву божественной бессмыслице веры свой человеческий разум, --- <<Я верю, потому что это нелепо>>  --- она принесла в жертву священному козлу свою человеческую природу и свое человеческое достоинство.}

\phantomsection
\addcontentsline{toc}{subsection}{К лекции девятой}
\subsection*{К лекции девятой}

\hypertarget{7}{(7)} \hyperlink{b7}{Впрочем, как известно, и христианская церковь принесла своей вере или, что то же, своему богу достаточное количество кровавых жертв. И если <<христианское государство>>  а стало быть, и христианское уголовное судопроизводство является лишь креатурой христианской веры, то и по сей день христиане приносят своей вере, или --- что, как сказано, одно и то же --- своему богу, кровавые человеческие жертвы в лице каждого бедного грешника, которого они тащат на эшафот. Ведь известно же, судя, по крайней мере, по газетам, что <<христианнейший>> король прусский из одних только религиозных соображений отказался отменить смертную казнь!}

\hypertarget{8}{(8)} \hyperlink{b8}{Когда, например, в 356 году в Риме свирепствовала заразная болезнь, то, как рассказывает Ливий в 5-й книге, устроено было впервые празднество <<лектистерний>>  то есть обед богов, и праздновалось притом восемь дней, чтобы умилостивить богов. И эта щедрость распространялась не только на богов, но и на людей. Во всем городе были открыты двери, все предлагалось для общественного пользования, знакомые и незнакомые приглашались к столу, воздерживались от всяких процессов и споров, дружески беседовали даже с врагами, снимали с пленников их цепи. Когда же в 359 году в Рим пришло известие о том, что, наконец, после 10-летней осады взяты Веи, то по этому поводу, как рассказывает Ливий в той же книге, была такая чрезвычайная радость, что еще до решения сената все храмы были полны римских матерей, которые благодарили богов, и сенат распорядился, чтобы в течение 4-х дней в течение большего количества дней, чем в предыдущие войны --- молились богам и благодарили их.}

\phantomsection
\addcontentsline{toc}{subsection}{К лекции одинадцатой}
\subsection*{К лекции одинадцатой} 

 \hypertarget{9}{(9)}  \hyperlink{b9}{Так, ученый исследователь Э. Рет (Е. Roth) в согласии с моими собственными результатами, к которым я пришел только другим путем, говорит в уже указанном сочинении об египетском и зороастровском вероучениях: <<Всем древним религиям обще то явление, что имена богов вначале были не чем иным, как простыми именами нарицательными, потому что они означали лишь вещи ветер, воду, огонь и т. п. --- и понятие личного существа еще совсем с ними не было связано. Это понятие развилось лишь впоследствии и мало-помалу из тех свойств, которые приписывались божественному существу, и таким образом произошло его собственное имя от одного из тех прозвищ, которые первоначально прилагались в большом числе к божественному существу для обозначения его различных свойств. Чем ближе поэтому понятие бога стоит к своим исходным моментам, тем оно делается все более неопределенным, так что имя бога в конце концов растворяется в простом имени вещи или в слове, обозначающем свойство>>.}

\phantomsection
\addcontentsline{toc}{subsection}{К лекции двенадцатой}
\subsection*{К лекции двенадцатой} 

\hypertarget{10}{(10)} \hyperlink{b10}{Приведенная здесь цитата взята из примечаний Диониса Фоссия к сочинению Маймонида <<Об идолопоклонстве>>. Тот смысл, в котором я ее здесь употребил, правда, не выражен там буквально, но если эту цитату сопоставить с другими, например с цитатами из книги Эйзенменгера о иудействе приведенными в <<Сущности христианства>>  где буквально значится, что мир существует лишь ради иудейства, то. можно убедиться, что она имеет указанный смысл.}

\hypertarget{11}{(11)} \hyperlink{b11}{Как нельзя вывести из монотеистического бога, как существа, от природы существенно отличного, многообразие природы и существующие вообще в ней различия, точно так же нельзя вывести из него в частности многообразие человеческой природы и существующие в ней различия с их последствием правом на существование различных религий. Из единства монотеистического существа, существующего в мыслях, следует лишь единство и одинаковость людей, а следовательно, и единство веры. Существующие в человеческой природе различия и ее многообразие, на которых основываются религиозная терпимость и индифферентизм, берут свое начало лишь из политеистического принципа чувственного воззрения. Что я не единственный человек, что кроме меня имеются еще и другие люди, это ведь говорит мне лишь чувство, лишь природа; внутренний же квакерский свет, бог, отличный от природы, сущность разума, отделенная от чувств, говорит мне лишь, что существую Я, один, и требует поэтому от другого человека, если бы таковой нашелся, чтобы он думал и верил, как Я, ибо перед реальностью монотеистического единства исчезает реальность различия, реальность другого, она лишь простая иллюзия чувств: <<Все то, что не бог, есть ничто, то есть все то, что не есть Я, есть ничто>>. Если поэтому с верой в единого бога сочетается терпимость по отношению к инаковерующим, то в основе этого бога лежит многообразное и терпимое существо природы. <<Натурализм, --- говорит К. Ф. Бардт в своей ,,Оценке естественной религии`` 1791 г., --- по своей природе ведет к терпимости и свободе. Он ведь сам не что иное, как вера в субъективную истину>> и так далее <<Но сторонник положительной религии считает лишь свою веру истинной, потому что бог, мол, открыл ее ему, и не может, стало быть, равнодушно взирать на различия, потому что для него каждое различие есть отступление от того единственного, во что бог, как он полагает, приказал верить>>. <<Могу ли я любить того, кого мои бог ненавидит и кого мои бог на веки вечные передал дьяволу?>> Но что или кто является богом естественной религии? <<Бог любви, который находит свое собственное блаженство в том, чтобы оказывать благодеяния своим созданиям и делать их счастливыми>>\dots <<Если бог есть любовь\dots в таком случае человеколюбец есть полное подобие бога>>. Но кто любит какое-либо существо, тот признает его индивидуальность. Кто любит цветы --- любит все цветы, радуется их бесконечному разнообразию и дает каждому цветку то, что отвечает его индивидуальной природе. Но что является принципом, или причиной, этих бесконечных различий и индивидуальностей, раскрываемых перед нами чувствами? Природа, сущность которой и есть это разнообразие и индивидуальность, потому что она не духовное, то есть абстрактное, метафизическое, существо, как бог. Бог, разумеется, также изображается, как <<бесконечное множество различий>>  но это множество заимствовано лишь от природы и ее воззрения. Что же такое, стало быть, бог естественной религии? Не что иное, как природа, но представленная в виде личного, чувствующего, благожелательного существа, не что иное, следовательно, как антропоморфизм природы. Должен я по этому поводу также заметить, что не только язычники, но и христиане --- отнюдь не только пантеисты --- постоянно соединяют природу с богом и даже их друг с другом отождествляют, то есть ставят природу на место бога. Вот несколько примеров. И. Барклай говорит в своем <<Зерцале душ>>:<<В обычаях этих народов выступает богатство природы, которое дало возникнуть под покровом сходства внешних проявлений столь многим различным привычкам и направлениям воли>>. Даже Меланхтон в своей <<Психологии>> говорит о желчном пузыре: <<Устрояющая природа его мудро спрятала>>  --- и далее о легком: >>\dotsс какой целью природа расположила легкое вокруг сердца, можно усмотреть из его функций>>. И Эразм в своем <<Собрании пословиц>> объясняет выражение <<сражаться с богами>> таким образом: по примеру титанов сражаться с богами --- значит противоборствовать природе?}

\phantomsection
\addcontentsline{toc}{subsection}{К лекции тринадцатой}
\subsection*{К лекции тринадцатой} 

\hypertarget{12}{(12)} \hyperlink{b12}{Это особенно явствует из представления о смерти вообще, величайшем зле в глазах необразованного человека. Человек первоначально не знает, что такое смерть, и еще менее, какова ее причина. Человек --- абсолютный эгоист; он не может себе мыслить отрицания своих желаний, и, следовательно, конца своей жизни, ибо он ведь хочет жить. Он вообще ничего не знает о природе, ничего о существе, отличном от человеческого существа и воли; как же мог бы он представить себе смерть как нечто естественное или даже необходимое? Смерть поэтому имеет для него человеческую, личную, произвольную причину; но смерть есть бедствие, некое зло, стало быть, причина ее есть зависть богов, не желающих человеку ни счастья, ни радости (<<Ты завистлив, Гадес!>> говорится в одной эпиграмме Эринны), или гнев богов из-за какого-нибудь причиненного им оскорбления (так, например, жители острова Тонга верят, по свидетельству В. Маринера в его книге <<Известия об островах Дружбы, или островах Тонга>>  что всякая человеческая беда причинена богами за нерадение о религиозных обязанностях), или одна лишь злоба духов и людей, находящихся с ними в сношениях, волшебников. Луллы (в провинции Чако) приписывали, по свидетельству Шарлевуа (История Парагвая, т. 1), все болезни, за исключением ветряной оспы, злобе невидимого животного, не отличающегося, впрочем, от <<духа>>; чикиты же, по свидетельству того же Шарлевуа (т. 2), наоборот, верили, что женщины являются причиною всех болезней. У кафров если волшебник, повелевающий стихиями, не может вызвать дождя, то это значит, что в этом отсутствии дождя повинен какой-нибудь человек, который потом намечается волшебником и убивается (Ausland, май 1849). Так, например, кхандами в Гондване смерть приписывается магическим силам отдельных лиц и богов, ибо смерть согласно их вере не есть необходимый удел человека, который, собственно говоря, бессмертен (совсем как у христиан) и которого настигает смерть лишь в том случае, если он оскорбил какое-либо божество, либо если лица, желающие ему зла и располагающие сверхъестественными силами, причиняют ему смерть. Все, например, смертные случаи, происшедшие от нападения тигров, приписываются подобным лицам, потому что тигр согласно верованиям кхандов (а также и христиан, по крайней мере, правоверных) создан для пользы людей, но разгневанные боги или волшебники пользуются им для своих целей (Ausland, январь 1849 года). Из этих представлений о причине и существе смерти и всех других зол вытекают и человеческие жертвоприношения и все другие беды, которые человек причиняет себе и другим, побуждаемый религией. Разумеется, не только из них, потому что какое бесчисленное количество людей не истребляла только огнем и мечом одна лишь вера в бессмертие! Богу нравится смерть человека, то ли из зависти и чувства мести, то ли из какого другого личного побуждения, стало быть, надлежит в его честь и для его удовольствия убивать людей. Но всего очевиднее наслаждается кровью человека бог войны, ибо лишь от смерти врага зависит победа, милостивый подарок бога войны; неудивительно, стало быть, что этому богу в особенности приносили человеческие жертвы. Богу вообще доставляют удовольствие страдания и муки человека, каковы бы ни были причины этого, и, следовательно, для того, чтобы ему нравиться, чтобы завоевать его расположение, необходимо добровольными жертвами и муками предупредить недобровольные.} 

\hypertarget{13}{(13)} \hyperlink{b13}{Дословно, впрочем, в переводе А. Шлегеля, это значит: я есмь вечное время (le temps infail-lible согласно Вилькинсу во французском переводе 1787 года), я --- всевидящая и всепожирающая смерть, я --- начало будущего.}

\phantomsection
\addcontentsline{toc}{subsection}{К лекции четырнадцатой}
\subsection*{К лекции четырнадцатой}

\hypertarget{14}{(14)} \hyperlink{b14}{<<Ты согласен, стало быть, с бессмысленным мнением номиналистов, которые не признают никакой другой всеобщности, кроме понятий и имен? Да, но я полагаю, что я тем самым присоединяюсь к очень разумному мнению; ибо скажи, ради бога, ты, который признаешь общие сущности и притом признаешь их существующими, что ты воспринимаешь в мире, что не было бы единично? В высшей степени единичен бог (singulanssinius est deus), единичны все его существа, данный ангел, данное солнце, данный камень, короче говоря, нет ничего, что не было бы отдельным существом. Эта мысль, впрочем, встречается также и у других, например у Скалигера. Ты говоришь, что есть, например, человеческая природа, которая всеобща. Но в чем же выражается эта всеобщая природа? Я, по крайней мере, вижу данную человеческую природу Платона, данную человеческую природу Сократа, и всегда эти природы единичны. Если ты более проницателен, то скажи же мне, в чем ты усматриваешь другую, всеобщую природу? так как имеется много индивидуумов, говоришь ты, то имеется, стало быть, во всех них одна общая природа. Так? но как ты это доказываешь? Мне, по крайней мере, довольно, что я имею одну единичную, и тебе довольно, что бы ты ни говорил, одной единичной; что касается меня, то я не вижу никакой природы, которая была бы нам обоим обща, которая была бы одна и та же в тебе и во мне. У тебя есть твое тело, твоя душа, твои отдельные части, твои дарования, у меня также есть мои собственные. Что же такое, следовательно, эта природа, которая одинакова во мне и в тебе?.. Ты говоришь, и притом с большим успехом: не присуща ли та же человеческая природа всем людям даже тогда, когда никто об этом не думает? Но та природа, которая и в самом деле присуща многим, не является ли она и в самом деле всеобщей? Я, разумеется, признаю, что человеческая природа присуща многим даже тогда, когда никто об этом не думает, но я при этом прибавляю, что она многообразна. Ты хотел сказать, что она одна, чтобы утверждать ее всеобщность, но я говорю, что она многообразна, чтобы утверждать существование отдельных натур\dots Скажи, пожалуйста, когда говорится: Платон --- человек, есть ли в этом тезисе сам Платон человек или кто другой? Разумеется, не кто другой, как он сам; и точно так же, когда говорится: Сократ --- человек, то человеком здесь является не кто другой, как Сократ сам или существо, от него отличное; так как поэтому человеческая природа является принадлежностью их обоих, то она не одного рода, а двух родов. Итак, возразишь ты мне, это пустая тавтология, когда говорят: Платон-человек, потому что это само говорит за себя. Я отвечаю, что каждое положение, чтобы быть истинным, должно быть идентичным, потому что ничего нельзя говорить такого о предмете, что не являлось бы им самим или не находилось бы в нем>> (Гассенди, <<Парадоксальные упражнения против Аристотеля>>). Конечно, существует всеобщее, но поскольку оно существует, а не является простым мысленным существом, оно есть не всеобщее, а единичное, индивидуальное, так что можно с таким же правом сказать вместе с реалистами, что оно существует, как и с номиналистами, что его нет. Человечество существует в людях, каждый есть человек; но каждый есть особый, от других отличный, индивидуальный человек. И ты можешь лишь в мыслях, но не в действительности отделить то, чем я отличаюсь от других, от того, в чем я с ними схож, следовательно, индивидуальное от всеобщего, не превращая меня в ничто. Действительное есть абсолютное, неотличимое единство; во мне нет ни одной точки, ни одного атома, которые бы не были индивидуальны. Правильно говорит уже поэтому Лейбниц в своей схоластической диссертации <<О природе индивидуума>>  что принцип индивидуации каждого индивидуума это вся его собственная сущность. То, что теологи говорят о боге, а именно, что в нем идентичны субъект и предикат, бытие и сущность, что о нем ничто не может быть сказано, кроме того, что он есть,-это поистине относится и к индивидуальности, к действительности. Но мышление отделяет то, в чем я схож с другими, от того, чем я от них отличаюсь, благодаря чему я индивидуум, следовательно, предикат от субъекта, имя прилагательное от имени существительного, и делает его самого существительным по той простой причине, что как для его природы --- потому что индивидуум, субъект не может этого воспринять --- так и для его назначения прилагательное есть главное. Поэтому и бог для абстрактного мышления есть главный предмет, главное существо, хотя, как я это показал в этих лекциях, как и в другом месте, он не что иное, как Thesaurus eruditionis scholasticae, Lexicon philosophi, Catholicon seu lexicon ex diversis rebus contractum, то есть собрание имен, имен прилагательных без существа, без материи, без субстанции, собрание, которое, несмотря на это, делается субстанцией --- и притом высшей субстанцией. С точки зрения абстрактного, уже переполненного всеобщностями мышления, выведение общего из единичного представляется неразумным, бессмысленным, ибо с общим в мышлении соединяется понятие существенного и необходимого, с единичным же --- понятие случайного, исключительного, безразличного. Мышление подводит, например, бесконечное множество друг с другом рядом лежащих песчинок под общее и коллективное понятие: песчаная куча. Образуя это понятие, я собираю одним взглядом песчинки в кучу, не делая между ними различия, и обозначаю в противоположность к этой куче, как будто бы она была нечто самостоятельное, песчинки, которые я в мыслях или руками одну за другой отбрасываю как отдельные, случайно тут находящиеся, несущественные, потому что они могут быть отброшены без того, чтобы куча перестала быть кучей. Но не являются ли и прочие песчинки кучи отдельными единицами? Что же такое куча, как не множество именно отдельных единиц? Не уничтожается ли она сама, если я отбрасыванию отдельных песчинок не ставлю никакого предела? Но где этот предел? Там, где мыслителю становится скучно возиться с единичными песчинками. Он одним произвольным прыжком перескакивает от песчинок к песчаной куче, то есть от отдельного к общему. Обще --- бесконечное, абсолютное мысли, единично --- бесконечное, абсолютное чувственности, действительности, ибо существует не только данное отдельное, но и все отдельное, но все отдельное неуловимо, ибо оно имеет свое бытие лишь в бесконечности времени и пространства. Ограничено данное место, но кроме него есть бесчисленное количество других мест, уничтожающих его ограниченность; ограничено данное время, но этот предел его теряется в потоке прошлого и будущего времени. Но как мышление, по крайней мере абстрактное, устраняет этот предел? Качественным изменением понятий; оно ограниченности данного места противопоставляет вездесущность, то есть не пространственное бытие, ограниченности данного времени --- вечность, то есть не временное бытие. Так мышление вообще без дальних слов перескакивает от отдельного к общему и делает его самостоятельным существом, от первого существенно отличным. <<Люди гибнут, но человечество остается>>. В самом деле? Но где же остается человечество, если нет людей? Кто же, таким образом, те <<люди, которые гибнут>>? Те, кто уже умер и кто живет? Но кто же --- человечество, которое остается? Грядущие люди. Но мышление или человек, когда он мыслит, принимает всюду, как это мы видим на данном примере, любую определенную сумму за всю сумму, нескольких индивидуумов за всех и ставит поэтому на место этих пропущенных будущих индивидуумов, с которыми он в мыслях уже разделался, покончил, родовое понятие, человечество. Голова есть палата представителей вселенной, родовое понятие --- представитель, заместитель индивидуумов, которые в своей бесконечной действительности не находят себе места в голове. Но именно потому, что родовое понятие есть представитель индивидуумов и что мы при словах --- индивидуумы, отдельные --- думаем лишь о тех или других отдельных, нам представляется --- по крайней мере, в том случае, если мы имеем голову, переполненную родовыми понятиями, и воззрение действительности нам стало чуждым --- как нельзя более естественным и разумным выводить отдельное из общего, то есть действительное из абстрактного, сущее из мыслимого, природу из бога. Тем не менее с этим выведением дело обстоит таким же образом, как со средневековой государственно-правовой фикцией, которая делает верхушку государства его фундаментом, согласно которой император, ведь император есть родовое понятие в области политики, в Риме только император назывался и был общественным лицом, остальные все были лицами частными, --- император есть источник и основа всякого права, всякой власти, всякого благородства, тогда как первоначально или согласно действительной истории происхождения имело место как раз обратное: <<власть масс>>  то есть по понятиям старых времен --- <<власть свободных людей>>  предшествовала монархическому принципу.}

\hypertarget{15}{(15)} \hyperlink{b15}{При мышлении и высказывании, когда уже ради одной преемственности мыслей целое разрывают на части, коим придают самостоятельное существование, когда у индивидуума вырывают желудок из тела, сердце из груди, мозг из головы, когда образуется застывшая идея изолированной индивидуальности, то есть простого призрака, продукта схоластической мысли, --- возможно, разумеется, и обратное, а именно, чтобы индивидуум имел своей предпосылкой общее понятие; ибо что такое индивидуум без содержания, без качеств, талантов или сил, которые делают человека человеком, но которые мы в мыслях об индивидууме различаем и делаем самостоятельными, как родовые понятия? То же, что нож, от которого при абстрагировании взяли прочь клинок. Конечно, идея или дело, ради которого я живу, не гибнет вместе со мною; конечно, разум не перестает существовать, если я перестаю думать, но лишь потому, что другие индивидуумы подхватывают это дело, другие индивидуумы думают вместо меня. <<Индивидуумы меняются, интересы остаются>>  но только потому, что другие имеют тот же интерес, что и я, и так же, как и я, хотят быть образованными, свободными, счастливыми людьми.}

\phantomsection
\addcontentsline{toc}{subsection}{К лекции шестнадцатой}
\subsection*{К лекции шестнадцатой} 

\hypertarget{16}{(16)} \hyperlink{b16}{О моих в этих лекциях высказанных политических взглядах только следующее коротенькое замечание. Уже Аристотель в своей <<Политике>>  трактующей почти все вопросы современности, но, как это само собою разумеется, трактующей их в духе древнего мира, говорит, что нужно быть не только знакомым с лучшим государственным устройством, но и знать, для каких людей оно годится, ибо и лучшее не для всех подходит. Если поэтому мне указывают с исторической, то есть связанной с временем и пространством, точки зрения на конституционную монархию, разумеется на истинную, как на единственно для нас подходящую, возможную и потому разумную государственную форму, то я с этим вполне согласен. Если же независимо от пространства и времени, то есть данного определенного времени (и тысячелетия являются определенным временем), данного определенного места (ведь и Европа есть только место, точка в мире), монархию изображают как единственно или абсолютно разумную государственную форму, то я протестую против этого и утверждаю, что в гораздо большей степени республика, разумеется демократическая, является той государственной формой, которая непосредственно представляется разуму, как соответствующая человеческому существу и, следовательно, истинная, что конституционная монархия есть птолемеевская, республика же --- коперниковская система политики и что поэтому в будущем человечества Коперник так же победит Птолемея в политике, как он его уже победил в астрономии, хотя некогда птолемеевская система мира также выдавалась философами и учеными за непоколебимую <<научную истину>>.}

\phantomsection
\addcontentsline{toc}{subsection}{К лекции двадцатой}
\subsection*{К лекции двадцатой}

\hypertarget{17}{(17)} \hyperlink{b17}{То же самое относится впрочем не только к язычникам, но и к древним израильтянам. Когда даниты отняли у Михея его идола, он вслед им крикнул: <<Вы взяли моих богов (или согласно другим --- моего бога), которых я сделал>>. Впрочем, отнюдь не один только скульптор (пластический делатель изображений), но и --- и притом в особенности духовный --- делатель изображений --- поэт --- есть делатель богов. Достаточно вспомнить только Гомера и Гезиода! Овидий в четвертой книге своих писем с Понта говорит буквально следующее: <<Боги делаются также в стихах (или при помощи стихов) (или поэтами)>>. Di quoque carminibiis (si fas esfc dicere) fiunfc. Когда утверждают, что религиозный человек почитает не самое изображение или самую статую как бога, но лишь бога в них, то это различение лишь в той мере обосновано, что бог существует также и вне статуи и изображения, а именно в голове, в духовном мире религиозного человека, лишь в той мере, стало быть, в какой вообще существует различие между существом, как существом чувственным, действительным, и им же, как существом представленным, духовным. Но вне этого это различение лишено основания. То именно, в чем человек почитает бога, есть его истинный, действительный бог; бог же, имеющий свое бытие над этим и вне этого, есть лишь призрак представления. Так, протестантизм, по крайней мере старый, ортодоксальный, находит и почитает бога в Библии, то есть он почитает Библию как бога. Протестант почитает, разумеется, не книгу, как книгу, подобно тому, как король ашантиев в Африке почитал Коран, хотя и не понимал в нем ни слова; он почитает ее содержание, слово божие, то слово, в котором бог высказал свое существо; но ведь это слово существует, по крайней мере не искаженное, только в Библии. Слово божие есть и божья мысль, божья воля, божье мнение, --- следовательно, божье существо; содержание священного писания поэтому есть содержание существа божьего. <<Нужно сделать все, --- говорит Лютер в своей проповеди, которую он держал в пасхальный понедельник 1530 вода в Кобурге, --- чтобы мы знали пользу и употребление писания, а именно, чтобы мы знали, что оно есть свидетельство во всех своих частях о Христе и к тому же высшее свидетельство, далеко превосходящее все знамения и чудеса, как на это указывает Христос в притче о богатом (Лука, 16, 29-31): у них Моисей и пророки, и если они и им не верят, то воистину они еще много меньше поверят, если бы кто из мертвых воскрес. Мертведы могут нас обмануть, но этого не может сделать писание. Вот это-то обстоятельство и заставляет нас так высоко ценить писание, и притом Христос сам считает его в данном случае лучшим свидетельством. Следовательно, он хотел сказать: вы читаете пророков и все же не верите? Правда, это бумага и чернила, и тем не менее это самое важное свидетельство. Так и Христос больше считает нужным ссылаться на него, чем на свое появление>> и так далее Кто же после этого станет удивляться тому, что в протестантской церкви <<сила божественного слова>>  или <<божественная сила священного писания>>  сделалась главным предметом теологических споров, что препирались всячески из-за <<моральной, естественной, сверхъестественной, физической, сходной с физической, объективной, субъективной силы божественного слова>>  например, учили тому, что <<божественная и сверхъестественная сила, которой человек просвещается и обращается в истинную веру, находится не возле священного писания, а в нем самом (поп adesse scripturae, sed inesse) и что человек обращается в веру не силой, сосуществующей с писанием, а силой, существующей в самом писании (И. Р. Шлегель, История церкви восемнадцатого века), что определенно утверждалась божественность священного писания. Так, генеральный суперинтендент и старший пастор Г. Нитше написал в первой четверти восемнадцатого века две книги на тему о том, является ли священное писание самим богом.} 

\hypertarget{18}{(18)} \hyperlink{b18}{Разумеется, бог, как уже достаточно было показано, есть также и образ природы, ее запечатленное в образе существо, --- природа ведь есть первый, первоначальный предмет религии, остающийся ее постоянной основой, но человек, и именно стоящий на точке зрения религии, воображает, представляет себе природу лишь по мерке своего собственного существа, так что запечатленное в образе существо природы есть лишь опредмеченное существо человека.}

\phantomsection
\addcontentsline{toc}{subsection}{К лекции двадцать первой}
\subsection*{К лекции двадцать первой}

\hypertarget{19}{(19)} \hyperlink{b19}{Для горения необходима, разумеется, температура, различная соответственно различию горю чего материала, но и для поэзии нужна определенная температура, меняющаяся соответственно индивидуальным различиям, --- внутренняя и внешняя теплота, чтобы вызвать огонь воодушевления. Когда мы воспламеняемся духовно, мы воспламеняемся и физически; нам бывает жарко и при спокойном положении в холодной комнате. И, наоборот, физический огонь вызывает также поэтический. Там, где кровь стынет от холода, не бьется и пульс поэзии.}

\hypertarget{20}{(20)} \hyperlink{b20}{<<К фантастическим видениям лихорадочного больного, --- говорит Г. Банкрофт в своей ,,Истории Соединенных Штатов Америки``  --- прислушивается целая деревня или целое племя, и вся нация скорее отдала бы и свой урожай, и свои драгоценные меха, и свою охотничью добычу, и все прочее, чем воспротивилась бы исполнению сна. Сон должен быть выполнен, хотя бы он потребовал, чтобы женщины отдавались всем и каждому. Вера в мир духов, раскрывающийся через сновидения (вернее: вера в сновидения, которые представлялись человеку как духи, боги, сверхчеловеческие существа), была всеобща. На Верхнем Озере племяннику одной индианки приснилось, что он видит французскую собаку, и женщина среди зимы отправилась за 400 верст снежного пути, чтобы достать такую собаку>>. Что за героизм! А между тем он был вызван только сновидением!}

\hypertarget{21}{(21)} \hyperlink{b21}{Так и в неоднократно упоминавшейся <<Истории Парагвая>> Гуараниса рассказывается, что часто люди умирали из одного страха перед волшебством. Жители Бразилии также <<до такой степени боятся злых духов, что некоторые из них умирали при виде представившегося им привидения>> (Бастхольм, Исторические данные к вопросу об изучении человека в его диком и неразвитом состоянии, 4 часть).} 

\phantomsection
\addcontentsline{toc}{subsection}{К лекции двадцать второй}
\subsection*{К лекции двадцать второй}

\hypertarget{22}{(22)} \hyperlink{b22}{Бог выполняет то, что человек желает; он есть существо, отвечающее желаниям человека; он отличается от желания лишь тем, что в нем является действительностью то, что в том есть лишь возможность; он сам есть исполненное или близкое к исполнению желание, или опредмеченное и ставшее действительностью существо желания. Кедворт спрашивает в своей <<Интеллектуальной системе>>: <<Если нет бога, то откуда же происходит, что все люди хотят иметь бога?>> Но следует скорее спросить наоборот: если бог есть, то почему и зачем людям его еще желать? То, что есть, не составляет предмет желания; желание, чтобы был бог, есть как раз доказательство того, что его нет. <<У них (богов), --- говорит греческий поэт (Пиндар) у Плутарха, --- нет болезней, они не стареют, они не знают труда, они избавлены от переправы через Ахерон>>. Можно ли отчетливее выразить, что боги --- это желания людей? <<Ничего, --- говорит Веллей Патеркул, --- не могут люди желать от богов, ничего боги не могут людям дать\dots чего бы Август\dots не предоставил Римскому государству>>. <<То, что подлежит изучению, --- говорит Софокл (Плутарх, О счастии), --- то я изучаю; что может быть найдено, я ищу; то, чего можно желать (или желанное, достойное желания), я испрашиваю у богов>>. <<У Анны не было детей, господь замкнул ее чрево>>  то есть она была бесплодна. <<Тогда встала Анна и молилась господу: если ты своей служанке дашь сына, то я на всю жизнь отдам его господу. И бог внял ей. Господь исполнил мою просьбу, с которой я к нему обращалась. Она забеременела и родила сына, и назвала его Самуилом>>  --- то есть испрошенным у бога, Theaiteton, как Иосиф переводит Самуила (Клерикус, Комментарий к Самуилу). Клерикус замечает к этому месту, что когда идет речь о словах: <<господь замкнул ее чрево>>  не приходится думать о чуде, то есть об особенном действии всемогущества божьего, что, следовательно, и раскрытие ее чрева не было чудом. Однако что же такое бог, что такое молитва, если они не имеют другой силы и другого предназначения, как развивать уже сформированные природные зародыши? Вера не позволяет вдаваться ни в какие анатомически-физиологические вопросы и исследования. Согласно вере бог или божественная сила молитвы, сила благочестивого желания, была причиной беременности Анны. Бог, который не способен ничего создать, который может лишь высидеть яйца, снесенные натурализмом, не есть бог. Бог в такой же мере стоит над природой, он так же свободен, так же мало связан анатомически-физиологическими условиями, как и желание, как и фантазия человека. Одиссей --- чтобы привести еще несколько примеров и доказательств связи между богом и желанием --- говорит Эсмею: <<Да даруют тебе, друг, Зевс и другие бессмертные боги то, что ты всего больше желаешь, за то, что ты меня так хорошо принимаешь>>. А в двадцать первой песне Одиссеи коровник говорит Одиссею: <<Отец Зевс, о, если бы ты исполнил желание, и герой вернулся бы, и ему указывал бы путь бессмертный!>>. Юпитер в овидиевых Фастах говорит беотийскому крестьянину Гириею, гостеприимно угостившему его вместе с его братом Нептуном и с Меркурием: <<Если у тебя есть какое-нибудь пожелание, говори; ты все получишь, или все тебе будет дано>>. Старец ответил: <<У меня была дорогая супруга, но сейчас ее покрывает земля. Я поклялся вашим именем не касаться никакой женщины, кроме нее. Я держу свое слово, но мое сердце раздвоено, я охотно бы сделался отцом и не хочу сделаться супругом>>. Боги сообща исполнили его желание: они помочились в бычачью кожу, и из божественной мочи по прошествии десяти лун произошел маленький мальчик. Если мы оставим в стороне водянистые дополнения к этой басне, то она говорит нам то же самое, что при подобных же обстоятельствах говорит и Ветхий завет: <<Есть ли для Господа что невозможное?>>  то есть, есть ли что невозможное для силы воображения человеческого сердца и желания?}

\phantomsection
\addcontentsline{toc}{subsection}{К лекции двадцать третьей}
\subsection*{К лекции двадцать третьей}

\hypertarget{23}{(23)} \hyperlink{b23}{Я не могу удержаться от того, чтобы не включить в эти примечания в высшей степени интересный по своей простоте и искренности индусский гимн к воде из Ригведы (Статья Кольбрука о священном писании индусов. Перев. Л. Полея, с приложением отрывков из древнейшей религиозной поэзии индусов). <<Воды, богини, напояющие коров, вас призываю я; мы должны приносить жертвы рекам. В воде заключено бессмертие (нектар), в воде -целительная сила; вы, жрецы, неустанно славьте воду! Сома возвестил мне, что в воде все целительные средства, что Агни (огонь) все осчастливливает и что вода все исцеляет. Воды! Пропитайте мое тело целебными средствами, уничтожающими болезни, чтобы я мог долго еще созерцать свет солнца. Воды! Возьмите от меня прочь все, что есть злого во мне, насилия, которые я совершил, проклятия и ложь, которые я произнес. Сегодня я почтил воды, я соединился (купаясь) с сущностью воды. Приди ты, Агни, наделенный водою, окружи меня блеском!>>} 

\phantomsection
\addcontentsline{toc}{subsection}{К лекции двадцать четвертой}
\subsection*{К лекции двадцать четвертой}

\hypertarget{24}{(24)} \hyperlink{b24}{Поскольку родители --- существа частные, боги же --- публичные, касающиеся и охватывающие все государство, всех граждан, то, конечно, первые стоят ниже вторых, ибо дом или семейство (то есть то или другое семейство) может, как говорит Валерий Максим, быть уничтожено без того, чтобы государство погибло, между тем как гибель города или государства необходимо ведет за собою гибель всех пенатов. Поэтому в ряду обязанностей Цицерон уделяет обязанностям по отношению к богам первое место, обязанностям по отношению к отечеству- второе, обязанностям по отношению к родителям третье. Но различие в чине или ранге не составляет различия по существу. Кроме того, первое в шкале мыслей не есть первое в шкале природы. Источник святости отечества есть святость собственного очага, пенатов, отцов, а источник святости богов --- святость отечества, ибо ведь главная основа их почитания та, что они боги отечества, что они di romani (римские боги), но прежде, чем возник Рим, не было и римских богов. <<Что более неприкосновенно, --- говорит Цицерон, или автор ,,Речи про себя``  --- что во всякой религии более охраняется, как не дом каждого гражданина?.. Это убежище так неприкосновенно, что никому не дозволено кого-либо вырвать из него>>. Какой контраст между этим преклонением языческого государства перед святостью домашнего права и той грубостью, тем бесстыдством, с которым христианское государство и к тому же на основании самых легких подозрений, врывается в дом, как вор, и тащит хозяина дома в тюрьму.}

\phantomsection
\addcontentsline{toc}{subsection}{К лекции двадцать пятой}
\subsection*{К лекции двадцать пятой}

\hypertarget{25}{(25)} \hyperlink{b25}{Так как древние язычники и, в частности, греки смотрели на все не только телесные, но и духовные блага и силы, как на богов или на дары богов, и сознавали, что без добродетели и рассудка или мудрости нет счастья (<<пагубна, --- говорит, например, Гезиод, --- для бедного смертного несправедливость>>  и также Солон: <<я хочу, конечно, иметь богатства, но не несправедливым путем>>), то предметами их желаний и молитв были не только материальные, но и духовные блага. Ведь начинают же поэты свои песни постоянно молитвами, обращенными к богам! Впрочем, они не знают добродетели, независимой от внешних благ, отсюда и жалоба поэтов на несчастье бедности, так как она портит людей, понуждает их к низкому образу мыслей и действий (<<о, Плутос (богатство)! прекраснейший и любезнейший из богов, --- говорится, например, у Феогнида, --- с твоей помощью, если я теперь и плох, я буду хорошим человеком>>), и точно так же они не знают и счастья, независимого от телесных благ. Так, в одной греческой застольной песне-молитве, обращенной к Гигиене, богине здоровья, говорится: <<Без тебя никто не счастлив!>>. И даже Аристотель еще не знает добродетели и счастья, независимых от внешних <<временных благ>>.}

\hypertarget{26}{(26)} \hyperlink{b26}{Правда, язычники обожествляли и бедность, несчастье, болезнь. Различие лишь в том, что хорошее есть нечто желанное, а худое или злое нечто проклинаемое. Так, у Феогнида, например, говорится: <<О, жалкая бедность! Отчего ты не хочешь уйти к другому человеку, отчего ты любишь меня против моей воли? Уйди же от меня!>>.}

\hypertarget{27}{(27)} \hyperlink{b27}{Так как я в <<Сущности христианства>> и в других своих сочинениях не морализировал, не подымал воя по поводу греха и даже не посвятил ему особой главы с точным обозначением его имени в заголовке, то мои критики меня упрекнули в том, что я не понял христианства. Но, как и в других кардинальных вопросах, --- это, разумеется, представляет собою утверждение без доказательства, но у меня нет ни времени, ни охоты для подобного рода доказательств, для несущественной и беспредметной критики, --- как, следовательно, и в других кардинальных вопросах мои остроумные противники мне поставили в упрек как раз мой правильный взгляд и мой такт, так и в данном вопросе. Против моей воли разросшееся примечание к этому примечанию см. в конце, после э 28. Как добродетель, или мораль, не является сама по себе целью и предметом христианской любви, так же точно и порок, или грех, не есть сам по себе предмет христианской ненависти. Бог есть цель христианина; но бог не есть --- или, по крайней мере, есть не только моральное существо; только моральное существо есть голая абстракция, голое понятие, а понятие не имеет существования. Бог же согласно вере есть существо существующее, действительное существо. Бог, разумеется, свят, добр, безгрешен; он понимает свою моральную доброту или совершенство, но лишь потому, что он есть совокупность всех благ; он ведь не что иное, как олицетворенное и опредмеченное существо силы воображения, наделенный и украшенный всеми сокровищами, всеми благами и совершенством природы и человечества. Моральное совершенство в боге есть не кантовская добродетель, не добродетель в противоречии со склонностью, с влечением к счастью; бог как совокупность всех благ есть блаженство; кто поэтому имеет бога своей целью, тот, конечно, имеет своей целью безгрешность, моральное совершенство, но в то же время --- непосредственно и неотделимо --- и блаженство. <<Когда я, --- говорит, например, Августин в 10-й книге своей ,,Исповеди``, --- когда я тебя, моего бога, ищу, я ищу блаженной жизни. Бог у христиан означает высшее благо, но точно так же и vita aeterna --- вечная или блаженная жизнь --- означает высшее благо. Христианин никоим образом не отвергает один только грех, или грех сам по себе, но он отвергает в то же время и его условия, его причины, его сообщников, отвергает все соотношение вещей, при котором грех является необходимым звеном: мир, природу, плоть. Грех ли любить женщину? Нет; и однако на небе, этой цели христианских желаний, нет такой любви. Грех ли еда и питье? Нет; но это --- нечто не божественное, поэтому исключенное из идеала христианства. Сущность христианства, как я его совершенно верно философски обозначил в сочинении, носящем этот заголовок, есть субъективность в хорошем и худом смысле этого слова --- субъективность, то есть душа или личность человека, избавленная от пределов, поставленных природой, и тем самым, разумеется, освобожденная от радостей, но также и от тягот плоти или, вернее, обожествленное, неограниченное, сверхъестественное влечение к счастью.}

\phantomsection
\addcontentsline{toc}{subsection}{К лекции двадцать седьмой}
\subsection*{К лекции двадцать седьмой}

\hypertarget{28}{(28)} \hyperlink{b28}{Так, например, одна старинная христианская книга духовных песнопений говорит: <<Хочешь ты меня положить на одр болезни? Я хочу. Должен ли я быть в нужде? Я хочу\dots И предашь ты меня смерти? Я хочу. Да исполнится твоя воля, о боже! Хочешь ты меня иметь на небе? Господи, это есть исполнение всех моих желаний. Должен ли я затем отправиться в ад? Я знаю, господи, это не есть твое желание. Что твоя воля этого не должна желать, того пожелала смерть твоего сына>>. В другом песнопении Хр. Тиция значится: <<Помощь, которую он отложил, он еще не отменил; если он не помогает в любой момент, то он помогает, когда это нужно>>. <<Ни одно несчастье, --- говорится в другом песнопении, --- не длится вечно; оно, наконец, прекращается>>. И в другом: <<Как богу угодно, так пусть и будет, я оставляю заботы птичкам. Если сегодня ко мне в дом не придет счастье, оно будет у меня завтра. Что мне уготовано, то останется неприкосновенным, хотя бы оно исполнением затянулось; благодари бога с усердием; что должно быть, то будет. Он мое счастье устроит>>.}

\hyperlink{b28}{А в одном песнопении Н. Германа говорится: <<Будь предан господу богу, пусть он делает, как ему угодно, ему ничто не нравится, что нам не было бы полезно, он нам всем хочет всего доброго>>. Наконец, в песнопении П. Гергарда: <<Страдания христиан имеют благостный смысл; кто здесь временно плакал, не будет жаловаться вечно, его ожидает совершенная радость в саду Христа, которому одному известна его жизнь>>.}

(К примеч. (27)). Всякая антикритика бесполезна, скучна, противна, потому что критики в своем старании не попять писателя, а опровергнуть его принимают видимость за сущность, без критики делают словесное существенным, местное --- универсальным, частное --- характерным, временное --- постоянным, относительное --- безусловным, связывают то, что друг к другу не относится, необходимо же связанное разъединяют --- словом, все перемешивают произвольно и в беспорядке и поэтому предоставляют антикритике не философскую, а лишь филологическую работу по толкованию цитат. Критики возлагают на автора обязанность научить их чтению, прежде всего чтению книг, написанных с умом; ибо остроумная манера писать состоит между прочим в том, что она предполагает ум и в читателе, что она не говорит всего, что она предоставляет читателю самому сказать себе о тех взаимоотношениях, условиях и ограничениях, при которых данное положение только и имеет значение и может быть мыслимо. Если поэтому читатель из тупости ли или из желания во что бы то ни стало раскритиковать автора не заполнит эти пробелы, эти пустые промежутки, если он самодеятельно не восполнит автора, если все его понимание и рассудок направлены лишь против него, но не за него, то неудивительно, что сочинение, и без того беззащитное, уничтожается до конца критическим произволом. Так, чтобы подтвердить это мое суждение несколькими образчиками, я укажу на профессора Шадена, который существенным, окончательным исходным пунктом своей критики моего <<понятия мышления>> делает один момент в моем развитии --- одну рецензию от 1838 года, а затем сочетает его, но самым произвольным и некритическим образом с положениями противоположного содержания из моих позднейших сочинений. Чем является, например, на стр. 47, параграф 24-й из <<Основных положений>>  который начинается словами: <<Правда, еще считается признанным, что душа ощущает тождественность с самой собой>>? Органическим посредствующим звеном между мыслями 1838 года и позднейшими <<дополнениями, которые выявляют себя как нечто во всех отношениях удивительное и в большей или меньшей степени противоречащее прежним определениям>>  является прежде всего частью прямая, частью косвенная критика упомянутой рецензии и ее точки зрения в статье <<Против дуализма>>  где я устанавливаю психологический генезис представлений о сверхчувственности, о нематериальности души, где я объясняю, как происходит то, что человек не может согласовать действие мысли с действием мозга; далее, доказательство, подтверждаемое бесчисленными примерами, что сверхчувственное существо есть не что иное, как нечувственное (отвлеченное или воображаемое) чувственное; наконец, темой всех моих позднейших сочинении является человек как субъект мышления, тогда как прежде мышление само было для меня субъектом и рассматривалось мною как нечто самодовлеющее. Но через все эти посредствующие звенья мой некритический критик перескакивает, абстрагирует себе из нескольких произвольно подобранных положений противоположность между духом и материей, и строит вслед за тем на этой основе воздушный замок своей критики, производимой им над <<понятием мышления>>. Столь же произвольна и некритична и его критика <<понятия бытия>>. Так, например, он говорит: бытие <<превращается (у Ф.) в тень\dots принижается до одной части мыслящего, до его яйности. Неудержимо необходимым становится тезис: <<материю нельзя упразднить, не упразднив разума, нельзя признать, не признав разума>>. Боже, как подходит сюда это положение! Ведь это --- лишь обобщенный исторический факт. И как из него вывести растворение бытия в мышлении? <<Правда, еще говорят, --- продолжает критик, --- быть --- значит быть предметом>>  но при этом тотчас же прибавляет: --- Следовательно, бытие предполагает наличность сознания. Нечто есть действительное нечто лишь как объект сознания\dots Следовательно, сознание есть мерило всякого существования>>. Как может <<добросовестный>> критик просмотреть, что это положение есть критика фихтевского идеализма, потому что сейчас же в следующей фразе значится: <<так, в идеализме осуществляется сущность теологии!>>. До какой степени вся его критика не попадает в цель, видно, впрочем, уже из того, что он содержание моих сочинений сводит к абстрактным понятиям бытия и мышления, тогда как с моей точки зрения вся философия о мышлении без мыслящего существа, о бытии без сущего существа, раскрываемого только чувством, --- вся философия вообще, которая берет вещи не <<in flagrant!>>, есть пустая и бесплодная спекуляция; я ведь определенно на место бытия ставлю природу, на место мышления --- человека, и точно также не абстрактную, а драматическую психологию, то есть психологию в соединении с предметами, в которых психика человека выявляется во всей своей полноте, --- следовательно, лишь в своих предметных выражениях, в своих действиях. Господин фон-Шаден, наверное, убежден, что он меня опроверг, по крайней мере раскритиковал; я же говорю ему, что он обо мне фантазировал и притом очень дико. Еще несколько слов о <<критике>> господина профессора Шаллера. И на эту <<критику>>  если бы я захотел заняться настоящей антикритикой, я мог бы ответить филологическим расчленением моих собственных сочинений, ибо ее автор дал до такой степени мало соответствующую истине оценку даже моего формального существа, что всегда верно лишь то, что противоположно всем его суждениям и построениям, и в своей мелочно-критической злобе идет так далеко, что отрицает или во всяком случае порицает даже самые простые и очевидные положения, являющиеся лишь выраженными в словах историческими фактами, как, например, тот факт, что естественная религия есть первоначальная религия. Однако я оставляю в стороне все отдельные упреки, все противоречия, все бессмысленное, что мой критик частью выводит из моих мыслей, частью находит непосредственно в них выраженным. Я выдвигаю лишь один пункт, но это кардинальный пункт, вокруг которого все вертится. Это --- понятие индивидуума. Существенное различие между моей точкой зрения и точкой зрения, представленной моим критиком, заключается в следующем: он отличает род или общее от индивидуума, противопоставляет ему общее как само себя полагающее, то есть самостоятельное, объективное существо, поэтому индивидуум для него есть отрицательное, конечное, относительное, случайное, стало быть, позиция индивидуума есть позиция <<произвола, безнравственности, софистики>>; я же отождествляю род с индивидуумом, индивидуализирую общее, но поэтому обобщаю индивидуума, то есть расширяю понятие индивидуума, так что индивидуум для меня есть истинное, абсолютное существо. С точки зрения г. Шаллера, человек или индивидуум имеет в себе <<самое себя полагающую, в себе необходимую общность>>  благодаря чему он может практически и теоретически выйти за собственные пределы; имеет <<принципиальную общность <<Я>>  являющуюся основой речи, <<существенную общность, при посредстве которой он выводится за границы своих индивидуальных наклонностей>>  при посредстве которой он преодолевает свой <<индивидуальный произвол>>  как, например, в нравственности; благодаря чему индивидуум, как, например, <<при художественном воодушевлении увлекается идеей, а не своими собственными индивидуальными представлениями>>  благодаря чему, например, в знании мои мысли <<являются не только моими, но и выражают сущность, представляют собой энергию опосредствования>>. Мы имеем здесь, следовательно, два существа в человеке: общее и индивидуальное, тогда как, по моему мнению, индивидуальность охватывает всего человека, сущность человека --- одна, общая сущность сама есть сущность индивидуальная. Правда, человек в самом себе делает различия он ведь сам явственно состоит из отличающихся Друг от друга и даже противоположных органов и сил, --- но то, что он в самом себе отличает, в такой же мере принадлежит к его индивидуальности, в такой же мере является составной частью ее, как и то, от чего он это отличает. Если я борюсь с какой-либо наклонностью, то разве та сила, при помощи которой я борюсь, не является в такой же мере силою моей индивидуальности, как и моя наклонность, только силой особого рода? Выражение: выйти за свои пределы, преодолеть самого себя, находит себе объяснение в других выражениях, как-то: превзойти самого себя. Может ли, в самом деле, индивидуум превзойти самого себя? Не есть ли это превосходящее лишь моя, только теперь созревшая, развившаяся, индивидуальная сила или способность? Но большинство людей превращают слова в сущности. Голова, местопребывание интеллекта, есть нечто совершенно иное, чем живот, местопребывание материальных страстей и потребностей. Но распространяется ли мое существо лишь до пупка, а не до головы? Есть ли только содержание моего чрева содержание моей индивидуальности? Разве я в голове уже больше не я? Не обнаруживается ли мое я более всего там? Разве мышление не есть индивидуальная деятельность, <<индивидуальное состояние>>? Почему же в таком случае оно заставляет меня так напрягаться? Не является ли голова мыслителя, то есть человека, который делает индивидуальную деятельность мышления своей главной и характерной для него задачей, отличной от не мыслящей головы? Или вы, г. профессор, быть может, полагаете, что Фихте философствовал в противоречии со своей индивидуальной наклонностью, что Гете и Рафаэль творили в противоречии с их индивидуальными наклонностями? Но что же делает художника художником, как не то, что его индивидуальные наклонности, представления и воззрения являются художественными? И что же такое идея художника, которой он вдохновляется, как не <<более или менее неопределенный образ другого индивидуума>>  то есть в данном случае произведения искусства, <<или другого индивидуального состояния>> искусства? Что же представляют собою вообще <<индивидуальные наклонности и представления>>? Это --- представления и наклонности, не принадлежащие к данной профессии, к данной точке зрения, к данному делу, но являющиеся такими же существенными, такими же положительными, как и другие. Я сочиняю, например, стихотворение в возвышенном духе, тогда как мне за это время приходят в голову различные комические сцены, к которым я вообще питаю особую склонность, и нарушают мой творческий полет; это --- представления <<индивидуальные>>  которые я должен держать от себя вдалеке, устранять, если я хочу выполнить свою задачу; но они перестают ими быть, как только я их самих делаю предметом особого произведения искусства, как только я уделяю им надлежащее место. 

Перед нами --- живописец; для него в его искусстве заключается основа и опора его материального и духовного или морального существования; но кроме этого призвания, этой, так сказать, супруги, выбранной им по склонности и общественно признанной, имеет он и другие влечения; он также и любитель музыки, верховой езды, охоты и так далее; из-за них он пренебрегает своим подлинным искусством и тем губит себя и свою семью. Эти влечения в данном случае являются, конечно, <<индивидуальными наклонностями>>; но разве они сами по себе заслуживают отвержения? Разве не имеют они признанного, объективного существования в других индивидуумах? Разве нет наездников, музыкантов, охотников по призванию и профессии? Одна из служанок застает случайно открытой шкатулку с драгоценными украшениями своей госпожи; она видит там массу ценных колец; у нее вырывается восклицание: ах! если бы я только могла свои лишенные украшений пальцы украсить столь чудесно. Соблазнительный случаи превращает желание в действие --- бедняга крадет и попадает в тюремный рабочий дом. Является ли эта склонность к драгоценному камню или золотому кольцу сама по себе <<индивидуальной>> и --- что то же самое с точки зрения наших спекулятивных философов --- подлежащей преодолению, греховной, наказуемой? Нет; потому что та же склонность у владелицы драгоценностей считается правомерной, причем предмет этой склонности признается неприкосновенною собственностью. Сквозь золото и драгоценные камни, которыми украшена корона главы государства, мы видим <<индивидуальную наклонность>> несчастной служанки к нарядам и украшениям в виде <<всеобщей страсти>>. Каждый человек вообще имеет массу желаний, наклонностей, вожделений, которым он не может позволить проявиться, потому что они находятся в противоречии с его общественным существом, с его профессией, с его существованием, с его положением, желаний и наклонностей, которые ведут поэтому у него эфемерное, микроскопическое, сперматозоидное существование, ибо у него для их удовлетворения не достает пространства и времени или других средств, но которые у других индивидуумов играют весьма значительную роль. Но умозаключение, делаемое из отрицания этих желаний и наклонностей к <<самое себя полагающей общности>>  к чистому призраку без наклонностей, без желаний, без индивидуальности, есть не что иное, как старый, лишь прикрытый логическими формулами или фразами дуалистический и фантастический скачок или умозаключение от мира к не мирскому существу, от материи к нематериальному, от тела --- к существу, лишенному телесности, потому что существо, которому я приношу в жертву эти наклонности и желания, есть само не что иное, как индивидуальная или даже самая, что ни на есть, индивидуальная наклонность и самое, что ни на есть, индивидуальное призвание, которым я оказал предпочтение перед другими, развив его усердием и упражнением до высоты мастерства и доведя до общественного признания; различие вообще между <<индивидуальным>> и общим --- относительно, неуловимо, причем то, что во мне является частным лицом, в других есть лицо общественное, общее. Не были ли вы сами, г-н профессор, раньше приват-доцентом? Но что такое приват-доцент? Индивидуум, желанию которого преподавать университетские <<власти>> из ученого чванства и высокомерия не хотят дать осуществиться как незаконной <<индивидуальной склонности>>? Но вы теперь, слава богу, профессор, и ваша прежняя частная склонность стала теперь для вас даже служебной обязанностью, <<нравственной необходимостью>>. Но какая же разница между <<тогда>> и <<теперь>>? Как профессор не хочет ничего знать о том, что он был некогда приват-доцентом, так же точно и обязанность, однажды отделившись от жизни и взобравшись на кафедру абстрактной морали, не хочет знать того, что она произошла из <<индивидуальной склонности>> человека. Откуда же, в самом деле, ведет свое происхождение, например, закон, а следовательно, и обязанность не убивать? От <<категорического императива>>? Да, но этот категорический императив гласит: я не хочу умереть, я хочу жить, и то, что я хочу, то ты --- должен, а именно: оставить меня жить. Откуда закон, а следовательно, и обязанность не красть? От самополагающейся общности? Почему бы не от самосадящегося зада? Владеть (besitzen) --- значит на чем-нибудь сидеть (sitzen), а сидеть нельзя, не имея седалища. Ты не должен красть --- означает на самом деле не что иное, как то, что ты не должен вырывать сидение из-под моей индивидуальной склонности и произвола, безразлично, есть ли это сиденье софа или соломенный мешок, королевский трон или папский ночной горшок, ты не должен воровать его из-под моего зада, этого последнего аргумента и фундамента права собственности! Откуда происходит то, что в немецких законах охота играет такую важную роль, что кража или убийство прирученного оленя карается строже, чем убийство раба? Из <<индивидуальной склонности>> немцев к охоте. Но что является несправедливым, варварским в германских законах об охоте? Склонность к охоте? Нет! Но то, что важные господа признают лишь собственную склонность законной властью, у всех же остальных они ту же склонность осуждают как лишь индивидуальную, в духе наших философов. <<Князья и дворяне, --- говорит Себ. Мюнстер в ,,немецкой истории Вирта``, --- все увлекаются охотой и полагают, что она принадлежит им одним в силу давнего обычая и дарованной свободы, но другим они запрещают под угрозой лишения глаз охотиться на оленей, серн, зайцев, а в некоторых местах запрещают даже под угрозой лишения головы>>. Но откуда ведет свое происхождение <<спекулятивная философия>> с ее полемикой против индивидуального произвола, индивидуальных склонностей, индивидуальных представлений или мыслей? Она происходит прямым путем из казармы, или, что приблизительно то же самое: казарма ведь не что иное, как секуляризованные монастыри средневековья, из иезуитских коллегий. Человек казармы --- все равно, военный он или духовный, католик или протестант --- не должен, как он хочет и как ему следует сообразно его индивидуальности, есть, пить, ходить, спать, не должен соответственно действовать, чувствовать, мыслить; нет! всякий индивидуальный произвол уничтожен, то есть уничтожено всякое мышление, всякое чувствование, всякое хотение; ибо тот, кто отнимает у меня мою собственную, или индивидуальную, волю, тот не оставляет никакой воли, и кто не признает за мною права на собственные мысли, права на мой индивидуальный разум, тот и вообще отрицает за мною право на мысли и разум, ибо не существует общего разума, как нет общего желудка, хотя каждый точно так же имеет желудок, как и орган мышления или способность к мышлению. 

Предоставим слово иезуитам, чтобы убедиться, что иезуитизм есть бессознательный оригинал и идеал для наших спекулятивных философов, как он является и сознательным идеалом и оригиналом для наших отъявленных консервативных государственных искусников. Иезуит, говорится в правилах общества Иисуса, оказывает сопротивление естественной наклонности, присущей всем людям: иметь свое собственное суждение и следовать ему (письмо св. Игнатия <<О добродетели послушания>>); он должен со слепым послушанием отказаться от собственного мнения и убеждения; он должен быть, как палка, являющаяся безвольным орудием нашей руки, или как труп, с которым можно делать, что хочешь (<<Краткое изложение устава>>  э 35, 36). Совершенно верно! Уничтожение <<индивидуального произвола>>  стало быть уничтожение произвольного движения, есть уничтожение жизни. Спекулятивный философ, подобно иезуиту, подобно монархисту, --- смертельный враг жизни, ибо он превыше всякой меры любит <<порядок и спокойствие>>  чтобы не быть потревоженным в своих мыслях; жизнь же по существу беспокойна, беспорядочна, анархична, ее также не уловишь ограниченными понятиями философа, как и не справишься с ней ограниченными законами монарха. Но что же такое то общее, которому иезуит приносит в жертву спою индивидуальность, склонность, произвол и разум? Что такое это одинаковое, тождественное --- так как все должны одинаково знать, одинаково говорить согласно указанным правилам --- что такое, повторяю я, это одинаковое, тождественное у отдельных иезуитских индивидуумов? Это одинаковое, это общее есть не что иное, как воля, <<индивидуальный произвол>> настоятеля, который для иезуита есть наместник бога, то есть сам бог, то есть то же самое, что монарх для монархиста. Иезуит должен, говорит св. Игнатий, не только то же самое желать, но и то же самое чувствовать, что и настоятель, и его суждению подчинять свое. Вот видите, господин профессор, что отрицание одной индивидуальности есть утверждение другой, что общее есть то индивидуальное, которое, однако, имеет власть господствовать над другими индивидуумами, потому ли, что оно насильственно подавляет их индивидуальность, или потому, что соответствует их индивидуальной склонности, ибо иезуитизм точно так же предполагает особенную способность и склонность. <<Священное писание>> --- чтобы привести еще другой пример --- есть для христианина само писание; <<дух глаголет>>  говорит Лютер к стиху 40-го псалма: <<в книге мною написано>>  точно он не знает ни о какой другой книге (хотя мир ими полон), кроме этой книги <<Священного писания>>). Но разве <<Священное писание>>  которому христианин приносит в жертву свой субъективный или <<индивидуальный>> разум, не есть также индивидуальная книга? Разве представления Библии одинаковы с представлениями Корана, Вед, <<Зенд-Авесты>>? То, что является общим по отношению к христианину, не индивидуально ли по отношению к магометанину или индусу? Не превратилось ли то, что для наших верующих предков имело значение <<слова божьего>>  давно уже в человеческое слово? И как относительно различие между общим и индивидуальным! Что в данном месте и в данное время имеет значение <<индивидуального произвола>>  то в другом месте и в другое время есть общий закон. И что сегодня или здесь есть субъективное еретическое мнение, то там или завтра есть священный символ веры. У нас сейчас республика и анархический произвол, монархическая власть и законность тождественны; но у римлян монархическое было обозначением беззакония, произвола, распутства, высокомерия; там говорилось: <<Царская власть есть преступление>>. И не подтверждено ли это изречение историей, даже и германской? Не возникла ли и у нас монархия, хотя и в согласии с желаниями и интересами толпы и в противовес злу аристократического многовластия, из индивидуального властолюбия, индивидуальной жадности, индивидуальной страсти к убийствам? Не явилась ли у нас смертная казнь --- по крайней мере, для платежеспособных, свободных людей --- лишь вместе с королевской властью? (Вирт, Германская история). И не является ли в монархии --- по крайней мере, в настоящей, абсолютной монархии --- индивидуальный произвол монарха общим законом, его индивидуальная склонность --- общим обычаем? Не говорится ли: <<L'Etat c'est moi1>> и <<quails rex, talis>>. grex>>? Так и Линий в книге 5 говорит о <<толпе, которая почти всегда похожа на правителя>>. Правда, существует одно и притом весьма реальное различие между общим и индивидуальным, но никоим образом не в духе и не к пользе наших политических и спекулятивных абсолютистов. Индивидуальное есть --- согласно духу языка --- лишь то, что данный индивидуум или несколько индивидуумов в противоположность другим индивидуумам имеют и хотят, а общее есть то, что каждая личность, но в отдельности, каждый индивидуум, но индивидуальным образом, имеет и хочет, ибо, например, каждый имеет голову, но свою собственную, индивидуальную голову, каждый --- волю, но свою собственную, индивидуальную волю. И общее поэтому есть единичное, индивидуальное, но так как каждый его имеет, то мышление абстрагирует его от отдельных экземпляров, отождествляет его и выставляет как вещь для себя, но вещь, общую всем, --- представление, из которого затем получаются все дальнейшие мучительно схоластические и идеалистические затруднения и вопросы о взаимоотношении между общим и единичным. Короче говоря, мышление полагает дискретное действительности как непрерывное, бесконечную многократность жизни как тождественное однократное. Познание существенного, нестираемого различия между мышлением и жизнью (или действительностью) есть начало всякой премудрости в мышлении и жизни. Только различение является здесь истинной связью. Мы отличаем государство --- я имею в виду не современное государство, имеющее свое существование лишь в государственно-нивелированных индивидуумах, но государство вообще, --- мы отличаем нацию от индивидуумов. Но что же такое государство, что такое нация, если я не принимаю во внимание индивидуумов, образующих это государство, эту нацию? Государство есть не что иное, как то, чего хотят все, нация --- не что иное, как то, что все собою представляют, или, по крайней мере, чего хочет и что собою представляет большинство, потому что только большинство решает, только это, хотя и совершенно неопределенное и относительное, мерило имеет для нас сознательно и бессознательно --- значение меры всеобщности. 

Ни один закон, говорит у Ливия Катон в своей речи в защиту закона Оппия не удовлетворяет одинаково всех; поэтому дело идет о том, полезен ли он большинству и целому. Какое преступление, говорит Цицерон или кто другой, кто был автором сочинения <<К Гереннию>>  может быть сравнимо с преступлением государственной измены или измены отечеству? При всех других преступлениях ущерб касается лишь отдельных лиц или немногих, это же преступление причиняет самое страшное несчастье всем гражданам, разрушает счастье всех. Древние германцы не знали преступления против величества, но лишь <<преступление против нации>> (Эйхгорн, История германского государства и права). Но кто был этой нацией? Все свободные немцы. <<О сравнительно маловажных вещах совещаются между собою знатнейшие или князья, о более важных --- все>> (Тацит). <<При обсуждении некоторых вопросов каждый отдельный правомочный имел кроме права принимать участие в обсуждении еще и право абсолютного вето>> (Вирт, в цитируемое сочинение). <<Я не перестану, --- пишет Брут Цицерону, --- стремиться освободить наше государство от рабства. Если это дело мне удастся, то мы все будем радоваться, если же нет, то я все же буду рад, ибо какими поступками или мыслями мне завершить мою жизнь, как не теми, которые имеют своей целью освобождение моих сограждан?>> Следовательно, кто живет и умирает с идеей свободы, тот думает лишь о свободных людях, о свободных индивидуумах, хотя бы он о том или другом индивидууме как раз и не думал. Но разве вы, мой добрейший г-н профессор, думаете, что я, противопоставляя единичное всеобщему, индивидуальное родовому, я имею в виду лишь данное единичное и исключаю другое единичное, имею в виду этих индивидуумов и исключаю других, что я, стало быть, защищаю монархический и аристократический принцип, который до сих пор заявлял себя как общее и господствовал над миром? Как можете вы мне приписать подобную нелепость! Мой принцип охватывает всех индивидуумов: прошедших, настоящих и будущих: точка зрения индивидуальности есть точка зрения бесконечности и универсальности, <<дурной>> в смысле исполненного предрассудков и завистливого понимания, но весьма хорошей в смысле жизни, ибо это единственно творческая и производительная бесконечность и универсальность. В практическом отношении индивидуализм есть социализм, но социализм не в смысле французского социализма, упраздняющего индивидуальность или --- что одно и то же и что является лишь абстрактным ее выражением --- свободу. В заключение еще одно только слово о роде в естественноисторическом отношении. <<Животные в период течки, очевидно, демонстрируют как некую реальность, общность рода>>? Ни в малой мере. Течка у животных, сила полового влечения даже у человека демонстрирует нам не что-либо иное, как то, что демонстрирует нам и всякое другое сильное влечение. Гнев, задетое чувство самосохранения, неудовлетворенная потребность в еде, голод приводят к тем же последствиям, как и неудовлетворенное половое влечение, а именно: они приводят животных и людей в настоящую ярость и неистовство. Ведь говорится уже у Гомера о голоде: 

\begin{quote}
    
Нет страшней ничего и неукротимей, чем голод; 

Властно смертного он о себе вспоминать заставляет, 

Даже такого, чей дух отягчен унылой печалью. 

Так и со мной! Хоть грущу, а все ж постоянно 

Голод еды и питья требует с бешенством ярым: 

Только насытив его, я о муках других вспоминаю. 

\end{quote}

Поэтому, если течка демонстрирует реальность родовой всеобщности, то есть общего понятия, то и ярость голода демонстрирует родовую всеобщность моего желудка, бешенство гнева по поводу какого-либо нанесенного мне оскорбления или ранения --- родовую общность моего Я. Но половое влечение так мало дружит с философией, в особенности со спекулятивной, и так мало говорит в пользу реальности общих понятий, что можно сказать скорее обратное, а именно: что оно выражает доведенную до крайнего напряжения реальность индивидуальности, ибо лишь в нем --- в половом влечении --- находит индивидуальность свое завершение, в нем уходит полностью в плоть. Половое различие есть цветение, кульминационная точка индивидуальности, самое чувствительное место, точка чести индивидуальности, половое влечение есть самое честолюбивое и гордое влечение, влечение стать творцом, автором. Наивысшее сознание своей личности человек как духовно, так и физически испытывает лишь там, где он автор, ибо лишь тут заложено его отличие от других, лишь в этом месте создает он нечто новое, в остальном же он, лишенный мысли, лишенный своего Я, механический <<повторитель>>. Чем крупнее человек, тем в большей степени он индивидуальность. Чем менее духовно одарены индивидуумы, чем ниже они стоят, тем меньше различаются они между собой, тем в меньшей степени они индивидуальности. Что половое влечение имеет своим объектом существо, которое этому моему индивидуальному влечению как раз соответствует, --- это у полового влечения обще с другими влечениями. Природа вообще улавливается и воспринимается только через посредство самой себя, то есть через посредство однородного родственного: воздух --- через посредство легкого, этого, так сказать, наиболее воздушного органа, свет через посредство глаза, органа света, звук --- через посредство эластичных, приходящих в колебательное состояние орудий слуха, твердое, материальноечерез грубое орудие материалистического органа осязания, съедобное, питательное --- через органы еды. Процесс дыхания есть поэтому процесс оплодотворения легкого воздухом или его кислородом, зрение --- процесс оплодотворения глаза или глазных нервов светом. И это оплодотворение легкого воздухом, глаза --- светом, прочих влечении или их органов --- предметами столь же плодотворно, как и так называемое подлинное оплодотворение, причем каждое влечение производит продукт, соответствующий ему и его объекту. Творческая деятельность --- ведь сущность природы, сущность жизни. Легкое как орган дыхания производит горение, глаз как Друг света создает световые картины, половое же влечение, в качестве влечения женского и мужского, производит лишь особи мужского и женского пола. Но разве производителем является индивидуум? Разве не бог или род творит, производит детей? Почему же в таком случае столь многие индивидуумы гибнут при рождении детей, при родах? Откуда происходит известная подавленность после акта оплодотворения, если мое собственное существо не принимает в нем участия? Откуда происходит индивидуальное сходство детей с их родителями, если принципом размножения является род, <<самополагающаяся общность>>  а не индивидуальность? Конечно, я не могу дать жизнь детям, если у меня для этого отсутствует какое-либо известное или неизвестное органическое условие или способность; но я также не могу видеть, слышать, ходить, есть, мочиться, если у меня для этого отсутствуют необходимые органические предпосылки и способности, я вообще ничего не могу и являюсь пустым именем, если от меня возьмут другую часть моего Я, мое Не-я --- природу. <<В добрый час, будем рождать детей>>  --- сказал Карнеад своей новобрачной, но с таким же правом мы можем сказать, если мы страдаем олигурией: <<В добрый час, будем мочиться>> (да позволено будет так выразиться). Когда Лютер, у которого были камни, во время одной поездки мог мочиться, то он сказал: <<Радость заставила меня называть эту воду,-которую при других обстоятельствах считают самой ничтожной, самой драгоценной для меня>>  приписывая причину этого силе слез и молитв или, что то же самое, божественному милосердию. <<В эту ночь бог проявил на мне чудо и еще проявляет его благодаря молитвам благочестивых людей>>. Пусть спекулятивные, религиозные и политические враги человеческой индивидуальности позволят вымыть им голову этой драгоценной, более того, божественной водой Лютера. Им придется тогда утверждать, что мочеиспускание, так же как и деторождение, есть действие рода, или общего призрака, или же признать, что природа соединила в одном органе мочеиспускание и деторождение, чтобы наглядным образом показать, что деторождение является делом индивидуумов в такой же степени, как и мочеиспускание. Я, впрочем, уже раньше высказывался на эту тему; не хочу, однако, как само собою разумеется, ни у кого отнимать свободу, по произволу ограничивать понятие индивидуума, взять у него кишки из живота и вслед за тем опять набить пустое чучело богом, безымянной субстанцией или еще каким-либо чудовищем спекулятивной фантазии. Точно так же не хочу я этими своими замечаниями лишить моих противников и их публику удовольствия верить, что это их изображение моей личности есть мое существо, что их карикатура на меня есть мой портрет. 

\end{document}
